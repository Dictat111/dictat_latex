\documentclass{exam-zh}
\usepackage{siunitx}
\usepackage{array} % 用于指定列宽

% 定义新的列类型,指定列宽为 0.80cm 且内容居中
\newcolumntype{C}{>{\centering\arraybackslash}p{0.80cm}} %设置表格宽度

% 显示答案
\examsetup{
  page/size=a4paper,
  paren/show-paren=true,
  paren/show-answer=true,
  paren/text-color=red,
  fillin/show-answer=true,
  fillin/text-color=blue,
  solution/text-color=blue,
  solution/show-solution=show-stay,
  page/foot-content=第;页,共;页,
}

%学生版
\ExamPrintAnswerSet{
    page/size=a4paper,             % 试卷页面尺寸设为A4
    paren/show-paren=true,        % 显示选择题答案括号(如"( )")
    paren/show-answer=false,      % 隐藏选择题括号内答案
    paren/text-color=red,         % 选择题答案文字颜色(若显示则为红色)
    fillin/show-answer=false,     % 隐藏填空题答案
    fillin/text-color=blue,       % 填空题答案文字颜色(若显示则为蓝色)
    fillin/no-answer-type=none,   % 填空题隐藏答案时无占位符(去除黑三角)
    solution/show-solution=hide,  % 隐藏解答题解答内容
}

% 不显示答案(如需隐藏答案,可注释上方“显示答案”配置,启用下方代码)]


% \examsetup{
%   page/size=a4paper,             % 试卷页面尺寸设为A4
%   paren/show-paren=true,        % 显示选择题答案括号(如"( )")
%   paren/show-answer=false,      % 隐藏选择题括号内答案
%   paren/text-color=red,         % 选择题答案文字颜色(若显示则为红色)
%   fillin/show-answer=false,     % 隐藏填空题答案
%   fillin/text-color=blue,       % 填空题答案文字颜色(若显示则为蓝色)
%   fillin/no-answer-type=none,   % 填空题隐藏答案时无占位符(去除黑三角)
%   solution/show-solution=hide,  % 隐藏解答题解答内容
%   % solution/blank-type=manual,  % (可选)隐藏解答时手动加空白
%   % solution/blank-vsep=60ex plus 1ex minus 1ex % (可选)控制空白高度
% }
% \ExamPrintAnswerSet[\geometry{showframe}]{
% page/size=a3paper,
% solution/show-solution=show-stay,
% paren/show-paren=true,
% }

\everymath{\displaystyle}

\usepackage{tasks}
\usepackage{tabularx}


\begin{document} 

\title{杭州市中职数学思维等级证书\\红段答题($~$)}
\maketitle

\information{
考试时间: 45分钟,
满分: 100分
}

\section{15小题,每小题4分,共60分}

% 必须加载array宏包(用于自定义列宽和格式)
% \usepackage{array}

\begin{table}[h]
    \centering
    % 表格总宽度设为文本行宽,16列自动分配宽度
    % X类型列会根据内容自动调整宽度,保持表格整体占满行宽
    \begin{tabularx}{\linewidth}{|>{\centering\arraybackslash}X|*{15}{>{\centering\arraybackslash}X|}}
      \hline
      题号&1 & 2 & 3 & 4 & 5 & 6 & 7 & 8 & 9 & 10 & 11 & 12 & 13 & 14 & 15 \\
      \hline
      答案 &  &   &   &   &   &   &   &   &   &    &    &    &    &    &    \\
      \hline
    \end{tabularx}
    % 宽度说明:
    % 表格总宽度 = 文本行宽(\linewidth)
    % 16列按比例自动分配宽度(第1列"题号"与其他题目列宽度协调分配)
    % 无需手动计算列宽,自动适配不同页面设置
\end{table}






\begin{table}[h]
    \centering
    % 表格宽度设置为文本行宽(\linewidth)
    % X类型列会自动分配剩余宽度,实现表格整体占满行宽
    \begin{tabularx}{\linewidth}{|>{\centering\arraybackslash}X|>{\centering\arraybackslash}X|>{\centering\arraybackslash}X|>{\centering\arraybackslash}X|}
        \hline
        题号 & 答案 & 题号 & 答案 \\
        \hline
        16 &  & 19 &  \\
        \hline
        17 &  & 20 &  \\
        \hline
        18 &  & 21 &  \\
        \hline
    \end{tabularx}
    \caption{适配文本行宽的题目答案表}
\end{table}


% \clearpage
\begin{problem}[points = 10]
某商场试销一种成本为$60$元/件的服装,经试销发现,销售量$y$(件)与销售单价$x$(元)符合一次函数 $y = k x+b$ 关系,且当$x=65$时,$y=55$;当$x = 17$时,$y=45$.
    \begin{enumerate}
        \item 写出一次函数 $y = k x +b $的解析式.
        \item 若该商场获得的利润不低于$500$元,试确定销售单价$x$的取值范围.

    \end{enumerate}
\end{problem}
\begin{solution}
    \begin{enumerate}
        \item $y = -x +120$.\score{2}
        \item 设销售单价为$x$元时,该商场获得的利润为$z$元,由题意得\[
        z = (x-60) (-x+120) = -x^2 +180x-7200,\]
        令 $z = -x^2 +180x-7200 \ge 500$, 则 $x^2 -180+7700 \le 0$, 解得 $70 \le x\le 110$,
        所以x的取值范围是$[70,110]$.\score{8}

    
    \end{enumerate}
        
\end{solution}


\end{document}
