\documentclass[12pt,a4paper]{ctexart}
\usepackage{amsmath, amssymb, geometry}
\usepackage{multicol}
\usepackage{graphicx}
\geometry{margin=2.2cm}

\begin{document}

\begin{center}
    \LARGE \textbf{杭州市中职数学思维等级证书红段测试题}\\[6pt]
    \large 考试时间:45分钟 \quad 满分:100分
\end{center}

\vspace{1em}

\section*{一、选择题(每题4分,共60分)}

\begin{enumerate}
    \item 已知集合 $A = \{x \in \mathbb{N} \mid x < 5\}$,集合 $B = \{3,4,6\}$,则 $A \cup B =$()
    \begin{multicols}{4}
    \begin{enumerate}
        \item $\{3,4,6\}$ 
        \item $\{0,1,2,3,4,6\}$ 
        \item $\{1,2,3,4,5,6\}$ 
        \item $\{0,1,2,3,4,5,6\}$
    \end{enumerate}
    \end{multicols}

    \item 已知 $a > b$,$c > 0$,下列结论正确的是()
    \begin{multicols}{2}
    \begin{enumerate}
        \item $a - c < b - c$
        \item $ac > bc$
        \item $a + c < b + c$
        \item $\dfrac{a}{c} < \dfrac{b}{c}$
    \end{enumerate}
    \end{multicols}

    \item 倾斜角为 $\dfrac{\pi}{6}$,在 $y$ 轴上的截距为 $4$,求直线的方程()
    \begin{multicols}{2}
    \begin{enumerate}
        \item $x - \sqrt{3}y + 4\sqrt{3} = 0$
        \item $x + \sqrt{3}y + 4\sqrt{3} = 0$
        \item $x - \sqrt{3}y - 4\sqrt{3} = 0$
        \item $x + \sqrt{3}y - 4\sqrt{3} = 0$
    \end{enumerate}
    \end{multicols}

    \item 已知 $x \in (0,\pi)$,则 $\cos x = -\dfrac{1}{2}$ 的解为()
    \begin{multicols}{2}
    \begin{enumerate}
        \item $\dfrac{\pi}{3}$
        \item $\dfrac{2\pi}{3}$
        \item $\dfrac{\pi}{6}$
        \item $\dfrac{5\pi}{6}$
    \end{enumerate}
    \end{multicols}

    \item $\tan x > 0$ 是 $x$ 在第一或第三象限的()
    \begin{multicols}{2}
    \begin{enumerate}
        \item 充分不必要条件
        \item 必要不充分条件
        \item 充要条件
        \item 既不充分又不必要条件
    \end{enumerate}
    \end{multicols}

    \item 函数 $f(x)=\sqrt{\lg(x + 3)}+\dfrac{1}{\sqrt{3 - x}}$ 的定义域为()
    \begin{multicols}{2}
    \begin{enumerate}
        \item $(-3,3)$
        \item $(-3,3]$
        \item $[-3,3)$
        \item $(-3,4]$
    \end{enumerate}
    \end{multicols}

    \item 方程 $\sqrt{(x - 3)^2 + y^2}+\sqrt{(x + 3)^2 + y^2}=8$ 表示的()
    \begin{multicols}{2}
    \begin{enumerate}
        \item 线段
        \item 椭圆
        \item 直线
        \item 双曲线
    \end{enumerate}
    \end{multicols}

    \item 二次函数 $y = x^2 - 2x - 4$ 在区间 $[-1,2]$ 上的最大值、最小值分别为()
    \begin{multicols}{2}
    \begin{enumerate}
        \item $-1, -5$
        \item $-1, -4$
        \item $-4, -5$
        \item $-5, -4$
    \end{enumerate}
    \end{multicols}

    \item 如果角 $\alpha$ 的终边过点 $(5,m)$,且 $\cos\alpha=\dfrac{5}{13}$,则 $\tan(\pi - \alpha)=$()
    \begin{multicols}{2}
    \begin{enumerate}
        \item $-\dfrac{12}{5}$
        \item $\pm\dfrac{12}{5}$
        \item $\dfrac{5}{12}$
        \item $\pm\dfrac{5}{12}$
    \end{enumerate}
    \end{multicols}

    \item 已知向量 $\boldsymbol{a}=(4,-2)$,$\boldsymbol{b}=(2,3)$,则 $|\boldsymbol{a} + 2\boldsymbol{b}|=$()
    \begin{multicols}{2}
    \begin{enumerate}
        \item $(8,4)$
        \item $4\sqrt{5}$
        \item $(8,8)$
        \item $8\sqrt{2}$
    \end{enumerate}
    \end{multicols}

    \item 已知二次函数 $y=(x - a)(x + b)$ 的图象如图所示,则函数 $y = b + \log_{a}x$ 的大致图象是()
    
    (选项为图形,略)

    \item 已知 $\log_{a}\frac{1}{2}=m$,$\log_{a}3 = n$,则 $a^{m + n}=$()
    \begin{multicols}{2}
    \begin{enumerate}
        \item $\dfrac{3}{2}$
        \item $\dfrac{2}{3}$
        \item $6$
        \item $\dfrac{1}{6}$
    \end{enumerate}
    \end{multicols}

    \item 圆 $x^2 + y^2 - 6x - 8y + 16 = 0$ 与直线 $x + y - 5 = 0$ 相交于 $A,B$ 两点,则 $|AB|=$()
    \begin{multicols}{2}
    \begin{enumerate}
        \item $2\sqrt{2}$
        \item $4\sqrt{2}$
        \item $3\sqrt{2}$
        \item $5\sqrt{2}$
    \end{enumerate}
    \end{multicols}

    \item 5封信投入6个邮筒,每个邮筒最多投1封,共有不同的投法种数为()
    \begin{multicols}{2}
    \begin{enumerate}
        \item $6$
        \item $30$
        \item $120$
        \item $7776$
    \end{enumerate}
    \end{multicols}

    \item 抛掷两颗均匀的骰子,出现点数之和为 $7$ 的概率是()
    \begin{multicols}{2}
    \begin{enumerate}
        \item $\dfrac{1}{6}$
        \item $\dfrac{1}{9}$
        \item $\dfrac{1}{12}$
        \item $\dfrac{1}{3}$
    \end{enumerate}
    \end{multicols}
\end{enumerate}

\section*{二、填空题(每题4分,共24分)}

\begin{enumerate}
    \setcounter{enumi}{15}
    \item 已知函数 
    \[
    f(x)=
    \begin{cases}
        x^2 + x - 4, & x \leq 0 \\
        \log_{4}x, & x > 0
    \end{cases}
    \]
    则函数 $f(x)$ 的图像与 $x$ 轴交点坐标为 \underline{\hspace{3cm}}。

    \item 已知 $\sin(x + \pi)\cos(x - \dfrac{\pi}{2}) = 4a + 1$,则实数 $a$ 的取值范围是 \underline{\hspace{3cm}}。

    \item 若直线 $4x + 5y - 5 = 0$ 和直线 $ax - y + 2 = 0$ 垂直,则实数 $a$ 为 \underline{\hspace{3cm}}。

    \item 若直线 $\dfrac{x}{a}+\dfrac{y}{b}=1\ (a > 0,b > 0)$ 经过点 $(4,3)$,则 $4a + 3b$ 的最小值为 \underline{\hspace{3cm}}。

    \item 若圆锥的轴截面是面积为 $16$ 的等边三角形,则此圆锥的侧面积为 \underline{\hspace{3cm}}。

    \item 计算:
    \[
    4\log_{4}\dfrac{1}{16}+\left(\dfrac{64}{27}\right)^{\frac{1}{3}}+\lg1000+(\sqrt{3} - 1)\ln e=
    \underline{\hspace{3cm}}。
    \]
\end{enumerate}

\section*{三、解答题(每题8分,共16分)}

\begin{enumerate}
    \setcounter{enumi}{21}
    \item (8分)已知椭圆的中心在原点,焦点在 $x$ 轴上,长轴长为 $4\sqrt{2}$,离心率为 $\dfrac{\sqrt{2}}{2}$,它与直线 $y = x - 3$ 相交于 $A,B$ 两点,且 $F_1$ 是椭圆的左焦点。求:
    \begin{enumerate}
        \item 椭圆的标准方程;
        \item $\triangle AF_1B$ 的面积。
    \end{enumerate}

    \item (8分)某中职学校根据教育部《国家学生体质健康标准》对学生进行体质健康测试。已知某学生投掷的实心球从起掷点开始,到着地点为止,在空中运动的高度 $y$(米)与水平距离 $x$(米)符合二次函数关系。

    如图所示,点 $A$ 为二次函数图像与 $y$ 轴的交点,点 $B$ 为实心球的着地点,点 $C$ 为最高点。测得:
    \[
    OA = \frac{7}{5}\text{米}, \quad OC = 4\text{米}, \quad OB = 9\text{米}.
    \]
    求:
    \begin{enumerate}
        \item 点 $A,B$ 的坐标;
        \item 高度 $y$ 与水平距离 $x$ 的二次函数解析式;
        \item 实心球在运动过程中离地面的最大高度。
    \end{enumerate}
\end{enumerate}

\end{document}
