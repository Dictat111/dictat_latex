\documentclass{exam-zh}
\usepackage{siunitx}
\usepackage{array}
\usepackage{verbatim} % 用于代码框展示解析
\usepackage{amsmath,amssymb}

% 页面与格式配置
\newcolumntype{C}{>{\centering\arraybackslash}p{0.80cm}}
\examsetup{
  page/size=a4paper,
  paren/text-color=red,
  fillin/text-color=blue,
  solution/text-color=blue,
}

\everymath{\displaystyle}
\usepackage{tasks}
\usepackage{tabularx}

\begin{document}

\title{专升本高等数学测试卷——答案与详细解析}
\maketitle

\information{
  适用范围:专升本数学(极限、连续、一元导数、不定积分)\\
  解析说明:每题解析含“考点”“思路”“步骤”三部分,关键公式标注清晰
}

% 一、选择题解析
\section*{一、选择题(每题3分,共30分)}

\subsection*{1. 极限 $\lim\limits_{x \to 0} \dfrac{\sin 5x}{x}$ 的值}
\noindent \textbf{答案:C(5)}
\begin{verbatim}
【考点】重要极限($\lim\limits_{t \to 0} \dfrac{\sin t}{t} = 1$)
【思路】通过“凑项”将原式变形为重要极限的形式,令 $t = 5x$,当 $x \to 0$ 时 $t \to 0$。
【解析步骤】
$\lim\limits_{x \to 0} \dfrac{\sin 5x}{x} = \lim\limits_{x \to 0} \left( \dfrac{\sin 5x}{5x} \times 5 \right)$
= $5 \times \lim\limits_{t \to 0} \dfrac{\sin t}{t}$(令 $t = 5x$)
= $5 \times 1 = 5$
\end{verbatim}

\subsection*{2. 函数 $f(x) = \dfrac{x^2 - 4}{x - 2}$ 在 $x = 2$ 处的极限}
\noindent \textbf{答案:A(4)}
\begin{verbatim}
【考点】分式函数极限(可去间断点)
【思路】$x \to 2$ 时分子分母均为0($\frac{0}{0}$型),先因式分解消去零因子。
【解析步骤】
分子因式分解:$x^2 - 4 = (x - 2)(x + 2)$
原式 = $\lim\limits_{x \to 2} \dfrac{(x - 2)(x + 2)}{x - 2}$($x \neq 2$,可消去 $x - 2$)
= $\lim\limits_{x \to 2} (x + 2) = 2 + 2 = 4$
\end{verbatim}

\subsection*{3. 极限 $\lim\limits_{x \to \infty} \left(1 + \dfrac{3}{x}\right)^{2x}$ 的值}
\noindent \textbf{答案:D($e^6$)}
\begin{verbatim}
【考点】重要极限($\lim\limits_{t \to \infty} \left(1 + \dfrac{1}{t}\right)^t = e$)
【思路】令 $t = \dfrac{x}{3}$(即 $x = 3t$),将原式凑成重要极限形式。
【解析步骤】
原式 = $\lim\limits_{x \to \infty} \left(1 + \dfrac{3}{x}\right)^{2x}$
= $\lim\limits_{t \to \infty} \left(1 + \dfrac{1}{t}\right)^{2 \times 3t}$(令 $t = \dfrac{x}{3}$,$x \to \infty$ 时 $t \to \infty$)
= $\lim\limits_{t \to \infty} \left[ \left(1 + \dfrac{1}{t}\right)^t \right]^6$
= $e^6$(由重要极限 $\lim\limits_{t \to \infty} \left(1 + \dfrac{1}{t}\right)^t = e$)
\end{verbatim}

\subsection*{4. 函数 $f(x) = \dfrac{1}{\ln(x - 1)}$ 的连续区间}
\noindent \textbf{答案:B($(1,2) \cup (2, +\infty)$)}
\begin{verbatim}
【考点】函数连续性(定义域与间断点)
【思路】连续区间 = 定义域 - 间断点,需满足:①分母不为0;②对数真数大于0。
【解析步骤】
1. 对数真数要求:$x - 1 > 0 \implies x > 1$
2. 分母不为0要求:$\ln(x - 1) \neq 0 \implies x - 1 \neq 1 \implies x \neq 2$
3. 综上,定义域为 $x > 1$ 且 $x \neq 2$,即连续区间为 $(1,2) \cup (2, +\infty)$
\end{verbatim}

\subsection*{5. 设 $y = x^3 \cos x$,则 $y'$}
\noindent \textbf{答案:C($3x^2 \cos x - x^3 \sin x$)}
\begin{verbatim}
【考点】导数的乘积法则($(uv)' = u'v + uv'$)
【思路】令 $u = x^3$,$v = \cos x$,分别求导后代入乘积法则。
【解析步骤】
1. 求 $u'$:$u = x^3$,由幂函数求导公式 $(x^n)' = nx^{n-1}$,得 $u' = 3x^2$
2. 求 $v'$:$v = \cos x$,由基本导数公式 $(\cos x)' = -\sin x$,得 $v' = -\sin x$
3. 乘积法则:$y' = u'v + uv' = 3x^2 \cdot \cos x + x^3 \cdot (-\sin x)$
= $3x^2 \cos x - x^3 \sin x$
\end{verbatim}

\subsection*{6. 设 $y = \ln(1 + x^2)$,则 $dy$}
\noindent \textbf{答案:A($\dfrac{2x}{1 + x^2}dx$)}
\begin{verbatim}
【考点】复合函数求导与微分($dy = y'dx$)
【思路】先求 $y'$(复合函数:令 $u = 1 + x^2$,$y = \ln u$),再乘 $dx$ 得 $dy$。
【解析步骤】
1. 复合函数求导:$y' = \dfrac{d}{du}(\ln u) \cdot \dfrac{d}{dx}(u)$(链式法则)
   - $\dfrac{d}{du}(\ln u) = \dfrac{1}{u}$
   - $\dfrac{d}{dx}(u) = \dfrac{d}{dx}(1 + x^2) = 2x$
   - 故 $y' = \dfrac{1}{1 + x^2} \cdot 2x = \dfrac{2x}{1 + x^2}$
2. 微分公式:$dy = y'dx = \dfrac{2x}{1 + x^2}dx$
\end{verbatim}

\subsection*{7. 曲线 $y = x^2 - 4x + 3$ 在点 $(2, -1)$ 处的切线方程}
\noindent \textbf{答案:B($y = -1$)}
\begin{verbatim}
【考点】导数的几何意义(切线斜率 = 该点导数)
【思路】1. 求切线斜率 $k = y'\big|_{x=2}$;2. 用点斜式 $y - y_0 = k(x - x_0)$ 写方程。
【解析步骤】
1. 求导数:$y = x^2 - 4x + 3$,$y' = 2x - 4$
2. 切线斜率:$k = y'\big|_{x=2} = 2 \times 2 - 4 = 0$(斜率为0,切线水平)
3. 点斜式:$y - (-1) = 0 \times (x - 2) \implies y + 1 = 0 \implies y = -1$
\end{verbatim}

\subsection*{8. 函数 $f(x) = x^3 - 3x + 1$ 在区间 $[-2, 2]$ 上的极小值}
\noindent \textbf{答案:D(-1)}
\begin{verbatim}
【考点】函数极值(导数判极值:一阶导数为0,二阶导数定正负)
【思路】1. 求驻点($f'(x) = 0$);2. 判驻点是否为极值点;3. 计算极值。
【解析步骤】
1. 求一阶导数:$f'(x) = 3x^2 - 3 = 3(x^2 - 1)$
2. 求驻点:令 $f'(x) = 0 \implies x^2 - 1 = 0 \implies x = 1$ 或 $x = -1$(均在 $[-2,2]$ 内)
3. 二阶导数判极值:$f''(x) = 6x$
   - 当 $x = 1$ 时:$f''(1) = 6 \times 1 = 6 > 0$,故 $x=1$ 是极小值点
   - 当 $x = -1$ 时:$f''(-1) = 6 \times (-1) = -6 < 0$,故 $x=-1$ 是极大值点
4. 计算极小值:$f(1) = 1^3 - 3 \times 1 + 1 = 1 - 3 + 1 = -1$
\end{verbatim}

\subsection*{9. 不定积分 $\int (x^2 + \sin x) dx$}
\noindent \textbf{答案:C($\dfrac{x^3}{3} - \cos x + C$)}
\begin{verbatim}
【考点】不定积分的基本公式与可加性($\int (u + v)dx = \int udx + \int vdx$)
【思路】分别积分两项,再相加,注意加积分常数 $C$。
【解析步骤】
1. 积分第一项:$\int x^2 dx$,由幂函数积分公式 $\int x^n dx = \dfrac{x^{n+1}}{n+1} + C$($n \neq -1$)
   得 $\int x^2 dx = \dfrac{x^3}{3} + C_1$
2. 积分第二项:$\int \sin x dx$,由基本积分公式 $\int \sin x dx = -\cos x + C_2$
3. 合并结果:$\int (x^2 + \sin x)dx = \dfrac{x^3}{3} - \cos x + (C_1 + C_2)$
   令 $C = C_1 + C_2$,得 $\dfrac{x^3}{3} - \cos x + C$
\end{verbatim}

\subsection*{10. 不定积分 $\int x e^{x^2} dx$}
\noindent \textbf{答案:A($\dfrac{1}{2} e^{x^2} + C$)}
\begin{verbatim}
【考点】换元积分法(第一类换元:凑微分)
【思路】令 $u = x^2$,则 $du = 2x dx \implies x dx = \dfrac{1}{2} du$,代入积分式。
【解析步骤】
1. 凑微分:令 $u = x^2$,则 $du = 2x dx \implies x dx = \dfrac{1}{2} du$
2. 代入原式:$\int x e^{x^2} dx = \int e^u \cdot \dfrac{1}{2} du = \dfrac{1}{2} \int e^u du$
3. 基本积分:$\int e^u du = e^u + C$,故 $\dfrac{1}{2} \int e^u du = \dfrac{1}{2} e^u + C$
4. 回代 $u = x^2$:$\dfrac{1}{2} e^{x^2} + C$
\end{verbatim}


% 二、填空题解析
\section*{二、填空题(每题3分,共30分)}

\subsection*{1. 极限 $\lim\limits_{x \to 1} \dfrac{x^2 - 2x + 1}{x - 1}$}
\noindent \textbf{答案:0}
\begin{verbatim}
【考点】$\frac{0}{0}$型极限(因式分解消零因子)
【解析】
分子因式分解:$x^2 - 2x + 1 = (x - 1)^2$
原式 = $\lim\limits_{x \to 1} \dfrac{(x - 1)^2}{x - 1} = \lim\limits_{x \to 1} (x - 1) = 1 - 1 = 0$
\end{verbatim}

\subsection*{2. 极限 $\lim\limits_{x \to 0} \dfrac{\sqrt{1 + x} - 1}{x}$}
\noindent \textbf{答案:$\dfrac{1}{2}$}
\begin{verbatim}
【考点】$\frac{0}{0}$型极限(分子有理化)
【解析】
分子有理化(乘共轭式 $\sqrt{1 + x} + 1$):
原式 = $\lim\limits_{x \to 0} \dfrac{(\sqrt{1 + x} - 1)(\sqrt{1 + x} + 1)}{x(\sqrt{1 + x} + 1)}$
= $\lim\limits_{x \to 0} \dfrac{(1 + x) - 1}{x(\sqrt{1 + x} + 1)}$(分子用平方差公式)
= $\lim\limits_{x \to 0} \dfrac{x}{x(\sqrt{1 + x} + 1)} = \lim\limits_{x \to 0} \dfrac{1}{\sqrt{1 + x} + 1}$
= $\dfrac{1}{\sqrt{1 + 0} + 1} = \dfrac{1}{2}$
\end{verbatim}

\subsection*{3. 函数 $f(x) = \begin{cases} x + 1, & x \leq 0 \\ e^x + a, & x > 0 \end{cases}$ 在 $x = 0$ 处连续,求 $a$}
\noindent \textbf{答案:0}
\begin{verbatim}
【考点】分段函数连续性(左右极限相等且等于函数值)
【解析】
1. 左极限($x \to 0^-$):$\lim\limits_{x \to 0^-} f(x) = \lim\limits_{x \to 0^-} (x + 1) = 0 + 1 = 1$
2. 右极限($x \to 0^+$):$\lim\limits_{x \to 0^+} f(x) = \lim\limits_{x \to 0^+} (e^x + a) = e^0 + a = 1 + a$
3. 连续性条件:左极限 = 右极限,即 $1 = 1 + a \implies a = 0$
\end{verbatim}

\subsection*{4. 函数 $f(x) = x^2 - 4x + 5$ 的单调递增区间}
\noindent \textbf{答案:$(2, +\infty)$(或 $[2, +\infty)$)}
\begin{verbatim}
【考点】函数单调性(导数符号判单调区间)
【解析】
1. 求导数:$f'(x) = 2x - 4$
2. 单调递增条件:$f'(x) > 0$(可导函数递增 $\iff$ 导数正,端点不影响区间单调性)
   解不等式:$2x - 4 > 0 \implies x > 2$
3. 故单调递增区间为 $(2, +\infty)$(包含端点 $[2, +\infty)$ 也正确,常规用开区间)
\end{verbatim}

\subsection*{5. 设 $y = \sin(3x - 1)$,则 $y''$}
\noindent \textbf{答案:$-9\sin(3x - 1)$}
\begin{verbatim}
【考点】复合函数的二阶导数
【解析】
1. 求一阶导数 $y'$(
