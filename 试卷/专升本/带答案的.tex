\documentclass{exam-zh}
\usepackage{siunitx}
\usepackage{array} % 用于指定列宽

% 定义新的列类型,指定列宽为 0.80cm 且内容居中
\newcolumntype{C}{>{\centering\arraybackslash}p{0.80cm}} %设置表格宽度

% % 显示答案
% \examsetup{
%   page/size=a4paper,
%   paren/show-paren=true,
%   paren/show-answer=true,
%   paren/text-color=red,
%   fillin/show-answer=true,
%   fillin/text-color=blue,
%   solution/text-color=blue,
%   solution/show-solution=show-stay,
% }

% 不显示答案(如需隐藏答案,可注释上方“显示答案”配置,启用下方代码)
\examsetup{
  page/size=a4paper,
  paren/show-paren=true,
  paren/show-answer=false,
  paren/text-color=red,
  fillin/show-answer=false,
  fillin/text-color=blue,
  fillin/no-answer-type=none, %去除横线上的三角
  solution/show-solution=hide,
  solution/blank-type=manual, % 不显示solution 的时候添加空白
  solution/blank-vsep=60ex plus 1ex minus 1ex %控制添加空白的大小
}

\everymath{\displaystyle}

\usepackage{tasks}
\usepackage{tabularx}

\begin{document} 

\title{专升本高等数学测试卷——极限、连续与一元函数微积分}
\maketitle

\information{
班级\underline{\hspace{6em}},
姓名\underline{\hspace{6em}},
学号\underline{\hspace{6em}}
}

\section{选择题:本题共 10 小题,每小题 3 分,共 30 分。}

\begin{question}
    极限 $\lim\limits_{x \to 0} \dfrac{\sin 5x}{x}$ 的值为\paren[C]
    \begin{choices}
    \item $0$
    \item $1$
    \item $5$
    \item $\dfrac{1}{5}$
    \end{choices}
\end{question}

\begin{question}
    函数 $f(x) = \dfrac{x^2 - 4}{x - 2}$ 在 $x = 2$ 处的极限为\paren[A]
    \begin{choices}
    \item $4$
    \item $2$
    \item $0$
    \item 不存在
    \end{choices}
\end{question}

\begin{question}
  极限 $\lim\limits_{x \to \infty} \left(1 + \dfrac{3}{x}\right)^{2x}$ 的值为\paren[D]
  \begin{choices}
  \item $e^3$
  \item $e^2$
  \item $e^5$
  \item $e^6$
  \end{choices}
\end{question}

\begin{question}
  函数 $f(x) = \dfrac{1}{\ln(x-1)}$ 的连续区间是\paren[B]
  \begin{choices}
  \item $(1, +\infty)$
  \item $(1,2) \cup (2, +\infty)$
  \item $[1,2) \cup (2, +\infty)$
  \item $(1,2]$
  \end{choices}
\end{question}

\begin{question}
  设 $y = x^3 \cos x$,则 $y'$ 等于\paren[C]
  \begin{choices}
  \item $3x^2 \cos x$
  \item $-x^3 \sin x$
  \item $3x^2 \cos x - x^3 \sin x$
  \item $3x^2 \cos x + x^3 \sin x$
  \end{choices}
\end{question}

\begin{question}
  设 $y = \ln(1 + x^2)$,则 $dy$ 等于\paren[A]
  \begin{choices}
  \item $\dfrac{2x}{1 + x^2}dx$
  \item $\dfrac{2x}{1 + x^2}$
  \item $\dfrac{1}{1 + x^2}dx$
  \item $\dfrac{1}{1 + x^2}$
  \end{choices}
\end{question}

\begin{question}
  曲线 $y = x^2 - 4x + 3$ 在点 $(2, -1)$ 处的切线方程为\paren[B]
  \begin{choices}
  \item $y = 2x - 5$
  \item $y = -1$
  \item $x = 2$
  \item $y = 4x - 9$
  \end{choices}
\end{question}

\begin{question}
  函数 $f(x) = x^3 - 3x + 1$ 在区间 $[-2, 2]$ 上的极小值为\paren[D]
  \begin{choices}
  \item $3$
  \item $-1$
  \item $-3$
  \item $-1$
  \end{choices}
\end{question}

\begin{question}
  不定积分 $\int (x^2 + \sin x) dx$ 等于\paren[C]
  \begin{choices}
  \item $\dfrac{x^3}{3} + \cos x + C$
  \item $2x - \cos x + C$
  \item $\dfrac{x^3}{3} - \cos x + C$
  \item $2x + \cos x + C$
  \end{choices}
\end{question}

\begin{question}
  不定积分 $\int x e^{x^2} dx$ 等于\paren[A]
  \begin{choices}
  \item $\dfrac{1}{2} e^{x^2} + C$
  \item $e^{x^2} + C$
  \item $2 e^{x^2} + C$
  \item $\dfrac{1}{2} e^x + C$
  \end{choices}
\end{question}

\section{填空题:本题共 10 小题,每小题 3 分,共 30 分。}

\begin{question}
  极限 $\lim\limits_{x \to 1} \dfrac{x^2 - 2x + 1}{x - 1} =$ \fillin[$0$].
\end{question}

\begin{question}
  极限 $\lim\limits_{x \to 0} \dfrac{\sqrt{1 + x} - 1}{x} =$ \fillin[$\dfrac{1}{2}$].
\end{question}

\begin{question}
  若函数 $f(x) = \begin{cases} 
  x + 1, & x \leq 0 \\
  e^x + a, & x > 0 
  \end{cases}$ 在 $x = 0$ 处连续,则 $a =$ \fillin[$0$].
\end{question}
% %
\begin{question}
    函数 $f(x) = x^2 - 4x + 5$ 的单调递增区间是 \fillin [$\lbrack 2, +\infty)$].
  \end{question}
  
\begin{question}
  设 $y = \sin(3x - 1)$,则 $y'' =$ \fillin[$-9\sin(3x - 1)$].
\end{question}

\begin{question}
  曲线 $y = x^3 - 3x$ 上切线斜率为 $0$ 的点的坐标是 \fillin[$(1, -2)$ 和 $(-1, 2)$].
\end{question}

\begin{question}
  函数 $f(x) = x + \dfrac{1}{x}$ 在区间 $[1, 2]$ 上的最大值为 \fillin[$\dfrac{5}{2}$].
\end{question}

\begin{question}
  不定积分 $\int \dfrac{1}{x^2} dx =$ \fillin[$-\dfrac{1}{x} + C$].
\end{question}

\begin{question}
  不定积分 $\int \cos(2x + 3) dx =$ \fillin[$\dfrac{1}{2}\sin(2x + 3) + C$].
\end{question}

\begin{question}
  不定积分 $\int x \ln x dx =$ \fillin[$\dfrac{1}{2}x^2 \ln x - \dfrac{1}{4}x^2 + C$].
\end{question}

% 选择题填涂区域
\section*{一、选择题}
\begin{table}[h]
\centering
\begin{tabular}{|C|C|C|C|C|C|C|C|C|C|}
  \hline
  1 & 2 & 3 & 4 & 5 & 6 & 7 & 8 & 9 & 10 \\
  \hline
    &   &   &   &   &   &   &   &   &    \\
  \hline
\end{tabular}
\end{table}

% 填空题答题区域
\section*{一、填空题}
\begin{tasks}[label=\arabic*.](5)
  \task \underline{\hspace{2cm}}
  \task \underline{\hspace{2cm}}
  \task \underline{\hspace{2cm}}
  \task \underline{\hspace{2cm}}
  \task \underline{\hspace{2cm}}
  \task \underline{\hspace{2cm}}
  \task \underline{\hspace{2cm}}
  \task \underline{\hspace{2cm}}
  \task \underline{\hspace{2cm}}
  \task \underline{\hspace{2cm}}
\end{tasks}  

\clearpage

\section{解答题:本题共 4 小题,共 40 分。解答应写出文字说明、证明过程或者演算步骤。}

\begin{problem}[points = 10]
  计算下列极限:

  \begin{tasks}[label=(\arabic*)](2)
      \task $\lim\limits_{x \to 2} \dfrac{x^2 - 5x + 6}{x^2 - 4}$;
      \task $\lim\limits_{x \to 0} \dfrac{1 - \cos 2x}{x \sin x}$.
  \end{tasks}
\end{problem}

\begin{solution}\\
  (1) 
  \begin{align*}
  \lim\limits_{x \to 2} \dfrac{x^2 - 5x + 6}{x^2 - 4} &= \lim\limits_{x \to 2} \dfrac{(x-2)(x-3)}{(x-2)(x+2)} \\
  &= \lim\limits_{x \to 2} \dfrac{x-3}{x+2} = \dfrac{2-3}{2+2} = -\dfrac{1}{4}.
  \end{align*}

  (2)
    \begin{align*}
  \lim\limits_{x \to 0} \dfrac{1 - \cos 2x}{x \sin x} &= \lim\limits_{x \to 0} \dfrac{2\sin^2 x}{x \sin x} \\
  &= \lim\limits_{x \to 0} \dfrac{2\sin x}{x} = 2.
    \end{align*}
\end{solution}

\begin{problem}[points = 10]
  设函数 $f(x) = \begin{cases} 
  x^2, & x \leq 1 \\
  ax + b, & x > 1 
  \end{cases}$,试确定 $a, b$ 的值,使 $f(x)$ 在 $x = 1$ 处可导。
\end{problem}

\begin{solution}
  函数在 $x=1$ 处可导的必要条件是在该点连续:\\
  $\lim\limits_{x \to 1^-} f(x) = 1^2 = 1$,$\lim\limits_{x \to 1^+} f(x) = a \cdot 1 + b = a + b$,\\
  由连续性得 $a + b = 1$。

  函数在 $x=1$ 处可导需左右导数相等:\\
  左导数:$f'_-(1) = \lim\limits_{h \to 0^-} \dfrac{(1+h)^2 - 1}{h} = \lim\limits_{h \to 0^-} (2 + h) = 2$,\\
  右导数:$f'_+(1) = \lim\limits_{h \to 0^+} \dfrac{a(1+h) + b - 1}{h} = a$,\\
  由可导性得 $a = 2$,代入 $a + b = 1$ 得 $b = -1$。

  故 $a = 2$,$b = -1$ 时,$f(x)$ 在 $x = 1$ 处可导。
\end{solution}

\clearpage

\begin{problem}[points = 10]
  求下列函数的导数或微分:
  \begin{tasks}[label=(\arabic*)](2)
    \task 设 $y = x^2 e^x$,求 $y'$;
    \task 设 $y = \ln \sqrt{1 + x^2}$,求 $dy$。
  \end{tasks}
\end{problem}

\begin{solution}
  (1) 由乘积法则:
  \[
  y' = (x^2)' e^x + x^2 (e^x)' = 2x e^x + x^2 e^x = e^x (x^2 + 2x)
  \]

  (2) 先化简函数:$y = \dfrac{1}{2} \ln(1 + x^2)$,则
  \[
  y' = \dfrac{1}{2} \cdot \dfrac{2x}{1 + x^2} = \dfrac{x}{1 + x^2}
  \]
  故 $dy = \dfrac{x}{1 + x^2} dx$
\end{solution}

\begin{problem}[points = 10]
  计算下列不定积分:
  \begin{tasks}[label=(\arabic*)](2)
    \task $\int x \cos 2x dx$;
    \task $\int \ln x dx$。
  \end{tasks}
\end{problem}

\begin{solution}
  (1) 使用分部积分法,设 $u = x$,$dv = \cos 2x dx$,则 $du = dx$,$v = \dfrac{1}{2} \sin 2x$
  \[
  \int x \cos 2x dx = \dfrac{1}{2} x \sin 2x - \dfrac{1}{2} \int \sin 2x dx = \dfrac{1}{2} x \sin 2x + \dfrac{1}{4} \cos 2x + C
  \]

  (2) 使用分部积分法,设 $u = \ln x$,$dv = dx$,则 $du = \dfrac{1}{x} dx$,$v = x$
  \[
  \int \ln x dx = x \ln x - \int x \cdot \dfrac{1}{x} dx = x \ln x - x + C
  \]
\end{solution}

\end{document}
