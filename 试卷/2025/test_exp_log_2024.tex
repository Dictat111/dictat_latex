\documentclass{exam-zh}
\usepackage{siunitx}
\usepackage{array} % 用于指定列宽

% 这个在win11上是没有问题的,因为那个exam-zh的包用的是 exam-zh 的 2024-02-15 版本(以文档为例) 现在的问题是要怎么更新包
%我手动下载了包,然后替换了原来的文件,然后用 sudo texhash 更新了库


% 定义新的列类型,指定列宽为 0.80cm 且内容居中
\newcolumntype{C}{>{\centering\arraybackslash}p{0.80cm}} %设置表格宽度

% doubao 绘制试卷网格的提示词
%我需要latex绘制表格,一共有20个,分为两层,上面一层写上1到10的数字,下面是空的,网格一样大小,指定列宽为0.80cm

\title{高一数学统测测试卷——第五章 指数函数与对数函数}
% \examsetup{
%   paren/show-paren=true,%显示 paren 命令的括号
%   paren/show-answer=true,%显示答案
%   fillin/no-answer-type=none,
%   fillin/width = 5em % 设置填空横线长度,默认是3em
% }

%显示答案
\examsetup{
  page/size=a4paper,
  paren/show-paren=true,
  paren/show-answer=true,
  paren/text-color=red,
  fillin/show-answer=true,
  fillin/text-color=blue,
  solution/text-color=blue,
  solution/show-solution=show-stay,
}


% 不显示答案
% \examsetup{
%   page/size=a4paper,
%   paren/show-paren=true,
%   paren/show-answer=false,
%   paren/text-color=red,
%   fillin/show-answer=false,
%   fillin/text-color=blue,
%   fillin/no-answer-type=none, %去除横线上的三角
%   solution/show-solution=hide,
%   solution/blank-type=manual, % 不显示solution 的时候添加空白
%   solution/blank-vsep=60ex plus 1ex minus 1ex %控制添加空白的大小
% }


\everymath{\displaystyle}





\usepackage{tasks}
\usepackage{tabularx}
%要怎么绘制填空题和选择题的填涂区域呢?
%怎么统计
 \begin{document} 

\maketitle
 \information{
班级\underline{\hspace{6em}},
姓名\underline{\hspace{6em}},
学号\underline{\hspace{6em}}
}

    \section{选择题:本题共 10 小题,每小题 3 分,共 30 分。}



    \begin{question}
        计算 $(-64)^{\frac{1}{3}} = $\paren[B]
        \begin{choices}
        \item $4$
        \item $-4$
        \item $\frac{1}{4}$
        \item $-\frac{1}{4}$
        \end{choices}
    \end{question}



    \begin{question}
        计算 $ 2^{\frac{1}{2}} \times 2^{\frac{1}{3}} \div 2^{\frac{1}{4}}    = $\paren[D]
        \begin{choices}
        \item $2^{\frac{13}{12}}$
        \item $2^{\frac{2}{3}}$
        \item $\frac{12}{7}$
        \item $\sqrt[12]{2^7}$
        \end{choices}
    \end{question}




    \begin{question}
      若指数函数 $f(x) = a^x$ 的图像经过$(2,2)$,则$f(8) = $\paren[B]
      \begin{choices}
      \item $8$
      \item $16$
      \item $32$
      \item $64$
      \end{choices}
    \end{question}



    
    \begin{question}
      函数 $y =  a^{x-3} +1 (a>0 \text{~且~}  a \neq 1 )$ 的图像恒过定点\paren[B]
      \begin{choices}
      \item $(3,1)$
      \item $(3,2)$
      \item $(3,3)$
      \item $(3,-1)$
      \end{choices}
    \end{question}


    \begin{question}
      下列函数在其定义域内单调递减的是\paren[C]
      \begin{choices}
      \item $y =\frac{1}{2} x $
      \item $y = 2^x$
      \item $y = 2^{-x}$
      \item $y = -x^{-1}$
      \end{choices}
    \end{question}


    
    \begin{question}
      设$(\frac{1}{2})^b < (\frac{1}{2})^a < 1 $,则下列选项正确的是\paren[D]
      \begin{choices}
      \item $b<a<1 $
      \item $b>a>1 $
      \item $b<a<0 $
      \item $b>a>0 $
      \end{choices}
    \end{question}

% 6




\begin{question}
  下列等式正确的是\paren[C]
  \begin{choices}
  \item $\lg 5 + \lg 3 = \lg 8$
  \item $\lg 5 - \lg 2 = \lg 3$
  \item $\lg \frac{1}{100}  = - 2$
  \item $\lg 5 = \frac{\ln 10}{\ln 2}$
  \end{choices}
\end{question}

% 7

\begin{question}
  已知 $\lg a $ 和 $\lg b$ 分别是方程 $x^2+x-3  = 0$ 的两个根, 则 $ab$ = \paren[C]
  \begin{choices}
  \item $10$
  \item $1$
  \item $\frac{1}{10}$
  \item $100$
  \end{choices}
\end{question}
% 8

    
\begin{question}
  若$a = \log_4 2 , b = \log_2 5 , c = \log_4 9$, 则 \paren[C]
  \begin{choices}
  \item $a<b<c$
  \item $c<b<a$
  \item $a<c<b$
  \item $b<c<a$
  \end{choices}
\end{question}



\begin{question}
  下列对数函数在区间$(0,+\infty)$内为减函数的是 \paren[B]
  \begin{choices}
  \item $y = \log_{\frac{5}{4}} x$
  \item $y = \log_{\frac{1}{3}} x$
  \item $y = \ln x$
  \item $y = \lg x$
  \end{choices}
\end{question}


\section{填空题:本题共 10 小题,每小题 3 分,共 30 分。}






        \begin{question}
          已知 $3^a = 5,3^b =2$,则 $3^{2a - 3b }  =$ \fillin[$\frac{25}{8}$].
        \end{question}

        \begin{question}
          化简:$\sqrt{(e - \pi)^2}=$\fillin[$\pi - e $].
        \end{question}

        \begin{question}
          若指数函数$y = (2a+3)^x$ 在$\mathbb{R}$上是减函数,则$a$的取值范围是 \fillin[$(-\frac{3}{2},-1)$].
        \end{question}

        \begin{question}
          当 $x\in [0,1]$ 时, 函数 $f(x) = 3^x+2$的值域为 \fillin[$[3,5]$].
        \end{question}


        \begin{question}
          解方程$\log_{\sqrt{2}} x = 4$,  $x = $\fillin[$4$].
        \end{question}


        \begin{question}
          计算$e^{\ln 5} = $\fillin[$5$].
        \end{question}


        \begin{question}
          若$\lg 2 =a , \lg 3 = b ,$则 $\log_2 24 = $    \fillin[$\frac{3a+b}{a}$ 或 $3 + \frac{b}{a}$](用$a,b$来表示).
        \end{question}

        \begin{question}
          若$\log_2(\log_3 x) = 1 $则 $x = $    \fillin[$9$].
        \end{question}


        \begin{question}
          计算 $2025^{\log_{2025}2} \times \ln e^{-1}=$      \fillin[$-2$].
        \end{question}
        \begin{question}
         函数 $f(x) = \frac{\sqrt{2-x}}{\lg(x+3)}$的定义域是  \fillin[$(-3,-2) \cup {( -2,2]}$].
        \end{question}




\section*{一、选择题}
\begin{table}[h]
\centering
\begin{tabular}{|C|C|C|C|C|C|C|C|C|C|}
  \hline
  1 & 2 & 3 & 4 & 5 & 6 & 7 & 8 & 9 & 10 \\
  \hline
    &   &   &   &   &   &   &   &   &    \\
  \hline
\end{tabular}
\end{table}


% 15个问题时
% \begin{table}[h]
%   \centering
%   \begin{tabular}{|C|C|C|C|C|C|C|C|C|C|C|C|C|C|C|}
%     \hline
%     1 & 2 & 3 & 4 & 5 & 6 & 7 & 8 & 9 & 10 & 11 & 12 & 13 & 14 & 15 \\
%     \hline
%       &   &   &   &   &   &   &   &   &    &    &    &    &    &    \\
%     \hline
%   \end{tabular}
%   \end{table}



\section*{一、填空题}
\begin{tasks}[label=\arabic*.](5)
  \task \underline{\hspace{2cm}}
  \task \underline{\hspace{2cm}}
  \task \underline{\hspace{2cm}}
  \task \underline{\hspace{2cm}}
  \task \underline{\hspace{2cm}}
  \task \underline{\hspace{2cm}}
  \task \underline{\hspace{2cm}}
  \task \underline{\hspace{2cm}}
  \task \underline{\hspace{2cm}}
  \task \underline{\hspace{2cm}}
\end{tasks}  



\clearpage









\section{解答题:本题共 4 小题,共 40 分。解答应写出文字说明、证明过程或者演算步骤。}

        % \begin{problem}[points = 12]
        %   已知函数 $f(x) = x (1 - \ln x)$。讨论 $f(x)$ 的单调性。
        %   设 $a$,$b$ 为两个不相等的正数,且 $b \ln a - a \ln b = a - b$,
        %   证明:$2 < \frac{1}{a} + \frac{1}{b} < \eu$.
        % \end{problem}


        % \begin{problem}[points = 10000]
        %   证明希尔伯特第五问题.
        % \end{problem}

        \begin{problem}[points = 10]
          求下列各式的值

          \begin{tasks}[label=(\arabic*)](2)
              \task $\log_{3} 18 - \log_3 2 + \lg \frac{1}{8} - \lg 125 +(\sqrt{3} -\sqrt{2})^{\ln 1}$;
              \task $\left(\lg5\right)^{2}+\lg4\cdot\lg5+\left(\lg2\right)^{2}$.
          \end{tasks}
        \end{problem}


        \begin{solution}\\

          (1) 
          \begin{align*}
             = &\log_3 9 + \log \frac{1}{1000} +1 \\
             = &2 - 3 +1 \\
             = & 0.
          \end{align*}

          (2)
            \begin{align*}
              = & \left(\lg5\right)^{2}+\lg2^2\cdot\lg5+\left(\lg2\right)^{2}\\
               = & \left(\lg5\right)^{2}+2 \cdot \lg2\cdot\lg5+\left(\lg2\right)^{2}\\
               = & (\lg 5 + \lg 2)^2\\
               = & 1.
          \end{align*}

        \end{solution}
        % \vspace{10cm} %添加书写空间




        
        \begin{problem}[points = 10]
          求下列函数定义域:

          \begin{tasks}[label=(\arabic*)](2)
              \task $f(x) = \sqrt{1-(\frac{1}{2})^x}$;
              \task $f(x) = \log_7 \frac{1}{1-3x}$.

          \end{tasks}
        \end{problem}


        
        \begin{solution}
          \\\
      (1)$[0,+\infty]$;\\
      (2)$(-\infty,\frac{1}{3})$.
        \end{solution}


\clearpage


        \begin{problem}[points = 10]
          求下列方程的解:
          $$
          \lg (\frac{1000}{x}) \cdot \lg (10 x) = -5.
          $$
        \end{problem}
        % \vspace{10cm} %添加书写空间
        \begin{solution}
原式等价于
          \begin{align*}
               &(3 - \lg x)(1+\lg x ) = -5 \\
             \Leftrightarrow  & (\lg x)^2 - 2 \lg x  -8  = 0 \\
             \Leftrightarrow  & (\lg x -4)  (\lg x +2 )   = 0 \\
          \end{align*}
          则 $\lg x = 4$ 或 $\lg = -2$, 则 $x = 10000$ 或 $x = \frac{1}{100}$ .

          \vspace{3cm}

        \end{solution}











        \begin{problem}[points = 10]
          已知函数 $f(x) = \lg(ax^2 +2 x +1)$ 的值域为 $\mathbf{R}$,求 实数 $a$ 的取值范围.
        \end{problem}


        \begin{solution}
          当 $a = 0$ 时,符合题意;

          当 $a<0$ 时,不符合题意;


          当 $a>0$ 时,$\Delta = 2^2 - 4 a = 4-4a \ge 0$,即 $a \le 1$的时候符合题意.

          综上所述,$a$的取值范围是 $[0,1]$.
          
                  \end{solution}
          


 \end{document}