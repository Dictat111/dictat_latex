\documentclass[twocolumn]{ctexart}
\usepackage{setspace} %用于控制行间距离
\usepackage{xeCJK}
\usepackage{xpinyin} %使用拼音
\usepackage[
    a4paper,
    left=1.5cm,
    right=1.5cm,
    top=2cm,
    bottom=2cm,
    headheight=15pt,
    headsep=10pt,
    footskip=20pt
]{geometry}
\newcommand{\二号}{\fontsize{21pt}{\baselineskip}\selectfont}%设置二号为21pt,使用默认行距的字体设置。
\newcommand{\三号}{\fontsize{15.75pt}{\baselineskip}\selectfont}
\newcommand{\四号}{\fontsize{14pt}{\baselineskip}\selectfont}
% \newcommand{\poetry}[2]{
%      \begin{flushleft}\phantomsection
%          \songti \三号{#1}
%          \addcontentsline{toc}{subsection}{#1}\\ %\addcontentsline 用于手动把内容添加到目录里。
%          \kaishu \四号{#2}
%     \end{flushleft}
% \vspace{1cm}
% }






% 定义一个计数器
\newcounter{poetrycounterall} % 用于记录每位诗人文体的不同.
\setcounter{poetrycounterall}{0} % 初始化为 0
\newcounter{poetrycounter}[section] % 用于记录每位诗人文体的不同.
\setcounter{poetrycounter}{0} % 初始化为 0
\newcounter{poetrycountersub}[subsection] % 用于记录每位诗人文体的不同.
\setcounter{poetrycountersub}{0} % 初始化为 0

\newcommand{\poetry}[2]{
    \stepcounter{poetrycounterall}%添加计数器
    \stepcounter{poetrycounter}%添加计数器
    \stepcounter{poetrycountersub}%添加计数器
    \begin{flushleft}\phantomsection
        \CJKfontspec{STSong}  \三号{#1}
        \addcontentsline{toc}{subsubsection}{#1}\\ %\addcontentsline 用于手动把内容添加到目录里。
        \vspace{0.2cm}
        \begin{spacing}{0.8} % 设置行间距为 1.5 倍
            \CJKfontspec{KaiTi} \四号{#2}
        \end{spacing}
    \end{flushleft}
    % \vspace{0.3cm}
    \vspace{-2em} %消除flushleft自带行的空白
    }



\usepackage[bookmarks,hypertexnames=false,debug]{hyperref}

\begin{document}

\tableofcontents   

\newpage

\clearpage
\section{魏晋南北朝}
\subsection{古诗十九首}

\poetry {行行重行行}{
行行重行行,与君生别离。\\
相去万余里,各在天一涯。\\
道路阻且长,会面安可知?\\
胡马依北风,越鸟巢南枝。\\
相去日已远,衣带日已缓。\\
浮云蔽白日,游子不顾返。\\
思君令人老,岁月忽已晚。\\
弃捐勿复道,努力加餐饭。\\
}
\poetry {迢迢牵牛星}{
迢迢牵牛星,皎皎河汉女。\\
纤纤擢素手,札札弄机杼。\\
终日不成章,泣涕零如雨。\\
河汉清且浅,相去复几许。\\
盈盈一水间,脉脉不得语。\\
}

\poetry {青青河畔草}{
青青河畔草,郁郁园中柳。\\
盈盈楼上女,皎皎当窗牖。\\
娥娥红粉妆,纤纤出素手。\\
昔为倡家女,今为荡子妇。\\
荡子行不归,空床难独守。\\
}

\poetry {涉江采芙蓉}{
涉江采芙蓉,兰泽多芳草。\\
采之欲遗谁?所思在远道。\\
还顾望旧乡,长路漫浩浩。\\
同心而离居,忧伤以终老。\\
}
\subsection{南朝民歌}
\poetry {西洲曲}{
忆梅下西洲,折梅寄江北。\\
单衫杏子红,双鬓鸦雏色。\\
西洲在何处?两桨桥头渡。\\
日暮伯劳飞,风吹乌桕树。\\
树下即门前,门中露翠钿。\\
开门郎不至,出门采红莲。\\
采莲南塘秋,莲花过人头。\\
低头弄莲子,莲子清如水。\\
置莲怀袖中,莲心彻底红。\\
忆郎郎不至,仰首望飞鸿。\\
鸿飞满西洲,望郎上青楼。\\
楼高望不见,尽日栏杆头。\\
栏杆十二曲,垂手明如玉。\\
卷帘天自高,海水摇空绿。\\
海水梦悠悠,君愁我亦愁。\\
南风知我意,吹梦到西洲。\\
}
\subsection{陶渊明}
\poetry{饮酒·其五}{
结庐在人境,而无车马喧。\\
问君何能尔?心远地自偏。\\
采菊东篱下,悠然见南山。\\
山气日夕佳,飞鸟相与还。\\
此中有真意,欲辨已忘言。\\
}

\poetry{归园田居·其一}{
少无适俗韵,性本爱丘山。\\
误落尘网中,一去三十年。\\
羁鸟恋旧林,池鱼思故渊。\\
开荒南野际,守拙归园田。\\
方宅十余亩,草屋八九间。\\
榆柳荫后檐,桃李罗堂前。\\
暧暧远人村,依依墟里烟。\\
狗吠深巷中,鸡鸣桑树颠。\\
户庭无尘杂,虚室有余闲。\\
久在樊笼里,复得返自然。\\
}

\poetry {归园田居・其三}{
种豆南山下,草盛豆苗稀。\\
晨兴理荒秽,带月荷锄归。\\
道狭草木长,夕露沾我衣。\\
衣沾不足惜,但使愿无违。\\
}

\poetry {读山海经・其十}{
精卫衔微木,将以填沧海。\\
刑天舞干戚,猛志固常在。\\
同物既无虑,化去不复悔。\\
徒设在昔心,良辰讵可待?\\
}


% \section{李白}
% \poetry{静夜思}{床前明月光,疑是地上霜。\\
举头望明月,低头思故乡。\\
}

\poetry{夜下征虏亭}{船下广陵\footnote{在今扬州}去,月明征虏亭\footnote{在今南京}。\\
山花如绣颊,江火似流萤。}
\poetry{劳劳亭}{天下伤心处,劳劳送客亭。\\
春风知别苦,不遣柳条青。\\
}

\poetry{独坐敬亭山}{众鸟高飞尽,孤云独去闲。\\
相看两不厌,只有敬亭山。\\
}

\poetry{玉阶怨}{玉阶生白露,夜久侵罗袜。\\
却下水晶帘,玲珑望秋月。\\
}

\poetry{夏日山中}{懒摇白羽扇,裸\xpinyin*{袒}青林中。\\
脱巾挂石壁,露顶洒松风。\\
}

\poetry{夜宿山寺}{危楼高百尺,手可摘星辰。\\
不敢高声语,恐惊天上人。\\
}

\poetry{秋浦歌十七首 十四}{炉火照天地,红星乱紫烟。\\
\xpinyin*{赧}\footnote{赧,因为羞愧而脸红,这里指的是脸被炉火映红了.}郎明月夜,歌曲动寒川。\\
}

\poetry{秋浦歌十七首 十五}{白发三千丈,缘愁似个长。\\
不知明镜里,何处得秋霜。\\
}


\poetry{怨情}{美人卷珠帘,深坐蹙\footnote{或作:颦}蛾眉。\\
但见泪痕湿,不知心恨谁?\\
}


\newcounter{libaiwujuecot}
\setcounter{libaiwujuecot}{\thepoetrycounter}

% 
\poetry{宣城见杜鹃花}{
蜀国曾闻子规鸟,宣城还见杜鹃花。\\
一叫一回肠一断,三春三月忆三巴。\\
}
\poetry{黄鹤楼送孟浩然之广陵}{故人西辞黄鹤楼,烟花三月下扬州。\\
孤帆远影碧空尽,惟见长江天际流。\\
}

\poetry{早发白帝城}{朝辞白帝彩云间,千里江陵一日还。\\
两岸猿声啼不住,轻舟已过万重山。\\
}
\poetry{越中览古}{越王勾践破吴归,义士还家尽锦衣。\footnote{义士 一作:战士}\\
宫女如花满春殿,只今惟有鹧鸪飞。\\
}
\poetry{清平调·云想衣裳花想容}{云想衣裳花想容,春风拂槛露华浓。\\
若非群玉山头见,会向瑶台月下逢。\\
}


\poetry{清平调·一枝红艳露凝香}{一枝红艳露凝香,云雨巫山枉断肠。\\
借问汉宫谁得似,可怜飞燕倚新妆。\\
}

\poetry{清平调·名花倾国两相欢}{名花倾国两相欢,常得君王带笑看。\\
解释春风无限恨,沉香亭北倚栏杆。\\
}

\poetry{与史郎中钦听黄鹤楼上吹笛}{一为迁客去长沙,西望长安不见家。\\
黄鹤楼中吹玉笛,江城五月落梅花。\\
}

\poetry{山中问答}{问余何意栖碧山,笑而不答心自闲。\\
桃花流水\xpinyin*{窅}然去,别有天地非人间。\\
}

\poetry{山中与幽人对酌}{两人对酌山花开,一杯一杯复一杯。\\
我醉欲眠卿且去,明朝有意抱琴来。\\
}

\poetry{从军行}{百战沙场碎铁衣,城南已合数重围。\\
突营射杀呼延将,独领残兵千骑归。\\
}

\poetry{峨眉山月歌}{峨眉山月半轮秋,影入平羌江水流。\\
夜发清溪向三峡,思君不见下渝州。\\
}

\poetry{赠汪伦}{李白乘舟将欲行,忽闻岸上踏歌声。\\
桃花潭水深千尺,不及汪伦送我情。\\
}

\poetry{望天门山}{天门中断楚江开,碧水东流至北回。\\
两岸青山相对出,孤帆一片日边来。\\
}

\poetry{春夜洛城闻笛}{谁家玉笛暗飞声,散入春风满洛城。\\
此夜曲中闻折柳,何人不起故园情。\\
}



\poetry{客中作}{兰陵美酒郁金香,玉碗盛来琥珀光。\\
但使主人能醉客,不知何处是他乡。\\
}
% 
\subsection{五言律诗}
\poetry{渡荆门送别}{渡远荆门外,来从楚国游。\\
山随平野尽,江入大荒流。\\
月下飞天镜,云生结海楼。\\
仍怜故乡水,万里送行舟。\\
}

\poetry{送友人}{青山横北郭,白水绕东城。\\
此地一为别,孤蓬万里征。\\
浮云游子意,落日故人情。\\
挥手自兹去,萧萧班马鸣。\\
}

\poetry{赠孟浩然}{吾爱孟夫子,风流天下闻。\\
红颜弃轩冕,白首卧松云。\\
醉月频中圣,迷花不事君。\\
高山安可仰,徒此\xpinyin*{揖}清芬。\\
}
\poetry{秋登宣城谢\xpinyin*{眺}北楼}{江城如画里,山晚望晴空\footnote{晚 一作:晓}。\\
两水夹明镜,双桥落彩虹。\\
人烟寒橘柚,秋色老梧桐。\\
谁念北楼上,临风怀谢公。\\
}

\poetry{访戴天山道士不遇}{犬吠水声中,桃花带露浓。\\
树深时见鹿,溪午不闻钟。\\
野竹分青霭,飞泉挂碧峰。\\
无人知所去,愁倚两三松。\\
}

\poetry{夜泊牛\xpinyin*{渚}怀古}{牛渚西江夜,青天无片云。\\
登舟望秋月,空忆谢将军。\\
余亦能高咏,斯人不可闻。\\
明朝挂帆席,枫叶落纷纷。\\
}

\poetry{塞下曲六首·其一}{五月天山雪,无花只有寒。\\
笛中闻折柳,春色未曾看。\\
晓战随金鼓,宵眠抱玉鞍。\\
愿将腰下剑,直为斩楼兰。\\
}
\poetry{送友人入蜀}{
见说蚕丛路,崎岖不易行。\\
山从人面起,云傍马头生。\\
芳树笼秦栈,春流遶蜀城。\\
升沉应已定,不必问君平。\\
}
\poetry{宿五松山下荀媪家}{
我宿五松下,寂寥无所欢。\\
田家秋作苦,邻女夜舂寒。\\
跪进雕胡饭,月光明素盘。\\
令人惭漂母,三谢不能餐。\\
}
\poetry{听蜀僧浚弹琴}{\footnote{或写成\xpinyin*{\CJKfontspec{KaiTi} 濬},这两个是异体字}蜀僧抱绿绮,西下峨眉峰。\\
为我一挥手,如听万壑松。\\
客心洗流水,余响入霜钟。\\
不觉碧山暮,秋云暗几重。\\
}

\poetry{沙丘城下寄杜甫}{
我来竟何事,高卧沙丘城。\\
城边有古树,日夕连秋声。\\
鲁酒不可醉,齐歌空复情。\\
思君若汶水,浩荡寄南征。\\
}
% 

\subsection{七言律诗}


\poetry{上李邕}{大鹏一日同风起,扶摇直上九万里。\\
假令风歇时下来,犹能簸却沧溟水。\\
世人见我恒殊调,闻余大言皆冷笑。\\
宣父犹能畏后生,丈夫未可轻年少。\\
}

\poetry{登金陵凤凰台}{凤凰台上凤凰游,凤去台空江自流。\\
吴宫花草埋幽径,晋代衣冠成古丘。\\
三山半落青天外,一水中分白鹭洲。\\
总为浮云能蔽日,长安不见使人愁。\\
}

\poetry{金陵城西楼月下吟}{
金陵夜寂凉风发,独上高楼望吴越。\\
白云映水摇空城,白露垂珠滴秋月。\\
月下沉吟久不归,古来相接眼中稀。\\
解道澄江净如练,令人长忆谢玄晖。\\
}
% 
\subsection{古风}
\poetry{古风 一}{
大雅久不作,吾衰竟谁陈。\\
王风委蔓草,战国多荆榛。\\
龙虎相啖食,兵戈逮狂秦。\\
正声何微茫,哀怨起骚人。\\
扬马激颓波,开流荡无垠。\\
废兴虽万变,宪章亦已沦。\\
自从建安来,绮丽不足珍。\\
圣代复元古,垂衣贵清真。\\
群才属休明,乘运共跃鳞。\\
文质相炳焕,众星罗秋旻。\\
我志在删述,垂辉暎千春。\\
希圣如有立,绝笔于获麟。\\
}
\poetry{古风 三}{
秦皇扫六合,虎视何雄哉。\\
飞剑决浮云,诸侯尽西来。\\
明断自天启,大略驾群才。\\
收兵铸金人,函谷正东开。\\
铭功会稽岭,骋望琅琊台。\\
刑徒七十万,起土骊山隈。\\
尚采不死药,茫然使心哀。\\
连弩射海鱼,长鲸正崔嵬。\\
额鼻象五岳,扬波喷云雷。\\
鬐鬣蔽青天,何由睹蓬莱。\\
徐[巿]载秦女,楼船几时迥。\\
但见三泉下,金棺葬寒灰。\\
}

\poetry{古风 十九}{
西岳莲花山,迢迢见明星。\\
素手把芙蓉,虚步蹑太清。\\
霓裳曳广带,飘拂升天行。\\
邀我登云台,高揖卫叔卿。\\
恍恍与之去,驾鸿凌紫冥。\\
俯视洛阳川,茫茫走胡兵。\\
流血涂野草,豺狼尽冠缨。\\
}

\poetry{金乡送韦八之西京}{
客自长安来,还归长安去。\\
狂风吹我心,西挂咸阳树。\\
此情不可道,此别何时遇。\\
望望不见君,连山起烟雾。\\
}
\poetry{相和歌辞 梁甫吟}{
长啸梁甫吟,何时见阳春?\\
君不见朝歌屠叟辞棘津,八十西来钓渭滨。\\
宁羞白发照渌水,逢时吐气思经纶。\\
广张三千六百钧,风雅暗与文王亲。\\
大贤虎变愚不测,当年颇似寻常人。\\
君不见高阳酒徒起草中,长揖山东隆准公。\\
入门不拜骋雄辨,两女辍洗来趋风。\\
东下齐城七十二,指麾楚汉如旋蓬。\\
狂生落拓尚如此,何况壮士当群雄。\\
我欲攀龙见明主,雷公砰訇震天鼓。\\
帝旁投壶多玉女,三时大笑开电光。\\
倏烁晦冥起风雨,阊阖九门不可通。\\
以额叩关阍者怒,白日不照吾精诚。\\
杞国无事忧天倾,{豸契}貐磨牙竞人肉。\\
驺虞不折生草茎,手接飞猱搏雕虎。\\
侧足焦原未言苦,智者可卷愚者豪。\\
世人见我轻鸿毛,力排南山三壮士,\\齐相杀之费二桃。\\
吴楚弄兵无剧孟,亚夫咍尔为徒劳。\\
梁父吟,梁父吟,声正悲。\\
张公两龙剑,神物合有时。\\
风云感会起屠钓,大人{山儿}屼当安之。\\
}

\poetry{相和歌辞 子夜四时歌四首 冬歌}{
明朝驿使发,一夜絮征袍。\\
素手抽针冷,那堪把剪刀。\\
裁缝寄远道,几日到临洮。\\
}

\poetry{杂曲歌辞 长干行二首 一}{
妾发初覆额,折花门前剧。\\
郎骑竹马来,绕床弄青梅。\\
同居长干里,两小无嫌猜。\\
十四为君妇,羞颜尚不开。\\
低头向暗壁,千唤不一回。\\
十五始展眉,愿同尘与灰。\\
常存抱柱信,岂上望夫台。\\
十六君远行,瞿塘滟预堆。\\
五月不可触,猿鸣天上哀。\\
门前迟行迹,一一生绿苔。\\
苔深不能扫,落叶秋风早。\\
八月蝴蝶来,双飞西园草。\\
感此伤妾心,坐愁红颜老。\\
早晚下三巴,预将书报家。\\
相迎不道远,直至长风沙。\\
}
\poetry{妾薄命}{汉帝重阿娇,贮之黄金屋。\\
咳唾落九天,随风生珠玉。\\
宠极爱还歇,妒深情却疏。\\
长门一步地,不肯暂回车。\\
雨落不上天,水覆难再收。\\
君情与妾意,各自东西流。\\
昔日芙蓉花,今成断根草。\\
以色事他人,能得几时好。\\
}

\poetry{北风行}{烛龙栖寒门,光曜犹旦开。\\
日月照之何不及此?\\惟有北风号怒天上来。\\
燕山雪花大如席,片片吹落轩辕台。\\
幽州思妇十二月,停歌罢笑双蛾摧。\\
倚门望行人,念君长城苦寒良可哀。\\
别时提剑救边去,遗此虎文金鞞靫。\\
中有一双白羽箭,蜘蛛结网生尘埃。\\
箭空在,人今战死不复回。\\
不忍见此物,焚之已成灰。\\
黄河捧土尚可塞,北风雨雪恨难裁。\\
}



\poetry{把酒问月·故人贾淳令予问之}{青天有月来几时?我今停杯一问之。\\
人攀明月不可得,月行却与人相随。\\
皎如飞镜临丹阙,绿烟灭尽清辉发。\\
但见宵从海上来,宁知晓向云间没。\\
白兔捣药秋复春,嫦娥孤栖与谁邻?\\
今人不见古时月,今月曾经照古人。\\
古人今人若流水,共看明月皆如此。\\
唯愿当歌对酒时,月光长照金樽里。\\
}

\poetry{侠客行}{赵客缦胡缨,吴钩霜雪明。\\
银鞍照白马,飒\xpinyin*{遝}如流星。\\
十步杀一人,千里不留行。\\
事了拂衣去,深藏身与名。\\
闲过信陵饮,脱剑膝前横。\\
将炙啖朱亥,持觞劝侯嬴。\\
三杯吐然诺,五岳倒为轻。\\
眼花耳热后,意气素霓生。\\
救赵挥金槌,邯郸先震惊。\\
千秋二壮士,煊赫大梁城。\\
纵死侠骨香,不惭世上英。\\
谁能书合下,白首太玄经。\\
}

\poetry{古风 十}{齐有倜傥生,鲁连特高妙。\\
明月出海底,一朝开光曜。\\
却秦振英声,后世仰末照。\\
意轻千金赠,愿向平原笑。\\
吾亦澹荡人,拂衣可同调。\\
}

\poetry{横吹曲辞 关山月}{明月出天山,苍茫云海间。\\
长风几万里,吹度玉门关。\\
汉下白登道,胡窥青海湾。\\
由来征战地,不见有人还。\\
戍客望边色,思归多苦颜。\\
高楼当此夜,叹息未应闲。\\
}

\poetry{长相思·其一}{长相思,在长安。\\
络纬秋啼金井阑,微霜凄凄簟色寒。\\
孤灯不明思欲绝,卷帷望月空长叹。\\
美人如花隔云端!\\
上有青冥之长天,下有渌水之波澜。\\
天长路远魂飞苦,梦魂不到关山难。\\
长相思,摧心肝!\\
}

\poetry{长相思·其二}{日色欲尽花含烟,月明如素愁不眠。\footnote{如素 一作:欲素}\\
赵瑟初停凤凰柱,蜀琴欲奏鸳鸯弦。\\
此曲有意无人传,愿随春风寄燕然。\\
忆君迢迢隔青天,\\昔日横波目,今作流泪泉。\\
不信妾肠断,归来看取明镜前。(肠断 一作:断肠)\\
}
\poetry{长相思·其三}{
美人在时花满堂,美人去后花余床。\\
床中绣被卷不寝,至今三载犹闻香。\\
香亦竟不灭,人亦竟不来。\\
相思黄叶落,白露点青苔。\\
}
\poetry{三五七言}{秋风清,秋月明。\\落叶聚还散,寒鸦栖复惊。\\
相思相见知何日,此时此夜难为情。\\
}

\poetry{子夜吴歌}{长安一片月,万户捣衣声。\\
秋风吹不尽,总是玉关情。\\
何日平胡虏,良人罢远征。\\
}

\poetry{古朗月行}{小时不识月,呼作白玉盘。\\
又疑瑶台镜,飞在白云端。\\
仙人垂两足,桂树作团团。\\
白兔捣药成,问言与谁餐。\\
蟾蜍蚀圆影,大明夜已残。\\
羿昔落九乌,天人清且安。\\
阴精此沦惑,去去不足观。\\
忧来其如何,凄怆摧心肝。\\
}









\poetry{下终南山过斛斯山人宿置酒}{暮从碧山下,山月随人归。\\
却顾所来径,苍苍横翠微。\\
相携及田家,童稚开荆扉。\\
绿竹入幽径,青萝拂行衣。\\
欢言得所憩,美酒聊共挥。\\
长歌吟松风,曲尽河星稀。\\
我醉君复乐,陶然共忘机。\\
}


\poetry{月下独酌四首·其一}{花间一壶酒,独酌无相亲。\\
举杯邀明月,对影成三人。\\
月既不解饮,影徒随我身。\\
暂伴月将影,行乐须及春。\\
我歌月徘徊,我舞影零乱。\\
醒时同交欢,醉后各分散。(同交欢 一作:相交欢)\\
永结无情游,相期邈云汉。\\
}

\poetry{月下独酌四首 二}{天若不爱酒,酒星不在天。\\
地若不爱酒,地应无酒泉。\\
天地既爱酒,爱酒不愧天。\\
已闻清比圣,复道浊如贤。\\
贤圣既已饮,何必求神仙。\\
三杯通大道,一斗合自然。\\
但得酒中趣,勿为醒者传。\\
}

\poetry{春思}{燕草如碧丝,秦桑低绿枝。\\
当君怀归日,是妾断肠时。\\
春风不相识,何事入罗帏。\\
}

\poetry{宣州谢脁楼饯别校书叔云}{弃我去者,昨日之日不可留;\\
乱我心者,今日之日多烦忧。\\
长风万里送秋雁,对此可以酣高楼。\\
蓬莱文章建安骨,中间小谢又清发。\\
俱怀逸兴壮思飞,欲上青天揽明月。\\
抽刀断水水更流,举杯消愁愁更愁。\\
人生在世不称意,明朝散发弄扁舟。\\
}

\poetry{庐山谣寄卢侍御虚舟}{我本楚狂人,凤歌笑孔丘。\\
手持绿玉杖,朝别黄鹤楼。\\
五岳寻仙不辞远,一生好入名山游。\\
庐山秀出南斗傍,屏风九叠云锦张。\\
影落明湖青黛光,金阙前开二峰长,\\
银河倒挂三石梁。\\
香炉瀑布遥相望,回崖沓嶂凌苍苍。\\
翠影红霞映朝日,鸟飞不到吴天长。\\
登高壮观天地间,大江茫茫去不还。\\
黄云万里动风色,白波九道流雪山。\\
好为庐山谣,兴因庐山发。\\
闲窥石镜清我心,谢公行处苍苔没。\\
早服还丹无世情,琴心三叠道初成。\\
遥见仙人彩云里,手把芙蓉朝玉京。\\
先期汗漫九\xpinyin*{垓}上,愿接卢敖游太清。\\
}

\poetry{梦游天姥吟留别}{海客谈瀛洲,烟涛微茫信难求;\\
越人语天姥,云霞明灭或可睹。\\
天姥连天向天横,势拔五岳掩赤城。\\
天台四万八千丈,对此欲倒东南倾。\\
我欲因之梦吴越,一夜飞度镜湖月。\footnote{ 度 通:渡}\\
湖月照我影,送我至剡溪。\\
谢公宿处今尚在,渌水荡漾清猿啼。\\
脚著谢公屐,身登青云梯。\\
半壁见海日,空中闻天鸡。\\
千岩万转路不定,迷花倚石忽已暝。\\
熊咆龙吟殷岩泉,栗深林兮惊层巅。\\
云青青兮欲雨,水澹澹兮生烟。\\
列缺霹雳,丘峦崩摧。\\
洞天石扉,訇然中开。\\
青冥浩荡不见底,日月照耀金银台。\\
霓为衣兮风为马,云之君兮纷纷而来下。\\
虎鼓瑟兮鸾回车,仙之人兮列如麻。\\
忽魂悸以魄动,恍惊起而长嗟。\\
惟觉时之枕席,失向来之烟霞。\\
世间行乐亦如此,古来万事东流水。\\
别君去兮何时还?且放白鹿青崖间,须行即骑访名山。\\
安能摧眉折腰事权贵,使我不得开心颜!\\
}

\poetry{金陵酒肆留别}{风吹柳花满店香,吴姬压酒劝客尝。\\
金陵子弟来相送,欲行不行各尽觞。\\
请君试问东流水,别意与之谁短长。\\
}




\poetry{秋浦歌}{白发三千丈,缘愁似个长。\\
不知明镜里,何处得秋霜?\\
}



\poetry{行路难·金樽清酒斗十千}{金樽清酒斗十千,玉盘珍羞直万钱。\\
停杯投箸不能食,拔剑四顾心茫然。\\
欲渡黄河冰塞川,将登太行雪满山。\\
闲来垂钓碧溪上,忽复乘舟梦日边。\\
行路难!行路难!多歧路,今安在?\\
长风破浪会有时,直挂云帆济沧海。\\
}


% \newcounter{libaiqijuecot}
% \setcounter{libaiqijuecot}{\thepoetrycountersub}

% 一共收入了李白 \thepoetrycounter  首诗。

% % 五言绝句有 \thelibaiwujuecot 首.七言绝句有 \thelibaiqijuecot 首.
% \clearpage


% \section{杜甫}
% 
\poetry{绝句二首 一}{
迟日江山丽,春风花草香。\\
泥融飞燕子,沙暖睡鸳鸯。\\
}
\poetry{八阵图}{
功盖三分国,名成八阵图。\\
江流石不转,遗恨失吞吴。\\
}
% 
\poetry{横吹曲辞 前出塞九首 六}{
挽弓当挽彊,用箭当用长。\\
射人先射马,擒贼先擒王。\\
杀人亦有限,列国自有疆。\\
茍能制侵陵,岂在多杀伤。\\
}

\poetry{望岳}{
岱宗夫如何,齐鲁青未了。\\
造化钟神秀,阴阳割昏晓。\\
荡胸生曾云,决眦入归鸟。\\
会当凌绝顶,一览众山小。\\
}
\poetry{房兵曹胡马诗}{
胡马大宛名,锋棱瘦骨成。\\
竹批双耳峻,风入四蹄轻。\\
所向无空阔,真堪托死生。\\
骁腾有如此,万里可横行。\\
}
\poetry{画鹰}{
素练风霜起,苍鹰画作殊。\\
㧐身思狡兔,侧目似愁胡。\\
绦旋光堪\xpinyin*{擿},轩楹势可呼。\\
何当击凡鸟,毛血洒平芜。\\
}


\poetry{春日忆李白}{
白也诗无敌,飘然思不群。\\
清新庾开府,俊逸鲍参军。\\
渭北春天树,江东日暮云。\\
何时一尊酒,重与细论文。\\
}

\poetry{对雪}{
战哭多新鬼,愁吟独老翁。\\
乱云低薄暮,急雪舞回风。\\
瓢弃尊无绿,炉存火似红。\\
数州消息断,愁坐正书空。\\
}
\poetry{月夜}{
今夜\xpinyin*{鄜}州月,闺中只独看。\\
遥怜小儿女,未解忆长安。\\
香雾云鬟湿,清辉玉臂寒。\\
何时倚虚幌,双照泪痕干。\\
}
\poetry{春望}{
国破山河在,城春草木深。\\
感时花溅泪,恨别鸟惊心。\\
烽火连三月,家书抵万金。\\
白头搔更短,浑欲不胜簪。\\
}
\poetry{天末忆李白}{
凉风起天末,君子意如何。\\
鸿雁几时到,江湖秋水多。\\
文章憎命达,魑魅喜人过。\\
应共冤魂语,投诗赠汩罗。\\
}
\poetry{岁暮}{
岁暮远为客,边隅还用兵。\\
烟尘犯雪岭,鼓角动江城。\\
天地日流血,朝廷谁请缨。\\
济时敢爱死,寂寞壮心惊。\\
}
\poetry{春夜喜雨}{
好雨知时节,当春乃发生。\\
随风潜入夜,润物细无声。\\
野径云俱黑,江船火独明。\\
晓看红湿处,花重锦官城。\\
}
\poetry{琴台}{
茂陵多病后,尚爱卓文君。\\
酒肆人间世,琴台日暮云。\\
野花留宝靥,蔓草见罗裙。\\
归凤求皇意,寥寥不复闻。\\
}
\poetry{水槛遣心二首 一}{
去郭轩楹敞,无村眺望赊。\\
澄江平少岸,幽树晚多花。\\
细雨鱼儿出,微风燕子斜。\\
城中十万户,此地两三家。\\
}
\poetry{奉济驿重送严公四韵}{
远送从此别,青山空复情。\\
几时杯重把,昨夜月同行。\\
列郡讴歌惜,三朝出入荣。\\
江村独归处,寂寞养残生。\\
}
\poetry{不见}{
不见李生久,佯狂真可哀。\\
世人皆欲杀,吾意独怜才。\\
敏捷诗千首,飘零酒一杯。\\
匡山读书处,头白好归来。\\
}
\poetry{春远}{
肃肃花絮晚,菲菲红素轻。\\
日长唯鸟雀,春远独柴荆。\\
数有关中乱,何曾剑外清。\\
故乡归不得,地入亚夫营。\\
}
\poetry{旅夜书怀}{
细草微风岸,危樯独夜舟。\\
星垂平野阔,月涌大江流。\\
名岂文章著,官因老病休。\\
飘飘何所似,天地一沙鸥。\\
}
\poetry{江汉}{
江汉思归客,乾坤一腐儒。\\
片云天共远,永夜月同孤。\\
落日心犹壮,秋风病欲疏。\\
古来存老马,不必取长途。\\
}
\poetry{月圆}{
孤月当楼满,寒江动夜\xpinyin*{扉}。\\
委波金不定,照席绮逾依。\\
未缺空山静,高悬列宿稀。\\
故园松桂发,万里共清辉。\\
}
\poetry{孤雁}{
孤雁不饮啄,飞鸣声念群。\\
谁怜一片影,相失万重云。\\
望尽似犹见,哀多如更闻。\\
野鸦无意绪,鸣噪自纷纷。\\
}
\poetry{登岳阳楼}{
昔闻洞庭水,今上岳阳楼。\\
吴楚东南坼,乾坤日夜浮。\\
亲朋无一字,老病有孤舟。\\
戎马关山北,凭轩涕泗流。\\
}
% 
\poetry{贫交行}{
翻手作云覆手雨,纷纷轻薄何须数。\\
不见管鲍贫时交,此道今人弃如土。\\
}
\poetry{赠李白}{
秋来相顾尚飘蓬,未就丹砂愧葛洪。\\
痛饮狂歌空度日,飞扬跋扈为谁雄。\\
}
\poetry{赠花卿}{
锦城丝管日纷纷,半入江风半入云。\\
此曲只应天上有,人间能得几回闻。\\
}
\poetry{绝句漫兴九首 五}{
肠断春江欲尽头,杖藜徐步立芳洲。\\
颠狂柳絮随风舞,轻薄桃花逐水流。\\
}
\poetry{绝句漫兴九首 七}{
\xpinyin*{糁}径杨花铺白\xpinyin*{毡},点溪荷叶叠青钱。\\
笋根稚子无人见,沙上\xpinyin*{凫}雏傍母眠。\\
}
\poetry{江畔独步寻花七绝句 五}{
黄师塔前江水东,春光懒困倚微风。\\
桃花一\xpinyin*{簇}开无主,可爱深红爱浅红。\\
}
\poetry{江畔独步寻花七绝句 六}{
黄四娘家花满蹊,千朵万朵压枝低。\\
留连戏蝶时时舞,自在娇莺恰恰啼。\\
}
\poetry{江畔独步寻花七绝句 七}{
不是爱花即肯死,只恐花尽老相催。\\
繁枝容易纷纷落,嫩叶商量细细开。\\
}
\poetry{戏为六绝句 一}{
\xpinyin*{庾}信文章老更成,凌云健笔意纵横。\\
今人嗤点流传赋,不觉前贤畏后生。\\
}
\poetry{戏为六绝句 二}{
杨王卢骆当时体,轻薄为文\xpinyin*{哂}未休。\\
尔曹身与名俱灭,不废江河万古流。\\
}

\poetry{绝句四首 三}{
两个黄鹂鸣翠柳,一行白鹭上青天。\\
窗含西岭千秋雪,门泊东吴万里船。\\
}
\poetry{江南逢李龟年}{
歧王宅里寻常见,崔九堂前几度闻。\\
正是江南好风景,落花时节又逢君。\\
}
% \poetry{虢国夫人}{
% \xpinyin*{虢}国夫人承主恩,平明上马入宫门。\\
% 却嫌脂粉涴颜色,澹埽蛾眉朝至尊。\footnote{一作《张祜集》灵台二首之一}\\
% }
% 
\poetry{悲陈陶}{
孟冬十郡良家子,血作陈陶泽中水。\\
野旷天清无战声,四万义军同日死。\\
群胡归来血洗箭,仍唱胡歌饮都市。\\
都人回面北向啼,日夜更望官军至。\\
}
\poetry{九日蓝田崔氏庄}{
老去悲秋强自宽,兴来今日尽君欢。\\
羞将短发还吹帽,笑倩旁人为正冠。\\
蓝水远从千涧落,玉山高并两峰寒。\\
明年此会知谁健,醉把茱萸仔细看。\\
}
\poetry{曲江二首 一}{
一片花飞减却春,风飘万点正愁人。\\
且看欲尽花经眼,莫厌伤多酒入唇。\\
江上小堂巢翡翠,花边高冢卧麒麟。\\
细推物理须行乐,何用浮名绊此身。\\
}
\poetry{曲江二首 二}{
朝回日日典春衣,每日江头尽醉归。\\
酒债寻常行处有,人生七十古来稀。\\
穿花\xpinyin*{蛱}蝶深深见,点水蜻蜓款款飞。\\
传语风光共流转,暂时相赏莫相违。\\
}
\poetry{蜀相}{
丞相祠堂何处寻,锦官城外柏森森。\\
映阶碧草自春色,隔叶黄鹂空好音。\\
三顾频烦天下计,两朝开济老臣心。\\
出师未捷身先死,长使英雄泪满襟。\\
}



\poetry{狂夫}{
万里桥西一草堂,百花潭水即沧浪。\\
风含翠筿娟娟静,雨裛红蕖冉冉香。\\
厚禄故人书断绝,恒饥稚子色凄凉。\\
欲填沟壑唯疏放,自笑狂夫老更狂。\\
}


\poetry{江村}{
清江一曲抱村流,长夏江村事事幽。\\
自去自来堂上燕,相亲相近水中鸥。\\
老妻画纸为棋局,稚子敲针作钓钩。\\
多病所须唯药物,微躯此外更何求。\\
}
\poetry{恨别}{
洛城一别四千里,胡骑长驱五六年。\\
草木变衰行剑外,兵戈阻绝老江边。\\
思家步月清宵立,忆弟看云白日眠。\\
闻道河阳近乘胜,司徒急为破幽燕。\\
}

\poetry{客至}{
舍南舍北皆春水,但见群鸥日日来。\\
花径不曾缘客扫,蓬门今始为君开。\\
盘餐市远无兼味,樽酒家贫只旧\xpinyin*{醅}。\\
肯与邻翁相对饮,隔篱呼取尽余杯。\\
}
\poetry{江上值水如海势聊短述}{
为人性僻耽佳句,语不惊人死不休。\\
老去诗篇浑漫兴,春来花鸟莫深愁。\\
新添水槛供垂钓,故著浮\xpinyin*{槎}替入舟。\\
焉得思如陶谢手,令渠述作与同游。\\
}
\poetry{送韩十四江东觐省}{
兵戈不见老莱衣,叹息人间万事非。\\
我已无家寻弟妹,君今何处访庭\xpinyin*{闱}。\\
黄牛峡静滩声转,白马江寒树影稀。\\
此别应须各努力,故乡犹恐未同归。\\
}


\poetry{闻官军收河南河北}{
剑外忽传收\xpinyin*{蓟}北,初闻涕泪满衣裳。\\
却看妻子愁何在,漫卷诗书喜欲狂。\\
白日放歌须纵酒,青春作伴好还乡。\\
即从巴峡穿巫峡,便下襄阳向洛阳。\\
}


\poetry{登高}{
风急天高猿啸哀,渚清沙白鸟飞回。\\
无边落木萧萧下,不尽长江衮衮来。\\
万里悲秋常作客,百年多病独登台。\\
艰难苦恨繁霜鬓,潦倒新停浊酒杯。\\
}

\poetry{登楼}{
花近高楼伤客心,万方多难此登临。\\
锦江春色来天地,玉垒浮云变古今。\\
北极朝廷终不改,西山寇盗莫相侵。\\
可怜后主还祠庙,日暮聊为梁甫吟。\\
}
\poetry{宿府}{
清秋幕府井梧寒,独宿江城蜡炬残。\\
永夜角声悲自语,中天月色好谁看。\\
风尘荏苒音书绝,关塞萧条行路难。\\
已忍伶俜十年事,强移栖息一枝安。\\
}

\poetry{阁夜}{
岁暮阴阳催短景,天涯霜雪霁寒宵。\\
五更鼓角声悲壮,三峡星河影动摇。\\
野哭几家闻战伐,夷歌数处起渔樵。\\
卧龙跃马终黄土,人事音书漫寂寥。\\
}

\poetry{白帝}{
白帝城中云出门,白帝城下雨翻盆。\\
高江急峡雷霆斗,翠木苍藤日月昏。\\
戎马不如归马逸,千家今有百家存。\\
哀哀寡妇诛求尽,恸哭秋原何处村。\\
}

\poetry{秋兴八首 一}{
玉露凋伤枫树林,巫山巫峡气萧森。\\
江间波浪兼天涌,塞上风云接地阴。\\
丛菊两开他日泪,孤舟一系故园心。\\
寒衣处处催刀尺,白帝城高急暮砧。\\
}
\poetry{秋兴八首 三}{
千家山郭静朝晖,一日江楼坐翠微。\\
信宿渔人还泛泛,清秋燕子故飞飞。\\
匡衡抗疏功名薄,刘向传经心事违。\\
同学少年多不贱,五陵衣马自轻肥。\\
}
\poetry{秋兴八首 四}{
闻道长安似弈棋,百年世事不胜悲。\\
王侯第宅皆新主,文武衣冠异昔时。\\
直北关山金鼓振,征西车马羽书迟。\\
鱼龙寂寞秋江冷,故国平居有所思。\\
}
\poetry{秋兴八首 五}{
蓬莱宫阙对南山,承露金茎霄汉间。\\
西望瑶池降王母,东来紫气满函关。\\
云移雉尾开宫扇,日绕龙鳞识圣颜。\\
一卧沧江惊岁晚,几回青琐照朝班。\\
}
\poetry{秋兴八首 六}{
\xpinyin*{瞿}唐峡口\xpinyin*{曲}江头,万里风烟接素秋。\\
花萼夹城通御气,芙蓉小苑入边愁。\\
朱帘绣柱围黄鹤,锦缆牙\xpinyin*{樯}起白鸥。\\
回首可怜歌舞地,秦中自古帝王州。\\
}
\poetry{秋兴八首 七}{
昆明池水汉时功,武帝\xpinyin*{旌}旗在眼中。\\
织女机丝虚月夜,石鲸鳞甲动秋风。\\
波漂\xpinyin*{菰}米沈云黑,露冷莲房坠粉红。\\
关塞极天唯鸟道,江湖满地一渔翁。\\
}
\poetry{咏怀古迹五首 一}{
支离东北风尘际,漂泊西南天地间。\\
三峡楼台淹日月,五溪衣服共云山。\\
\xpinyin*{羯}胡事主终无赖,词客衰时且未还。\\
\xpinyin*{庾}信平生最萧瑟,暮年诗赋动江关。\\
}
\poetry{咏怀古迹五首 二}{
摇落深知宋玉悲,风流儒雅亦吾师。\\
\xpinyin*{怅}望千秋一洒泪,萧条异代不同时。\\
江山故宅空文藻,云雨荒台岂梦思。\\
最是楚宫俱泯灭,舟人指点到今疑。\\
}
\poetry{咏怀古迹五首 三}{
群山万壑赴荆门,生长明妃尚有村。\\
一去紫台连朔漠,独留青冢向黄昏。\\
画图\xpinyin{省}{xing3}识春风面,环佩空归月夜魂。\\
千载琵琶作胡语,分明怨恨曲中论。\\
}
\poetry{咏怀古迹五首 四}{
蜀主窥吴幸三峡,崩年亦在永安宫。\\
翠华想像空山里,玉殿虚无野寺中。\\
古庙杉松巢水鹤,岁时伏腊走村翁。\\
武侯祠屋常邻近,一体君臣祭祀同。\\
}
\poetry{咏怀古迹五首 五}{
诸葛大名垂宇宙,宗臣遗像肃清高。\\
三分割据\xpinyin*{纡}筹策,万古云霄一羽毛。\\
伯仲之间见伊吕\footnote{伊尹和吕尚},指挥若定失萧曹\footnote{使萧何和曹参失色}。\\
福移汉祚难恢复,志决身歼军务劳。\\
}




% \clearpage
% \section{王维}
% 
\poetry{息夫人}{
莫以今时宠,难忘旧日恩。\\
看花满眼泪,不共楚王言。\\
}
\poetry{辋川集 鹿柴}{
空山不见人,但闻人语响。\\
返景入深林,复照青苔上。\\
}
\poetry{辋川集 栾家濑}{
飒飒秋雨中,浅浅石溜泻。\\
跳波自相溅,白鹭惊复下。\\
}
\poetry{辋川集 白石滩}{
清浅白石滩,绿蒲向堪把。\\
家住水东西,浣纱明月下。\\
}
\poetry{辋川集 竹里馆}{
独坐幽篁里,弹琴复长啸。\\
深林人不知,明月来相照。\\
}
\poetry{辋川集 辛夷坞}{
木末芙蓉花,山中发红萼。\\
涧户寂无人,纷纷开且落。\\
}
\poetry{皇甫岳云溪杂题五首 鸟鸣涧}{
人闲桂花落,夜静春山空。\\
月出惊山鸟,时鸣春涧中。\\
}
\poetry{送别}{
山中相送罢,日暮掩柴扉。\\
春草明年绿,王孙归不归。\\
}
\poetry{杂诗三首 二}{
君自故乡来,应知故乡事。\\
来日绮窗前,寒梅着花未。\\
}
\poetry{崔兴宗写真咏}{
画君年少时,如今君已老。\\
今时新识人,知君旧时好。\\
}
\poetry{相思}{
红豆生南国,秋来发故枝。\\
愿君多采撷,此物最相思。\\
}
\poetry{书事}{
轻阴阁小雨,深院昼慵开。\\
坐看苍苔色,欲上人衣来。\\
}
\poetry{阙题二首 一}{
荆谿白石出,天寒红叶稀。\\
山路元无雨,空翠湿人衣。\\
}
\poetry{田园乐七首 四}{
萋萋春草秋绿,落落长松夏寒。\\
牛羊自归村巷,童稚不识衣冠。\\
}
\poetry{田园乐七首 五}{
山下孤烟远村,天边独树高原。\\
一瓢颜回陋巷,五柳先生对门。\\
}
\poetry{田园乐七首 六}{
桃红复含宿雨,柳绿更带朝烟。\\
花落家童未埽,莺啼山客犹眠。\\
}
% 
\poetry{新晴野望}{
新晴原野旷,极目无氛垢。\\
郭门临渡头,村树连谿口。\\
白水明田外,碧峰出山后。\\
农月无闲人,倾家事南亩。\\
}
\poetry{辋川闲居赠裴秀才迪}{
寒山转苍翠,秋水日潺湲。\\
倚杖柴门外,临风听暮蝉。\\
渡头余落日,墟里上孤烟。\\
复值接舆醉,狂歌五柳前。\\
}
\poetry{酬张少府}{
晚年唯好静,万事不关心。\\
自顾无长策,空知返旧林。\\
松风吹解带,山月照弹琴。\\
君问穷通理,渔歌入浦深。\\
}
\poetry{送梓州李使君}{
万壑树参天,千山响杜鹃。\\
山中一夜雨,树杪百重泉。\\
汉女输橦布,巴人讼芋田。\\
文翁翻教授,不敢倚先贤。\\
}
\poetry{过香积寺}{
不知香积寺,数里入云峰。\\
古木无人迳,深山何处钟。\\
泉声咽危石,日色冷青松。\\
薄暮空潭曲,安禅制毒龙。\\
}
\poetry{山居秋暝}{
空山新雨后,天气晚来秋。\\
明月松间照,清泉石上流。\\
竹喧归浣女,莲动下渔舟。\\
随意春芳歇,王孙自可留。\\
}
\poetry{终南别业}{
中岁颇好道,晚家南山陲。\\
兴来每独往,胜事空自知。\\
行到水穷处,坐看云起时。\\
偶然值林叟,谈笑无还期。\\
}
\poetry{山居即事}{
寂寞掩柴扉,苍茫对落晖。\\
鹤巢松树遍,人访荜门稀。\\
绿竹含新粉,红莲落故衣。\\
渡头烟火起,处处采菱归。\\
}
\poetry{终南山}{
太乙近天都,连山接海隅。\\
白云回望合,青霭入看无。\\
分野中峰变,阴晴众壑殊。\\
欲投人处宿,隔水问樵夫。\\
}
\poetry{观猎}{
风劲角弓鸣,将军猎渭城。\\
草枯鹰眼疾,雪尽马蹄轻。\\
忽过新丰市,还归细柳营。\\
回看射雕处,千里暮云平。\\
}
\poetry{汉江临泛}{
楚塞三湘接,荆门九派通。\\
江流天地外,山色有无中。\\
郡邑浮前浦,波澜动远空。\\
襄阳好风日,留醉与山翁。\\
}
\poetry{使至塞上}{
单车欲问边,属国过居延。\\
征蓬出汉塞,归雁入胡天。\\
大漠孤烟直,长河落日圆。\\
萧关逢候吏,都护在燕然。\\
}

\poetry{秋夜独坐}{
独坐悲双鬓,空堂欲二更。\\
雨中山果落,灯下草虫鸣。\\
白发终难变,黄金不可成。\\
欲知除老病,唯有学无生。\\
}

% 
\poetry{少年行四首 一}{
新丰美酒斗十千,咸阳游侠多少年。\\
相逢意气为君饮,系马高楼垂柳边。\\
}
\poetry{少年行四首 二}{
出身仕汉羽林郎,初随骠骑战渔阳。\\
孰知不向边庭苦,纵死犹闻侠骨香。\\
}
\poetry{九月九日忆山东兄弟}{
独在异乡为异客,每逢佳节倍思亲。\\
遥知兄弟登高处,遍插茱萸少一人。\\
}
\poetry{送王尊师归蜀中拜埽}{
大罗天上神仙客,濯锦江头花柳春。\\
不为碧鸡称使者,唯令白鹤报乡人。\\
}
\poetry{渭城曲}{
渭城朝雨浥轻尘,客舍青青杨柳春。\\
劝君更尽一杯酒,西出阳关无故人。\\
}
\poetry{送沈子归江东}{
杨柳渡头行客稀,罟师荡桨向临圻。\\
唯有相思似春色,江南江北送君归。\\
}
\poetry{菩提寺禁裴迪来相看说逆贼等凝碧池上作音乐供奉人等举声便一时泪下私成口号诵示裴迪}{
万户伤心生野烟,百僚何日更朝天。\\
秋槐叶落空宫里,凝碧池头奏管弦。\\
}

% \poetry{横吹曲辞 出塞}{
居延城外猎天骄,白草连天野火烧。\\
暮云空碛时驱马,秋日平原好射雕。\\
护羌校尉朝乘障,破虏将军夜渡辽。\\
玉靶角弓珠勒马,汉家将赐霍嫖姚。\\
}
\poetry{和贾舍人早朝大明宫之作}{
绛帻鸡人送晓筹,尚衣方进翠云裘。\\
九天阊阖开宫殿,万国衣冠拜冕旒。\\
日色才临仙掌动,香烟欲傍衮龙浮。\\
朝罢须裁五色诏,佩声归向凤池头。\\
}
\poetry{春日与裴迪过新昌里访吕逸人不遇}{
桃源一向绝风尘,柳市南头访隐沦。\\
到门不敢题凡鸟,看竹何须问主人。\\
城上青山如屋里,东家流水入西邻。\\
闭户著书多岁月,种松皆老作龙鳞。\\
}
\poetry{酌酒与裴迪}{
酌酒与君君自宽,人情翻覆似波澜。\\
白首相知犹按剑,朱门先达笑弹冠。\\
草色全经细雨湿,花枝欲动春风寒。\\
世事浮云何足问,不如高卧且加餐。\\
}
\poetry{积雨辋川庄作}{
积雨空林烟火迟,蒸藜炊黍饷东菑。\\
漠漠水田飞白鹭,阴阴夏木啭黄鹂。\\
山中习静观朝槿,松下清斋折露葵。\\
野老与人争席罢,海鸥何事更相疑。\\
}
% 
\poetry{送綦毋潜落第还乡}{
圣代无隐者,英灵尽来归。\\
遂令东山客,不得顾采薇。\\
既至君门远,孰云吾道非。\\
江淮度寒食,京洛缝春衣。\\
置酒临长道,同心与我违。\\
行当浮桂櫂,未几拂荆扉。\\
远树带行客,孤村当落晖。\\
吾谋适不用,勿谓知音稀。\\
}
\poetry{青谿}{
言入黄花川,每逐青谿水。\\
随山将万转,趣途无百里。\\
声喧乱石中,色静深松里。\\
漾漾泛菱荇,澄澄映葭苇。\\
我心素已闲,清川澹如此。\\
请留盘石上,垂钓将已矣。\\
}
\poetry{横吹曲辞 陇头吟}{
长安少年游侠客,夜上戍楼看太白。\\
陇头明月迥临关,陇上行人夜吹笛。\\
关西老将不胜愁,驻马听之双泪流。\\
身经大小百余战,麾下偏裨万户侯。\\
苏武才为典属国,节旄空尽海西头。\\
}
\poetry{西施咏}{
    艳色天下重,西施宁久微。\\
    朝仍越溪女,暮作吴宫妃。\\
    贱日岂殊众,贵来方悟稀。\\
    邀人傅香粉,不自著罗衣。\\
    君宠益娇态,君怜无是非。\\
    当时浣纱伴,莫得同车归。\\
    持谢邻家子,效颦安可希。\\
    }
\poetry{陇头吟}{
长安少年游侠客,夜上戍楼看太白。\\
陇头明月迥临关,陇上行人夜吹笛。\\
关西老将不胜愁,驻马听之双泪流。\\
身经大小百余战,麾下偏裨万户侯。\\
苏武才为典属国,节旄落尽海西头。\\
}
\poetry{老将行}{
少年十五二十时,步行夺得胡马射。\\
射杀中山白额虎,肯数邺下黄须儿。\\
一身转战三千里,一剑曾当百万师。\\
汉兵奋迅如霹雳,虏骑崩腾畏蒺藜。\\
卫青不败由天幸,李广无功缘数奇。\\
自从弃置便衰朽,世事磋跎成白首。\\
昔时飞箭无全目,今日垂杨生左肘。\\
路傍时卖故侯瓜,门前学种先生柳。\\
苍茫古木连穷巷,寥落寒山对虚牖。\\
誓令疏勒出飞泉,不似颍川空使酒。\\
贺兰山下阵如云,羽檄交驰日夕闻。\\
节使三河募年少,诏书五道出将军。\\
试拂铁衣如雪色,聊持宝剑动星文。\\
愿得燕弓射天将,耻令越甲鸣吴军。\\
莫嫌旧日云中守,犹堪一战取功勋。\\
}
\poetry{桃源行}{
渔舟逐水爱山春,两岸桃花夹去津。\\
坐看红树不知远,行尽青溪不见人。\\
山口潜行始隈隩,山开旷望旋平陆。\\
遥看一处攒云树,近入千家散花竹。\\
樵客初传汉姓名,居人未改秦衣服。\\
居人共住武陵源,还从物外起田园。\\
月明松下房栊静,日出云中鸡犬喧。\\
惊闻俗客争来集,竞引还家问都邑。\\
平明闾巷埽花开,薄暮渔樵乘水入。\\
初因避地去人间,及至成仙遂不还。\\
峡里谁知有人事,世中遥望空云山。\\
不疑灵境难闻见,尘心未尽思乡县。\\
出洞无论隔山水,辞家终拟长游衍。\\
自谓经过旧不迷,安知峰壑今来变。\\
当时只记入山深,青溪几曲到云林。\\
春来遍是桃花水,不辨仙源何处寻。\\
}
\poetry{洛阳女儿行}{
洛阳女儿对门居,才可容颜十五余。\\
良人玉勒乘骢马,侍女金盘鲙鲤鱼。\\
画阁朱楼尽相望,红桃绿柳垂檐向。\\
罗帏送上七香车,宝扇迎归九华帐。\\
狂夫富贵在青春,意气骄奢剧季伦。\\
自怜碧玉亲教舞,不惜珊瑚持与人。\\
春窗曙灭九微火,九微片片飞花璅。\\
戏罢曾无理曲时,妆成秪是熏香坐。\\
城中相识尽繁华,日夜经过赵李家。\\
谁怜越女颜如玉,贫贱江头自浣纱。\\
}
\poetry{送秘书晁监还日本国}{
积水不可极,安知沧海东。\\
九州何处远,万里若乘空。\\
向国唯看日,归帆但信风。\\
鳌身暎天黑,鱼眼射波红。\\
乡树扶桑外,主人孤岛中。\\
别离方异域,音信若为通。\\
}

% \clearpage




% \section{白居易}
% 
\poetry{夜雪}{
已讶衾枕冷,复见窗户明。\\
夜深知雪重,时闻折竹声。\\
}
\poetry{南浦别}{
南浦凄凄别,西风袅袅秋。\\
一看肠一断,好去莫回头。\\
}
\poetry{问刘十九}{
绿蚁新醅酒,红泥小火炉。\\
晚来天欲雪,能饮一杯无。\\
}
% \poetry{赋得古原草送别}{
离离原上草,一岁一枯荣。\\
野火烧不尽,春风吹又生。\\
远芳侵古道,晴翠接荒城。\\
又送王孙去,萋萋满别情。\\
}
% \poetry{夜筝}{
紫袖红弦明月中,自弹自感暗低容。\\
弦凝指咽声停处,别有深情一万重。\\
}
\poetry{惜牡丹花二首 一}{
惆怅阶前红牡丹,晚来唯有两枝残。\\
明朝风起应吹尽,夜惜衰红把火看。\\
}
\poetry{村夜}{
霜草苍苍虫切切,村南村北行人绝。\\
独出前门望野田,月明荞麦花如雪。\\
}


\poetry{邯郸冬至夜思家}{
邯郸驿里逢冬至,抱膝灯前影伴身。\\
想得家中夜深坐,还应说着远行人。\\
}
\poetry{大林寺桃花}{
人间四月芳菲尽,山寺桃花始盛开。\\
长恨春归无觅处,不知转入此中来。\\
}

\poetry{春词}{
低花树暎小妆楼,春入眉心两点愁。\\
斜倚栏干背鹦鹉,思量何事不回头。\\
}





\poetry{后宫词}{
泪湿罗巾梦不成,夜深前殿按歌声。\\
红颜未老恩先断,斜倚薰笼坐到明。\\
}




\poetry{白云泉}{
天平山上白云泉,云自无心水自闲。\\
何必奔冲山下去,更添波浪向人间。\\
}

\poetry{暮江吟}{
一道残阳铺水中,半江瑟瑟半江红。\\
可怜九月初三夜,露似真珠月似弓。\\
}
\poetry{杂曲歌辞 浪淘沙 四}{
借问江湖与海水,何似君情与妾心。\\
相恨不如潮有信,相思始觉海非深。\\
}


\poetry{寄湘灵}{
泪眼凌寒冻不流,每经高处即回头。\\
遥知别后西楼上,应凭阑干独自愁。\\
}
% \poetry{放言五首 三}{
赠君一法决狐疑,不用钻龟与祝蓍。\\
试玉要烧三日满,辨材须待七年期。\\
周公恐惧流言后,王莽谦恭未篡时。\\
向使当初身便死,一生真伪复谁知。\\
}
\poetry{春题湖上}{
湖上春来似画图,乱峰围绕水平铺。\\
松排山面千重翠,月点波心一颗珠。\\
碧毯线头抽早稻,青罗裙带展新蒲。\\
未能抛得杭州去,一半句留是此湖。\\
}
\poetry{钱唐湖春行}{
孤山寺北贾亭西,水面初平云脚低。\\
几处早莺争暖树,谁家新燕啄春泥。\\
乱花渐欲迷人眼,浅草才能没马蹄。\\
最爱湖东行不足,绿杨阴里白沙堤。\\
}

\poetry{望月有感}{
\footnote{原名为:自河南经乱关内阻饥兄弟离散各在一处因望月有感聊书所怀寄上浮梁大兄于潜七兄乌江十五兄兼示符离及下邽弟妹}时难年饥世业空,弟兄羇旅各西东。\\
田园寥落干戈后,骨肉流离道路中。\\
吊影分为千里雁,辞根散作九秋蓬。\\
共看明月应垂泪,一夜乡心五处同。\\ %望月有感
}


\poetry{采石墓}{
采石江边李白坟,绕田无限草连云。\\
可怜荒垅穷泉骨,曾有惊天动地文。\\
但是诗人多薄命,就中沦落不过君。\\
}













% \clearpage






% 
\section{苏轼}
\poetry{浣溪沙·山下兰芽短浸溪}{
    山下兰芽短浸溪。
    松间沙路净无泥。
    萧萧暮雨子规啼。\\
    谁道人生无再少,门前流水尚能西。
    休将白发唱黄鸡。}


\poetry{水龙吟·似花还似非花}{  
似花还似非花,也无人惜从教坠。\\
抛家傍路,思量却是,无情有思。\\
萦损柔肠,困酣娇眼,欲开还闭。\\
梦随风万里,寻郎去处,又还被、莺呼起。\\
不恨此花飞尽,恨西园、落红难缀。\\
晓来雨过,遗踪何在,一池萍碎。\\
春色三分,二分尘土,一分流水。\\
细看来,不是杨花点点,是离人泪。  
}

\poetry{满庭芳·蜗角虚名}{蜗角虚名,蝇头微利,算来著甚乾忙。\\
事皆前定,谁弱又谁强。\\
且趁闲身未老,尽放我、些子疏狂。\\
百年里,浑教是醉,三万六千场。\\
思量。\\
能几许,忧愁风雨,一半相妨。\\
又何须,抵死说短论长。\\
幸对清风皓月,苔茵展、云幕高张。\\
江南好,千锺美酒,一曲《满庭芳》。\\
}


\poetry{水调歌头·落日绣帘卷}{落日绣帘卷,亭下水连空。\\
知君为我,新作窗户湿青红。\\
长记平山堂上,敧枕江南烟雨,渺渺没孤鸿。\\
认得醉翁语,山色有无中。\\
一千顷,都镜净,倒碧峰。\\
忽然浪起,掀舞一叶白头翁。\\
堪笑兰台公子,未解庄生天籁,刚道有雌雄。\\
一点浩然气,千里快哉风。\\
}

\poetry{水调歌头·明月几时有}{明月几时有,把酒问青天。\\
不知天上宫阙,今夕是何年。\\
我欲乘风归去,又恐琼楼玉宇,高处不胜寒。\\
起舞弄清影,何似在人间。\\
转朱阁,低绮户,照无眠。\\
不应有恨,何事长向别时圆。\\
人有悲欢离合,月有阴晴圆缺,此事古难全。\\
但愿人长久,千里共婵娟。\\
}

\poetry{念奴娇·大江东去}{大江东去,浪淘尽、千古风流人物。\\
故垒西边人道是,三国周郎赤壁。\\
乱石穿空,惊涛拍岸,卷起千堆雪。\\
江山如画,一时多少豪杰。\\
遥想公瑾当年,小乔初嫁了,雄姿英发。\\
羽扇纶巾谈笑间,强虏灰飞烟灭。\\
故国神游,多情应笑,我早生华发。\\
人间如梦,一尊还酹江月。\\
}

\poetry{西江月·世事一场大梦}{世事一场大梦,人生几度秋凉。\\
夜来风叶已鸣廊。\\
看取眉头鬓上。\\
酒贱常愁客少,月明多被云妨。\\
中秋谁与共孤光。\\
把盏凄然北望。\\
}

\poetry{西江月·三过平山堂下}{三过平山堂下,半生弹指声中。\\
十年不见老仙翁。\\
壁上龙蛇飞动。\\
欲吊文章太守,仍歌杨柳春风。\\
休言万事转头空。\\
未转头时皆梦。
}


\poetry{临江仙·一别都门三改火}{一别都门三改火,天涯踏尽红尘。\\
依然一笑作春温。\\
无波真古井,有节是秋筠。\\
惆怅孤帆连夜发,送行淡月微云。\\
尊前不用翠眉颦。\\
人生如逆旅,我亦是行人。\\
}

\poetry{临江仙·夜饮东坡醒复醉}{夜饮东坡醒复醉,归来仿佛三更。\\
家童鼻息已雷鸣。\\
敲门都不应,倚杖听江声。\\
长恨此身非我有,何时忘却营营。\\
夜阑风静縠纹平。\\
小舟从此逝,江海寄馀生。\\
}

\poetry{鹧鸪天}{林断山明竹隐墙。\\
乱蝉衰草小池塘。\\
翻空白鸟时时见,照水红蕖细细香。\\
村舍外,古城旁。\\
杖藜徐步转斜阳。\\
殷勤昨夜三更雨,又得浮生一日凉。\\
}

\poetry{少年游}{去年相送,馀杭门外,飞雪似杨花。\\
今年春尽,杨花似雪,犹不见还家。\\
对酒卷帘邀明月,风露透窗纱。\\
恰似姮娥怜双燕,分明照、画梁斜。\\
}

\poetry{定风波}{莫听穿林打叶声。\\
何妨吟啸且徐行。\\
竹杖芒鞋轻胜马。\\
谁怕。\\
一蓑烟雨任平生。\\
料峭春风吹酒醒。\\
微冷。\\
山头斜照却相迎。\\
回首向来潇洒处。\\
归去。\\
也无风雨也无晴。\\
}

\poetry{定风波}{常羡人间琢玉郎。\\
天应乞与点酥娘。\\
尽道清歌传皓齿。\\
风起。\\
雪飞炎海变清凉。\\
万里归来颜愈少。\\
微笑。\\
笑时犹带岭梅香。\\
试问岭南应不好。\\
却道。\\
此心安处是吾乡。\\
}

\poetry{南乡子}{霜降水痕收。\\
浅碧鳞鳞露远洲。\\
酒力渐消风力软,飕飕。\\
破帽多情却恋头。\\
佳节若为酬。\\
但把清尊断送秋。\\
万事到头都是梦,休休。\\
明日黄花蝶也愁。
}

\poetry{南乡子}{东武望馀杭。\\
云海天涯两杳茫。\\
何日功成名遂了,还乡。\\
醉笑陪公三万场。\\
不用诉离觞。\\
痛饮从来别有肠。\\
今夜送归灯火冷,河塘。\\
堕泪羊公却姓杨。\\
}

\poetry{望江南・忆江南}{春未老,风细柳斜斜。\\
试上超然台上看,半壕春水一城花。\\
烟雨暗千家。\\
寒食后,酒醒却咨嗟。\\
休对故人思故国,且将新火试新茶。\\
诗酒趁年华。\\
}

\poetry{卜算子}{蜀客到江南,长忆吴山好。\\
吴蜀风流自古同,归去应须早。\\
还与去年人,共藉西湖草。\\
莫惜尊前仔细看,应是容颜老。\\
}

\poetry{卜算子}{缺月挂疏桐,漏断人初静。\\
时见幽人独往来,缥缈孤鸿影。\\
惊起却回头,有恨无人省。\\
拣尽寒枝不肯栖,寂寞沙洲冷。\\
}

\poetry{贺新郎}{乳燕飞华屋。\\
悄无人、桐阴转午,晚凉新浴。\\
手弄生绡白团扇,扇手一时似玉。\\
渐困倚、孤眠清熟。\\
帘外谁来推绣户,枉教人、梦断瑶台曲。\\
又却是,风敲竹。\\
石榴半吐红巾蹙。\\
待浮花、浪蕊都尽,伴君幽独。\\
秾艳一枝细看取,芳心千重似束。\\
又恐被、秋风惊绿。\\
若待得君来向此,花前对酒不忍触。\\
共粉泪,两蔌蔌。\\
}

\poetry{洞仙歌}{冰肌玉骨,自清凉无汗。\\
水殿风来暗香满。\\
绣帘开、一点明月窥人,人未寝、敧枕钗横鬓乱。\\
起来携素手,庭户无声,时见疏星渡河汉。\\
试问夜如何,夜已三更,金波淡、玉绳低转。\\
但屈指、西风几时来,又不道、流年暗中偷换。\\
}

\poetry{江神子・江城子}{凤凰山下雨初晴。\\
水风清。\\
晚霞明。\\
一朵芙蕖,开过尚盈盈。\\
何处飞来双白鹭,如有意,慕娉婷。\\
忽闻江上弄哀筝。\\
苦含情,遣谁听。\\
烟敛云收,依约是湘灵。\\
欲待曲终寻问取,人不见,数峰青。\\
}

\poetry{江神子・江城子}{老夫聊发少年狂。\\
左牵黄。\\
右擎苍。\\
锦帽貂裘,千骑卷平冈。\\
为报倾城随太守,亲射虎,看孙郎。\\
酒酣胸胆尚开张。\\
鬓微霜。\\
又何妨。\\
持节云中,何日遣冯唐。\\
会挽雕弓如满月,西北望,射天狼。\\
}

\poetry{江神子・江城子}{天涯流落思无穷。\\
既相逢。\\
却匆匆。\\
携手佳人,和泪折残红。\\
为问东风馀几许,春纵在,与谁同。\\
隋堤三月水溶溶。\\
背归鸿。\\
去吴中。\\
回首彭城,清泗与淮通。\\
寄我相思千点泪,流不到,楚江东。\\
}

\poetry{江神子・江城子}{十年生死两茫茫。\\
不思量。\\
自难忘。\\
千里孤坟,无处话凄凉。\\
纵使相逢应不识,尘满面,鬓如霜。\\
夜来幽梦忽还乡。\\
小轩窗。\\
正梳妆。\\
相顾无言,惟有泪千行。\\
料得年年断肠处,明月夜,短松冈。\\
}

\poetry{蝶恋花}{花褪残红青杏小。\\
燕子飞时,绿水人家绕。\\
枝上柳绵吹又少。\\
天涯何处无芳草。\\
墙里秋千墙外道。\\
墙外行人,墙里佳人笑。\\
笑渐不闻声渐悄。\\
多情却被无情恼。\\
}

\poetry{蝶恋花}{簌簌无风花自亸。\\
寂寞园林,柳老樱桃过。\\
落日多情还照坐。\\
山青一点横云破。\\
路尽河回千转柁。\\
系缆渔村,月暗孤灯火。\\
凭仗飞魂招楚些。\\
我思君处君思我。\\
}

\poetry{永遇乐}{明月如霜,好风如水,清景无限。\\
曲港跳鱼,圆荷泻露,寂寞无人见。\\
紞如三鼓,铿然一叶,黯黯梦云惊断。\\
夜茫茫,重寻无处,觉来小园行遍。\\
天涯倦客,山中归路,望断故园心眼。\\
燕子楼空,佳人何在,空锁楼中燕。\\
古今如梦,何曾梦觉,但有旧欢新怨。\\
异时对,黄楼夜景,为余浩叹。\\
}

\poetry{行香子}{清夜无尘。\\
月色如银。\\
酒斟时、须满十分。\\
浮名浮利,虚苦劳神。\\
叹隙中驹,石中火,梦中身。\\
虽抱文章,开口谁亲。\\
且陶陶、乐尽天真。\\
几时归去,作个闲人。\\
对一张琴,一壶酒,一溪云。\\
}

\poetry{行香子}{携手江村。
梅雪飘裙。情何限、处处消魂。\\
故人不见,旧曲重闻。\\
向望湖楼,孤山寺,涌金门。\\
寻常行处,题诗千首,绣罗衫、与拂红尘。\\
别来相忆,知是何人。\\
有湖中月,江边柳,陇头云。\\
}
一共收入了苏轼 \thepoetrycounter  首诗。

% \clearpage
% 
\section{纳兰性德}
\poetry{浣溪沙·身向云山那畔行}{身向云山那畔行,北风吹断马嘶声\\
深秋远塞若为情\\
一抹晚烟荒戍垒,半竿斜日旧关城\\
古今幽恨几时平\\
}

\poetry{浣溪沙·已惯天涯莫浪愁}{已惯天涯莫浪愁,寒云衰草渐成秋\\
漫因睡起又登楼\\
伴我萧萧惟代马,笑人寂寂有牵牛,劳人只合一生休\\
}



\poetry{浣溪沙·谁道飘零不可怜}{西郊冯氏园看海棠,因忆《香严词》有感\\
谁道飘零不可怜,旧游时节好花天\\
断肠人去自经年\\
一片晕红才著雨,几丝柔绿乍和烟\\
倩魂销尽夕阳前\\
}

\poetry{浣溪沙·残雪凝辉冷画屏}{残雪凝辉冷画屏,落梅横笛已三更,更无人处月胧明\\
我是人间惆怅客,知君何事泪纵横,断肠声里忆平生\\}






\poetry{摊破浣溪沙·风絮飘残已化萍}{风絮飘残已化萍,泥莲刚倩藕丝萦\\
珍重别拈香一瓣,记前生\\
人到情多情转薄,而今真个悔多情\\
又到断肠回首处,泪偷零\\
}

\poetry{摊破浣溪沙·葬名花}{林下荒苔道韫家,生怜玉骨委尘沙\\
愁向风前无处说,数归鸦\\
半世浮萍随逝水,一宵冷雨葬名花\\
魂是柳绵吹欲碎,绕天涯\\
}

\poetry{长相思·山一程}{山一程,水一程,身向榆关那畔行,夜深千帐灯\\
风一更,雪一更,聒碎乡心梦不成,故园无此声\\
}

\poetry{菩萨蛮·问君何事轻离别}{问君何事轻离别,一年能几团圆月\\
杨柳乍如丝\\
故园春尽时\\
春归归不得,两桨松花隔\\
旧事逐寒潮,啼鹃恨未消\\
}

\poetry{菩萨蛮·朔风吹散三更雪}{朔风吹散三更雪,倩魂犹恋桃花月\\
梦好莫催醒,由他好处行\\
无端听画角,枕畔红冰薄\\
塞马一声嘶,残星拂大旗\\
}

\poetry{菩萨蛮·萧萧几叶风兼雨}{萧萧几叶风兼雨,离人偏识长更苦\\
欹枕数秋天,蟾蜍下早弦\\
夜寒惊被薄,泪与灯花落\\
无处不伤心,轻尘在玉琴\\
}

\poetry{菩萨蛮·催花未歇花奴鼓}{催花未歇花奴鼓,酒醒已见残红舞\\
不忍覆余觞,临风泪数行\\
粉香看又别,空剩当时月\\
月也异当时,凄清照鬓丝\\
}

\poetry{菩萨蛮·新寒中酒敲窗雨}{新寒中酒敲窗雨,残香细袅秋情绪\\
才道莫伤神,青衫湿一痕\\
无聊成独卧,弹指韶光过\\
记得别伊时,桃花柳万丝\\
}

\poetry{菩萨蛮·黄云紫塞三千里}{黄云紫塞三千里,女墙西畔啼乌起\\
落日万山寒,萧萧猎马还\\
笳声听不得,入夜空城黑\\
秋梦不归家,残灯落碎花\\
}




\poetry{画堂春·一生一代一双人}{
一生一代一双人,争教两处销魂。\\
相思相望不相亲,天为谁春。\\
浆向蓝桥易乞,药成碧海难奔。\\
若容相访饮牛津,相对忘贫。\\
}


\poetry{临江仙·寒柳}{飞絮飞花何处是,层冰积雪摧残,疏疏一树五更寒\\
爱他明月好,憔悴也相关\\
最是繁丝摇落后,转教人忆春山\\
湔裙梦断续应难\\
西风多少恨,吹不散眉弯}

\poetry{蝶恋花·又到绿杨曾折处}{又到绿杨曾折处,不语垂鞭,踏遍清秋路\\
衰草连天无意绪,雁声远向萧关去\\
不恨天涯行役苦,只恨西风,吹梦成今古\\
明日客程还几许,沾衣况是新寒雨\\
}

\poetry{蝶恋花·辛苦最怜天上月}{辛苦最怜天上月,一夕如环,夕夕都成玦\\
若似月轮终皎洁,不辞冰雪为卿热\\
无那尘缘容易绝,燕子依然,软踏帘钩说\\
唱罢秋坟愁未歇,春丛认取双栖蝶\\
}

\poetry{蝶恋花·出塞}{今古河山无定据\\
画角声中,牧马频来去\\
满目荒凉谁可语\\
西风吹老丹枫树\\
从前幽怨应无数\\
铁马金戈,青冢黄昏路\\
一往情深深几许\\
深山夕照深秋雨\\
}

一共收入了纳兰性德 \thepoetrycounter  首诗。

% \clearpage
% \section{李贺}
\poetry{梦天}{
老兔寒蟾泣天色,云楼半开壁斜白。\\
玉轮轧露湿团光,鸾珮相逢桂香陌。\\
黄尘清水三山下,更变千年如走马。\\
遥望齐州九点烟,一泓海水杯中泻。\\
}
\poetry{苦昼短}{
飞光飞光,劝尔一杯酒。\\
吾不识青天高,黄地厚。\\
唯见月寒日暖,来煎人寿。\\
食熊则肥,食蛙则瘦。\\
神君何在,太一安有。\\
天东有若木,下置衔烛龙。\\
吾将斩龙足,嚼龙肉。\\
使之朝不得回,夜不得伏。\\
自然老者不死,少者不哭。\\
何为服黄金,吞白玉。\\
谁似任公子,云中骑碧驴。\\
刘彻茂陵多滞骨,嬴政梓棺费鲍鱼。\\
}
\poetry{鼓吹曲辞 将进酒}{
瑠璃锺,琥珀浓,小槽酒滴真珠红。\\
烹龙炮凤玉脂泣,罗屏绣幕围香风。\\
吹龙笛,击鼍鼓,皓齿歌,细腰舞。\\
况是青春日将暮,桃花乱落如红雨。\\
劝君终日酩酊醉,酒不到刘伶坟上土。\\
}


\poetry{相和歌辞 雁门太守行}{
黑云压城城欲摧,甲光向月金鳞开。\\
角声满天秋色里,塞上燕支凝夜紫。\\
半卷红旗临易水,霜重鼓寒声不起。\\
报君黄金台上意,提携玉龙为君死。\\
}
\poetry{杂歌谣辞 苏小小歌}{
幽兰露,如啼眼。\\
无物结同心,烟花不堪翦。\\
草如茵,松如盖。\\
风为裳,水为佩。\\
油壁车,久相待。\\
冷翠烛,劳光彩。\\
西陵下,风吹雨。\\
}
\poetry{李凭箜篌引}{
吴丝蜀桐张高秋,空白凝云颓不流。\\
江娥啼竹素女愁,李凭中国弹箜篌。\\
昆山玉碎凤皇叫,芙蓉泣露香兰笑。\\
十二门前融冷光,二十三丝动紫皇。\\
女娲炼石补天处,石破天惊逗秋雨。\\
梦入坤山教神妪,老鱼跳波瘦蛟舞。\\
吴质不眠倚桂树,露脚斜飞湿寒兔。\\
}
\poetry{雁门太守行}{
黑云压城城欲摧,甲光向日金鳞开。\\
角声满天秋色里,塞上燕脂凝夜紫。\\
半卷红旗临易水,霜重鼓寒声不起。\\
报君黄金台上意,提携玉龙为君死。\\
}
\poetry{秦王饮酒}{
秦王骑虎游八极,剑光照空天自碧。\\
羲和敲日玻璃声,劫灰飞尽古今平。\\
龙头泻酒邀酒星,金槽琵琶夜枨枨。\\
洞庭雨脚来吹笙,酒酣喝月使倒行。\\
银云栉栉瑶殿明,宫门掌事报一更。\\
花楼玉凤声娇狞,海绡红文香浅清。\\
黄鹅跌舞千年觥,仙人烛树蜡烟轻,清琴醉眼泪泓泓。\\
}
\poetry{南园十三首 五}{
男儿何不带吴钩,收取关山五十州。\\
请君暂上凌烟阁,若个书生万户侯。\\
}
\poetry{南园十三首 六}{
寻章摘句老雕虫,晓月当帘挂玉弓。\\
不见年年辽海上,文章何处哭秋风。\\
}
\poetry{马诗二十三首 四}{
此马非凡马,房星本是星。\\
向前敲瘦骨,犹自带铜声。\\
}
\poetry{马诗二十三首 五}{
大漠山如雪,燕山月似钩。\\
何当金络脑,快走踏清秋。\\
}

一共有 \thepoetrycounterall 首.

\end{document}


