\poetry{鼓吹曲辞 艾如张}{
锦襜褕,绣裆襦,强强饮啄哺尔雏。\\
陇东卧穟满风雨,莫信龙媒陇西去。\\
齐人织网如素空,张在野春平碧中。\\
网丝漠漠无形影,误尔触之伤首红。\\
艾叶绿花谁翦刻,中藏祸机不可测。\\
}
\poetry{鼓吹曲辞 上之回}{
上之回,大旗喜。\\
悬虹彗,挞凤尾。\\
剑匣破,舞蛟龙。\\
蚩尤死,鼓逢逢。\\
天高庆雷齐坠地,地无惊烟海千里。\\
}
\poetry{鼓吹曲辞 巫山高}{
碧丛丛,高插天,大江翻澜神曳烟。\\
楚魂寻梦风飔然,晓风飞雨生苔钱。\\
瑶姬一去一千年,丁香笻竹啼老猿。\\
古祠近月蟾桂寒,椒花坠红湿云间。\\
}
\poetry{鼓吹曲辞 将进酒}{
将进酒,将进酒。\\
酒中有毒鸩主父,言之主父伤主母。\\
母为妾地父妾天,仰天俯地不忍言。\\
佯为僵踣主父前,主父不知加妾鞭。\\
旁人知妾为主说,主将泪洗鞭头血。\\
推摧主母牵下堂,扶妾遣升堂上床。\\
将进酒,酒中无毒令主寿。\\
愿主回思归主母,遣妾如此事主父。\\
妾为此事人偶知,自惭不密方自悲。\\
主今颠倒安置妾,贪天僭地谁不为。\\
}
\poetry{鼓吹曲辞 将进酒}{
瑠璃锺,琥珀浓,小槽酒滴真珠红。\\
烹龙炮凤玉脂泣,罗屏绣幕围香风。\\
吹龙笛,击鼍鼓,皓齿歌,细腰舞。\\
况是青春日将暮,桃花乱落如红雨。\\
劝君终日酩酊醉,酒不到刘伶坟上土。\\
}
\poetry{相和歌辞 箜篌引}{
公乎公乎,提壶将焉如?屈平沈湘不足慕,徐衍入海诚为愚。\\
公乎公乎,床有菅席盘有鱼。\\
北里有贤兄,东邻有小姑。\\
陇亩油油黍与葫,瓦甒浊醪蚁浮浮。\\
黍可食,醪可饮,公乎公乎其奈居。\\
被发奔流竟何如?贤兄小姑哭呜呜。\\
}
\poetry{相和歌辞 江南曲}{
汀洲白蘋草,柳恽乘马归。\\
江头樝树香,岸上蝴蝶飞。\\
酒杯若叶露,玉轸蜀桐虚。\\
朱楼通水陌,沙暖一双鱼。\\
}
\poetry{相和歌辞 日出行}{
白日下崐崘,发光如舒丝。\\
徒照葵藿心,不照游子悲。\\
折折黄河曲,日从中央转。\\
旸谷耳曾闻,若木眼不见。\\
奈何铄石,胡为销人。\\
羿弯弓属矢那不中,足令久不得奔,讵教晨光夕昏。\\
}
\poetry{相和歌辞 蜀国弦}{
枫香晚华静,锦水南山影。\\
惊石坠猿哀,竹云愁半岭。\\
凉月生秋浦,玉沙鳞鳞光。\\
谁家红泪客,不忍过瞿塘。\\
}
\poetry{相和歌辞 铜雀妓}{
佳人一壶酒,秋容满千里。\\
石马卧新烟,忧来何所似?歌声且潜弄,陵树风自起。\\
长裾压高台,泪眼看花机。\\
}
\poetry{相和歌辞 长歌续短歌}{
长歌破衣襟,短歌断白发。\\
秦王不可见,旦夕成内热。\\
渴饮壶中酒,饥拔陇头粟。\\
凄凄四月兰,千里一时绿。\\
夜峰何离离,月明落石底。\\
裴回沿石寻,照出高峰外。\\
不得与之游,歌成鬓先改。\\
}
\poetry{相和歌辞 猛虎行}{
长戈莫舂,彊弩莫烹。\\
乳孙哺子,教得生狞。\\
举头为城,掉尾为旌。\\
东海黄公,愁见夜行。\\
道逢驺虞,牛哀不平。\\
生何用尺刀,壁上雷鸣。\\
泰山之下,妇人哭声。\\
官家有程,吏不敢听。\\
}
\poetry{相和歌辞 难忘曲}{
夹道开洞门,弱杨低画戟。\\
帘影竹叶起,箫声吹日色。\\
蜂语绕妆镜,拂蛾学春碧。\\
乱系丁香梢,满阑花向夕。\\
}
\poetry{相和歌辞 塘上行}{
藕花凉露湿,花缺藕根涩。\\
飞下雌鸳鸯,塘水声溢溢。\\
}
\poetry{相和歌辞 安乐宫}{
深井桐乌起,尚复牵清水。\\
未盥邵陵王,瓶中弄长翠。\\
新城安乐宫,宫如凤皇翅。\\
歌回蜡版鸣,大绾提壶使。\\
绿繁悲水曲,茱萸别秋子。\\
}
\poetry{相和歌辞 雁门太守行}{
黑云压城城欲摧,甲光向月金鳞开。\\
角声满天秋色里,塞上燕支凝夜紫。\\
半卷红旗临易水,霜重鼓寒声不起。\\
报君黄金台上意,提携玉龙为君死。\\
}
\poetry{相和歌辞 神弦曲}{
西山日没东山昏,旋风吹马马踏云。\\
画弦素管声浅繁,花裙綷{纟蔡}步秋尘。\\
桂叶刷风桂坠子,青狸哭血寒狐死。\\
古壁彩虬金帖尾,雨工骑入秋潭水。\\
百年老鸮成木魅,笑声碧火巢中起。\\
}
\poetry{相和歌辞 神弦别曲}{
巫山小女隔云别,松花春风山上发。\\
绿盖独穿香迳归,白马花竿前孑孑。\\
蜀江风澹水如罗,堕兰谁泛相经过。\\
南山桂树为君死,云衫残污红脂花。\\
}
\poetry{相和歌辞 莫愁曲}{
草生陇坂下,鸦噪城堞头。\\
何人此城里,城角栽石榴。\\
青丝系五马,黄金络双牛。\\
白鱼驾莲船,夜作十里游。\\
归来无人识,暗上沈香楼。\\
罗床倚瑶瑟,残月倾帘钩。\\
今日槿花落,明朝梧树秋。\\
若负平生意,何名作莫愁。\\
}
\poetry{相和歌辞 大堤曲}{
妾家住横塘,红纱满桂香。\\
青云教绾头上髻,明月与作耳边珰。\\
莲风起,江畔春。\\
大堤上,留北人。\\
郎食鲤鱼尾,妾食猩猩唇。\\
莫指襄阳道,绿浦归帆少。\\
今日菖蒲花,明朝枫树老。\\
}
\poetry{相和歌辞 江南弄}{
江中绿雾起凉波,天上叠𪩘红嵯峨。\\
水风浦云生老竹,渚瞑蒲帆如一幅。\\
鲈鱼千头酒百斛,酒中倒卧南山绿。\\
吴歈越吟未终曲,江上团团帖寒玉。\\
}
\poetry{相和歌辞 上云乐}{
飞香走红满天春,花龙盘盘上紫云。\\
三千宫女列金屋,五十弦瑟海上闻。\\
大江碎碎银沙路,嬴女机中断烟素。\\
断烟素,缝舞衣,八月一日君前舞。\\
}
\poetry{舞曲歌辞 章和二年中}{
云萧素,风拂拂,麦芒如篲黍和粟。\\
关中父老百领襦,关东吏人乏诟租。\\
健犊春耕土膏黑,菖蒲丛丛沿水脉。\\
殷勤为我下田鉏,百钱携赏丝桐客。\\
游春漫光坞花白,野林散香神降席。\\
拜神得寿献天子,七星贯断姮娥死。\\
}
\poetry{舞曲歌辞 公莫舞歌}{
方花古础排九楹,刺豹淋血盛银罂。\\
华筵鼓吹无桐竹,长刀直立割鸣筝。\\
横楣麤锦生红纬,日炙锦嫣王未醉。\\
腰下三看宝玦光,项庄掉箾拦前起。\\
材官小臣公莫舞,座上真人赤龙子。\\
芒砀云瑞抱天回,咸阳王气清如水。\\
铁枢铁楗重束关,大旗五丈撞双环。\\
汉王今日须秦印,绝膑刳肠臣不论。\\
}
\poetry{舞曲歌辞 拂舞辞}{
吴娥声绝天,空云闲裴回。\\
门外满车马,亦须生绿苔。\\
尊有乌程酒,劝君千万寿。\\
全胜汉武锦楼上,晓望晴寒饮花露。\\
东方日不破,天光无老时。\\
丹成作蛇乘白雾,千年重化玉井龟。\\
从蛇作龟二千载,吴堤绿草年年在。\\
背有八卦称神仙,邪鳞顽甲滑腥涎。\\
}
\poetry{琴曲歌辞 湘妃}{
筠竹千年老不死,长伴秦娥盖湘水。\\
蛮娘吟弄满寒空,九山静绿泪花红。\\
离鸾别凤烟梧中,巫云蜀雨遥相通。\\
幽愁秋气上青枫,凉夜波间吟古龙。\\
}
\poetry{琴曲歌辞 走马引}{
我有辞乡剑,玉锋堪截云。\\
襄阳走马客,意气自生春。\\
朝嫌剑光静,暮嫌剑花冷。\\
能持剑向人,不解持照身。\\
}
\poetry{琴曲歌辞  渌水辞}{
今宵好风月,阿侯在何处?为有倾城色,翻成足愁苦。\\
东湖采莲叶,南湖拔蒲根。\\
未持寄小姑,且持感愁魂。\\
}
\poetry{杂曲歌辞 神仙曲}{
碧峰海面藏灵书,上帝拣作神仙居。\\
晴时笑语闻空虚,鬬乘巨浪骑鲸鱼。\\
春罗翦字邀王母,共宴红楼最深处。\\
鹤羽冲风过海迟,不如却使青龙去。\\
犹疑王母不相许,垂露娃鬟更传语。\\
}
\poetry{杂曲歌辞 少年乐}{
芳草落花如锦地,二十长游醉乡里。\\
红缨不重白马骄,垂柳金丝香拂水。\\
吴娥未笑花不开,绿鬓耸堕兰云起。\\
陆郎倚醉牵罗袂,夺得宝钗金翡翠。\\
}
\poetry{杂曲歌辞 浩歌}{
南风吹山作平地,帝遣天吴移海水。\\
王母桃花千遍红,彭祖巫咸几回死。\\
青毛骢马参差钱,娇春杨柳含细烟。\\
筝人劝我金屈卮,神血未凝身问谁。\\
不须浪饮丁督护,世上英雄本无主。\\
买丝绣作平原君,有酒惟浇赵州土。\\
漏催水咽玉蟾蜍,卫娘发薄不胜梳。\\
看见秋眉换深绿,二十男儿那刺促。\\
}
\poetry{杂曲歌辞 摩多楼子}{
玉塞去金人,二万四千里。\\
风吹沙作云,一时渡辽水。\\
天白水如练,甲丝双串断。\\
行行莫苦辛,城月犹残半。\\
晓气朔烟上,趢趗胡马蹄。\\
行人听水别,隔陇长东西。\\
}
\poetry{杂曲歌辞 堂堂}{
堂堂复堂堂,红脱梅灰香。\\
十年粉蠹生画梁,饥虫不食推碎黄。\\
蕙花已老桃叶长,禁院悬帘隔御光。\\
华清源中礜石汤,裴回百凤随君王。\\
}
\poetry{杂曲歌辞 十二月乐辞 正月}{
上楼迎春新春归,暗黄著柳宫漏迟。\\
薄薄淡霭弄野姿,寒绿幽泥生短丝。\\
锦床晓卧玉肌冷,露脸未开对朝暝。\\
官街柳带不堪折,早晚菖蒲胜绾结。\\
}
\poetry{杂曲歌辞 十二月乐辞 二月}{
二月饮酒采桑津,宜男草生兰笑人。\\
蒲如交剑风如薰,劳劳胡燕怨酣春。\\
薇帐逗烟生绿尘,金翅峨髻愁暮云。\\
沓飒起舞真珠裙,津头送别唱流水。\\
酒客背寒南山死。\\
}
\poetry{杂曲歌辞 十二月乐辞 三月}{
东方风来满眼春,花城柳暗愁几人,复宫深殿竹风起。\\
新翠舞襟静如水,光风转蕙百余里。\\
暖雾驱云扑天地,军装宫妓扫蛾浅。\\
摇摇锦旗夹城暖,曲水飘香去不归。\\
梨花落尽成秋苑。\\
}
\poetry{杂曲歌辞 十二月乐辞 四月}{
晓凉暮凉树如盖,千山浓绿生云外。\\
依微香雨青氛氲,腻叶蟠花照曲门。\\
金塘闲水摇碧漪,老景沉重无惊飞,堕红残萼暗参差。\\
}
\poetry{杂曲歌辞 十二月乐辞 五月}{
雕玉押帘上,轻縠笼虚门。\\
井汲铅华水,扇织鸳鸯文。\\
回雪舞凉殿,甘露洗空绿。\\
罗袖从徊翔,香汗霑宝粟。\\
}
\poetry{杂曲歌辞 十二月乐辞 六月}{
裁生罗,伐湘竹。\\
帔拂疏霜簟秋玉,炎炎红镜东方开。\\
晕如车轮上徘徊,啾啾赤帝骑龙来。\\
}
\poetry{杂曲歌辞 十二月乐辞 七月}{
星依云渚冷,露滴盘中圆。\\
好花生木末,衰蕙愁空园。\\
夜天如玉砌,池叶极青钱。\\
仅厌舞衫薄,稍知花簟寒。\\
晓风何拂拂,北斗光阑干。\\
}
\poetry{杂曲歌辞 十二月乐辞 八月}{
孀妾怨长夜,独客梦归家。\\
傍檐虫缉丝,向壁灯垂花。\\
檐外月光吐,帘中树影斜。\\
悠悠飞露姿,点缀池中荷。\\
}
\poetry{杂曲歌辞 十二月乐辞 九月}{
离宫散萤天似水,竹黄池冷芙蓉死。\\
月缀金铺光脉脉,凉苑虚庭空澹白。\\
霜花飞飞风草草,翠锦斓斑满层道。\\
鸡人罢唱晓珑璁,鸦啼金井下疏桐。\\
}
\poetry{杂曲歌辞 十二月乐辞 十月}{
玉壶银箭稍难倾,釭花夜笑凝幽明。\\
碎霜斜舞上罗幕,烛笼两行照飞阁。\\
珠帷怨卧不成眠,金凤刺衣着体寒,长眉对月鬬弯环。\\
}
\poetry{杂曲歌辞 十二月乐辞 十一月}{
宫城团回凛严光,白光碎碎堕琼芳。\\
挝锺高饮千日酒,却天凝寒作君寿。\\
}
\poetry{杂曲歌辞 十二月乐辞 十二月}{
日脚淡光红洒洒,薄霜不销桂枝下。\\
依稀和气解冬严,已就长日辞长夜。\\
}
\poetry{杂曲歌辞 十二月乐辞 闰月}{
帝重光,年重时,七十二候回环推。\\
天官玉琯灰剩飞,今岁何长来岁迟。\\
王母移桃献天子,羲氏和氏迂龙辔。\\
}
\poetry{杂歌谣辞 李夫人歌}{
紫皇宫殿重重开,夫人飞入琼瑶台。\\
绿香绣帐何时歇,青云无光宫水咽。\\
翩联桂花坠秋月,孤鸾惊啼商丝发。\\
红璧阑珊悬佩珰,歌台小妓遥相望。\\
玉蟾滴水鸡人唱,露华兰叶参差光。\\
}
\poetry{杂歌谣辞 苏小小歌}{
幽兰露,如啼眼。\\
无物结同心,烟花不堪翦。\\
草如茵,松如盖。\\
风为裳,水为佩。\\
油壁车,久相待。\\
冷翠烛,劳光彩。\\
西陵下,风吹雨。\\
}
\poetry{杂歌谣辞 邺城童子谣}{
邺城中,暮尘起。\\
将黑丸,斫文吏。\\
棘为鞭,虎为马。\\
团团走,邺城下。\\
切玉剑,射日弓。\\
献何人,奉相公。\\
扶毂来,关右儿。\\
香扫涂,相公归。\\
}
\poetry{李凭箜篌引}{
吴丝蜀桐张高秋,空白凝云颓不流。\\
江娥啼竹素女愁,李凭中国弹箜篌。\\
昆山玉碎凤皇叫,芙蓉泣露香兰笑。\\
十二门前融冷光,二十三丝动紫皇。\\
女娲炼石补天处,石破天惊逗秋雨。\\
梦入坤山教神妪,老鱼跳波瘦蛟舞。\\
吴质不眠倚桂树,露脚斜飞湿寒兔。\\
}
\poetry{残丝曲}{
垂杨叶老莺哺儿,残丝欲断黄蜂归。\\
绿鬓年少金钗客,缥粉壶中沈琥珀。\\
花台欲暮春辞去,落花起作回风舞。\\
榆荚相催不知数,沈郎青钱夹城路。\\
}
\poetry{还自会稽歌}{
野粉椒壁黄,湿萤满梁殿。\\
台城应教人,秋衾梦铜辇。\\
吴霜点归鬓,身与塘蒲晚。\\
脉脉辞金鱼,羁臣守迍贱。\\
}
\poetry{出城寄权璩杨敬之}{
草暖云昏万里春,宫花拂面送行人。\\
自言汉剑当飞去,何事还车载病身。\\
}
\poetry{示弟}{
别弟三年后,还家一日余。\\
醁醽今夕酒,缃帙去时书。\\
病骨犹能在,人间底事无。\\
何须问牛马,抛掷任枭卢。\\
}
\poetry{竹}{
入水文光动,抽空绿影春。\\
露华生笋迳,苔色拂霜根。\\
织可承香汗,裁堪钓锦鳞。\\
三梁曾入用,一节奉王孙。\\
}
\poetry{同沈驸马赋得御沟水}{
入苑白泱泱,宫入正靥黄。\\
遶堤龙骨冷,拂岸鸭头香。\\
别馆惊残梦,停杯泛小觞。\\
幸因流浪处,暂得见何郎。\\
}
\poetry{始为奉礼忆昌谷山居}{
扫断马蹄痕,衙回自闭门。\\
长枪江米熟,小树枣花春。\\
向壁悬如意,当帘阅角巾。\\
犬书曾去洛,鹤病悔游秦。\\
土甑封茶叶,山杯锁竹根。\\
不知船上月,谁櫂满溪云。\\
}
\poetry{七夕}{
别浦今朝暗,罗帷午夜愁。\\
鹊辞穿线月,花入曝衣楼。\\
天上分金镜,人间望玉钩。\\
钱塘苏小小,更值一年秋。\\
}
\poetry{过华清宫}{
春月夜啼鸦,宫帘隔御花。\\
云生朱络暗,石断紫钱斜。\\
玉椀盛残露,银灯点旧纱。\\
蜀王无近信,泉上有芹芽。\\
}
\poetry{送沈亚之歌}{
吴兴才人怨春风,桃花满陌千里红。\\
紫丝竹断騘马小,家住钱塘东复东。\\
白藤交穿织书笈,短策齐裁如梵夹。\\
雄光宝矿献春卿,烟底蓦波乘一叶。\\
春卿拾材白日下,掷置黄金解龙马。\\
携笈归江重入门,劳劳谁是怜君者。\\
吾闻壮夫重心骨,古人三走无摧捽。\\
请君待旦事长鞭,他日还辕及秋律。\\
}
\poetry{咏怀二首 一}{
长卿怀茂陵,绿草垂石井。\\
弹琴看文君,春风吹鬓影。\\
梁王与武帝,弃之如断梗。\\
惟留一简书,金泥泰山顶。\\
}
\poetry{咏怀二首 二}{
日夕著书罢,惊霜落素丝。\\
镜中聊自笑,讵是南山期。\\
头上无幅巾,苦蘗已染衣。\\
不见清溪鱼,饮水得自宜。\\
}
\poetry{追和柳恽}{
汀洲白蘋草,柳恽乘马归。\\
江头樝树香,岸上蝴蝶飞。\\
酒杯箬叶露,玉轸蜀桐虚。\\
朱楼通水陌,沙暖一双鱼。\\
}
\poetry{春坊正字剑子歌}{
先辈匣中三尺水,曾入吴潭斩龙子。\\
隙月斜明刮露寒,练带平铺吹不起。\\
蛟胎皮老蒺藜刺,䴙鹈淬花白鹇尾。\\
直是荆轲一片心,莫教照见春坊字。\\
挼丝团金悬{罒/鹿}{罒/敕},神光欲截蓝田玉。\\
提出西方白帝惊,嗷嗷鬼母秋郊哭。\\
}
\poetry{贵公子夜阑曲}{
袅袅沈水烟,乌啼夜阑景。\\
曲沼芙蓉波,腰围白玉冷。\\
}
\poetry{雁门太守行}{
黑云压城城欲摧,甲光向日金鳞开。\\
角声满天秋色里,塞上燕脂凝夜紫。\\
半卷红旗临易水,霜重鼓寒声不起。\\
报君黄金台上意,提携玉龙为君死。\\
}
\poetry{大堤曲}{
妾家住横塘,红纱满桂香。\\
青云教绾头上髻,明月与作耳边珰。\\
莲风起,江畔春。\\
大堤上,留北人。\\
郎食鲤鱼尾,妾食猩猩唇。\\
莫指襄阳道,绿浦归帆少。\\
今日菖蒲花,明朝枫树老。\\
}
\poetry{蜀国弦}{
枫香晚花静,锦水南山影。\\
惊石坠猿哀,竹云愁半岭。\\
凉月生秋浦,玉沙粼粼光。\\
谁家红泪客,不忍过瞿塘。\\
}
\poetry{苏小小墓}{
幽兰露,如啼眼。\\
无物结同心,烟花不堪剪。\\
草如茵,松如盖。\\
风为裳,水为珮。\\
油壁车,夕相待。\\
冷翠烛,劳光彩。\\
西陵下,风吹雨。\\
}
\poetry{梦天}{
老兔寒蟾泣天色,云楼半开壁斜白。\\
玉轮轧露湿团光,鸾珮相逢桂香陌。\\
黄尘清水三山下,更变千年如走马。\\
遥望齐州九点烟,一泓海水杯中泻。\\
}
\poetry{唐儿歌}{
头玉硗硗眉刷翠,杜郎生得真男子。\\
骨重神寒天庙器,一双瞳人剪秋水。\\
竹马梢梢摇绿尾,银鸾睒光踏半臂。\\
东家娇娘求对值,浓笑画空作唐字。\\
眼大心雄知所以,莫忘作歌人姓李。\\
}
\poetry{绿章封事}{
青霓扣额呼宫神,鸿龙玉狗开天门。\\
石榴花发满溪津,溪女洗花染白云。\\
绿章封事咨元父,六街马蹄浩无主。\\
虚空风气不清泠,短衣小冠作尘土。\\
金家香衖千轮鸣,杨雄秋室无俗声。\\
愿携汉戟招书鬼,休令恨骨填蒿里。\\
}
\poetry{河南府试十二月乐词 正月}{
上楼迎春新春归,暗黄著柳宫漏迟。\\
薄薄淡霭弄野姿,寒绿幽风生短丝。\\
锦床晓卧玉肌冷,露脸未开对朝暝。\\
官街柳带不堪折,早晚菖蒲胜绾结。\\
}
\poetry{河南府试十二月乐词 二月}{
二月饮酒采桑津,宜男草生兰笑人。\\
蒲如交剑风如薰,劳劳胡燕怨酣春。\\
薇帐逗烟生绿尘,金翘峨髻愁暮云。\\
沓飒起舞真珠裙,津头送别唱流水,酒客背寒南山死。\\
}
\poetry{河南府试十二月乐词 三月}{
东方风来满眼春,花城柳暗愁杀人。\\
复宫深殿竹风起,新翠舞衿净如水。\\
光风转蕙百余里,暖雾驱云扑天地。\\
军装宫妓扫蛾浅,摇摇锦旗夹城暖。\\
曲水漂香去不归,梨花落尽成秋苑。\\
}
\poetry{河南府试十二月乐词 四月}{
晓凉暮凉树如盖,千山浓绿生云外。\\
依微香雨青氛氲,腻叶蟠花照曲门。\\
金塘闲水摇碧漪,老景沉重无惊飞,堕红残萼暗参差。\\
}
\poetry{河南府试十二月乐词 五月}{
雕玉押帘额,轻縠笼虚门。\\
井汲铅华水,扇织鸳鸯纹。\\
回雪舞凉殿,甘露洗空绿。\\
罗袖从徊翔,香汗沾宝粟。\\
}
\poetry{河南府试十二月乐词 六月}{
裁生罗,伐湘竹。\\
帔拂疏霜簟秋玉,炎炎红镜东方开。\\
晕如车轮上裴回,啾啾赤帝骑龙来。\\
}
\poetry{河南府试十二月乐词 七月}{
星依云渚冷,露滴盘中圆。\\
好花生木末,衰蕙愁空园。\\
夜天如玉砌,池叶极青钱。\\
仅厌舞衫薄,稍知花簟寒。\\
晓风何拂拂,北斗光阑干。\\
}
\poetry{河南府试十二月乐词 八月}{
孀妾怨夜长,独客梦归家。\\
傍檐虫缉丝,向壁灯垂花。\\
帘外月光吐,帘内树影斜。\\
悠悠飞露姿,点缀池中荷。\\
}
\poetry{河南府试十二月乐词 九月}{
离宫散萤天似水,竹黄池冷芙蓉死。\\
月缀金铺光脉脉,凉苑虚庭空澹白。\\
露花飞飞风草草,翠锦斓斑满层道。\\
鸡人罢唱晓珑璁{王匆},鸦啼金井下疏桐。\\
}
\poetry{河南府试十二月乐词 十月}{
玉壶银箭稍难倾,釭花夜笑凝幽明。\\
碎霜斜舞上罗幕,烛龙两行照飞阁。\\
珠帷怨卧不成眠,金凤刺衣着体寒,长眉对月鬬弯环。\\
}
\poetry{河南府试十二月乐词 十一月}{
宫城团回凛严光,白天碎碎堕琼芳。\\
挝锺高饮千日酒,却天凝寒作君寿。\\
}
\poetry{河南府试十二月乐词 十二月}{
日脚淡光红洒洒,薄霜不销桂枝下。\\
依稀和气排冬严,已就长日辞长夜。\\
}
\poetry{河南府试十二月乐词 闰月}{
帝重光,年重时。\\
七十二候回环推,天官玉琯灰剩飞。\\
今岁何长来岁迟,王母移桃献天子。\\
}
\poetry{天上谣}{
天河夜转漂回星,银浦流云学水声。\\
玉宫桂树花未落,仙妾采香垂珮缨。\\
秦妃卷帘北窗晓,窗前植桐青凤小。\\
王子吹笙鹅管长,呼龙耕烟种瑶草。\\
粉霞红绶藕丝裙,青洲步拾兰苕春。\\
东指羲和能走马,海尘新生石山下。\\
}
\poetry{浩歌}{
南风吹山作平地,帝遣天吴移海水。\\
王母桃花千遍红,彭祖巫咸几回死。\\
青毛骢马参差钱,娇春杨柳含细烟。\\
筝人劝我金屈卮,神血未凝身问谁。\\
不须浪饮丁都护,世上英雄本无主。\\
买丝绣作平原君,有酒惟浇赵州土。\\
漏催水咽玉蟾蜍,卫娘发薄不胜梳。\\
看见秋眉换新绿,二十男儿那刺促。\\
}
\poetry{秋来}{
桐风惊心壮士苦,衰灯络纬啼寒素。\\
谁看青简一编书,不遣花虫粉空蠹。\\
思牵今夜肠应直,雨冷香魂吊书客。\\
秋坟鬼唱鲍家诗,恨血千年土中碧。\\
}
\poetry{帝子歌}{
洞庭明月一千里,凉风雁啼天在水。\\
九节菖蒲石上死,湘神弹琴迎帝子。\\
山头老桂吹古香,雌龙怨吟寒水光。\\
沙浦走鱼白石郎,闲取真珠掷龙堂。\\
}
\poetry{秦王饮酒}{
秦王骑虎游八极,剑光照空天自碧。\\
羲和敲日玻璃声,劫灰飞尽古今平。\\
龙头泻酒邀酒星,金槽琵琶夜枨枨。\\
洞庭雨脚来吹笙,酒酣喝月使倒行。\\
银云栉栉瑶殿明,宫门掌事报一更。\\
花楼玉凤声娇狞,海绡红文香浅清。\\
黄鹅跌舞千年觥,仙人烛树蜡烟轻,清琴醉眼泪泓泓。\\
}
\poetry{洛姝真珠}{
真珠小娘下清廓,洛苑香风飞绰绰。\\
寒鬓斜钗玉燕光,高楼唱月敲悬珰。\\
兰风桂露洒幽翠,红弦袅云咽深思。\\
花袍白马不归来,浓蛾叠柳香唇醉。\\
金鹅屏风蜀山梦,鸾裾凤带行烟重。\\
八骢笼晃脸差移,日丝繁散曛罗洞。\\
市南曲陌无秋凉,楚腰卫鬓四时芳。\\
玉喉窱窱排空光,牵云曳雪留陆郎。\\
}
\poetry{李夫人歌}{
紫皇宫殿重重开,夫人飞入琼瑶台。\\
绿香绣帐何时歇,青云无光宫水咽。\\
翩联桂花坠秋月,孤鸾惊啼商丝发。\\
红壁阑珊悬珮珰,歌台小妓遥相望。\\
玉蟾滴水鸡人唱,露华兰叶参差光。\\
}
\poetry{走马引}{
我有辞乡剑,玉锋堪截云。\\
襄阳走马客,意气自生春。\\
朝嫌剑花净,暮嫌剑光冷。\\
能持剑向人,不解持照身。\\
}
\poetry{湘妃}{
筠竹千年老不死,长伴秦娥盖湘水。\\
蛮娘吟弄满寒空,九山静绿泪花红。\\
离鸾别凤烟梧中,巫云蜀雨遥相通。\\
幽愁秋气上青枫,凉夜波间吟古龙。\\
}
\poetry{南园十三首 一}{
花枝草蔓眼中开,小白长红越女腮。\\
可怜日暮嫣香落,嫁与春风不用媒。\\
}
\poetry{南园十三首 二}{
宫北田塍晓气酣,黄桑饮露窣宫帘。\\
长腰健妇偷攀折,将𫗪吴王八茧蚕。\\
}
\poetry{南园十三首 三}{
竹里缲丝挑网车,青蝉独噪日光斜。\\
桃胶迎夏香琥珀,自课越佣能种瓜。\\
}
\poetry{南园十三首 四}{
三十未有二十余,白日长饥小甲蔬。\\
桥头长老相哀念,因遗戎韬一卷书。\\
}
\poetry{南园十三首 五}{
男儿何不带吴钩,收取关山五十州。\\
请君暂上凌烟阁,若个书生万户侯。\\
}
\poetry{南园十三首 六}{
寻章摘句老雕虫,晓月当帘挂玉弓。\\
不见年年辽海上,文章何处哭秋风。\\
}
\poetry{南园十三首 七}{
长卿牢落悲空舍,曼倩诙谐取自容。\\
见买若耶溪水剑,明朝归去事猿公。\\
}
\poetry{南园十三首 八}{
春水初生乳燕飞,黄蜂小尾扑花归。\\
窗含远色通书幌,鱼拥香钩近石矶。\\
}
\poetry{南园十三首 九}{
泉沙耎卧鸳鸯暖,曲岸回篙舴艋迟。\\
泻酒木栏椒叶盖,病容扶起种菱丝。\\
}
\poetry{南园十三首 十}{
边让今朝忆蔡邕,无心裁曲卧春风。\\
舍南有竹堪书字,老去溪头作钓翁。\\
}
\poetry{南园十三首 十一}{
长峦谷口倚嵇家,白昼千峰老翠华。\\
自履藤鞋收石蜜,手牵苔絮长莼花。\\
}
\poetry{南园十三首 十二}{
松溪黑水新龙卵,桂洞生硝旧马牙。\\
谁遣虞卿裁道帔,轻绡一匹染朝霞。\\
}
\poetry{南园十三首 十三}{
小树开朝迳,长茸湿夜烟。\\
柳花惊雪浦,麦雨涨溪田。\\
古刹疏钟度,遥岚破月悬。\\
沙头敲石火,烧竹照渔船。\\
}
\poetry{金铜仙人辞汉歌}{
茂陵刘郎秋风客,夜闻马嘶晓无迹。\\
画栏桂树悬秋香,三十六宫土花碧。\\
魏官牵车指千里,东关酸风射眸子。\\
空将汉月出宫门,忆君清泪如铅水。\\
衰兰送客咸阳道,天若有情天亦老。\\
携盘独出月荒凉,渭城已远波声小。\\
}
\poetry{古悠悠行}{
白景归西山,碧华上迢迢。\\
今古何处尽,千岁随风飘。\\
海沙变成石,鱼沫吹秦桥。\\
空光远流浪,铜柱从年消。\\
}
\poetry{黄头郎}{
黄头郎,捞拢去不归。\\
南浦芙蓉影,愁红独自垂。\\
水弄湘娥珮,竹啼山露月。\\
玉瑟调青门,石云湿黄葛。\\
沙上蘼芜花,秋风已先发。\\
好持扫罗荐,香出鸳鸯热。\\
}
\poetry{马诗二十三首 一}{
龙脊贴连钱,银蹄白踏烟。\\
无人织锦韂,谁为铸金鞭。\\
}
\poetry{马诗二十三首 二}{
腊月草根甜,天街雪似盐。\\
未知口硬软,先拟蒺藜衔。\\
}
\poetry{马诗二十三首 三}{
忽忆周天子,驱车上玉山。\\
鸣驺辞凤苑,赤骥最承恩。\\
}
\poetry{马诗二十三首 四}{
此马非凡马,房星本是星。\\
向前敲瘦骨,犹自带铜声。\\
}
\poetry{马诗二十三首 五}{
大漠山如雪,燕山月似钩。\\
何当金络脑,快走踏清秋。\\
}
\poetry{马诗二十三首 六}{
饥卧骨查牙,粗毛刺破花。\\
鬣焦朱色落,发断锯长麻。\\
}
\poetry{马诗二十三首 七}{
西母酒将阑,东王饭已干。\\
君王若燕去,谁为拽车辕。\\
}
\poetry{马诗二十三首 八}{
赤兔无人用,当须吕布骑。\\
吾闻果下马,羇策任蛮儿。\\
}
\poetry{马诗二十三首 九}{
飂叔去匆匆,如今不豢龙。\\
夜来霜压栈,骏骨折西风。\\
}
\poetry{马诗二十三首 十}{
摧榜渡乌江,神骓泣向风。\\
}
\poetry{马诗二十三首 十一}{
内马赐宫人,银鞯刺骐驎。\\
午时盐坂上,蹭蹬溘风尘。\\
}
\poetry{马诗二十三首 十二}{
批竹初攒耳,桃花未上身。\\
他时须搅阵,牵去借将军。\\
}
\poetry{马诗二十三首 十三}{
宝玦谁家子,长闻侠骨香。\\
堆金买骏骨,将送楚襄王。\\
}
\poetry{马诗二十三首 十四}{
香幞赭罗新,盘龙蹙镫鳞。\\
回看南陌上,谁道不逢春。\\
}
\poetry{马诗二十三首 十五}{
不从桓公猎,何能伏虎威。\\
一朝沟陇出,看取拂云飞。\\
}
\poetry{马诗二十三首 十六}{
唐剑斩隋公,拳毛属太宗。\\
莫嫌金甲重,且去捉{风旋}风。\\
}
\poetry{马诗二十三首 十七}{
白铁锉青禾,砧间落细莎。\\
世人怜小颈,金埒畏长牙。\\
}
\poetry{马诗二十三首 十八}{
伯乐向前看,旋毛在腹间。\\
祗今掊白草,何日蓦青山。\\
}
\poetry{马诗二十三首 十九}{
萧寺驮经马,元从竺国来。\\
空知有善相,不解走章台。\\
}
\poetry{马诗二十三首 二十}{
重围如燕尾,宝剑似鱼肠。\\
欲求千里脚,先采眼中光。\\
}
\poetry{马诗二十三首 二十一}{
暂系腾黄马,仙人上彩楼。\\
须鞭玉勒吏,何事谪高州。\\
}
\poetry{马诗二十三首 二十二}{
汗血到王家,随鸾撼玉珂。\\
少君骑海上,人见是青骡。\\
}
\poetry{马诗二十三首 二十三}{
武帝爱神仙,烧金得紫烟。\\
厩中皆肉马,不解上青天。\\
}
\poetry{申胡子觱篥歌}{
颜热感君酒,含嚼芦中声。\\
花娘𥮾绥妥,休睡芙蓉屏。\\
谁截太平管,列点排空星。\\
直贯开花风,天上驱云行。\\
今夕岁华落,令人惜平生。\\
心事如波涛,中坐时时惊。\\
朔客骑白马,剑弝悬兰缨。\\
俊健如生猱,肯拾蓬中萤。\\
}
\poetry{老夫采玉歌}{
采玉采玉须水碧,琢作步摇徒好色。\\
老夫饥寒龙为愁,蓝溪水汽无清白。\\
夜雨冈头食蓁子,杜鹃口血老夫泪。\\
蓝溪之水厌生人,身死千年恨溪水。\\
斜山柏风雨如啸,泉脚挂绳青袅袅。\\
村寒白屋念娇婴,古台石磴悬肠草。\\
}
\poetry{伤心行}{
咽咽学楚吟,病骨伤幽素。\\
秋姿白发生,木叶啼风雨。\\
灯青兰膏歇,落照飞蛾舞。\\
古壁生凝尘,羇魂梦中语。\\
}
\poetry{湖中曲}{
长眉越沙采兰若,桂叶水葓春漠漠。\\
横船醉眠白昼闲,渡口梅风歌扇薄。\\
燕钗玉股照青渠,越王娇郎小字书。\\
蜀纸封巾报云鬓,晚漏壶中水淋尽。\\
}
\poetry{黄家洞}{
雀步蹙沙声促促,四尺角弓青石镞。\\
黑幡三点铜鼓鸣,高作猿啼摇箭箙。\\
彩巾缠踍幅半斜,溪头簇队暎葛花。\\
山潭晚雾吟白鼍,竹蛇飞蠹射金沙。\\
闲驱竹马缓归家,官军自杀容州槎。\\
}
\poetry{屏风曲}{
蝶栖石竹银交关,水凝绿鸭瑠璃钱。\\
团回六曲抱膏兰,将鬟镜上掷金蝉。\\
沈香火暖茱萸烟,酒觥绾带新承懽。\\
月风吹露屏外寒,城上乌啼楚女眠。\\
}
\poetry{南山田中行}{
秋野明,秋风白,塘水漻漻虫啧啧。\\
云根苔藓山上石,冷红泣露娇啼色。\\
荒畦九月稻叉牙,蛰萤低飞陇迳斜。\\
石脉水流泉滴沙,鬼灯如漆点松花。\\
}
\poetry{贵主征行乐}{
奚骑黄铜连锁甲,罗旗香干金画叶。\\
中军留醉河阳城,娇嘶紫燕踏花行。\\
春营骑将如红玉,走马捎鞭上空绿。\\
女垣素月角咿咿,牙帐未开分锦衣。\\
}
\poetry{酒罢张大彻索赠诗}{
长鬣张郎三十八,天遣裁诗花作骨。\\
往还谁是龙头人,公主遣秉鱼须笏。\\
太行青草上白衫,匣中章奏密如蚕。\\
金门石阁知卿有,豸角鸡香早晚含。\\
陇西长吉摧颓客,酒阑感觉中区窄。\\
葛衣断碎赵城秋,吟诗一夜东方白。\\
}
\poetry{罗浮山父与葛篇}{
依依宜织江雨空,雨中六月兰台风。\\
博罗老仙时出洞,千岁石床啼鬼工。\\
蛇毒浓凝洞堂湿,江鱼不食衔沙立。\\
欲剪箱中一尺天,吴娥莫道吴刀涩。\\
}
\poetry{仁和里杂叙皇甫湜}{
大人乞马癯乃寒,宗人贷宅荒厥垣。\\
横庭鼠径空土涩,出篱大枣垂朱残。\\
安定美人截黄绶,脱落缨裾瞑朝酒。\\
还家白笔未上头,使我清声落人后。\\
枉辱称知犯君眼,排引才升强𫄠断。\\
洛风送马入长关,阖扇未开逢猰犬。\\
那知坚都相草草,客枕幽单看春老。\\
归来骨薄面无膏,疫气冲头鬓茎少。\\
欲雕小说干天官,宗孙不调为谁怜。\\
明朝下元复西道,崆峒叙别长如天。\\
}
\poetry{宫娃歌}{
蜡光高悬照纱空,花房夜𢭏红守宫。\\
象口吹香毾{登毛}暖,七星挂城闻漏板。\\
寒入罘罳殿影昏,彩鸾帘额著霜痕。\\
啼蛄吊月钩栏下,屈膝铜铺锁阿甄。\\
梦入家门上沙渚,天河落处长洲路。\\
愿君光明如太阳,放妾骑鱼撇波去。\\
}
\poetry{堂堂}{
堂堂复堂堂,红脱梅灰香。\\
十年粉蠹生画梁,饥虫不食推碎黄。\\
蕙花已老桃叶长,禁院悬帘隔御光。\\
华清源中礜石汤,裴回白凤随君王。\\
}
\poetry{勉爱行二首送小季之庐山 一}{
洛郊无俎豆,弊厩惭老马。\\
小雁过𬬻峰,影落楚水下。\\
长船倚云泊,石镜秋凉夜。\\
岂解有乡情,弄月聊呜哑。\\
}
\poetry{勉爱行二首送小季之庐山 二}{
别柳当马头,官槐如兔目。\\
欲将千里别,持我易斗粟。\\
南云北云空脉断,灵台经络悬春线。\\
青轩树转月满床,下国饥儿梦中见。\\
维尔之昆二十余,年来持镜颇有须。\\
辞家三载今如此,索米王门一事无。\\
荒沟古水光如刀,庭南拱柳生蛴螬。\\
江干幼客真可念,郊原晚吹悲号号。\\
}
\poetry{致酒行}{
零落栖迟一杯酒,主人奉觞客长寿。\\
主父西游困不归,家人折断门前柳。\\
吾闻马周昔作新丰客,天荒地老无人识。\\
空将笺上两行书,直犯龙颜请恩泽。\\
我有迷魂招不得,雄鸡一声天下白。\\
少年心事当拏云,谁念幽寒坐呜呃。\\
}
\poetry{长歌续短歌}{
长歌破衣襟,短歌断白发。\\
秦王不可见,旦夕成内热。\\
渴饮壶中酒,饥拔陇头粟。\\
凄凄四月阑,千里一时绿。\\
夜峰何离离,明月落石底。\\
裴回沿石寻,照出高峰外。\\
不得与之游,歌成鬓先改。\\
}
\poetry{公莫舞歌}{
方花古础排九楹,刺豹淋血盛银罂。\\
华筵鼓吹无桐竹,长刀直立割鸣筝。\\
横楣麤锦生红纬,日炙锦嫣王未醉。\\
腰下三看宝玦光,项庄掉箾栏前起。\\
材官小臣公莫舞,座上真人赤龙子。\\
芒砀云端抱天回,咸阳王气清如水。\\
铁枢铁楗重束关,大旗五丈撞双镮。\\
汉王今日颁秦印,绝膑刳肠臣不论。\\
}
\poetry{昌谷北园新笋四首 一}{
箨落长竿削玉开,君看母笋是龙材。\\
更容一夜抽千尺,别却池园数寸泥。\\
}
\poetry{昌谷北园新笋四首 二}{
斫取青光写楚辞,腻香春粉黑离离。\\
无情有恨何人见,露压烟啼千万枝。\\
}
\poetry{昌谷北园新笋四首 三}{
家泉石眼两三茎,晓看阴根紫脉生。\\
今年水曲春沙上,笛管新篁拔玉青。\\
}
\poetry{昌谷北园新笋四首 四}{
古竹老梢惹碧云,茂陵归卧叹清贫。\\
风吹千亩迎雨啸,鸟重一枝入酒尊。\\
}
\poetry{恼公}{
宋玉愁空断,娇饶粉自红。\\
歌声春草露,门掩杏花丛。\\
注口樱桃小,添眉桂叶浓。\\
晓奁妆秀靥,夜帐减香筒。\\
钿镜飞孤鹊,江图画水葓。\\
陂陀梳碧凤,腰袅带金虫。\\
杜若含清露,河蒲聚紫茸。\\
月分蛾黛破,花合靥朱融。\\
发重疑盘雾,腰轻乍倚风。\\
密书题豆蔲,隐语笑芙蓉。\\
莫锁茱萸匣,休开翡翠笼。\\
弄珠惊汉燕,烧蜜引胡蜂。\\
醉缬抛红网,单罗挂绿蒙。\\
数钱教姹女,买药问巴賨。\\
匀脸安斜雁,移灯想梦熊。\\
肠攒非束竹,胘急是张弓。\\
晚树迷新蝶,残蜺忆断虹。\\
古时填渤澥,今日凿崆峒。\\
绣沓褰长幔,罗裙结短封。\\
心摇如舞鹤,骨出似飞龙。\\
井槛淋清漆,门铺缀白铜。\\
隈花开兔径,向壁印狐踪。\\
玳瑁钉帘薄,琉璃叠扇烘。\\
象床缘素柏,瑶席卷香葱。\\
细管吟朝幌,芳醪落夜枫。\\
宜男生楚巷,栀子发金墉。\\
龟甲开屏涩,鹅毛渗墨浓。\\
黄庭留卫瓘,绿树养韩冯。\\
鸡唱星悬柳,鸦啼露滴桐。\\
黄娥初出座,宠妹始相从。\\
蜡泪垂兰烬,秋芜扫绮栊。\\
吹笙翻旧引,沽酒待新丰。\\
短珮愁填粟,长弦怨削菘。\\
曲池眠乳鸭,小阁睡娃僮。\\
褥缝𥮾双线,钩縚辫五总。\\
蜀烟飞重锦,峡雨溅轻容。\\
拂镜羞温峤,薰衣避贾充。\\
鱼生玉藕下,人在石莲中。\\
含水弯蛾翠,登楼选马鬃。\\
使君居曲陌,园令住临邛。\\
桂火流苏暖,金炉细炷通。\\
春迟王子态,莺啭谢娘慵。\\
玉漏三星曙,铜街五马逢。\\
犀株防胆怯,银液镇心忪。\\
跳脱看年命,琵琶道吉凶。\\
王时应七夕,夫位在三宫。\\
无力涂云母,多方带药翁。\\
符因青鸟送,囊用绛纱缝。\\
汉苑寻官柳,河桥阂禁钟。\\
月明中妇觉,应笑画堂空。\\
}
\poetry{感讽五首 一}{
合浦无明珠,龙洲无木奴。\\
足知造化力,不给使君须。\\
越妇未织作,吴蚕始蠕蠕。\\
县官骑马来,狞色虬紫须。\\
怀中一方板,板上数行书。\\
不因使君怒,焉得诣尔庐。\\
越妇拜县官,桑牙今尚小。\\
会待春日晏,丝车方掷掉。\\
越妇通言语,小姑具黄粱。\\
县官踏飡去,簿吏复登堂。\\
}
\poetry{感讽五首 二}{
奇俊无少年,日车何躃躃。\\
我待纡双绶,遗我星星发。\\
都门贾生墓,青蝇久断绝。\\
寒食摇扬天,愤景长肃杀。\\
皇汉十二帝,唯帝称睿哲。\\
一夕信竖儿,文明永沦歇。\\
}
\poetry{感讽五首 三}{
南山何其悲,鬼雨洒空草。\\
长安夜半秋,风前几人老。\\
低迷黄昏径,袅袅青栎道。\\
月午树无影,一山唯白晓。\\
漆炬迎新人,幽圹萤扰扰。\\
}
\poetry{感讽五首 四}{
星尽四方高,万物知天曙。\\
己生须己养,荷担出门去。\\
君平久不反,康伯循国路。\\
晓思何𫍢𫍢,阛阓千人语。\\
}
\poetry{感讽五首 五}{
石根秋水明,石畔秋草瘦。\\
侵衣野竹香,蛰蛰垂叶厚。\\
岑中月归来,蟾光挂空秀。\\
桂露对仙娥,星星下云逗。\\
凄凉栀子落,山璺泣清漏。\\
下有张仲蔚,披书案将朽。\\
}
\poetry{三月过行宫}{
渠水红繁拥御墙,风娇小叶学娥妆。\\
垂帘几度青春老,堪锁千年白日长。\\
}
\poetry{追和何谢铜雀妓}{
佳人一壶酒,秋容满千里。\\
石马卧新烟,忧来何所似。\\
歌声且潜弄,陵树风自起。\\
长裾压高台,泪眼看花机。\\
}
\poetry{送秦光禄北征}{
北虏胶堪折,秋沙乱晓鼙。\\
髯胡频犯塞,骄气似横霓。\\
灞水楼船渡,营门细柳开。\\
将军驰白马,豪彦骋雄材。\\
箭射欃枪落,旗悬日月低。\\
榆稀山易见,甲重马频嘶。\\
天远星光没,沙平草叶齐。\\
风吹云路火,雪污玉关泥。\\
屡断呼韩颈,曾然董卓脐。\\
太常犹旧宠,光禄是新隮。\\
宝玦麒麟起,银壶狒狖啼。\\
桃花连马发,彩絮扑鞍来。\\
呵臂悬金斗,当唇注玉罍。\\
清苏和碎蚁,紫腻卷浮杯。\\
虎鞹先蒙马,鱼肠且断犀。\\
{走参}{走覃}西旅狗,蹙额北方奚。\\
守帐然香暮,看鹰永夜栖。\\
黄龙就别镜,青冢念阳台。\\
周处长桥役,侯调短弄哀。\\
钱塘阶凤羽,正室擘鸾钗。\\
内子攀琪树,羌儿奏落梅。\\
今朝擎剑去,何日刺蛟回。\\
}
\poetry{酬荅二首 一}{
金鱼公子夹衫长,密装腰鞓割玉方。\\
行处春风随马尾,柳花偏打内家香。\\
}
\poetry{酬荅二首 二}{
雍州二月梅池春,御水䴔䴖暖白蘋。\\
试问酒旗歌板地,今朝谁是拗花人。\\
}
\poetry{画角东城}{
河转曙萧萧,鸦飞睥睨高。\\
帆长摽越甸,壁冷挂吴刀。\\
淡菜生寒日,鲕鱼潠白涛。\\
水花霑抹额,旗鼓夜迎潮。\\
}
\poetry{谢秀才有妾缟练改从于人秀才留之不得[后]生感忆座人制诗嘲谢贺复继四首 一}{
谁知泥忆云,望断梨花春。\\
荷丝制机练,竹叶剪花裙。\\
月明啼阿姊,灯暗会良人。\\
也识君夫婿,金鱼挂在身。\\
}
\poetry{谢秀才有妾缟练改从于人秀才留之不得[后]生感忆座人制诗嘲谢贺复继四首 二}{
铜镜立青鸾,燕脂拂紫绵。\\
腮花弄暗粉,眼尾泪侵寒。\\
碧玉破不复,瑶琴重拨弦。\\
今日非昔日,何人敢正看。\\
}
\poetry{谢秀才有妾缟练改从于人秀才留之不得[后]生感忆座人制诗嘲谢贺复继四首 三}{
洞房思不禁,蜂子作花心。\\
灰暖残香炷,发冷青虫簪。\\
夜遥灯焰短,睡熟小屏深。\\
好作鸳鸯梦,南城罢𢭏砧。\\
}
\poetry{谢秀才有妾缟练改从于人秀才留之不得[后]生感忆座人制诗嘲谢贺复继四首 四}{
寻常轻宋玉,今日嫁文鸯。\\
戟干横龙簴,刀环倚桂窗。\\
邀人裁半袖,端坐据胡床。\\
泪湿红轮重,栖乌上井梁。\\
}
\poetry{昌谷读书示巴童}{
虫响灯光薄,宵寒药气浓。\\
君怜垂翅客,辛苦尚相从。\\
}
\poetry{巴童荅}{
巨鼻宜山褐,庞眉入苦吟。\\
非君唱乐府,谁识怨秋深。\\
}
\poetry{代崔家送客}{
行盖柳烟下,马蹄白翩翩。\\
恐随行处尽,何忍重扬鞭。\\
}
\poetry{出城}{
雪下桂花稀,啼乌被弹归。\\
关水乘驴影,秦风帽带垂。\\
入乡试万里,无印自堪悲。\\
卿卿忍相问,镜中双泪姿。\\
}
\poetry{莫种树}{
园中莫种树,种树四时愁。\\
独睡南床月,今秋似去秋。\\
}
\poetry{将发}{
东床卷席罢,[濩]落将行去。\\
秋白遥遥空,日满门前路。\\
}
\poetry{追赋画江潭苑四首 一}{
吴苑晓苍苍,宫衣水溅黄。\\
小鬟红粉薄,骑马珮珠长。\\
路指台城迥,罗薰袴褶香。\\
行云霑翠辇,今日似襄王。\\
}
\poetry{追赋画江潭苑四首 二}{
宝袜菊衣单,蕉花密露寒。\\
水光兰泽叶,带重剪刀钱。\\
角暖盘弓易,靴长上马难。\\
泪痕霑寝帐,匀粉照金鞍。\\
}
\poetry{追赋画江潭苑四首 三}{
剪翅小鹰斜,[縚]根玉镟花。\\
秋垂妆钿粟,箭箙钉文牙。\\
●●啼深竹,䴔䴖老湿沙。\\
宫官烧蜡火,飞烬污铅华。\\
}
\poetry{追赋画江潭苑四首 四}{
十骑簇芙蓉,宫衣小队红。\\
练香熏宋鹊,寻箭踏卢龙。\\
旗湿金铃重,霜干玉镫空。\\
今朝画眉早,不待景阳钟。\\
}
\poetry{潞州张大宅病酒遇江使寄上十四兄}{
秋至昭关后,当知赵国寒。\\
系书随短羽,写恨破长笺。\\
病客眠清晓,疏桐坠绿鲜。\\
城鸦啼粉堞,军吹压芦烟。\\
岸帻褰沙幌,枯塘卧折莲。\\
木窗银迹画,石磴水痕钱。\\
旅酒侵愁肺,离歌绕懦弦。\\
诗封两条泪,露折一枝兰。\\
莎老沙鸡泣,松干瓦兽残。\\
觉骑燕地马,梦载楚溪船。\\
椒桂倾长席,鲈鲂斫玳筵。\\
岂能忘旧路,江岛滞佳年。\\
}
\poetry{难忘曲}{
夹道开洞门,弱杨低画戟。\\
帘影竹华起,箫声吹日色。\\
蜂语绕妆镜,拂蛾学春碧。\\
乱系丁香梢,满栏花向夕。\\
}
\poetry{贾公闾贵婿曲}{
朝衣不须长,分花对袍缝。\\
嘤嘤白马来,满脑黄金重。\\
今朝香气苦,珊瑚涩难枕。\\
且要弄风人,暖蒲沙上饮。\\
燕语踏帘钩,日虹屏中碧。\\
潘令在河阳,无人死芳色。\\
}
\poetry{夜饮朝眠曲}{
觞酣出座东方高,腰横半解星劳劳。\\
柳苑鸦啼公主醉,薄露压花蕙园气。\\
玉转湿丝牵晓水,熟粉生香琅玕紫。\\
夜饮朝眠断无事,楚罗之帏卧皇子。\\
}
\poetry{王濬墓下作}{
人间无阿童,犹唱水中龙。\\
白草侵烟死,秋梨遶地红。\\
古书平黑石,袖剑断青铜。\\
耕势鱼鳞起,坟科马鬣封。\\
菊花垂湿露,棘径卧干蓬。\\
松柏愁香涩,南原几夜风。\\
}
\poetry{客游}{
悲满千里心,日暖南山石。\\
不谒承明庐,老作平原客。\\
四时别家庙,三年去乡国。\\
旅歌屡弹铗,归问时裂帛。\\
}
\poetry{崇义里滞雨}{
落莫谁家子,来感长安秋。\\
壮年抱羁恨,梦泣生白头。\\
瘦马秣败草,雨沫飘寒沟。\\
南宫古帘暗,湿景传签筹。\\
家山远千里,云脚天东头。\\
忧眠枕剑匣,客帐梦封侯。\\
}
\poetry{冯小怜}{
湾头见小怜,请上琵琶弦。\\
破得春风恨,今朝直几钱。\\
裙垂竹叶带,鬓湿杏花烟。\\
玉冷红丝重,齐宫妾驾鞭。\\
}
\poetry{赠陈商}{
长安有男儿,二十心已朽。\\
楞伽堆案前,楚辞系肘后。\\
人生有穷拙,日暮聊饮酒。\\
秪今道已塞,何必须白首。\\
凄凄陈述圣,披褐鉏俎豆。\\
学为尧舜文,时人责衰偶。\\
柴门车辙冻,日下榆影瘦。\\
黄昏访我来,苦节青阳皱。\\
太华五千仞,劈地抽森秀。\\
旁古无寸寻,一上戛牛斗。\\
公卿纵不怜,宁能锁吾口。\\
李生师太华,大坐看白昼。\\
逢霜作朴{木敕},得气为春柳。\\
礼节乃相去,憔悴如刍狗。\\
风雪直斋坛,墨组贯铜绶。\\
臣妾气态间,唯欲承箕帚。\\
天眼何时开,古剑庸一吼。\\
}
\poetry{钓鱼诗}{
秋水钓红渠,仙人待素书。\\
菱丝萦独茧,蒲米蛰双鱼。\\
斜竹垂清沼,长纶贯碧虚。\\
饵悬春[蜥蜴],钩坠小蟾蜍。\\
詹子情无限,龙阳恨有余。\\
为看烟浦上,楚女泪霑裾。\\
}
\poetry{奉和二兄罢使遣马归延州}{
空留三尺剑,不用一丸泥。\\
马向沙场去,人归故国来。\\
笛愁翻陇水,酒喜沥春灰。\\
锦带休惊雁,罗衣尚鬬鸡。\\
还吴已渺渺,入郢莫凄凄。\\
自是桃李树,何畏不成蹊。\\
}
\poetry{荅赠}{
本是张公子,曾名萼绿华。\\
沈香熏小像,杨柳伴啼鸦。\\
露重金泥冷,杯阑玉树斜。\\
琴堂沽酒客,新买后园花。\\
}
\poetry{题赵生壁}{
大妇然竹根,中妇舂玉屑。\\
冬暖拾松枝,日烟坐蒙灭。\\
木藓青桐老,石井水声发。\\
曝背卧东亭,桃花满肌骨。\\
}
\poetry{感春}{
日暖自萧条,花悲北郭骚。\\
榆穿莱子眼,柳断舞儿腰。\\
上幕迎神燕,飞丝送百劳。\\
胡琴今日恨,急语向檀槽。\\
}
\poetry{仙人}{
弹琴石壁上,翻翻一仙人。\\
手持白鸾尾,夜扫南山云。\\
鹿饮寒涧下,鱼归清海滨。\\
当时汉武帝,书报桃花春。\\
}
\poetry{河阳歌}{
染罗衣,秋蓝难着色。\\
不是无心人,为作台邛客。\\
花烧中潬城,颜郎身已老。\\
惜许两少年,抽心似春草。\\
今日见银牌,今夜鸣玉䜩。\\
牛头高一尺,隔坐应相见。\\
月从东方来,酒从东方转。\\
觥船饫口红,[蜜]炬千枝烂。\\
}
\poetry{花游曲}{
春柳南陌态,冷花寒露姿。\\
今朝醉城外,拂镜浓扫眉。\\
烟湿愁车重,红油覆画衣。\\
舞裙香不暖,酒色上来迟。\\
}
\poetry{春昼}{
朱城报春更漏转,光风催兰吹小殿。\\
草细堪梳,柳长如线。\\
卷衣秦帝,扫粉赵燕。\\
日含画幕,蜂上罗荐。\\
平阳花坞,河阳花县。\\
越妇搘机,吴蚕作茧。\\
菱汀系带,荷塘倚扇。\\
江南有情,塞北无恨。\\
}
\poetry{安乐宫}{
深井桐乌起,尚复牵情水。\\
未盥邵陵瓜,缾中弄长翠。\\
新成安乐宫,宫如凤皇翅。\\
歌回蜡板鸣,左悺提壶使。\\
绿蘩悲水曲,茱萸别秋子。\\
}
\poetry{胡蝶飞}{
杨花扑帐春云热,龟甲屏风醉眼缬。\\
东家胡蝶西家飞,白骑少年今日归。\\
}
\poetry{梁公子}{
风彩出萧家,本是菖蒲花。\\
南塘连子熟,洗马走江沙。\\
御笺银沫冷,长簟凤窠斜。\\
种柳营中暗,题书赐馆娃。\\
}
\poetry{牡丹种曲}{
莲枝未长秦蘅老,走马驮金𣃁春草。\\
水灌香泥却月盆,一夜绿房迎白晓。\\
美人醉语园中烟,晚华已散蝶又阑。\\
梁王老去罗衣在,拂袖风吹蜀国弦。\\
归霞帔拖蜀帐昏,嫣红落粉罢承恩。\\
檀郎谢女眠何处,楼台月明燕夜语。\\
}
\poetry{后园凿井歌}{
井上辘轳床上转,水声繁,弦声浅。\\
情若何,荀奉倩。\\
城头日,长向城头住。\\
一日作千年,不须流下去。\\
}
\poetry{开愁歌}{
秋风吹地百草干,华容碧影生晚寒。\\
我当二十不得意,一心愁谢如枯兰。\\
衣如飞鹑马如狗,临岐击剑生铜吼。\\
旗亭下马解秋衣,请贳宜阳一壶酒。\\
壶中唤天云不开,白昼万里闲凄迷。\\
主人劝我养心骨,莫受俗物相填{豕灰}。\\
}
\poetry{秦宫诗}{
越罗衫袂迎春风,玉刻麒麟腰带红。\\
楼头曲宴仙人语,帐底吹笙香雾浓。\\
人间酒暖春茫茫,花枝入帘白日长。\\
飞窗复道传筹饮,十夜铜盘腻烛黄。\\
秃衿小袖调鹦鹉,紫绣麻{韦叚}踏哮虎。\\
斫桂烧金待晓筵,白鹿青苏夜半煮。\\
桐英永巷骑新马,内屋深屏生色画。\\
开门烂用水衡钱,卷起黄河向身泻。\\
皇天厄运犹曾裂,秦宫一生花底活。\\
鸾篦夺得不还人,醉睡氍毺满堂月。\\
}
\poetry{古邺城童子谣效王粲刺曹操}{
邺城中,暮尘起。\\
将黑丸,斫文吏。\\
棘为鞭,虎为马。\\
团团走,邺城下。\\
切玉剑,射日弓。\\
献何人,奉相公。\\
扶毂来,关右儿。\\
香扫涂,相公归。\\
}
\poetry{杨生青花紫石砚歌}{
端州石工巧如神,踏天磨刀割紫云。\\
佣刓抱水含满唇,暗洒苌弘冷血痕。\\
纱帷昼暖墨花春,轻沤漂沫松麝薰。\\
干腻薄重立脚匀,数寸光秋无日昏。\\
圆毫促点声静新,孔砚宽顽何足云。\\
}
\poetry{房中思}{
新桂如蛾眉,秋风吹小绿。\\
行轮出门去,玉銮声断续。\\
月轩下风露,晓庭自幽涩。\\
谁能事贞素,卧听莎鸡泣。\\
}
\poetry{石城晓}{
月落大堤上,女垣栖乌起。\\
细露湿团红,寒香解夜醉。\\
女牛渡天河,柳烟满城曲。\\
上客留断缨,残蛾鬬双绿。\\
春帐依微蝉翼罗,横茵突金隐体花。\\
帐前轻絮鹤毛起,欲说春心无所似。\\
}
\poetry{苦昼短}{
飞光飞光,劝尔一杯酒。\\
吾不识青天高,黄地厚。\\
唯见月寒日暖,来煎人寿。\\
食熊则肥,食蛙则瘦。\\
神君何在,太一安有。\\
天东有若木,下置衔烛龙。\\
吾将斩龙足,嚼龙肉。\\
使之朝不得回,夜不得伏。\\
自然老者不死,少者不哭。\\
何为服黄金,吞白玉。\\
谁似任公子,云中骑碧驴。\\
刘彻茂陵多滞骨,嬴政梓棺费鲍鱼。\\
}
\poetry{章和二年中}{
云萧索,田风拂拂,麦芒如篲黍如粟。\\
关中父老百领襦,关东吏人乏诟租。\\
健犊春耕土膏黑,菖蒲丛丛沿水脉。\\
殷勤为我下田租,百钱携偿丝桐客。\\
游春漫光坞花白,野林散香神降席。\\
拜神得寿献天子,七星贯断姮娥死。\\
}
\poetry{春归昌谷}{
束发方读书,谋身苦不早。\\
终军未乘传,颜子鬓先老。\\
天网信崇大,矫士常慅慅。\\
逸目骈甘华,羁心如荼蓼。\\
旱云二三月,岑岫相颠倒。\\
谁揭頳玉盘,东方发红照。\\
春热张鹤盖,兔目官槐小。\\
思焦面如病,尝胆肠似绞。\\
京国心烂漫,夜梦归家少。\\
发轫东门外,天地皆浩浩。\\
青树骊山头,花风满秦道。\\
宫台光错落,装尽偏峰峤。\\
细绿及团红,当路杂啼笑。\\
香风下高广,鞍马正华耀。\\
独乘鸡栖车,自觉少风调。\\
心曲语形影,祗身焉足乐。\\
岂能脱负檐,刻鹤曾无兆。\\
幽幽太华侧,老柏如建纛。\\
龙皮相排戛,翠羽更荡掉。\\
驱趋委憔悴,眺览强容貌。\\
花蔓阂行辀,縠烟暝深徼。\\
少健无所就,入门愧家老。\\
听讲依大树,观书临曲沼。\\
知非出柙虎,甘作藏雾豹。\\
韩鸟处[矰]缴,湘鯈在笼罩。\\
狭行无廓落,壮士徒轻躁。\\
}
\poetry{昌谷诗}{
昌谷五月稻,细青满平水。\\
遥峦相压叠,颓绿愁堕地。\\
光洁无秋思,凉旷吹浮媚。\\
竹香满凄寂,粉节涂生翠。\\
草发垂恨鬓,光露泣幽泪。\\
层围烂洞曲,芳径老红醉。\\
攒虫锼古柳,蝉子鸣高邃。\\
大带委黄葛,紫蒲交狭涘。\\
石钱差复藉,厚叶皆蟠腻。\\
汰沙好平白,立马印青字。\\
晚鳞自遨游,瘦鹄暝单跱。\\
嘹嘹湿蛄声,咽源惊溅起。\\
纡缓玉真路,神娥蕙花里。\\
苔絮萦涧砾,山实垂頳紫。\\
小柏俨重扇,肥松突丹髓。\\
鸣流走响韵,垅秋拖光穟。\\
莺唱闵女歌,瀑悬楚练帔。\\
风露满笑眼,骈岩杂舒坠。\\
乱条迸石岭,细颈喧岛毖。\\
日脚埽昏翳,新云启华閟。\\
谧谧厌夏光,商风道清气。\\
高眠服玉容,烧桂祀天几。\\
雾衣夜披拂,眠坛梦真粹。\\
待驾栖鸾老,故宫椒壁圮。\\
鸿珑数铃响,羁臣发凉思。\\
阴藤束朱键,龙帐著魈魅。\\
碧锦帖花柽,香衾事残贵。\\
歌尘蠹木在,舞彩长云似。\\
珍壤割绣段,里俗祖风义。\\
邻凶不相杵,疫病无邪祀。\\
鲐皮识仁惠,丱角知腼耻。\\
县省司刑官,户乏诟租吏。\\
竹薮添堕简,石矶引钩饵。\\
溪湾转水带,芭蕉倾蜀纸。\\
岑光晃縠襟,孤景拂繁事。\\
泉尊陶宰酒,月眉谢郎妓。\\
丁丁幽钟远,矫矫单飞至。\\
霞𪩘殷嵯峨,危溜听争次。\\
淡蛾流平碧,薄月眇阴悴。\\
凉光入涧岸,廓尽山中意。\\
渔童下宵网,霜禽竦烟翅。\\
潭镜滑蛟涎,浮珠𪡋鱼戏。\\
风桐瑶匣瑟,萤星锦城使。\\
柳缀长缥带,篁掉短笛吹。\\
石根缘绿藓,芦笋抽丹渍。\\
漂旋弄天影,古桧拏云臂。\\
愁月薇帐红,罥云香蔓刺。\\
芒麦平百井,闲乘列千肆。\\
刺促成纪人,好学鸱夷子。\\
}
\poetry{铜驼悲}{
落魄三月罢,寻花去东家。\\
谁作送春曲,洛岸悲铜驼。\\
桥南多马客,北山饶古人。\\
客饮杯中酒,驼悲千万春。\\
生世莫徒劳,风吹盘上烛。\\
厌见桃株笑,铜驼夜来哭。\\
}
\poetry{自昌谷到洛后门}{
九月大野白,苍芩竦秋门。\\
寒凉十月末,露霰蒙晓昏。\\
澹色结昼天,心事填空云。\\
道上千里风,野竹蛇涎痕。\\
石涧涷波声,鸡叫清寒晨。\\
强行到东舍,解马投旧邻。\\
东家名廖者,乡曲传姓辛。\\
杖头非饮酒,吾请造其人。\\
始欲南去楚,又将西适秦。\\
襄王与武帝,各自留青春。\\
闻道兰台上,宋玉无归魂。\\
缃缥两行字,蛰虫蠹秋芸。\\
为探秦台意,岂命余负薪。\\
}
\poetry{七月一日晓入太行山}{
一夕遶山秋,香露溘蒙菉。\\
新桥倚云阪,候虫嘶露朴。\\
洛南今已远,越衾谁为熟。\\
石气何凄凄,老莎如短镞。\\
}
\poetry{秋凉诗寄正字十二兄}{
闭门感秋风,幽姿任契阔。\\
大野生素空,天地旷肃杀。\\
露光泣残蕙,虫响连夜发。\\
房寒寸辉薄,迎风绛纱折。\\
披书古芸馥,恨唱华容歇。\\
百日不相知,花光变凉节。\\
弟兄谁念虑,笺翰既通达。\\
青袍度白马,草简奏东阙。\\
梦中相聚笑,觉见半床月。\\
长思剧寻环,乱忧抵覃葛。\\
}
\poetry{艾如张}{
锦襜褕,绣裆襦。\\
强饮啄,哺尔雏。\\
陇东卧穟满风雨,莫信笼媒陇西去。\\
齐人织网如素空,张在野田平碧中。\\
网丝漠漠无形影,误尔触之伤首红。\\
艾叶绿花谁剪刻,中藏祸机不可测。\\
}
\poetry{上云乐}{
飞香走红满天春,花龙盘盘上紫云。\\
三千宫女列金屋,五十弦瑟海上闻。\\
天江碎碎银沙路,嬴女机中断烟素。\\
断烟素,缝衣缕,八月一日君前舞。\\
}
\poetry{摩多楼子}{
玉塞去金人,二万四千里。\\
风吹沙作云,一时渡辽水。\\
天白水如练,甲丝双串断。\\
行行莫苦辛,城月犹残半。\\
晓气朔烟上,趢趗胡马蹄。\\
行人临水别,隔陇长东西。\\
}
\poetry{猛虎行}{
长戈莫舂,长弩莫抨。\\
乳孙哺子,教得生狞。\\
举头为城,掉尾为旌。\\
东海黄公,愁见夜行。\\
道逢驺虞,牛哀不平。\\
何用尺刀,壁上雷鸣。\\
泰山之下,妇人哭声。\\
官家有程,吏不敢听。\\
}
\poetry{日出行}{
白日下昆仑,发光如舒丝。\\
徒照葵藿心,不照游子悲。\\
折折黄河曲,日从中央转。\\
旸谷耳曾闻,若木眼不见。\\
奈尔铄石,胡为销人,羿弯弓属矢那不中。\\
足令久不得奔,讵教晨光夕昏。\\
}
\poetry{苦篁调啸引}{
请说轩辕在时事,伶伦采竹二十四。\\
伶伦采之自昆丘,轩辕诏遣中分作十二。\\
伶伦以之正音律,轩辕以之调元气。\\
当时黄帝上天时,二十三管咸相随。\\
唯留一管人间吹,无德不能得此管,此管沈埋虞舜祠。\\
}
\poetry{拂舞歌辞}{
吴娥声绝天,空云闲裴回。\\
门外满车马,亦须生绿苔。\\
尊有乌程酒,劝君千万寿。\\
全胜汉武锦楼上,晓望晴寒饮花露。\\
东方日不破,天光无老时。\\
丹成作蛇乘白雾,千年重化玉井土。\\
从蛇作土一千载,吴堤绿草年年在。\\
背有八卦称神仙,邪鳞顽甲滑腥涎。\\
}
\poetry{夜坐吟}{
踏踏马蹄谁见过,眼看北斗直天河。\\
西风罗幕生翠波,铅华笑妾颦青蛾。\\
为君起唱长相思,帘外严霜皆倒飞。\\
明星烂烂东方陲,红霞梢出东南涯,陆郎去矣乘班骓。\\
}
\poetry{箜篌引}{
公乎公乎,提壶将焉如。\\
屈平沈湘不足慕,徐衍入海诚为愚。\\
公乎公乎,床有菅席盘有鱼。\\
北里有贤兄,东邻有小姑。\\
陇亩油油黍与葫,瓦甒浊醪蚁浮浮。\\
黍可食,醪可饮,公乎公乎其奈居。\\
被发奔流竟何如,贤兄小姑哭呜呜。\\
}
\poetry{巫山高}{
碧丛丛,高插天,大江翻澜神曳烟。\\
楚魂寻梦风飔然,晓风飞雨生苔钱。\\
瑶姬一去一千年,丁香筇竹啼老猿。\\
古祠近月蟾桂寒,椒花坠红湿云间。\\
}
\poetry{平城下}{
饥寒平城下,夜夜守明月。\\
别剑无玉花,海风断鬓发。\\
塞长连白空,遥见汉旗红。\\
青帐吹短笛,烟雾湿昼龙。\\
日晚在城上,依稀望城下。\\
风吹枯蓬起,城中嘶瘦马。\\
借问筑城吏,去关几千里。\\
惟愁裹尸归,不惜倒戈死。\\
}
\poetry{江南弄}{
江中绿雾起凉波,天上叠𪩘红嵯峨。\\
水风浦云生老竹,渚暝蒲帆如一幅。\\
鲈鱼千头酒百斛,酒中倒卧南山绿。\\
吴歈越吟未终曲,江上团团帖寒玉。\\
}
\poetry{荣华乐}{
鸢肩公子二十余,齿编贝,唇激朱。\\
气如虹霓,饮如建瓴。\\
走马夜归叫严更,径穿复道游椒房。\\
龙裘金玦杂花光,玉堂调笑金楼子。\\
台下戏学邯郸倡,口吟舌话称女郎。\\
锦祛绣面汉帝旁,得明珠十斛。\\
白璧一双,新诏垂金曳紫光煌煌。\\
马如飞,人如水。\\
九卿六官皆望履,将回日月先反掌。\\
欲作江河唯画地,峨峨虎冠上切云。\\
竦剑晨趋凌紫氛,绣段千寻贻皂隶。\\
黄金百镒贶家臣,十二门前张大宅。\\
晴春烟起连天碧,金铺缀日杂红光。\\
铜龙啮环似争力,瑶姬凝醉卧芳席。\\
海素笼窗空下隔,丹穴取凤充行庖。\\
貜貜如拳那足食,金蟾呀呀兰烛香。\\
军装武妓声琅珰,谁知花雨夜来过。\\
但见池台春草长,嘈嘈弦吹匝天开。\\
洪厓箫声遶天来,天长一矢贯双虎。\\
云弝绝骋聒旱雷,乱袖交竿管儿舞。\\
吴音绿鸟学言语,能教刻石平紫金。\\
解送刻毛寄新兔,三皇皇后七贵人。\\
五十校尉二将军,当时飞去逐彩云,化作今日京华春。\\
}
\poetry{相劝酒}{
羲和骋六辔,昼夕不曾闲。\\
弹乌崦嵫竹,抶马蟠桃鞭。\\
蓐收既断翠柳,青帝又造红兰。\\
尧舜至今万万岁,数子将为倾盖间。\\
青钱白璧买无端,丈夫快意方为欢。\\
臛蠵臛熊何足云,会须锺饮北海,箕踞南山。\\
歌淫淫,管愔愔,横波好送雕题金。\\
人生得意且如此,何用强知元化心。\\
相劝酒,终无辍,伏愿陛下鸿名终不歇。\\
子孙绵如石上葛,来长安,车骈骈。\\
中有梁冀旧宅,石崇故园。\\
}
\poetry{瑶华乐}{
穆天子,走龙媒,八辔冬珑逐天回。\\
五精扫地凝云开,高门左右日月环。\\
四方错镂棱层殷,舞霞垂尾长盘珊。\\
江澄海净神母颜,施红点翠照虞泉。\\
曳云拖玉下昆山,列斾如松,张盖如轮。\\
金风殿秋,清明发春。\\
八銮十乘,矗如云屯。\\
琼锺瑶席甘露文,玄霜绛雪何足云,薰梅染柳将赠君。\\
铅华之水洗君骨,与君相对作真质。\\
}
\poetry{北中寒}{
一方黑照三方紫,黄河冰合鱼龙死。\\
三尺木皮断文理,百石强车上河水。\\
霜花草上大如钱,挥刀不入迷蒙天。\\
争瀯海水飞凌喧,山瀑无声玉虹悬。\\
}
\poetry{梁台古愁}{
梁王台沼空中立,天河之水夜飞入。\\
台前鬬玉作蛟龙,绿粉扫天愁露湿。\\
撞钟饮酒行射天,金虎蹙裘喷血斑。\\
朝朝暮暮愁海翻,长绳系日乐当年。\\
芙蓉凝红得秋色,兰脸别春啼脉脉。\\
芦洲客雁报春来,寥落野篁秋漫白。\\
}
\poetry{公无出门}{
天迷迷,地密密。\\
熊虺食人魂,雪霜断人骨。\\
嗾犬狺狺相索索,舐掌偏宜佩兰客。\\
帝遣乘轩灾自息,玉星点剑黄金轭。\\
我虽跨马不得还,历阳湖波大如山。\\
毒虬相视振金环,狻猊猰{犭俞}吐嚵涎。\\
鲍焦一世披草眠,颜回廿九鬓毛斑。\\
颜回非血衰,鲍焦不违天。\\
天畏遭衔啮,所以致之然。\\
分明犹惧公不信,公看呵壁书问天。\\
}
\poetry{神弦别曲}{
巫山小女隔云别,春风松花山上发。\\
绿盖独穿香径归,白马花竿前孑孑。\\
蜀江风澹水如罗,堕兰谁泛相经过。\\
南山桂树为君死,云衫浅污红脂花。\\
}
\poetry{绿水词}{
今宵好风月,阿侯在何处。\\
为有倾人色,翻成足愁苦。\\
东湖采莲叶,南湖拔蒲根。\\
未持寄小姑,且持感愁魂。\\
}
\poetry{沙路曲}{
柳脸半眠丞相树,珮马钉铃踏沙路。\\
断烬遗香袅翠烟,烛骑啼乌上天去。\\
帝家玉龙开九关,帝前动笏移南山。\\
独垂重印押千官,金窠篆字红屈盘。\\
沙路归来闻好语,旱火不光天下雨。\\
}
\poetry{上之回}{
上之回,大旗喜。\\
悬红云,挞凤尾。\\
剑匣破,舞蛟龙。\\
蚩尤死,鼓逢逢。\\
天高庆雷齐坠地,地无惊烟海千里。\\
}
\poetry{高轩过}{
华裾织翠青如葱,金环压辔摇玲珑。\\
马蹄隐耳声隆隆,入门下马气如虹。\\
云是东京才子,文章钜公。\\
二十八宿罗心胸,九精照耀贯当中。\\
殿前作赋声摩空,笔补造化天无功。\\
庞眉书客感秋蓬,谁知死草生华风。\\
我今垂翅附冥鸿,他日不羞蛇作龙。\\
}
\poetry{贝宫夫人}{
丁丁海女弄金环,雀钗翘揭双翅关。\\
六宫不语一生闲,高悬银榜照青山。\\
长眉凝绿几千年,清凉堪老镜中鸾。\\
秋肌稍觉玉衣寒,空光帖妥水如天。\\
}
\poetry{兰香神女庙}{
古春年年在,闲绿摇暖云。\\
松香飞晚华,柳渚含日昏。\\
沙砌落红满,石泉生水芹。\\
幽篁画新粉,蛾绿横晓门。\\
弱蕙不胜露,山秀愁空春。\\
舞珮剪鸾翼,帐带涂轻银。\\
兰桂吹浓香,菱藕长莘莘。\\
看雨逢瑶姬,乘船值江君。\\
吹箫饮酒醉,结绶金丝裙。\\
走天呵白鹿,游水鞭锦鳞。\\
密发虚鬟飞,腻颊凝花匀。\\
团鬓分蛛巢,秾眉笼小唇。\\
弄蝶和轻妍,风光怯腰身。\\
深帏金鸭冷,奁镜幽凤尘。\\
踏雾乘风归,撼玉山上闻。\\
}
\poetry{送韦仁实兄弟入关}{
送客饮别酒,千觞无赭颜。\\
何物最伤心,马首鸣金镮。\\
野色浩无主,秋明空旷间。\\
坐来壮胆破,断目不能看。\\
行槐引西道,青梢长攒攒。\\
韦郎好兄弟,叠玉生文翰。\\
我在山上舍,一亩蒿硗田。\\
夜雨叫租吏,春声暗交关。\\
谁解念劳劳,苍突唯南山。\\
}
\poetry{洛阳城外别皇甫湜}{
洛阳吹别风,龙门起断烟。\\
冬树束生涩,晚紫凝华天。\\
单身野霜上,疲马飞蓬间。\\
凭轩一双泪,奉坠绿衣前。\\
}
\poetry{溪晚凉}{
白狐向月号山风,秋寒扫云留碧空。\\
玉烟青湿白如幢,银湾晓转流天东。\\
溪汀眠鹭梦征鸿,轻涟不语细游溶。\\
层岫回岑复叠龙,苦篁对客吟歌筒。\\
}
\poetry{官不来题皇甫湜先辈厅}{
官不来,官庭秋,老桐错干青龙愁。\\
书司曹佐走如牛,叠声问佐官来不。\\
官不来,门幽幽。\\
}
\poetry{长平箭头歌}{
漆灰骨末丹水沙,凄凄古血生铜花。\\
白翎金簳雨中尽,直余三脊残狼牙。\\
我寻平原乘两马,驿东石田蒿坞下。\\
风长日短星萧萧,黑旗云湿悬空夜。\\
左魂右魄啼肌瘦,酪瓶倒尽将羊炙。\\
虫栖雁病芦笋红,回风送客吹阴火。\\
访古汍澜收断镞,折锋赤璺曾刲肉。\\
南陌东城马上儿,劝我将金换簝竹。\\
}
\poetry{江楼曲}{
楼前流水江陵道,鲤鱼风起芙蓉老。\\
晓钗催鬓语南风,抽帆归来一日功。\\
鼍吟浦口飞梅雨,竿头酒旗换青苎。\\
萧骚浪白云差池,黄粉油衫寄郎主。\\
新槽酒声苦无力,南湖一顷菱花白。\\
眼前便有千里愁,小玉开屏见山色。\\
}
\poetry{塞下曲}{
胡角引北风,蓟门白于水。\\
天含青海道,城头月千里。\\
露下旗蒙蒙,寒金鸣夜刻。\\
蕃甲鏁蛇鳞,马嘶青冢白。\\
秋静见旄头,沙远席羁愁。\\
帐北天应尽,河声出塞流。\\
}
\poetry{染丝上春机}{
玉罂泣水桐花井,蒨丝沈水如云影。\\
美人懒态燕脂愁,春梭抛掷鸣高楼。\\
彩线结茸背复叠,白袷玉郎寄桃叶。\\
为君挑鸾作腰绶,愿君处处宜春酒。\\
}
\poetry{五粒小松歌}{
蛇子蛇孙鳞蜿蜿,新香几粒洪厓饭。\\
绿波浸叶满浓光,细束龙髯铰刀剪。\\
主人壁上铺州图,主人堂前多俗儒。\\
月明白露秋泪滴,石笋溪云肯寄书。\\
}
\poetry{塘上行}{
藕花凉露湿,花缺藕根涩。\\
飞下雌鸳鸯,塘水声溘溘。\\
}
\poetry{吕将军歌}{
吕将军,骑赤兔。\\
独携大胆出秦门,金粟堆边哭陵树。\\
北方逆气污青天,剑龙夜叫将军闲。\\
将军振袖挥剑锷,玉阙朱城有门阁。\\
榼榼银龟摇白马,傅粉女郎火旗下。\\
恒山铁骑请金枪,遥闻箙中花箭香。\\
西郊寒蓬叶如刺,皇天新栽养神骥。\\
厩中高桁排蹇蹄,饱食青刍饮白水。\\
圆苍低迷盖张地,九州人事皆如此。\\
赤山秀铤御时英,绿眼将军会天意。\\
}
\poetry{休洗红}{
休洗红,洗多红色浅。\\
卿卿骋少年,昨日殷桥见。\\
封侯早归来,莫作弦上箭。\\
}
\poetry{神弦曲}{
西山日没东山昏,旋风吹马马踏云。\\
画弦素管声浅繁,花裙綷{纟蔡}步秋尘。\\
桂叶刷风桂坠子,青狸哭血寒狐死。\\
古壁彩虬金帖尾,雨工骑入秋潭水。\\
百年老鸮成木魅,笑声碧火巢中起。\\
}
\poetry{野歌}{
鸦翎羽箭山桑弓,仰天射落衔芦鸿。\\
麻衣黑肥冲北风,带酒日晚歌田中。\\
男儿屈穷心不穷,枯荣不等嗔天公。\\
寒风又变为春柳,条条看即烟蒙蒙。\\
}
\poetry{神弦}{
女巫浇酒云满空,玉炉炭火香冬冬。\\
海神山鬼来座中,纸钱窸窣鸣{风旋}风。\\
相思木帖金舞鸾,攒蛾一喋重一弹。\\
呼星召鬼歆杯盘,山魅食时人森寒。\\
终南日色低平湾,神兮长在有无间。\\
神嗔神喜师更颜,送神万骑还青山。\\
}
\poetry{将进酒}{
琉璃锺,琥珀浓,小槽酒滴真珠红。\\
烹龙炮凤玉脂泣,罗屏绣幕围香风。\\
吹龙笛,击鼍鼓,皓齿歌,细腰舞。\\
况是青春日将暮,桃花乱落如红雨。\\
劝君终日酩酊醉,酒不到刘伶坟上土。\\
}
\poetry{美人梳头歌}{
西施晓梦绡帐寒,香鬟堕髻半沈檀。\\
辘轳咿哑转鸣玉,惊起芙蓉睡新足。\\
双鸾开镜秋水光,解鬟临镜立象床。\\
一编香丝云撒地,玉钗落处无声腻。\\
纤手却盘老鸦色,翠滑宝钗簪不得。\\
春风烂熳恼娇慵,十八鬟多无气力。\\
妆成{髼夆=妥}鬌敧不斜,云裾数步踏雁沙。\\
背人不语向何处,下堦自折樱桃花。\\
}
\poetry{月漉漉篇}{
月漉漉,波烟玉。\\
莎青桂花繁,芙蓉别江木。\\
粉态袷罗寒,雁羽铺烟湿。\\
谁能看石帆,乘船镜中入。\\
秋白鲜红死,水香莲子齐。\\
挽菱隔歌袖,绿刺罥银泥。\\
}
\poetry{京城}{
驱马出门意,牢落长安心。\\
两事谁向道,自作秋风吟。\\
}
\poetry{官街鼓}{
晓声隆隆催转日,暮声隆隆呼月出。\\
汉城黄柳暎新帘,柏陵飞燕埋香骨。\\
磓发千年日长白,孝武秦王听不得。\\
从君翠发芦花色,独共南山守中国。\\
几回天上葬神仙,漏声相将无断缘。\\
}
\poetry{许公子郑姬歌}{
许史世家外亲贵,宫锦千端买沈醉。\\
铜驼酒熟烘明胶,古堤大柳烟中翠。\\
桂开客花名郑袖,入洛闻香鼎门口。\\
先将芍药献妆台,后解黄金大如斗。\\
莫愁帘中许合欢,清弦五十为君弹。\\
弹声咽春弄君骨,骨兴牵人马上鞍。\\
两马八蹄踏兰苑,情如合竹谁能见。\\
夜光玉枕栖凤皇,袷罗当门刺纯线。\\
长翻蜀纸卷明君,转角含商破碧云。\\
自从小靥来东道,曲里长眉少见人。\\
相如冢上生秋柏,三秦谁是言情客。\\
蛾鬟醉眼拜诸宗,为谒皇孙请曹植。\\
}
\poetry{新夏歌}{
晓木千笼真蜡彩,落蕊枯香数分在。\\
阴枝秀牙卷缥茸,长风回气扶葱茏。\\
野家麦畦上新垅,长畛裴回桑柘重。\\
刺香满地菖蒲草,雨梁燕语悲身老。\\
三月摇扬入河道,天浓地浓柳梳扫。\\
}
\poetry{题归梦}{
长安风雨夜,书客梦昌谷。\\
怡怡中堂笑,小弟栽涧菉。\\
家门厚重意,望我饱饥腹。\\
劳劳一寸心,灯花照鱼目。\\
}
\poetry{经沙苑}{
野水泛长澜,宫牙开小蒨。\\
无人柳自春,草渚鸳鸯暖。\\
晴嘶卧沙马,老去悲啼展。\\
今春还不归,塞嘤折翅雁。\\
}
\poetry{出城别张又新酬李汉}{
李子别上国,南山崆峒春。\\
不闻今夕鼓,差慰煎情人。\\
赵壹赋命薄,马卿家业贫。\\
乡书何所报,紫蕨生石云。\\
长安玉桂国,戟带披侯门。\\
惨阴地自光,宝马踏晓昏。\\
腊春戏草苑,玉挽鸣{隐阝=车}辚。\\
绿网缒金铃,霞卷清池漘。\\
开贯泻蚨母,买冰防夏蝇。\\
时宜裂大袂,剑客车盘茵。\\
小人如死灰,心切生秋榛。\\
皇图跨四海,百姓拖长绅。\\
光明霭不发,腰龟徒甃银。\\
吾将噪礼乐,声调摩清新。\\
欲使十千岁,帝道如飞神。\\
华实自苍老,流采长倾湓。\\
没没暗𫜬舌,涕血不敢论。\\
今将下东道,祭酒而别秦。\\
六郡无剿儿,长刀谁拭尘。\\
地理阳无正,快马逐服辕。\\
二子美年少,调道讲清浑。\\
讥笑断冬夜,家庭疏筿穿。\\
曙风起四方,秋月当东悬。\\
赋诗面投掷,悲哉不遇人。\\
此别定霑臆,越布先裁巾。\\
}
\poetry{南园}{
方领蕙带折角巾,杜若已老兰苕春。\\
南山削秀蓝玉合,小雨归去飞凉云。\\
熟杏暖香梨叶老,草梢竹栅锁池痕。\\
郑公乡老开酒尊,坐泛楚奏吟招魂。\\
}
\poetry{假龙吟歌}{
石轧铜杯,吟咏枯瘁。\\
苍鹰摆血,白凤下肺。\\
桂子自落,云弄车盖。\\
木死沙崩恶谿岛,阿母得仙今不老。\\
窞中跳汰截清涎,隈壖卧水埋金爪。\\
崖蹬苍苔吊石发,江君掩帐筼筜折。\\
莲花去国一千年,雨后闻腥犹带{钅截}。\\
}
\poetry{感讽六首 一}{
人间春荡荡,帐暖香扬扬。\\
飞光染幽红,夸娇来洞房。\\
舞席泥金蛇,桐竹罗花床。\\
眼逐春瞑醉,粉随泪色黄。\\
王子下马来,曲沼鸣鸳鸯。\\
焉知肠车转,一夕巡九方。\\
}
\poetry{感讽六首 二}{
苦风吹朔寒,沙惊秦木折。\\
舞影逐空天,画鼓余清节。\\
蜀书秋信断,黑水朝波咽。\\
娇魂从回风,死处悬乡月。\\
}
\poetry{感讽六首 三}{
杂杂胡马尘,森森边士戟。\\
天教胡马战,晓云皆血色。\\
妇人携汉卒,箭箙囊巾帼。\\
不惭金印重,踉蹡腰鞬力。\\
恂恂乡门老,昨夜试锋镝。\\
走马遣书勋,谁能分粉墨。\\
}
\poetry{感讽六首 四}{
青门放弹去,马色连空郊。\\
何年帝家物,玉装鞍上摇。\\
去去走犬归,来来坐烹羔。\\
千金不了馔,狢肉称盘臊。\\
试问谁家子,乃老能佩刀。\\
西山白盖下,贤俊寒萧萧。\\
}
\poetry{感讽六首 五}{
晓菊泫寒露,似悲团扇风。\\
秋凉经汉殿,班子泣衰红。\\
本无辞辇意,岂见入空宫。\\
腰衱珮珠断,灰蝶生阴松。\\
}
\poetry{感讽六首 六}{
蝶飞红粉台,柳扫吹笙道。\\
十日悬户庭,九秋无衰草。\\
调歌送风转,杯池白鱼小。\\
水宴截香腴,菱科映青罩。\\
{艹/丰}蒙梨花满,春昏弄长啸。\\
唯愁苦花落,不悟世衰到。\\
抚旧唯销魂,南山坐悲峭。\\
}
\poetry{莫愁曲}{
草生龙坡下,鸦噪城堞头。\\
何人此城里,城角栽石榴。\\
青丝系五马,黄金络双牛。\\
白鱼驾莲船,夜作十里游。\\
归来无人识,暗上沈香楼。\\
罗床倚瑶瑟,残月倾帘钩。\\
今日槿花落,明朝桐树秋。\\
莫负平生意,何名何莫愁。\\
}
\poetry{夜来乐}{
红罗复帐金流苏,华灯九枝悬鲤鱼。\\
丽人暎月开铜铺,春水滴酒猩猩沽。\\
重一箧,香十株,赤金瓜子兼杂麸。\\
五丝封青凫,阿侯此笑千万余。\\
南轩汉转帘影疏,桐林哑哑挟子乌。\\
剑崖鞭节青石珠,白䯄吹湍凝霜须。\\
漏长送珮承明庐,倡楼嵯峨明月孤。\\
新客下马故客去,绿蝉秀黛重拂梳。\\
}
\poetry{嘲雪}{
昨日发葱岭,今朝下兰渚。\\
喜从千里来,乱笑含春语。\\
龙沙湿汉旗,凤扇迎秦素。\\
久别辽城鹤,毛衣已应故。\\
}
\poetry{春怀引}{
芳蹊密影成花洞,柳结浓烟花带重。\\
蟾蜍碾玉挂明弓,捍拨装金打仙凤。\\
宝枕垂云选春梦,钿合碧寒龙脑冻。\\
阿侯系锦觅周郎,凭仗东风好相送。\\
}
\poetry{白虎行}{
火乌日暗崩腾云,秦皇虎视苍生群。\\
烧书灭国无暇日,铸剑佩玦惟将军。\\
玉坛设醮思冲天,一世二世当万年。\\
烧丹未得不死药,拏舟海上寻神仙。\\
鲸鱼张鬣海波沸,耕人半作征人鬼。\\
雄豪气猛如焰烟,无人为决天河水。\\
谁最苦兮谁最苦,报人义士深相许。\\
渐离击筑荆卿歌,荆卿把酒燕丹语。\\
剑如霜兮胆如铁,出燕城兮望秦月。\\
天授秦封祚未移,衮龙衣点荆卿血。\\
朱旗卓地白虎死,汉皇知是真天子。\\
}
\poetry{有所思}{
去年陌上歌离曲,今日君书远游蜀。\\
帘外花开二月风,台前泪滴千行竹。\\
琴心与妾肠,此夜断还续。\\
想君白马悬雕弓,世间何处无春风。\\
君心未肯镇如石,妾颜不久如花红。\\
夜残高碧横长河,河上无梁空白波。\\
西风未起悲龙梭,年年织素攒双蛾。\\
江山迢遰无休绝,泪眼看灯乍明灭。\\
自从孤馆深锁窗,桂花几度圆还缺。\\
鸦鸦向晓鸣森木,风过池塘响丛玉。\\
白日萧条梦不成,桥南更问仙人卜。\\
}
\poetry{啁少年}{
青騘马肥金鞍光,龙脑入缕罗衫香。\\
美人狭坐飞琼觞,贫人唤云天上郎。\\
别起高楼临碧筿,丝曳红鳞出深沼。\\
有时半醉百花前,背把金丸落飞鸟。\\
自说生来未为客,一身美妾过三百。\\
岂知𣃁地种苗家,官税频催勿人织。\\
长得积玉夸豪毅,每揖闲人多意气。\\
生来不读半行书,只把黄金买身贵。\\
少年安得长少年,海波尚变为桑田。\\
荣枯递传急如箭,天公不肯于公偏。\\
莫道韶华镇长在,发白面皱专相待。\\
}
\poetry{高平县东私路}{
侵侵槲叶香,木花滞寒雨。\\
今夕山上秋,永谢无人处。\\
石磎远荒涩,棠实悬辛苦。\\
古者定幽寻,呼君作私路。\\
}
\poetry{神仙曲}{
碧峰海面藏灵书,上帝拣作神仙居。\\
清明笑语闻空虚,鬬乘巨浪骑鲸鱼。\\
春罗书字邀王母,共䜩红楼最深处。\\
鹤羽冲风过海迟,不如却使青龙去。\\
犹疑王母不相许,垂露娃鬟更传语。\\
}
\poetry{龙夜吟}{
鬈发胡儿眼睛绿,高楼夜静吹横竹。\\
一声似向天上来,月下美人望乡哭。\\
直排七点星藏指,暗合清风调宫征。\\
蜀道秋深云满林,湘江半夜龙惊起。\\
玉堂美人边塞情,碧窗皓月愁中听。\\
寒砧能𢭏百尺练,粉泪凝珠滴红线。\\
胡儿莫作陇头吟,隔窗暗结愁人心。\\
}
\poetry{昆仑使者}{
昆仑使者无消息,茂陵烟树生愁色。\\
金盘玉露自淋漓,元气茫茫收不得。\\
麒麟背上石文裂,虬龙鳞下红枝折。\\
何处偏伤万国心,中天夜久高明月。\\
}
\poetry{汉唐姬饮酒歌}{
御服沾霜露,天衢长蓁棘。\\
金隐秋尘姿,无人为带饰。\\
玉堂歌声寝,芳林烟树隔。\\
云阳台上歌,鬼哭复何益。\\
铁剑常光光,至凶威屡逼。\\
彊枭噬母心,犇厉索人魄。\\
相看两相泣,泪下如波激。\\
宁用清酒为,欲作黄泉客。\\
不说玉山颓,且无饮中色。\\
勉从天帝诉,天上寡沈厄。\\
无处张穗帷,如何望松柏。\\
妾身昼团团,君魂夜寂寂。\\
蛾眉自觉长,颈粉谁怜白。\\
矜持昭阳意,不肯看南陌。\\
}
\poetry{听颖师琴歌}{
别浦云归桂花渚,蜀国弦中双凤语。\\
芙蓉叶落秋鸾离,越王夜起游天姥。\\
暗珮清臣敲水玉,渡海蛾眉牵白鹿。\\
谁看挟剑赴长桥,谁看浸发题春竹。\\
竺僧前立当吾门,梵宫真相眉棱尊。\\
古琴大轸长八尺,峄阳老树非桐孙。\\
凉馆闻弦惊病客,药囊暂别龙须席。\\
请歌直请卿相歌,奉礼官卑复何益。\\
}
\poetry{谣俗}{
上林胡蝶小,试伴汉家君。\\
飞向南城去,误落石榴裙。\\
脉脉花满树,翾翾燕遶云。\\
出门不识路,羞问陌头人。\\
}
\poetry{静女春曙曲}{
嫩叶怜芳抱新蕊,泣露枝枝滴夭泪。\\
粉窗香咽颓晓云,锦堆花密藏春睡。\\
恋屏孔雀摇金尾,莺舌分明呼婢子。\\
冰洞寒龙半匣水,一只商鸾逐烟起。\\
}
\poetry{少年乐}{
芳草落花如锦地,二十长游醉乡里。\\
红缨不动白马骄,垂柳金丝香拂水。\\
吴娥未笑花不开,绿鬓耸堕兰云起。\\
陆郎倚醉牵罗袂,夺得宝钗金翡翠。\\
}
\poetry{句}{
不见山巅树,摧杌下为薪。\\
日睹井中泥,上出作埃尘。\\
情知一丘趣,不谢千里印。\\
倚剑登高台,悠悠送春目。\\
}
\poetry{白门前}{
白门前,大楼喜,悬虹云,挞龙尾。\\
剑破匣,舞蛟龙,蚩尤死,鼓龙蓬。\\
天堕地,无惊飞,海千里。\\
(《四部业刊》影印所以刊本《李贺歌诗编》)(此诗为《全唐诗》卷三九三李贺《上之回》一诗之别本,前八句仅数字不同,末四句,《上之回》为七言二句。\\
)-394-。\\
}
