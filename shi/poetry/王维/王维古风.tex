
\poetry{送綦毋潜落第还乡}{
圣代无隐者,英灵尽来归。\\
遂令东山客,不得顾采薇。\\
既至君门远,孰云吾道非。\\
江淮度寒食,京洛缝春衣。\\
置酒临长道,同心与我违。\\
行当浮桂櫂,未几拂荆扉。\\
远树带行客,孤村当落晖。\\
吾谋适不用,勿谓知音稀。\\
}
\poetry{青谿}{
言入黄花川,每逐青谿水。\\
随山将万转,趣途无百里。\\
声喧乱石中,色静深松里。\\
漾漾泛菱荇,澄澄映葭苇。\\
我心素已闲,清川澹如此。\\
请留盘石上,垂钓将已矣。\\
}
\poetry{横吹曲辞 陇头吟}{
长安少年游侠客,夜上戍楼看太白。\\
陇头明月迥临关,陇上行人夜吹笛。\\
关西老将不胜愁,驻马听之双泪流。\\
身经大小百余战,麾下偏裨万户侯。\\
苏武才为典属国,节旄空尽海西头。\\
}
\poetry{西施咏}{
    艳色天下重,西施宁久微。\\
    朝仍越溪女,暮作吴宫妃。\\
    贱日岂殊众,贵来方悟稀。\\
    邀人傅香粉,不自著罗衣。\\
    君宠益娇态,君怜无是非。\\
    当时浣纱伴,莫得同车归。\\
    持谢邻家子,效颦安可希。\\
    }
\poetry{陇头吟}{
长安少年游侠客,夜上戍楼看太白。\\
陇头明月迥临关,陇上行人夜吹笛。\\
关西老将不胜愁,驻马听之双泪流。\\
身经大小百余战,麾下偏裨万户侯。\\
苏武才为典属国,节旄落尽海西头。\\
}
\poetry{老将行}{
少年十五二十时,步行夺得胡马射。\\
射杀中山白额虎,肯数邺下黄须儿。\\
一身转战三千里,一剑曾当百万师。\\
汉兵奋迅如霹雳,虏骑崩腾畏蒺藜。\\
卫青不败由天幸,李广无功缘数奇。\\
自从弃置便衰朽,世事磋跎成白首。\\
昔时飞箭无全目,今日垂杨生左肘。\\
路傍时卖故侯瓜,门前学种先生柳。\\
苍茫古木连穷巷,寥落寒山对虚牖。\\
誓令疏勒出飞泉,不似颍川空使酒。\\
贺兰山下阵如云,羽檄交驰日夕闻。\\
节使三河募年少,诏书五道出将军。\\
试拂铁衣如雪色,聊持宝剑动星文。\\
愿得燕弓射天将,耻令越甲鸣吴军。\\
莫嫌旧日云中守,犹堪一战取功勋。\\
}
\poetry{桃源行}{
渔舟逐水爱山春,两岸桃花夹去津。\\
坐看红树不知远,行尽青溪不见人。\\
山口潜行始隈隩,山开旷望旋平陆。\\
遥看一处攒云树,近入千家散花竹。\\
樵客初传汉姓名,居人未改秦衣服。\\
居人共住武陵源,还从物外起田园。\\
月明松下房栊静,日出云中鸡犬喧。\\
惊闻俗客争来集,竞引还家问都邑。\\
平明闾巷埽花开,薄暮渔樵乘水入。\\
初因避地去人间,及至成仙遂不还。\\
峡里谁知有人事,世中遥望空云山。\\
不疑灵境难闻见,尘心未尽思乡县。\\
出洞无论隔山水,辞家终拟长游衍。\\
自谓经过旧不迷,安知峰壑今来变。\\
当时只记入山深,青溪几曲到云林。\\
春来遍是桃花水,不辨仙源何处寻。\\
}
\poetry{洛阳女儿行}{
洛阳女儿对门居,才可容颜十五余。\\
良人玉勒乘骢马,侍女金盘鲙鲤鱼。\\
画阁朱楼尽相望,红桃绿柳垂檐向。\\
罗帏送上七香车,宝扇迎归九华帐。\\
狂夫富贵在青春,意气骄奢剧季伦。\\
自怜碧玉亲教舞,不惜珊瑚持与人。\\
春窗曙灭九微火,九微片片飞花璅。\\
戏罢曾无理曲时,妆成秪是熏香坐。\\
城中相识尽繁华,日夜经过赵李家。\\
谁怜越女颜如玉,贫贱江头自浣纱。\\
}
\poetry{送秘书晁监还日本国}{
积水不可极,安知沧海东。\\
九州何处远,万里若乘空。\\
向国唯看日,归帆但信风。\\
鳌身暎天黑,鱼眼射波红。\\
乡树扶桑外,主人孤岛中。\\
别离方异域,音信若为通。\\
}