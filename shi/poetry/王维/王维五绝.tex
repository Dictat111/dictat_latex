
\poetry{息夫人}{
莫以今时宠,难忘旧日恩。\\
看花满眼泪,不共楚王言。\\
}
\poetry{辋川集 鹿柴}{
空山不见人,但闻人语响。\\
返景入深林,复照青苔上。\\
}
\poetry{辋川集 栾家濑}{
飒飒秋雨中,浅浅石溜泻。\\
跳波自相溅,白鹭惊复下。\\
}
\poetry{辋川集 白石滩}{
清浅白石滩,绿蒲向堪把。\\
家住水东西,浣纱明月下。\\
}
\poetry{辋川集 竹里馆}{
独坐幽篁里,弹琴复长啸。\\
深林人不知,明月来相照。\\
}
\poetry{辋川集 辛夷坞}{
木末芙蓉花,山中发红萼。\\
涧户寂无人,纷纷开且落。\\
}
\poetry{皇甫岳云溪杂题五首 鸟鸣涧}{
人闲桂花落,夜静春山空。\\
月出惊山鸟,时鸣春涧中。\\
}
\poetry{送别}{
山中相送罢,日暮掩柴扉。\\
春草明年绿,王孙归不归。\\
}
\poetry{杂诗三首 二}{
君自故乡来,应知故乡事。\\
来日绮窗前,寒梅着花未。\\
}
\poetry{崔兴宗写真咏}{
画君年少时,如今君已老。\\
今时新识人,知君旧时好。\\
}
\poetry{相思}{
红豆生南国,秋来发故枝。\\
愿君多采撷,此物最相思。\\
}
\poetry{书事}{
轻阴阁小雨,深院昼慵开。\\
坐看苍苔色,欲上人衣来。\\
}
\poetry{阙题二首 一}{
荆谿白石出,天寒红叶稀。\\
山路元无雨,空翠湿人衣。\\
}
\poetry{田园乐七首 四}{
萋萋春草秋绿,落落长松夏寒。\\
牛羊自归村巷,童稚不识衣冠。\\
}
\poetry{田园乐七首 五}{
山下孤烟远村,天边独树高原。\\
一瓢颜回陋巷,五柳先生对门。\\
}
\poetry{田园乐七首 六}{
桃红复含宿雨,柳绿更带朝烟。\\
花落家童未埽,莺啼山客犹眠。\\
}