\poetry{横吹曲辞 出塞 一}{
秦时明月汉时关,万里长征人未还。\\
但使龙城飞将在,不教胡马度阴山。\\
}
\poetry{相和歌辞 从军行 二}{
烽火城西百尺楼,黄昏独上海风秋。\\
更吹横笛关山月,谁解金闺万里愁。\\
}
\poetry{相和歌辞 从军行 三}{
琵琶起舞换新声,总是关山旧别情。\\
撩乱边愁弹不尽,高高秋月下长城。\\
}
\poetry{相和歌辞 从军行 四}{
青海长云暗雪山,孤城遥望雁门关。\\
黄沙百战穿金甲,不破楼兰终不还。\\
}
\poetry{相和歌辞 长信怨 一}{
金井梧桐秋叶黄,珠帘不卷夜来霜。\\
金炉玉枕无颜色,卧听南宫清漏长。\\
}
\poetry{相和歌辞 长信怨 二}{
奉帚平明金殿开,暂将团扇共裴回。\\
玉颜不及寒鸦色,犹带昭阳日影来。\\
}
\poetry{相和歌辞 采莲曲三首 二}{
荷叶罗裙一色裁,芙蓉向脸两边开。\\
乱入池中看不见,闻歌始觉有人来。\\
}
\poetry{从军行七首 一}{
烽火城西百尺楼,黄昏独上海风秋。\\
更吹羌笛关山月,无那金闺万里愁。\\
}
\poetry{从军行七首 二}{
琵琶起舞换新声,总是关山旧别情。\\
撩乱边愁听不尽,高高秋月照长城。\\
}
\poetry{从军行七首 四}{
青海长云暗雪山,孤城遥望玉门关。\\
黄沙百战穿金甲,不破楼兰终不还。\\
}
\poetry{从军行七首 五}{
大漠风尘日色昏,红旗半卷出辕门。\\
前军夜战洮河北,已报生擒吐谷浑。\\
}
\poetry{出塞二首 一}{
秦时明月汉时关,万里长征人未还。\\
但使龙城飞将在,不教胡马度阴山。\\
}
\poetry{采莲曲二首 二}{
荷叶罗裙一色裁,芙蓉向脸两边开。\\
乱入池中看不见,闻歌始觉有人来。\\
}
\poetry{春宫曲}{
昨夜风开露井桃,未央前殿月轮高。\\
平阳歌舞新承宠,帘外春寒赐锦袍。\\
}
\poetry{西宫春怨}{
西宫夜静百花香,欲卷珠帘春恨长。\\
斜抱云和深见月,朦胧树色隐昭阳。\\
}
\poetry{西宫秋怨}{
芙蓉不及美人妆,水殿风来珠翠香。\\
谁分含啼掩秋扇,空悬明月待君王。\\
}
\poetry{长信秋词五首 一}{
金井梧桐秋叶黄,珠帘不卷夜来霜。\\
熏笼玉枕无颜色,卧听南宫清漏长。\\
}
\poetry{长信秋词五首 三}{
奉帚平明金殿开,且将团扇暂裴回。\\
玉颜不及寒鸦色,犹带[昭]阳日影来。\\
}
\poetry{闺怨}{
闺中少妇不曾愁,春日凝妆上翠楼。\\
忽见陌头杨柳色,悔教夫婿觅封侯。\\
}
\poetry{听流人水调子}{
孤舟微月对枫林,分付鸣筝与客心。\\
岭色千重万重雨,断弦收与泪痕深。\\
}
\poetry{送魏二}{
醉别江楼橘柚香,江风引雨入舟凉。\\
忆君遥在潇湘月,愁听清猿梦里长。\\
}
\poetry{芙蓉楼送辛渐二首 一}{
寒雨连天夜入湖,平明送客楚山孤。\\
洛阳亲友如相问,一片冰心在玉壶。\\
}