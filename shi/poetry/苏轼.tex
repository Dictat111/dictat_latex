
\section{苏轼}
\poetry{浣溪沙·山下兰芽短浸溪}{
    山下兰芽短浸溪。
    松间沙路净无泥。
    萧萧暮雨子规啼。\\
    谁道人生无再少,门前流水尚能西。
    休将白发唱黄鸡。}


\poetry{水龙吟·似花还似非花}{  
似花还似非花,也无人惜从教坠。\\
抛家傍路,思量却是,无情有思。\\
萦损柔肠,困酣娇眼,欲开还闭。\\
梦随风万里,寻郎去处,又还被、莺呼起。\\
不恨此花飞尽,恨西园、落红难缀。\\
晓来雨过,遗踪何在,一池萍碎。\\
春色三分,二分尘土,一分流水。\\
细看来,不是杨花点点,是离人泪。  
}

\poetry{满庭芳·蜗角虚名}{蜗角虚名,蝇头微利,算来著甚乾忙。\\
事皆前定,谁弱又谁强。\\
且趁闲身未老,尽放我、些子疏狂。\\
百年里,浑教是醉,三万六千场。\\
思量。\\
能几许,忧愁风雨,一半相妨。\\
又何须,抵死说短论长。\\
幸对清风皓月,苔茵展、云幕高张。\\
江南好,千锺美酒,一曲《满庭芳》。\\
}


\poetry{水调歌头·落日绣帘卷}{落日绣帘卷,亭下水连空。\\
知君为我,新作窗户湿青红。\\
长记平山堂上,敧枕江南烟雨,渺渺没孤鸿。\\
认得醉翁语,山色有无中。\\
一千顷,都镜净,倒碧峰。\\
忽然浪起,掀舞一叶白头翁。\\
堪笑兰台公子,未解庄生天籁,刚道有雌雄。\\
一点浩然气,千里快哉风。\\
}

\poetry{水调歌头·明月几时有}{明月几时有,把酒问青天。\\
不知天上宫阙,今夕是何年。\\
我欲乘风归去,又恐琼楼玉宇,高处不胜寒。\\
起舞弄清影,何似在人间。\\
转朱阁,低绮户,照无眠。\\
不应有恨,何事长向别时圆。\\
人有悲欢离合,月有阴晴圆缺,此事古难全。\\
但愿人长久,千里共婵娟。\\
}

\poetry{念奴娇·大江东去}{大江东去,浪淘尽、千古风流人物。\\
故垒西边人道是,三国周郎赤壁。\\
乱石穿空,惊涛拍岸,卷起千堆雪。\\
江山如画,一时多少豪杰。\\
遥想公瑾当年,小乔初嫁了,雄姿英发。\\
羽扇纶巾谈笑间,强虏灰飞烟灭。\\
故国神游,多情应笑,我早生华发。\\
人间如梦,一尊还酹江月。\\
}

\poetry{西江月·世事一场大梦}{世事一场大梦,人生几度秋凉。\\
夜来风叶已鸣廊。\\
看取眉头鬓上。\\
酒贱常愁客少,月明多被云妨。\\
中秋谁与共孤光。\\
把盏凄然北望。\\
}

\poetry{西江月·三过平山堂下}{三过平山堂下,半生弹指声中。\\
十年不见老仙翁。\\
壁上龙蛇飞动。\\
欲吊文章太守,仍歌杨柳春风。\\
休言万事转头空。\\
未转头时皆梦。
}


\poetry{临江仙·一别都门三改火}{一别都门三改火,天涯踏尽红尘。\\
依然一笑作春温。\\
无波真古井,有节是秋筠。\\
惆怅孤帆连夜发,送行淡月微云。\\
尊前不用翠眉颦。\\
人生如逆旅,我亦是行人。\\
}

\poetry{临江仙·夜饮东坡醒复醉}{夜饮东坡醒复醉,归来仿佛三更。\\
家童鼻息已雷鸣。\\
敲门都不应,倚杖听江声。\\
长恨此身非我有,何时忘却营营。\\
夜阑风静縠纹平。\\
小舟从此逝,江海寄馀生。\\
}

\poetry{鹧鸪天}{林断山明竹隐墙。\\
乱蝉衰草小池塘。\\
翻空白鸟时时见,照水红蕖细细香。\\
村舍外,古城旁。\\
杖藜徐步转斜阳。\\
殷勤昨夜三更雨,又得浮生一日凉。\\
}

\poetry{少年游}{去年相送,馀杭门外,飞雪似杨花。\\
今年春尽,杨花似雪,犹不见还家。\\
对酒卷帘邀明月,风露透窗纱。\\
恰似姮娥怜双燕,分明照、画梁斜。\\
}

\poetry{定风波}{莫听穿林打叶声。\\
何妨吟啸且徐行。\\
竹杖芒鞋轻胜马。\\
谁怕。\\
一蓑烟雨任平生。\\
料峭春风吹酒醒。\\
微冷。\\
山头斜照却相迎。\\
回首向来潇洒处。\\
归去。\\
也无风雨也无晴。\\
}

\poetry{定风波}{常羡人间琢玉郎。\\
天应乞与点酥娘。\\
尽道清歌传皓齿。\\
风起。\\
雪飞炎海变清凉。\\
万里归来颜愈少。\\
微笑。\\
笑时犹带岭梅香。\\
试问岭南应不好。\\
却道。\\
此心安处是吾乡。\\
}

\poetry{南乡子}{霜降水痕收。\\
浅碧鳞鳞露远洲。\\
酒力渐消风力软,飕飕。\\
破帽多情却恋头。\\
佳节若为酬。\\
但把清尊断送秋。\\
万事到头都是梦,休休。\\
明日黄花蝶也愁。
}

\poetry{南乡子}{东武望馀杭。\\
云海天涯两杳茫。\\
何日功成名遂了,还乡。\\
醉笑陪公三万场。\\
不用诉离觞。\\
痛饮从来别有肠。\\
今夜送归灯火冷,河塘。\\
堕泪羊公却姓杨。\\
}

\poetry{望江南・忆江南}{春未老,风细柳斜斜。\\
试上超然台上看,半壕春水一城花。\\
烟雨暗千家。\\
寒食后,酒醒却咨嗟。\\
休对故人思故国,且将新火试新茶。\\
诗酒趁年华。\\
}

\poetry{卜算子}{蜀客到江南,长忆吴山好。\\
吴蜀风流自古同,归去应须早。\\
还与去年人,共藉西湖草。\\
莫惜尊前仔细看,应是容颜老。\\
}

\poetry{卜算子}{缺月挂疏桐,漏断人初静。\\
时见幽人独往来,缥缈孤鸿影。\\
惊起却回头,有恨无人省。\\
拣尽寒枝不肯栖,寂寞沙洲冷。\\
}

\poetry{贺新郎}{乳燕飞华屋。\\
悄无人、桐阴转午,晚凉新浴。\\
手弄生绡白团扇,扇手一时似玉。\\
渐困倚、孤眠清熟。\\
帘外谁来推绣户,枉教人、梦断瑶台曲。\\
又却是,风敲竹。\\
石榴半吐红巾蹙。\\
待浮花、浪蕊都尽,伴君幽独。\\
秾艳一枝细看取,芳心千重似束。\\
又恐被、秋风惊绿。\\
若待得君来向此,花前对酒不忍触。\\
共粉泪,两蔌蔌。\\
}

\poetry{洞仙歌}{冰肌玉骨,自清凉无汗。\\
水殿风来暗香满。\\
绣帘开、一点明月窥人,人未寝、敧枕钗横鬓乱。\\
起来携素手,庭户无声,时见疏星渡河汉。\\
试问夜如何,夜已三更,金波淡、玉绳低转。\\
但屈指、西风几时来,又不道、流年暗中偷换。\\
}

\poetry{江神子・江城子}{凤凰山下雨初晴。\\
水风清。\\
晚霞明。\\
一朵芙蕖,开过尚盈盈。\\
何处飞来双白鹭,如有意,慕娉婷。\\
忽闻江上弄哀筝。\\
苦含情,遣谁听。\\
烟敛云收,依约是湘灵。\\
欲待曲终寻问取,人不见,数峰青。\\
}

\poetry{江神子・江城子}{老夫聊发少年狂。\\
左牵黄。\\
右擎苍。\\
锦帽貂裘,千骑卷平冈。\\
为报倾城随太守,亲射虎,看孙郎。\\
酒酣胸胆尚开张。\\
鬓微霜。\\
又何妨。\\
持节云中,何日遣冯唐。\\
会挽雕弓如满月,西北望,射天狼。\\
}

\poetry{江神子・江城子}{天涯流落思无穷。\\
既相逢。\\
却匆匆。\\
携手佳人,和泪折残红。\\
为问东风馀几许,春纵在,与谁同。\\
隋堤三月水溶溶。\\
背归鸿。\\
去吴中。\\
回首彭城,清泗与淮通。\\
寄我相思千点泪,流不到,楚江东。\\
}

\poetry{江神子・江城子}{十年生死两茫茫。\\
不思量。\\
自难忘。\\
千里孤坟,无处话凄凉。\\
纵使相逢应不识,尘满面,鬓如霜。\\
夜来幽梦忽还乡。\\
小轩窗。\\
正梳妆。\\
相顾无言,惟有泪千行。\\
料得年年断肠处,明月夜,短松冈。\\
}

\poetry{蝶恋花}{花褪残红青杏小。\\
燕子飞时,绿水人家绕。\\
枝上柳绵吹又少。\\
天涯何处无芳草。\\
墙里秋千墙外道。\\
墙外行人,墙里佳人笑。\\
笑渐不闻声渐悄。\\
多情却被无情恼。\\
}

\poetry{蝶恋花}{簌簌无风花自亸。\\
寂寞园林,柳老樱桃过。\\
落日多情还照坐。\\
山青一点横云破。\\
路尽河回千转柁。\\
系缆渔村,月暗孤灯火。\\
凭仗飞魂招楚些。\\
我思君处君思我。\\
}

\poetry{永遇乐}{明月如霜,好风如水,清景无限。\\
曲港跳鱼,圆荷泻露,寂寞无人见。\\
紞如三鼓,铿然一叶,黯黯梦云惊断。\\
夜茫茫,重寻无处,觉来小园行遍。\\
天涯倦客,山中归路,望断故园心眼。\\
燕子楼空,佳人何在,空锁楼中燕。\\
古今如梦,何曾梦觉,但有旧欢新怨。\\
异时对,黄楼夜景,为余浩叹。\\
}

\poetry{行香子}{清夜无尘。\\
月色如银。\\
酒斟时、须满十分。\\
浮名浮利,虚苦劳神。\\
叹隙中驹,石中火,梦中身。\\
虽抱文章,开口谁亲。\\
且陶陶、乐尽天真。\\
几时归去,作个闲人。\\
对一张琴,一壶酒,一溪云。\\
}

\poetry{行香子}{携手江村。
梅雪飘裙。情何限、处处消魂。\\
故人不见,旧曲重闻。\\
向望湖楼,孤山寺,涌金门。\\
寻常行处,题诗千首,绣罗衫、与拂红尘。\\
别来相忆,知是何人。\\
有湖中月,江边柳,陇头云。\\
}
一共收入了苏轼 \thepoetrycounter  首诗。
