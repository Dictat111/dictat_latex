\poetry{相和歌辞 三阁词四首 一}{
贵人三阁上,日晏未梳头。\\
不应有恨事,娇甚却成愁。\\
}
\poetry{相和歌辞 三阁词四首 二}{
珠箔曲琼钩,子细见扬州。\\
北兵那得度,浪语判悠悠。\\
}
\poetry{相和歌辞 三阁词四首 三}{
沈香帖阁柱,金缕画门楣。\\
回首降幡下,已见黍离离。\\
}
\poetry{相和歌辞 三阁词四首 四}{
三人出眢井,一身登槛车。\\
朱门漫临水,不可见鲈鱼。\\
}
\poetry{琴曲歌辞 秋风引}{
何处秋风至?萧萧送雁群。\\
朝来入庭树,孤客最先闻。\\
}
\poetry{杂曲歌辞 荆州乐二首 一}{
渚宫杨柳暗,麦城朝雉飞。\\
可怜踏青伴,乘暖著轻衣。\\
}
\poetry{杂曲歌辞 荆州乐二首 二}{
今日好南风,商旅相催发。\\
沙头樯竿上,始见春江阔。\\
}
\poetry{杂曲歌辞 纪南歌}{
风烟纪南城,尘土荆门路。\\
天寒猎兽者,走上樊姬墓。\\
}
\poetry{杂曲歌辞 纥那曲 一}{
杨柳郁青青,竹枝无限情。\\
同郎一回顾,听唱纥那声。\\
}
\poetry{杂曲歌辞 纥那曲 二}{
踏曲兴无穷,调同词不同。\\
愿郎千万寿,长作主人翁。\\
}
\poetry{杂曲歌辞 忆江南 一}{
春过也,共惜艳阳年。\\
犹有桃花流水上,无辞竹叶醉樽前,惟待见青天。\\
}
\poetry{杂曲歌辞 忆江南 二}{
春去也,多谢洛城人。\\
弱柳从风疑举袂,丛兰裛露似霑巾,独笑亦含嚬。\\
}
\poetry{两如何诗谢裴令公赠别二首 一}{
一言一顾重,重何如。\\
今日陪游清洛苑,昔年别入承明庐。\\
}
\poetry{两如何诗谢裴令公赠别二首 二}{
一东一西别,别何如。\\
终期大冶再熔炼,愿托扶摇翔碧虚。\\
}
\poetry{和乐天春词依忆江南曲拍为句 一}{
春去也,多谢洛城人。\\
弱柳从风疑举袂,丛兰裛露似霑巾,独坐亦含嚬。\\
}
\poetry{和乐天春词依忆江南曲拍为句 二}{
春过也,笑惜艳阳年。\\
犹有桃花流水上,无辞竹叶醉樽前,惟待见青天。\\
}
\poetry{泽宫诗 一}{
秩秩泽宫,有的维鹄。\\
祁祁庶士,于以干禄。\\
}
\poetry{泽宫诗 二}{
彼鹄斯微,若止若翔。\\
千里之差,起于毫芒。\\
}
\poetry{泽宫诗 三}{
我矢既直,我弓既良。\\
依于高墉,因我不臧。\\
}
\poetry{泽宫诗 四}{
高墉伊何,维器与时。\\
视之以心,谁谓鹄微。\\
}
\poetry{荅乐天临都驿见赠}{
北固山边波浪,东都城里风尘。\\
世事不同心事,新人何似故人。\\
}
\poetry{再赠乐天}{
一政政官轧轧,一年年老骎骎。\\
身外名何足算,别来诗且同吟。\\
}
\poetry{酬杨侍郎凭见寄}{
十年毛羽催颓,一旦天书召回。\\
看看瓜时欲到,故侯也好归来。\\
}
\poetry{荆州歌二首 一}{
渚宫杨柳暗,麦城朝雉飞。\\
可怜踏青伴,乘暖著轻衣。\\
}
\poetry{荆州歌二首 二}{
今日好南风,商旅相催发。\\
沙头樯竿上,始见春江阔。\\
}
\poetry{纪南歌}{
风烟纪南城,尘土荆门路。\\
天寒多猎骑,走上樊姬墓。\\
}
\poetry{视刀环歌}{
常恨言语浅,不如人意深。\\
今朝两相视,脉脉万重心。\\
}
\poetry{三阁辞四首 一}{
贵人三阁上,日晏未梳头。\\
不应有恨事,娇甚却成愁。\\
}
\poetry{三阁辞四首 二}{
珠箔曲琼钩,仔细见扬州。\\
北兵那得度,浪话判悠悠。\\
}
\poetry{三阁辞四首 三}{
沉香帖阁柱,金缕画门楣。\\
回首降旛下,已见黍[离]离。\\
}
\poetry{三阁辞四首 四}{
三人出眢井,一身登槛车。\\
朱门漫临水,不可见鲈鱼。\\
}
\poetry{纥那曲二首 一}{
杨柳郁青青,竹枝无限情。\\
周郎一回顾,听唱纥那声。\\
}
\poetry{纥那曲二首 二}{
踏曲兴无穷,调同词不同。\\
愿郎千万寿,长作主人翁。\\
}
\poetry{淮阴行五首 一}{
簇簇淮阴市,竹楼缘岸上。\\
好日起樯竿,乌飞惊五两。\\
}
\poetry{淮阴行五首 二}{
今日转船头,金乌指西北。\\
烟波与春草,千里同一色。\\
}
\poetry{淮阴行五首 三}{
船头大铜镮,摩挲光阵阵。\\
早早使风来,沙头一眼认。\\
}
\poetry{淮阴行五首 四}{
何物令侬羡,羡郎船尾燕。\\
衔泥趁樯竿,宿食长相见。\\
}
\poetry{淮阴行五首 五}{
隔浦望行船,头昂尾幰幰。\\
无奈晚来时,清淮春浪软。\\
}
\poetry{浑侍中宅牡丹}{
径尺千余朵,人间有此花。\\
今朝见颜色,更不向诸家。\\
}
\poetry{咏红柿子}{
晓连星影出,晚带日光悬。\\
本因遗采掇,翻自保天年。\\
}
\poetry{秋风引}{
何处秋风至,萧萧送雁群。\\
朝来入庭树,孤客最先闻。\\
}
\poetry{柳花词三首 一}{
开从绿条上,散逐香风远。\\
故取花落时,悠扬占春晚。\\
}
\poetry{柳花词三首 二}{
轻飞不假风,轻落不委地。\\
撩乱舞晴空,发人无限思。\\
}
\poetry{柳花词三首 三}{
晴天暗暗雪,来送青春暮。\\
无意似多情,千家万家去。\\
}
\poetry{路傍曲}{
南山宿雨晴,春入凤凰城。\\
处处闻弦管,无非送酒声。\\
}
\poetry{君山怀古}{
属车八十一,此地阻长风。\\
千载威灵尽,赭山寒水中。\\
}
\poetry{庭竹}{
露涤铅粉节,风摇青玉枝。\\
依依似君子,无地不相宜。\\
}
\poetry{唐郎中宅与诸公同饮酒看牡丹}{
今日花前饮,甘心醉数杯。\\
但愁花有语,不为老人开。\\
}
\poetry{题寿安甘棠馆二首 一}{
公馆似仙家,池清竹径斜。\\
山禽忽惊起,冲落半岩花。\\
}
\poetry{题寿安甘棠馆二首 二}{
门前洛阳道,门里桃花路。\\
尘土与烟霞,其间十余步。\\
}
\poetry{古词二首 一}{
轩后初冠冕,前旒为蔽明。\\
安知从复道,然后见人情。\\
}
\poetry{古词二首 二}{
簿领乃俗士,清谈信古风。\\
吾观苏令绰,朱墨一何工。\\
}
\poetry{寓兴二首 一}{
常谈即至理,安事非常情。\\
寄语何平叔,无为轻老生。\\
}
\poetry{寓兴二首 二}{
世途多礼数,鹏鷃各逍遥。\\
何事陶彭泽,抛官为折腰。\\
}
\poetry{咏史二首 一}{
骠骑非无势,少卿终不去。\\
世道剧颓波,我心如砥柱。\\
}
\poetry{咏史二首 二}{
贾生明王道,卫绾工车戏。\\
同遇汉文时,何人居贵位。\\
}
\poetry{经檀道济故垒}{
万里长城坏,荒营野草秋。\\
秣陵多士女,犹唱白符鸠。\\
}
\poetry{伤段右丞}{
江海多豪气,朝廷有直声。\\
何言马蹄下,一旦是佳城。\\
}
\poetry{伤独孤舍人}{
昔别矜年少,今悲丧国华。\\
远来同社燕,不见早梅花。\\
}
\poetry{再伤庞尹}{
京兆归何处,章台空暮尘。\\
可怜鸾镜下,哭杀画眉人。\\
}
\poetry{敬酬微公见寄二首 一}{
凄凉沃州僧,憔悴柴桑宰。\\
别来二十年,唯余两心在。\\
}
\poetry{敬酬微公见寄二首 二}{
越江千里镜,越岭四时雪。\\
中有逍遥人,夜深观水月。\\
}
\poetry{鄂渚留别李二十一表臣大夫}{
高樯起行色,促柱动离声。\\
欲问江深浅,应如远别情。\\
}
\poetry{荅表臣赠别二首 一}{
昔为瑶池侣,飞舞集蓬莱。\\
今作江汉别,风雪一徘徊。\\
}
\poetry{荅表臣赠别二首 二}{
嘶马立未还,行舟路将转。\\
江头瞑色深,挥袖依稀见。\\
}
\poetry{始发鄂渚寄表臣二首 一}{
祖帐管弦绝,客帆西风生。\\
回车已不见,犹听马嘶声。\\
}
\poetry{始发鄂渚寄表臣二首 二}{
晓发柳林戍,遥城闻五鼓。\\
忆与故人眠,此时犹晤语。\\
}
\poetry{出鄂州界怀表臣二首 一}{
离席一挥杯,别愁今尚醉。\\
迟迟有情处,却恨江帆驶。\\
}
\poetry{出鄂州界怀表臣二首 二}{
梦觉疑连榻,舟行忽千里。\\
不见黄鹤楼,寒沙雪相似。\\
}
\poetry{和游房公旧竹亭闻琴绝句}{
尚有竹间路,永无綦下尘。\\
一闻流水曲,重忆餐霞人。\\
}
\poetry{西州李尚书知愚与元武昌有旧远示二篇吟之泫然因以继和二首 一}{
如何赠琴日,已是绝弦时。\\
无复双金报,空余挂剑悲。\\
}
\poetry{西州李尚书知愚与元武昌有旧远示二篇吟之泫然因以继和二首 二}{
宝匣从此闲,朱弦谁复调。\\
祗应随玉树,同向土中销。\\
}
\poetry{别苏州二首 一}{
三载为吴郡,临岐祖帐开。\\
虽非谢桀黠,且为一裴回。\\
}
\poetry{别苏州二首 二}{
流水阊门外,秋风吹柳条。\\
从来送客处,今日自魂销。\\
}
\poetry{罢和州游建康}{
秋水清无力,寒山暮多思。\\
官闲不计程,遍上南朝寺。\\
}
\poetry{九日登高}{
世路山河险,君门烟雾深。\\
年年上高处,未省不伤心。\\
}
\poetry{荅柳子厚}{
年方伯玉早,恨比四愁多。\\
会待休车骑,相随出罻罗。\\
}
\poetry{馆娃宫在旧郡西南砚石山前瞰姑苏台傍有采香径梁天监中置佛寺曰灵岩即故宫也信为绝境因赋二章 一}{
宫馆贮娇娃,当时意大夸。\\
艳倾吴国尽,笑入楚王家。\\
}
\poetry{馆娃宫在旧郡西南砚石山前瞰姑苏台傍有采香径梁天监中置佛寺曰灵岩即故宫也信为绝境因赋二章 二}{
月殿移椒壁,天花代舜华。\\
唯余采香径,一带绕山斜。\\
}
\poetry{纥那曲 一}{
杨柳郁青青,竹枝无限情。\\
同郎一回顾,听唱纥那声。\\
}
\poetry{纥那曲 二}{
蹋曲兴无穷,调同辞不同。\\
愿郎千万寿,长作主人翁。\\
}
\poetry{忆江南 一}{
春去也,多谢洛城人。\\
弱柳从风疑举袂,丛兰裛露似霑巾,独坐亦含嚬。\\
}
\poetry{忆江南 二}{
春去也,共惜艳阳年。\\
犹有桃花流水上,无辞竹叶醉尊前,惟待见青天。\\
}
\poetry{赠乐天}{
唯君比萱草,相见可忘忧。\\
(《白氏长庆集》卷三四《酬梦得比萱草见赠》自注引)。\\
}
\poetry{贫居咏怀赠乐天}{
若有金挥胜二疏。\\
(同前卷三五《酬梦得贫居咏怀见赠》自注引)。\\
}
\poetry{酬乐天}{
炼尽美少年。\\
(同前《梦得前所酬篇有炼尽美少年之句因思往事兼咏今怀重以长句答之》)。\\
}
