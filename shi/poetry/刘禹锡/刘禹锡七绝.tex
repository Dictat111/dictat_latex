
\poetry{杂曲歌辞 竹枝 二}{
山桃红花满上头,蜀江春水拍江流。\\
花红易衰似郎意,水流无限似侬愁。\\
}
\poetry{杂曲歌辞 竹枝 六}{
瞿塘嘈嘈十二滩,此中道路古来难。\\
长恨人心不如水,等闲平地起波澜。\\
}
\poetry{杂曲歌辞 竹枝 一}{
杨柳青青江水平,闻郎江上唱歌声。\\
东边日出西边雨,道是无情还有情。\\
}
\poetry{杂曲歌辞 浪淘沙 一}{
九曲黄河万里沙,浪淘风簸自天涯。\\
如今直上银河去,同到牵牛织女家。\\
}
\poetry{杂曲歌辞 浪淘沙 六}{
日照澄洲江雾开,淘金女伴满江隈。\\
美人首饰侯王印,尽是沙中浪底来。\\
}
\poetry{杂曲歌辞 潇湘神二曲 二}{
斑竹枝,斑竹枝,泪痕点点寄相思。\\
楚客欲听瑶瑟怨,潇湘深夜月明时。\\
}
\poetry{竹枝词二首 一}{
杨柳青青江水平,闻郎江上唱歌声。\\
东边日出西边雨,道是无晴却有晴。\\
}
\poetry{秋词二首 一}{
自古逢秋悲寂寥,我言秋日胜春朝。\\
晴空一鹤排云上,便引诗情到碧霄。\\
}
\poetry{秋词二首 二}{
山明水净夜来霜,数树深红出浅黄。\\
试上高楼清入骨,岂如春色嗾人狂。\\
}

\poetry{元和十一年自朗州召至京戏赠看花诸君子}{
紫陌红尘拂面来,无人不道看花回。\\
玄都观里桃千树,尽是刘郎去后栽。\\
}
\poetry{再游玄都观}{
百亩庭中半是苔,桃花净尽菜花开。\\
种桃道士归何处,前度刘郎今又来。\\
}
\poetry{望夫石}{
终日望夫夫不归,化为孤石苦相思。\\
望来已是几千载,只似当时初望时。\\
}
\poetry{金陵五题 石头城}{
山围故国周遭在,潮打空城寂寞回。\\
淮水东边旧时月,夜深还过女墙来。\\
}
\poetry{金陵五题 乌衣巷}{
朱雀桥边野草花,乌衣巷口夕阳斜。\\
旧时王谢堂前燕,飞入寻常百姓家。\\
}
\poetry{金陵五题 台城}{
台城六代竞豪华,结绮临春事最奢。\\
万户千门成野草,只缘一曲后庭花。\\
}
\poetry{赏牡丹}{
庭前芍药妖无格,池上芙蕖净少情。\\
唯有牡丹真国色,花开时节动京城。\\
}
\poetry{和乐天春词}{
新妆面面下朱楼,深锁春光一院愁。\\
行到中庭数花朵,蜻蜓飞上玉搔头。\\
}
\poetry{望洞庭}{
湖光秋月两相和,潭面无风镜未磨。\\
遥望洞庭山水翠,白银盘里一青螺。\\
}

\poetry{杨柳枝}{
春江一曲柳千条,二十年前旧板桥。\\
曾与美人桥上别,恨无消息到今朝。\\
}