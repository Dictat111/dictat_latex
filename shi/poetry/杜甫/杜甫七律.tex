
\poetry{悲陈陶}{
孟冬十郡良家子,血作陈陶泽中水。\\
野旷天清无战声,四万义军同日死。\\
群胡归来血洗箭,仍唱胡歌饮都市。\\
都人回面北向啼,日夜更望官军至。\\
}
\poetry{九日蓝田崔氏庄}{
老去悲秋强自宽,兴来今日尽君欢。\\
羞将短发还吹帽,笑倩旁人为正冠。\\
蓝水远从千涧落,玉山高并两峰寒。\\
明年此会知谁健,醉把茱萸仔细看。\\
}
\poetry{曲江二首 一}{
一片花飞减却春,风飘万点正愁人。\\
且看欲尽花经眼,莫厌伤多酒入唇。\\
江上小堂巢翡翠,花边高冢卧麒麟。\\
细推物理须行乐,何用浮名绊此身。\\
}
\poetry{曲江二首 二}{
朝回日日典春衣,每日江头尽醉归。\\
酒债寻常行处有,人生七十古来稀。\\
穿花\xpinyin*{蛱}蝶深深见,点水蜻蜓款款飞。\\
传语风光共流转,暂时相赏莫相违。\\
}
\poetry{蜀相}{
丞相祠堂何处寻,锦官城外柏森森。\\
映阶碧草自春色,隔叶黄鹂空好音。\\
三顾频烦天下计,两朝开济老臣心。\\
出师未捷身先死,长使英雄泪满襟。\\
}



\poetry{狂夫}{
万里桥西一草堂,百花潭水即沧浪。\\
风含翠筿娟娟静,雨裛红蕖冉冉香。\\
厚禄故人书断绝,恒饥稚子色凄凉。\\
欲填沟壑唯疏放,自笑狂夫老更狂。\\
}


\poetry{江村}{
清江一曲抱村流,长夏江村事事幽。\\
自去自来堂上燕,相亲相近水中鸥。\\
老妻画纸为棋局,稚子敲针作钓钩。\\
多病所须唯药物,微躯此外更何求。\\
}
\poetry{恨别}{
洛城一别四千里,胡骑长驱五六年。\\
草木变衰行剑外,兵戈阻绝老江边。\\
思家步月清宵立,忆弟看云白日眠。\\
闻道河阳近乘胜,司徒急为破幽燕。\\
}

\poetry{客至}{
舍南舍北皆春水,但见群鸥日日来。\\
花径不曾缘客扫,蓬门今始为君开。\\
盘餐市远无兼味,樽酒家贫只旧\xpinyin*{醅}。\\
肯与邻翁相对饮,隔篱呼取尽余杯。\\
}
\poetry{江上值水如海势聊短述}{
为人性僻耽佳句,语不惊人死不休。\\
老去诗篇浑漫兴,春来花鸟莫深愁。\\
新添水槛供垂钓,故著浮\xpinyin*{槎}替入舟。\\
焉得思如陶谢手,令渠述作与同游。\\
}
\poetry{送韩十四江东觐省}{
兵戈不见老莱衣,叹息人间万事非。\\
我已无家寻弟妹,君今何处访庭\xpinyin*{闱}。\\
黄牛峡静滩声转,白马江寒树影稀。\\
此别应须各努力,故乡犹恐未同归。\\
}


\poetry{闻官军收河南河北}{
剑外忽传收\xpinyin*{蓟}北,初闻涕泪满衣裳。\\
却看妻子愁何在,漫卷诗书喜欲狂。\\
白日放歌须纵酒,青春作伴好还乡。\\
即从巴峡穿巫峡,便下襄阳向洛阳。\\
}


\poetry{登高}{
风急天高猿啸哀,渚清沙白鸟飞回。\\
无边落木萧萧下,不尽长江衮衮来。\\
万里悲秋常作客,百年多病独登台。\\
艰难苦恨繁霜鬓,潦倒新停浊酒杯。\\
}

\poetry{登楼}{
花近高楼伤客心,万方多难此登临。\\
锦江春色来天地,玉垒浮云变古今。\\
北极朝廷终不改,西山寇盗莫相侵。\\
可怜后主还祠庙,日暮聊为梁甫吟。\\
}
\poetry{宿府}{
清秋幕府井梧寒,独宿江城蜡炬残。\\
永夜角声悲自语,中天月色好谁看。\\
风尘荏苒音书绝,关塞萧条行路难。\\
已忍伶俜十年事,强移栖息一枝安。\\
}

\poetry{阁夜}{
岁暮阴阳催短景,天涯霜雪霁寒宵。\\
五更鼓角声悲壮,三峡星河影动摇。\\
野哭几家闻战伐,夷歌数处起渔樵。\\
卧龙跃马终黄土,人事音书漫寂寥。\\
}

\poetry{白帝}{
白帝城中云出门,白帝城下雨翻盆。\\
高江急峡雷霆斗,翠木苍藤日月昏。\\
戎马不如归马逸,千家今有百家存。\\
哀哀寡妇诛求尽,恸哭秋原何处村。\\
}

\poetry{秋兴八首 一}{
玉露凋伤枫树林,巫山巫峡气萧森。\\
江间波浪兼天涌,塞上风云接地阴。\\
丛菊两开他日泪,孤舟一系故园心。\\
寒衣处处催刀尺,白帝城高急暮砧。\\
}
\poetry{秋兴八首 三}{
千家山郭静朝晖,一日江楼坐翠微。\\
信宿渔人还泛泛,清秋燕子故飞飞。\\
匡衡抗疏功名薄,刘向传经心事违。\\
同学少年多不贱,五陵衣马自轻肥。\\
}
\poetry{秋兴八首 四}{
闻道长安似弈棋,百年世事不胜悲。\\
王侯第宅皆新主,文武衣冠异昔时。\\
直北关山金鼓振,征西车马羽书迟。\\
鱼龙寂寞秋江冷,故国平居有所思。\\
}
\poetry{秋兴八首 五}{
蓬莱宫阙对南山,承露金茎霄汉间。\\
西望瑶池降王母,东来紫气满函关。\\
云移雉尾开宫扇,日绕龙鳞识圣颜。\\
一卧沧江惊岁晚,几回青琐照朝班。\\
}
\poetry{秋兴八首 六}{
\xpinyin*{瞿}唐峡口\xpinyin*{曲}江头,万里风烟接素秋。\\
花萼夹城通御气,芙蓉小苑入边愁。\\
朱帘绣柱围黄鹤,锦缆牙\xpinyin*{樯}起白鸥。\\
回首可怜歌舞地,秦中自古帝王州。\\
}
\poetry{秋兴八首 七}{
昆明池水汉时功,武帝\xpinyin*{旌}旗在眼中。\\
织女机丝虚月夜,石鲸鳞甲动秋风。\\
波漂\xpinyin*{菰}米沈云黑,露冷莲房坠粉红。\\
关塞极天唯鸟道,江湖满地一渔翁。\\
}
\poetry{咏怀古迹五首 一}{
支离东北风尘际,漂泊西南天地间。\\
三峡楼台淹日月,五溪衣服共云山。\\
\xpinyin*{羯}胡事主终无赖,词客衰时且未还。\\
\xpinyin*{庾}信平生最萧瑟,暮年诗赋动江关。\\
}
\poetry{咏怀古迹五首 二}{
摇落深知宋玉悲,风流儒雅亦吾师。\\
\xpinyin*{怅}望千秋一洒泪,萧条异代不同时。\\
江山故宅空文藻,云雨荒台岂梦思。\\
最是楚宫俱泯灭,舟人指点到今疑。\\
}
\poetry{咏怀古迹五首 三}{
群山万壑赴荆门,生长明妃尚有村。\\
一去紫台连朔漠,独留青冢向黄昏。\\
画图\xpinyin{省}{xing3}识春风面,环佩空归月夜魂。\\
千载琵琶作胡语,分明怨恨曲中论。\\
}
\poetry{咏怀古迹五首 四}{
蜀主窥吴幸三峡,崩年亦在永安宫。\\
翠华想像空山里,玉殿虚无野寺中。\\
古庙杉松巢水鹤,岁时伏腊走村翁。\\
武侯祠屋常邻近,一体君臣祭祀同。\\
}
\poetry{咏怀古迹五首 五}{
诸葛大名垂宇宙,宗臣遗像肃清高。\\
三分割据\xpinyin*{纡}筹策,万古云霄一羽毛。\\
伯仲之间见伊吕\footnote{伊尹和吕尚},指挥若定失萧曹\footnote{使萧何和曹参失色}。\\
福移汉祚难恢复,志决身歼军务劳。\\
}



