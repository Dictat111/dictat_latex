
\poetry{横吹曲辞 前出塞九首 六}{
挽弓当挽彊,用箭当用长。\\
射人先射马,擒贼先擒王。\\
杀人亦有限,列国自有疆。\\
茍能制侵陵,岂在多杀伤。\\
}

\poetry{望岳}{
岱宗夫如何,齐鲁青未了。\\
造化钟神秀,阴阳割昏晓。\\
荡胸生曾云,决眦入归鸟。\\
会当凌绝顶,一览众山小。\\
}
\poetry{房兵曹胡马诗}{
胡马大宛名,锋棱瘦骨成。\\
竹批双耳峻,风入四蹄轻。\\
所向无空阔,真堪托死生。\\
骁腾有如此,万里可横行。\\
}
\poetry{画鹰}{
素练风霜起,苍鹰画作殊。\\
㧐身思狡兔,侧目似愁胡。\\
绦旋光堪\xpinyin*{擿},轩楹势可呼。\\
何当击凡鸟,毛血洒平芜。\\
}


\poetry{春日忆李白}{
白也诗无敌,飘然思不群。\\
清新庾开府,俊逸鲍参军。\\
渭北春天树,江东日暮云。\\
何时一尊酒,重与细论文。\\
}

\poetry{对雪}{
战哭多新鬼,愁吟独老翁。\\
乱云低薄暮,急雪舞回风。\\
瓢弃尊无绿,炉存火似红。\\
数州消息断,愁坐正书空。\\
}
\poetry{月夜}{
今夜\xpinyin*{鄜}州月,闺中只独看。\\
遥怜小儿女,未解忆长安。\\
香雾云鬟湿,清辉玉臂寒。\\
何时倚虚幌,双照泪痕干。\\
}
\poetry{春望}{
国破山河在,城春草木深。\\
感时花溅泪,恨别鸟惊心。\\
烽火连三月,家书抵万金。\\
白头搔更短,浑欲不胜簪。\\
}
\poetry{天末忆李白}{
凉风起天末,君子意如何。\\
鸿雁几时到,江湖秋水多。\\
文章憎命达,魑魅喜人过。\\
应共冤魂语,投诗赠汩罗。\\
}
\poetry{岁暮}{
岁暮远为客,边隅还用兵。\\
烟尘犯雪岭,鼓角动江城。\\
天地日流血,朝廷谁请缨。\\
济时敢爱死,寂寞壮心惊。\\
}
\poetry{春夜喜雨}{
好雨知时节,当春乃发生。\\
随风潜入夜,润物细无声。\\
野径云俱黑,江船火独明。\\
晓看红湿处,花重锦官城。\\
}
\poetry{琴台}{
茂陵多病后,尚爱卓文君。\\
酒肆人间世,琴台日暮云。\\
野花留宝靥,蔓草见罗裙。\\
归凤求皇意,寥寥不复闻。\\
}
\poetry{水槛遣心二首 一}{
去郭轩楹敞,无村眺望赊。\\
澄江平少岸,幽树晚多花。\\
细雨鱼儿出,微风燕子斜。\\
城中十万户,此地两三家。\\
}
\poetry{奉济驿重送严公四韵}{
远送从此别,青山空复情。\\
几时杯重把,昨夜月同行。\\
列郡讴歌惜,三朝出入荣。\\
江村独归处,寂寞养残生。\\
}
\poetry{不见}{
不见李生久,佯狂真可哀。\\
世人皆欲杀,吾意独怜才。\\
敏捷诗千首,飘零酒一杯。\\
匡山读书处,头白好归来。\\
}
\poetry{春远}{
肃肃花絮晚,菲菲红素轻。\\
日长唯鸟雀,春远独柴荆。\\
数有关中乱,何曾剑外清。\\
故乡归不得,地入亚夫营。\\
}
\poetry{旅夜书怀}{
细草微风岸,危樯独夜舟。\\
星垂平野阔,月涌大江流。\\
名岂文章著,官因老病休。\\
飘飘何所似,天地一沙鸥。\\
}
\poetry{江汉}{
江汉思归客,乾坤一腐儒。\\
片云天共远,永夜月同孤。\\
落日心犹壮,秋风病欲疏。\\
古来存老马,不必取长途。\\
}
\poetry{月圆}{
孤月当楼满,寒江动夜\xpinyin*{扉}。\\
委波金不定,照席绮逾依。\\
未缺空山静,高悬列宿稀。\\
故园松桂发,万里共清辉。\\
}
\poetry{孤雁}{
孤雁不饮啄,飞鸣声念群。\\
谁怜一片影,相失万重云。\\
望尽似犹见,哀多如更闻。\\
野鸦无意绪,鸣噪自纷纷。\\
}
\poetry{登岳阳楼}{
昔闻洞庭水,今上岳阳楼。\\
吴楚东南坼,乾坤日夜浮。\\
亲朋无一字,老病有孤舟。\\
戎马关山北,凭轩涕泗流。\\
}