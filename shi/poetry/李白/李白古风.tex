
\subsection{古风}
\poetry{古风 一}{
大雅久不作,吾衰竟谁陈。\\
王风委蔓草,战国多荆榛。\\
龙虎相啖食,兵戈逮狂秦。\\
正声何微茫,哀怨起骚人。\\
扬马激颓波,开流荡无垠。\\
废兴虽万变,宪章亦已沦。\\
自从建安来,绮丽不足珍。\\
圣代复元古,垂衣贵清真。\\
群才属休明,乘运共跃鳞。\\
文质相炳焕,众星罗秋旻。\\
我志在删述,垂辉暎千春。\\
希圣如有立,绝笔于获麟。\\
}
\poetry{古风 三}{
秦皇扫六合,虎视何雄哉。\\
飞剑决浮云,诸侯尽西来。\\
明断自天启,大略驾群才。\\
收兵铸金人,函谷正东开。\\
铭功会稽岭,骋望琅琊台。\\
刑徒七十万,起土骊山隈。\\
尚采不死药,茫然使心哀。\\
连弩射海鱼,长鲸正崔嵬。\\
额鼻象五岳,扬波喷云雷。\\
鬐鬣蔽青天,何由睹蓬莱。\\
徐[巿]载秦女,楼船几时迥。\\
但见三泉下,金棺葬寒灰。\\
}

\poetry{古风 十九}{
西岳莲花山,迢迢见明星。\\
素手把芙蓉,虚步蹑太清。\\
霓裳曳广带,飘拂升天行。\\
邀我登云台,高揖卫叔卿。\\
恍恍与之去,驾鸿凌紫冥。\\
俯视洛阳川,茫茫走胡兵。\\
流血涂野草,豺狼尽冠缨。\\
}

\poetry{金乡送韦八之西京}{
客自长安来,还归长安去。\\
狂风吹我心,西挂咸阳树。\\
此情不可道,此别何时遇。\\
望望不见君,连山起烟雾。\\
}
\poetry{相和歌辞 梁甫吟}{
长啸梁甫吟,何时见阳春?\\
君不见朝歌屠叟辞棘津,八十西来钓渭滨。\\
宁羞白发照渌水,逢时吐气思经纶。\\
广张三千六百钧,风雅暗与文王亲。\\
大贤虎变愚不测,当年颇似寻常人。\\
君不见高阳酒徒起草中,长揖山东隆准公。\\
入门不拜骋雄辨,两女辍洗来趋风。\\
东下齐城七十二,指麾楚汉如旋蓬。\\
狂生落拓尚如此,何况壮士当群雄。\\
我欲攀龙见明主,雷公砰訇震天鼓。\\
帝旁投壶多玉女,三时大笑开电光。\\
倏烁晦冥起风雨,阊阖九门不可通。\\
以额叩关阍者怒,白日不照吾精诚。\\
杞国无事忧天倾,{豸契}貐磨牙竞人肉。\\
驺虞不折生草茎,手接飞猱搏雕虎。\\
侧足焦原未言苦,智者可卷愚者豪。\\
世人见我轻鸿毛,力排南山三壮士,\\齐相杀之费二桃。\\
吴楚弄兵无剧孟,亚夫咍尔为徒劳。\\
梁父吟,梁父吟,声正悲。\\
张公两龙剑,神物合有时。\\
风云感会起屠钓,大人{山儿}屼当安之。\\
}

\poetry{相和歌辞 子夜四时歌四首 冬歌}{
明朝驿使发,一夜絮征袍。\\
素手抽针冷,那堪把剪刀。\\
裁缝寄远道,几日到临洮。\\
}

\poetry{杂曲歌辞 长干行二首 一}{
妾发初覆额,折花门前剧。\\
郎骑竹马来,绕床弄青梅。\\
同居长干里,两小无嫌猜。\\
十四为君妇,羞颜尚不开。\\
低头向暗壁,千唤不一回。\\
十五始展眉,愿同尘与灰。\\
常存抱柱信,岂上望夫台。\\
十六君远行,瞿塘滟预堆。\\
五月不可触,猿鸣天上哀。\\
门前迟行迹,一一生绿苔。\\
苔深不能扫,落叶秋风早。\\
八月蝴蝶来,双飞西园草。\\
感此伤妾心,坐愁红颜老。\\
早晚下三巴,预将书报家。\\
相迎不道远,直至长风沙。\\
}
\poetry{妾薄命}{汉帝重阿娇,贮之黄金屋。\\
咳唾落九天,随风生珠玉。\\
宠极爱还歇,妒深情却疏。\\
长门一步地,不肯暂回车。\\
雨落不上天,水覆难再收。\\
君情与妾意,各自东西流。\\
昔日芙蓉花,今成断根草。\\
以色事他人,能得几时好。\\
}

\poetry{北风行}{烛龙栖寒门,光曜犹旦开。\\
日月照之何不及此?\\惟有北风号怒天上来。\\
燕山雪花大如席,片片吹落轩辕台。\\
幽州思妇十二月,停歌罢笑双蛾摧。\\
倚门望行人,念君长城苦寒良可哀。\\
别时提剑救边去,遗此虎文金鞞靫。\\
中有一双白羽箭,蜘蛛结网生尘埃。\\
箭空在,人今战死不复回。\\
不忍见此物,焚之已成灰。\\
黄河捧土尚可塞,北风雨雪恨难裁。\\
}



\poetry{把酒问月·故人贾淳令予问之}{青天有月来几时?我今停杯一问之。\\
人攀明月不可得,月行却与人相随。\\
皎如飞镜临丹阙,绿烟灭尽清辉发。\\
但见宵从海上来,宁知晓向云间没。\\
白兔捣药秋复春,嫦娥孤栖与谁邻?\\
今人不见古时月,今月曾经照古人。\\
古人今人若流水,共看明月皆如此。\\
唯愿当歌对酒时,月光长照金樽里。\\
}

\poetry{侠客行}{赵客缦胡缨,吴钩霜雪明。\\
银鞍照白马,飒\xpinyin*{遝}如流星。\\
十步杀一人,千里不留行。\\
事了拂衣去,深藏身与名。\\
闲过信陵饮,脱剑膝前横。\\
将炙啖朱亥,持觞劝侯嬴。\\
三杯吐然诺,五岳倒为轻。\\
眼花耳热后,意气素霓生。\\
救赵挥金槌,邯郸先震惊。\\
千秋二壮士,煊赫大梁城。\\
纵死侠骨香,不惭世上英。\\
谁能书合下,白首太玄经。\\
}

\poetry{古风 十}{齐有倜傥生,鲁连特高妙。\\
明月出海底,一朝开光曜。\\
却秦振英声,后世仰末照。\\
意轻千金赠,愿向平原笑。\\
吾亦澹荡人,拂衣可同调。\\
}

\poetry{横吹曲辞 关山月}{明月出天山,苍茫云海间。\\
长风几万里,吹度玉门关。\\
汉下白登道,胡窥青海湾。\\
由来征战地,不见有人还。\\
戍客望边色,思归多苦颜。\\
高楼当此夜,叹息未应闲。\\
}

\poetry{长相思·其一}{长相思,在长安。\\
络纬秋啼金井阑,微霜凄凄簟色寒。\\
孤灯不明思欲绝,卷帷望月空长叹。\\
美人如花隔云端!\\
上有青冥之长天,下有渌水之波澜。\\
天长路远魂飞苦,梦魂不到关山难。\\
长相思,摧心肝!\\
}

\poetry{长相思·其二}{日色欲尽花含烟,月明如素愁不眠。\footnote{如素 一作:欲素}\\
赵瑟初停凤凰柱,蜀琴欲奏鸳鸯弦。\\
此曲有意无人传,愿随春风寄燕然。\\
忆君迢迢隔青天,\\昔日横波目,今作流泪泉。\\
不信妾肠断,归来看取明镜前。(肠断 一作:断肠)\\
}
\poetry{长相思·其三}{
美人在时花满堂,美人去后花余床。\\
床中绣被卷不寝,至今三载犹闻香。\\
香亦竟不灭,人亦竟不来。\\
相思黄叶落,白露点青苔。\\
}
\poetry{三五七言}{秋风清,秋月明。\\落叶聚还散,寒鸦栖复惊。\\
相思相见知何日,此时此夜难为情。\\
}

\poetry{子夜吴歌}{长安一片月,万户捣衣声。\\
秋风吹不尽,总是玉关情。\\
何日平胡虏,良人罢远征。\\
}

\poetry{古朗月行}{小时不识月,呼作白玉盘。\\
又疑瑶台镜,飞在白云端。\\
仙人垂两足,桂树作团团。\\
白兔捣药成,问言与谁餐。\\
蟾蜍蚀圆影,大明夜已残。\\
羿昔落九乌,天人清且安。\\
阴精此沦惑,去去不足观。\\
忧来其如何,凄怆摧心肝。\\
}









\poetry{下终南山过斛斯山人宿置酒}{暮从碧山下,山月随人归。\\
却顾所来径,苍苍横翠微。\\
相携及田家,童稚开荆扉。\\
绿竹入幽径,青萝拂行衣。\\
欢言得所憩,美酒聊共挥。\\
长歌吟松风,曲尽河星稀。\\
我醉君复乐,陶然共忘机。\\
}


\poetry{月下独酌四首·其一}{花间一壶酒,独酌无相亲。\\
举杯邀明月,对影成三人。\\
月既不解饮,影徒随我身。\\
暂伴月将影,行乐须及春。\\
我歌月徘徊,我舞影零乱。\\
醒时同交欢,醉后各分散。(同交欢 一作:相交欢)\\
永结无情游,相期邈云汉。\\
}

\poetry{月下独酌四首 二}{天若不爱酒,酒星不在天。\\
地若不爱酒,地应无酒泉。\\
天地既爱酒,爱酒不愧天。\\
已闻清比圣,复道浊如贤。\\
贤圣既已饮,何必求神仙。\\
三杯通大道,一斗合自然。\\
但得酒中趣,勿为醒者传。\\
}

\poetry{春思}{燕草如碧丝,秦桑低绿枝。\\
当君怀归日,是妾断肠时。\\
春风不相识,何事入罗帏。\\
}

\poetry{宣州谢脁楼饯别校书叔云}{弃我去者,昨日之日不可留;\\
乱我心者,今日之日多烦忧。\\
长风万里送秋雁,对此可以酣高楼。\\
蓬莱文章建安骨,中间小谢又清发。\\
俱怀逸兴壮思飞,欲上青天揽明月。\\
抽刀断水水更流,举杯消愁愁更愁。\\
人生在世不称意,明朝散发弄扁舟。\\
}

\poetry{庐山谣寄卢侍御虚舟}{我本楚狂人,凤歌笑孔丘。\\
手持绿玉杖,朝别黄鹤楼。\\
五岳寻仙不辞远,一生好入名山游。\\
庐山秀出南斗傍,屏风九叠云锦张。\\
影落明湖青黛光,金阙前开二峰长,\\
银河倒挂三石梁。\\
香炉瀑布遥相望,回崖沓嶂凌苍苍。\\
翠影红霞映朝日,鸟飞不到吴天长。\\
登高壮观天地间,大江茫茫去不还。\\
黄云万里动风色,白波九道流雪山。\\
好为庐山谣,兴因庐山发。\\
闲窥石镜清我心,谢公行处苍苔没。\\
早服还丹无世情,琴心三叠道初成。\\
遥见仙人彩云里,手把芙蓉朝玉京。\\
先期汗漫九\xpinyin*{垓}上,愿接卢敖游太清。\\
}

\poetry{梦游天姥吟留别}{海客谈瀛洲,烟涛微茫信难求;\\
越人语天姥,云霞明灭或可睹。\\
天姥连天向天横,势拔五岳掩赤城。\\
天台四万八千丈,对此欲倒东南倾。\\
我欲因之梦吴越,一夜飞度镜湖月。\footnote{ 度 通:渡}\\
湖月照我影,送我至剡溪。\\
谢公宿处今尚在,渌水荡漾清猿啼。\\
脚著谢公屐,身登青云梯。\\
半壁见海日,空中闻天鸡。\\
千岩万转路不定,迷花倚石忽已暝。\\
熊咆龙吟殷岩泉,栗深林兮惊层巅。\\
云青青兮欲雨,水澹澹兮生烟。\\
列缺霹雳,丘峦崩摧。\\
洞天石扉,訇然中开。\\
青冥浩荡不见底,日月照耀金银台。\\
霓为衣兮风为马,云之君兮纷纷而来下。\\
虎鼓瑟兮鸾回车,仙之人兮列如麻。\\
忽魂悸以魄动,恍惊起而长嗟。\\
惟觉时之枕席,失向来之烟霞。\\
世间行乐亦如此,古来万事东流水。\\
别君去兮何时还?且放白鹿青崖间,须行即骑访名山。\\
安能摧眉折腰事权贵,使我不得开心颜!\\
}

\poetry{金陵酒肆留别}{风吹柳花满店香,吴姬压酒劝客尝。\\
金陵子弟来相送,欲行不行各尽觞。\\
请君试问东流水,别意与之谁短长。\\
}

\poetry{赠裴十四}{
朝见裴叔则,朗如行玉山。\\
黄河落天走东海,万里写入胸怀间。\\
身骑白鼋不敢度,金高南山买君顾。\\
裴回六合无相知,飘若浮云且西去。\\
}


\poetry{秋浦歌}{白发三千丈,缘愁似个长。\\
不知明镜里,何处得秋霜?\\
}



\poetry{行路难·金樽清酒斗十千}{金樽清酒斗十千,玉盘珍羞直万钱。\\
停杯投箸不能食,拔剑四顾心茫然。\\
欲渡黄河冰塞川,将登太行雪满山。\\
闲来垂钓碧溪上,忽复乘舟梦日边。\\
行路难!行路难!多歧路,今安在?\\
长风破浪会有时,直挂云帆济沧海。\\
}


\poetry{灞陵行送别}{
送君灞陵亭,灞水流浩浩。\\
上有无花之古树,下有伤心之春草。\\
我向秦人问路岐,云是王粲南登之古道。\\
古道连绵走西京,紫阙落日浮云生。\\
正当今夕断肠处,黄鹂愁绝不忍听。\\
}
\poetry{乌栖曲}{
姑苏台上乌栖时,吴王宫里醉西施。\\
吴歌楚舞欢未毕,青山欲衔半边日。(还有三句)\\
}
\poetry{经下邳圯桥怀张子房}{
子房未虎啸,破产不为家。\\
沧海得壮士,椎秦博浪沙。\\
报韩虽不成,天地皆振动。\\
潜匿游下邳,岂曰非智勇。\\
我来圯桥上,怀古钦英风。\\
惟见碧流水,曾无黄石公。\\
叹息此人去,萧条徐泗空。\\
}
\poetry{杂曲歌辞 远别离}{
远别离,古有皇英之二女,乃在洞庭之南,潇湘之浦。\\
海水直下万里深,谁人不言此离苦。\\
日惨惨兮云冥冥,猩猩啼烟兮鬼啸雨。\\
我纵言之将何补,皇穹窃恐不照余之忠诚。\\
雷冯冯兮欲吼怒,尧舜当之亦禅禹。\\
君失臣兮龙为鱼,权归臣兮鼠变虎。\\
或云尧幽囚,舜野死,九疑联绵皆相似,重瞳孤坟竟何是。\\
帝子泣兮绿云间,随风波兮去无还。\\
恸哭兮远望,见苍梧之深山。\\
苍梧山崩湘水绝,竹上之泪乃可灭。\\
}

\poetry{相和歌辞 猛虎行}{
朝作猛虎行,暮作猛虎吟。\\
肠断非关陇头水,泪下不为雍门琴。\\
旌旗缤纷两河道,战鼓惊山欲倾倒。\\
秦人半作燕地囚,胡马翻衔洛阳草。\\
一输一失关下兵,朝降夕叛幽蓟城。\\
巨鳌未斩海水动,鱼龙奔走安得宁。\\
颇似楚汉时,翻覆无定止。\\
朝过博浪沙,暮入淮阴市。\\
张良未遇韩信贫,刘项存亡在两臣。\\
暂到下邳受兵略,来投漂母作主人。\\
贤哲栖栖古如此,今时亦弃青云士。\\
有策不敢犯龙鳞,窜身南国避胡尘。\\
宝书长剑挂高阁,金鞍骏马散故人。\\
昨日方为宣城客,掣铃交通二千石。\\
有时六博快壮心,绕床三匝呼一掷。\\
楚人每道张旭奇,心藏风云世莫知。\\
三吴邦伯多顾盼,四海雄侠皆相推。\\
萧曹曾作沛中吏,攀龙附凤当有时。\\
溧阳酒楼三月春,杨花漠漠愁杀人。\\
胡人绿眼吹玉笛,吴歌白纻飞梁尘。\\
丈夫相见且为乐,槌牛挝鼓会众宾。\\
我从此去钓东海,得鱼笑寄情相亲。\\
}

\poetry{公无渡河}{
黄河西来决昆仑,咆哮万里触龙门。\\
波滔天,尧咨嗟,大禹理百川,儿啼不窥家。\\
杀湍烟洪水,九州始蚕麻。\\
其害乃去,茫然风沙。\\
被发之叟狂而痴,清晨临流欲奚为。\\
旁人不惜妻止之,公无渡河苦渡之。\\
虎可搏,河难凭,公果溺死流海湄。\\
有长鲸白齿若雪山,公乎公乎挂罥于其间,箜篌所悲竟不还。\\
}
\poetry{江上吟}{
木兰之枻沙棠舟,玉箫金管坐两头。\\
美酒尊中置千斛,载妓随波任去留。\\
仙人有待乘黄鹤,海客无心随白鸥。\\
屈平词赋悬日月,楚王台榭空山丘。\\
兴酣落笔摇五岳,诗成笑傲凌沧洲。\\
功名富贵若长在,汉水亦应西北流。\\
}