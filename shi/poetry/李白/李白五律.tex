
\subsection{五言律诗}
\poetry{渡荆门送别}{渡远荆门外,来从楚国游。\\
山随平野尽,江入大荒流。\\
月下飞天镜,云生结海楼。\\
仍怜故乡水,万里送行舟。\\
}

\poetry{送友人}{青山横北郭,白水绕东城。\\
此地一为别,孤蓬万里征。\\
浮云游子意,落日故人情。\\
挥手自兹去,萧萧班马鸣。\\
}

\poetry{赠孟浩然}{吾爱孟夫子,风流天下闻。\\
红颜弃轩冕,白首卧松云。\\
醉月频中圣,迷花不事君。\\
高山安可仰,徒此\xpinyin*{揖}清芬。\\
}
\poetry{秋登宣城谢\xpinyin*{眺}北楼}{江城如画里,山晚望晴空\footnote{晚 一作:晓}。\\
两水夹明镜,双桥落彩虹。\\
人烟寒橘柚,秋色老梧桐。\\
谁念北楼上,临风怀谢公。\\
}

\poetry{访戴天山道士不遇}{犬吠水声中,桃花带露浓。\\
树深时见鹿,溪午不闻钟。\\
野竹分青霭,飞泉挂碧峰。\\
无人知所去,愁倚两三松。\\
}

\poetry{夜泊牛\xpinyin*{渚}怀古}{牛渚西江夜,青天无片云。\\
登舟望秋月,空忆谢将军。\\
余亦能高咏,斯人不可闻。\\
明朝挂帆席,枫叶落纷纷。\\
}

\poetry{塞下曲六首·其一}{五月天山雪,无花只有寒。\\
笛中闻折柳,春色未曾看。\\
晓战随金鼓,宵眠抱玉鞍。\\
愿将腰下剑,直为斩楼兰。\\
}

\poetry{听蜀僧浚弹琴}{\footnote{或写成\xpinyin*{\CJKfontspec{KaiTi} 濬},这两个是异体字}蜀僧抱绿绮,西下峨眉峰。\\
为我一挥手,如听万壑松。\\
客心洗流水,余响入霜钟。\\
不觉碧山暮,秋云暗几重。\\
}