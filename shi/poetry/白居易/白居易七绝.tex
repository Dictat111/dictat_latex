\poetry{夜筝}{
紫袖红弦明月中,自弹自感暗低容。\\
弦凝指咽声停处,别有深情一万重。\\
}
\poetry{惜牡丹花二首 一}{
惆怅阶前红牡丹,晚来唯有两枝残。\\
明朝风起应吹尽,夜惜衰红把火看。\\
}
\poetry{村夜}{
霜草苍苍虫切切,村南村北行人绝。\\
独出前门望野田,月明荞麦花如雪。\\
}


\poetry{邯郸冬至夜思家}{
邯郸驿里逢冬至,抱膝灯前影伴身。\\
想得家中夜深坐,还应说着远行人。\\
}
\poetry{大林寺桃花}{
人间四月芳菲尽,山寺桃花始盛开。\\
长恨春归无觅处,不知转入此中来。\\
}

\poetry{春词}{
低花树暎小妆楼,春入眉心两点愁。\\
斜倚栏干背鹦鹉,思量何事不回头。\\
}





\poetry{后宫词}{
泪湿罗巾梦不成,夜深前殿按歌声。\\
红颜未老恩先断,斜倚薰笼坐到明。\\
}




\poetry{白云泉}{
天平山上白云泉,云自无心水自闲。\\
何必奔冲山下去,更添波浪向人间。\\
}

\poetry{暮江吟}{
一道残阳铺水中,半江瑟瑟半江红。\\
可怜九月初三夜,露似真珠月似弓。\\
}
\poetry{杂曲歌辞 浪淘沙 四}{
借问江湖与海水,何似君情与妾心。\\
相恨不如潮有信,相思始觉海非深。\\
}


\poetry{寄湘灵}{
泪眼凌寒冻不流,每经高处即回头。\\
遥知别后西楼上,应凭阑干独自愁。\\
}