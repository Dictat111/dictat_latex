\section{李贺}
\poetry{梦天}{
老兔寒蟾泣天色,云楼半开壁斜白。\\
玉轮轧露湿团光,鸾珮相逢桂香陌。\\
黄尘清水三山下,更变千年如走马。\\
遥望齐州九点烟,一泓海水杯中泻。\\
}
\poetry{苦昼短}{
飞光飞光,劝尔一杯酒。\\
吾不识青天高,黄地厚。\\
唯见月寒日暖,来煎人寿。\\
食熊则肥,食蛙则瘦。\\
神君何在,太一安有。\\
天东有若木,下置衔烛龙。\\
吾将斩龙足,嚼龙肉。\\
使之朝不得回,夜不得伏。\\
自然老者不死,少者不哭。\\
何为服黄金,吞白玉。\\
谁似任公子,云中骑碧驴。\\
刘彻茂陵多滞骨,嬴政梓棺费鲍鱼。\\
}
\poetry{鼓吹曲辞 将进酒}{
瑠璃锺,琥珀浓,小槽酒滴真珠红。\\
烹龙炮凤玉脂泣,罗屏绣幕围香风。\\
吹龙笛,击鼍鼓,皓齿歌,细腰舞。\\
况是青春日将暮,桃花乱落如红雨。\\
劝君终日酩酊醉,酒不到刘伶坟上土。\\
}


\poetry{相和歌辞 雁门太守行}{
黑云压城城欲摧,甲光向月金鳞开。\\
角声满天秋色里,塞上燕支凝夜紫。\\
半卷红旗临易水,霜重鼓寒声不起。\\
报君黄金台上意,提携玉龙为君死。\\
}
\poetry{杂歌谣辞 苏小小歌}{
幽兰露,如啼眼。\\
无物结同心,烟花不堪翦。\\
草如茵,松如盖。\\
风为裳,水为佩。\\
油壁车,久相待。\\
冷翠烛,劳光彩。\\
西陵下,风吹雨。\\
}
\poetry{李凭箜篌引}{
吴丝蜀桐张高秋,空白凝云颓不流。\\
江娥啼竹素女愁,李凭中国弹箜篌。\\
昆山玉碎凤皇叫,芙蓉泣露香兰笑。\\
十二门前融冷光,二十三丝动紫皇。\\
女娲炼石补天处,石破天惊逗秋雨。\\
梦入坤山教神妪,老鱼跳波瘦蛟舞。\\
吴质不眠倚桂树,露脚斜飞湿寒兔。\\
}
\poetry{雁门太守行}{
黑云压城城欲摧,甲光向日金鳞开。\\
角声满天秋色里,塞上燕脂凝夜紫。\\
半卷红旗临易水,霜重鼓寒声不起。\\
报君黄金台上意,提携玉龙为君死。\\
}
\poetry{秦王饮酒}{
秦王骑虎游八极,剑光照空天自碧。\\
羲和敲日玻璃声,劫灰飞尽古今平。\\
龙头泻酒邀酒星,金槽琵琶夜枨枨。\\
洞庭雨脚来吹笙,酒酣喝月使倒行。\\
银云栉栉瑶殿明,宫门掌事报一更。\\
花楼玉凤声娇狞,海绡红文香浅清。\\
黄鹅跌舞千年觥,仙人烛树蜡烟轻,清琴醉眼泪泓泓。\\
}

\poetry {致酒行}{
零落栖迟一杯酒,主人奉觞客长寿。\\
主父西游困不归,家人折断门前柳。\\
吾闻马周昔作新丰客,天荒地老无人识。\\
空将笺上两行书,直犯龙颜请恩泽。\\
我有迷魂招不得,雄鸡一声天下白。\\
少年心事当拏云,谁念幽寒坐呜呃。\\
}
\poetry{南园十三首 五}{
男儿何不带吴钩,收取关山五十州。\\
请君暂上凌烟阁,若个书生万户侯。\\
}
\poetry{南园十三首 六}{
寻章摘句老雕虫,晓月当帘挂玉弓。\\
不见年年辽海上,文章何处哭秋风。\\
}
\poetry{马诗二十三首 四}{
此马非凡马,房星本是星。\\
向前敲瘦骨,犹自带铜声。\\
}
\poetry{马诗二十三首 五}{
大漠山如雪,燕山月似钩。\\
何当金络脑,快走踏清秋。\\
}