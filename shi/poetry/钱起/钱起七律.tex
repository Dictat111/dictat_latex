\poetry{送马明府赴江陵}{
陶令南行心自永,江天极目澄秋景。\\
万室遥方犬不鸣,双凫下处人皆静。\\
清风高兴得湖山,门柳萧条双翟闲。\\
黄花满把应相忆,落日登楼北望还。\\
}
\poetry{卢龙塞行送韦掌记}{
雨雪纷纷黑山外,行人共指卢龙塞。\\
万里飞沙咽鼓鼙,三军杀气凝旌斾。\\
陈琳书记本翩翩,料敌张兵夺酒泉。\\
圣主好文兼好武,封侯莫比汉皇年。\\
}
\poetry{同程九早入中书}{
汉家贤相重英奇,蟠木何材也见知。\\
不意云霄能自致,空惊鹓鹭忽相随。\\
腊雪初明柏子殿,春光欲上万年枝。\\
独惭皇鉴明如日,未厌春光向玉墀。\\
}
\poetry{仲春宴王补阙城东小池}{
王孙兴至幽寻好,芳草春深景气和。\\
药院爱随流水入,山斋喜与白云过。\\
犹嫌巢鹤窥人远,不厌丛花对客多。\\
醉来倚玉无余事,目送归鸿笑复歌。\\
}
\poetry{夜宿灵台寺寄郎士元}{
西日横山含碧空,东方吐月满禅宫。\\
朝瞻双顶青冥上,夜宿诸天色界中。\\
石潭倒献莲花水,塔院空闻松柏风。\\
万里故人能尚尔,知君视听我心同。\\
}
\poetry{题郎士元半日吴村别业兼呈李长官}{
半日吴村带晚霞,闲门高柳乱飞鸦。\\
横云岭外千重树,流水声中一两家。\\
愁人昨夜相思苦,闰月今年春意赊。\\
自叹梅生头似雪,却怜潘令县如花。\\
}
\poetry{猷川雪后送僧粲临还京时避世卧疾}{
连步青溪几万重,有时共立在孤峰。\\
斋到盂空餐雪麦,经传金字坐云松。\\
呻吟独卧猷川水,振锡先闻长乐钟。\\
回望群山携手处,离心一一涕无从。\\
}
\poetry{和李员外扈驾幸温泉宫}{
未央月晓度疏钟,凤辇时巡出九重。\\
雪霁山门迎瑞日,云开水殿候飞龙。\\
经寒不入宫中树,佳气常薰仗外峰。\\
遥羡枚皋扈仙跸,偏承霄汉渥恩浓。\\
}
\poetry{长信怨}{
长信萤来一叶秋,蛾眉泪尽九重幽。\\
𫛛鹊观前明月度,芙蓉阙下绛河流。\\
鸳衾久别难为梦,凤管遥闻更起愁。\\
谁分昭阳夜歌舞,君王玉辇正淹留。\\
}
\poetry{送河南陆少府}{
云间陆生美且奇,银章朱绶暎金羁。\\
自料抱材将致远,宁嗟趋府暂牵卑。\\
东城社日催巢燕,上苑秋声散御梨。\\
朝夕诏书还柏署,行看飞隼集高枝。\\
}
\poetry{送李评事赴潭州使幕}{
湖南远去有余情,蘋叶初齐白芷生。\\
谩说简书催物役,遥知心赏缓王程。\\
兴过山寺先云到,啸引江帆带月行。\\
幕下由来贵无事,伫闻谈笑静黎甿。\\
}
\poetry{送李九贬南阳}{
玉柱金罍醉不欢,云山驿道向东看。\\
鸿声断续暮天远,柳影萧疏秋日寒。\\
霜降幽林霑蕙若,弦惊翰苑失鸳鸾。\\
秋来回首君门阻,马上应歌行路难。\\
}
\poetry{送裴頔侍御使蜀}{
柱史才年四十强,须髯玄发美清扬。\\
朝天绣服乘恩贵,出使星轺满路光。\\
锦水繁花添丽藻,峨嵋明月引飞觞。\\
多才自有云霄望,计日应追鸳鹭行。\\
}
\poetry{送韦信爱子归觐}{
离舟解缆到斜晖,春水东流燕北飞。\\
才子学诗趋露冕,棠花含笑待斑衣。\\
稍闻江树啼猿近,转觉山林过客稀。\\
借问还珠盈合浦,何如鲤也入庭闱。\\
}
\poetry{送兴平王少府游梁}{
旧识相逢情更亲,攀欢甚少怆离频。\\
黄绶罢来多远客,青山何处不愁人。\\
日斜官树闻蝉满,雨过关城见月新。\\
梁国遗风重词赋,诸侯应念马卿贫。\\
}
\poetry{送张员外出牧岳州}{
凤凰衔诏与何人,善政多才宠寇恂。\\
台上鸳鸾争送远,岳阳云树待行春。\\
自怜黄阁知音在,不厌彤幨出守频。\\
应笑冯唐衰且拙,世情相见白头新。\\
}
\poetry{送孙十尉温县}{
飞花落絮满河桥,千里伤心送客遥。\\
不惜芸香染黄绶,惟怜鸿羽下青霄。\\
云衢有志终骧首,吏道无媒且折腰。\\
急管繁弦催一醉,颓阳不驻引征镳。\\
}
\poetry{送锺评事应宏词下第东归}{
芳岁归人嗟转蓬,含情回首灞陵东。\\
蛾眉不入秦台镜,鹢羽还惊宋国风。\\
世事悠扬春梦里,年光寂寞旅愁中。\\
劝君稍尽离筵酒,千里佳期难再同。\\
}
\poetry{送严维尉河南}{
蕙叶青青花乱开,少年趋府下蓬莱。\\
甘泉未献扬雄赋,吏道何劳贾谊才。\\
征陌独愁飞盖远,离筵只惜暝钟催。\\
欲知别后相思处,愿植琼枝向柏台。\\
}
\poetry{送马员外拜官觐省}{
二十为郎事汉文,鸳雏骥子自为群。\\
笔精已许台中妙,剑术还令世上闻。\\
归觐屡经槐里月,出师常笑棘门军。\\
莫言来往朝天远,看取鸣鞘入断云。\\
}
\poetry{送冷朝阳擢第后归金陵觐省}{
莱子昼归今始好,潘园景色夏偏浓。\\
夕阳流水吟诗去,明月青山出竹逢。\\
兄弟相欢初让果,乡人争贺旧登龙。\\
佳期少别俄千里,云树愁看过几重。\\
}
\poetry{九日宴浙江西亭}{
诗人九日怜芳菊,筵客高斋宴浙江。\\
渔浦浪花摇素壁,西陵树色入秋窗。\\
木奴向熟悬金实,桑落新开泻玉缸。\\
四子醉时争讲习,笑论黄霸旧为邦。\\
}
\poetry{和王员外雪晴早朝}{
紫微晴雪带恩光,绕仗偏随鸳鹭行。\\
长信月留宁避晓,宜春花满不飞香。\\
独看积素凝清禁,已觉轻寒让太阳。\\
题柱盛名兼绝唱,风流谁继汉田郎。\\
}
\poetry{避暑纳凉}{
木槿花开畏日长,时摇轻扇倚绳床。\\
初晴草蔓缘新笋,频雨苔衣染旧墙。\\
十旬河朔应虚醉,八柱天台好纳凉。\\
无事始然知静胜,深垂纱帐咏沧浪。\\
}
\poetry{早夏}{
楚狂身世恨情多,似病如忧正是魔。\\
花萼败春多寂寞,叶阴迎夏已清和。\\
鹂黄好鸟摇深树,细白佳人著紫罗。\\
军旅阅诗裁不得,可怜风景遣如何。\\
}
\poetry{题嵩阳焦道士石壁}{
三峰花畔碧堂悬,锦里真人此得仙。\\
玉体才飞西蜀雨,霓裳欲向大罗天。\\
彩云不散烧丹灶,白鹿时藏种玉田。\\
幸入桃源因去世,方期丹诀一延年。\\
}
\poetry{题延州圣僧穴}{
定力无涯不可称,未知何代坐禅僧。\\
默默山门宵闭月,荧荧石壁昼然灯。\\
四时树长书经叶,万岁岩悬拄杖藤。\\
昔日舍身缘救鸽,今时出见有飞鹰。\\
}
\poetry{乐游原晴望上中书李侍郎}{
爽气朝来万里清,凭高一望九秋轻。\\
不知凤沼霖初霁,但觉尧天日转明。\\
四野山河通远色,千家砧杵共秋声。\\
遥想青云丞相府,何时开合引书生。\\
}
\poetry{幽居春暮书怀}{
自哂鄙夫多野性,贫居数亩半临湍。\\
谿云杂雨来茅屋,山雀将雏到药栏。\\
仙箓满床闲不厌,阴符在箧老羞看。\\
更怜童子宜春服,花里寻师指杏坛。\\
}
\poetry{谒许由庙}{
故向箕山访许由,林泉物外自清幽。\\
松上挂瓢枝几变,石间洗耳水空流。\\
绿苔唯见遮三径,青史空传谢九州。\\
缅想古人增叹惜,飒然云树满岩秋。\\
}
\poetry{过张成侍御宅}{
丞相幕中题凤人,文章心事每相亲。\\
从军谁谓仲宣乐,入室方知颜子贫。\\
杯里紫茶香代酒,琴中绿水静留宾。\\
欲知别后相思意,唯愿琼枝入梦频。\\
}
\poetry{酬考功杨员外见赠佳句}{
上林谏猎知才薄,尺组承恩愧命牵。\\
潢潦难滋沧海润,萤光空尽太阳前。\\
虚名滥接登龙士,野性宁忘种黍田。\\
相国无私人守朴,何辞老去上皇年。\\
}
\poetry{寄永嘉王十二}{
永嘉风景入新年,才子诗成定可怜。\\
梦里还乡不相见,天涯忆戴复谁传。\\
花倾晓露垂如泪,莺拂游丝断若弦。\\
愿得回风吹海雁,飞书一宿到君边。\\
}
\poetry{七盘岭阻寇闻李端公先到南楚}{
日暮穷途泪满襟,云天南望羡飞禽。\\
阮肠暗与孤鸿断,江水遥连别恨深。\\
明月既能通忆梦,青山何用隔同心。\\
秦楚眼看成绝国,相思一寄白头吟。\\
}
\poetry{酬赵给事相寻不遇留赠}{
谁忆颜生穷巷里,能劳马迹破春苔。\\
忽看童子扫花处,始愧夕郎题凤来。\\
斜景适随诗兴尽,好风才送珮声回。\\
岂无鸡黍期他日,惜此残春阻绿杯。\\
}
\poetry{山中酬杨补阙见过}{
日暖风恬种药时,红泉翠壁薜萝垂。\\
幽溪鹿过苔还静,深树云来鸟不知。\\
青琐同心多逸兴,春山载酒远相随。\\
却惭身外牵缨冕,未胜杯前倒接{罒/离}。\\
}
\poetry{同王𫓾起居程浩郎中韩翃舍人题安国寺用上人院}{
慧眼沙门真远公,经行宴坐有儒风。\\
香缘不绝簪裾会,禅想宁妨藻思通。\\
曙后𬬻烟生不灭,晴来堦色并归空。\\
狂夫入室无余事,唯与天花一笑同。\\
}
\poetry{寻司勋李郎中不遇}{
知己知音同舍郎,如何咫尺阻清扬。\\
每恨蒹葭傍芳树,多惭新燕入华堂。\\
重花不隔陈蕃榻,修竹能深夫子墙。\\
唯有早朝趋凤合,朝时怜羽接鸳行。\\
}
\poetry{赠张南史}{
紫泥何日到沧州,笑向东阳沈隐侯。\\
黛色晴峰云外出,縠文江水县前流。\\
使臣自欲论公道,才子非关厌薄游。\\
溪畔秋兰虽可佩,知君不得少停舟。\\
}
\poetry{暇日览旧诗因以题咏}{
逍遥心地得关关,偶被功名涴我闲。\\
有寿亦将归象外,无诗兼不恋人间。\\
何穷默识轻洪范,未丧斯文胜大还。\\
筐箧静开难似此,蕊珠春色海中山。\\
}
\poetry{汉武出猎}{
汉家无事乐时雍,羽猎年年出九重。\\
玉帛不朝金阙路,旌旗长绕彩霞峰。\\
且贪原兽轻黄屋,宁畏渔人犯白龙。\\
薄暮方归长乐观,垂杨几处绿烟浓。\\
}
\poetry{宴曹王宅}{
贤王驷马退朝初,小苑三春带雨余。\\
林沼葱茏多贵气,楼台隐暎接天居。\\
仙鸡引敌穿红药,宫燕衔泥落绮疏。\\
自叹平生相识愿,何如今日厕应徐。\\
}
\poetry{重赠赵给事}{
久飞鸳掖出时髦,耻负平生稽古劳。\\
玉树满庭家转贵,云衢独步位初高。\\
能迂驺驭寻蜗舍,不惜瑶华报木桃。\\
应念潜郎守贫病,常悲休沐对蓬蒿。\\
}
\poetry{赠阙下裴舍人}{
二月黄莺飞上林,春城紫禁晓阴阴。\\
长乐钟声花外尽,龙池柳色雨中深。\\
阳和不散穷途恨,霄汉长怀捧日新。\\
献赋十年犹未遇,羞将白发对华簪。\\
}
\poetry{登刘宾客高斋}{
能以功成疏宠位,不将心赏负云霞。\\
林间客散孙弘阁,城上山宜绮季家。\\
蝴蝶晴连池岸草,黄鹂晚出柳园花。\\
日陪鲤也趋文苑,谁道门生隔绛纱。\\
}
\poetry{哭辛霁}{
流水辞山花别枝,随风一去绝还期。\\
昨夜故人泉下宿,今朝白发镜中垂。\\
音徽寂寂空成梦,容范朝朝无见时。\\
旦暮余生几息在,不应存没未尝悲。\\
}
\poetry{和慕容法曹寻渔者寄城中故人}{
孤烟一点绿溪湄,渔父幽居即旧基。\\
饥鹭不惊收钓处,闲麛应乳负暄时。\\
茅斋对雪开尊好,穉子焚枯饭客迟。\\
胜事宛然怀抱里,顷来新得谢公诗。\\
}
\poetry{山花}{
山花照坞复烧溪,树树枝枝尽可迷。\\
野客未来枝畔立,流莺已向树边啼。\\
从容只是愁风起,眷恋常须向日西。\\
别有妖妍胜桃李,攀来折去亦成蹊。\\
}
\poetry{芝草(题拟)}{
岂如玉殿生三秀,讵有铜池出五云。\\
陌上尧樽倾北斗,楼前舜乐动南薰。\\
(《全芳备祖后集》卷十一《芝草》。\\
此四句为“七言律诗散联”)。\\
}
