\poetry{江行无题一百首 一}{
倾酒向涟漪,乘流东去时。\\
寸心同尺璧,投此报冯夷。\\
}
\poetry{江行无题一百首 二}{
江曲全萦楚,雪飞半自秦。\\
岘山回首望,如别故关人。\\
}
\poetry{江行无题一百首 三}{
浦烟函夜色,冷日转秋旻。\\
自有沈碑石,清光不照人。\\
}
\poetry{江行无题一百首 四}{
楚岸云空合,楚城人不来。\\
只今谁善舞,莫恨发阳台。\\
}
\poetry{江行无题一百首 五}{
行背青山郭,吟当白露秋。\\
风流无屈宋,空咏古荆州。\\
}
\poetry{江行无题一百首 六}{
晚来渔父喜,罾重欲收迟。\\
恐有长江使,金钱愿赎龟。\\
}
\poetry{江行无题一百首 七}{
去指龙沙路,徒悬象阙心。\\
夜凉无远梦,不为偶闻砧。\\
}
\poetry{江行无题一百首 八}{
霁云疏有叶,雨浪细无花。\\
稳放扁舟去,江天自有涯。\\
}
\poetry{江行无题一百首 九}{
好日当秋半,层波动旅肠。\\
已行千里外,谁与共秋光。\\
}
\poetry{江行无题一百首 十}{
润色非东里,官曹更建章。\\
宦游难自定,来唤櫂船郎。\\
}
\poetry{江行无题一百首 十一}{
夜江清未晓,徒惜月光沈。\\
不是因行乐,堪伤老大心。\\
}
\poetry{江行无题一百首 十二}{
翳日多乔木,维舟取束薪。\\
静听江叟语,俱是厌兵人。\\
}
\poetry{江行无题一百首 十三}{
箭漏日初短,汀烟草未衰。\\
雨余虽更绿,不是采蘋时。\\
}
\poetry{江行无题一百首 十四}{
山雨夜来涨,喜鱼跳满江。\\
岸沙平欲尽,垂蓼入船窗。\\
}
\poetry{江行无题一百首 十五}{
渚边新雁下,舟上独凄凉。\\
俱是南来客,怜君缀一行。\\
}
\poetry{江行无题一百首 十六}{
牵路沿江狭,沙崩岸不平。\\
尽知行处险,谁肯载时轻。\\
}
\poetry{江行无题一百首 十七}{
云密连江暗,风斜著物鸣。\\
一杯真战将,笑尔作愁兵。\\
}
\poetry{江行无题一百首 十八}{
柳拂斜开路,篱边数户村。\\
可能还有意,不掩向江门。\\
}
\poetry{江行无题一百首 十九}{
不识桓公渴,徒吟子美诗。\\
江清唯独看,心外更谁知。\\
}
\poetry{江行无题一百首 二十}{
憔悴异灵均,非谗作逐臣。\\
如逢渔父问,未是独醒人。\\
}
\poetry{江行无题一百首 二十一}{
水涵秋色静,云带夕阳高。\\
诗癖非吾病,何妨吮短毫。\\
}
\poetry{江行无题一百首 二十二}{
登舟非古岸,还似阻西陵。\\
箕伯无多少,回头讵不能。\\
}
\poetry{江行无题一百首 二十三}{
帆翅初张处,云鹏怒翼同。\\
莫愁千里路,自有到来风。\\
}
\poetry{江行无题一百首 二十四}{
秋云久无雨,江燕社犹飞。\\
却笑舟中客,今年未得归。\\
}
\poetry{江行无题一百首 二十五}{
佳节虽逢菊,浮生正似萍。\\
故山何处望,荒岸小长亭。\\
}
\poetry{江行无题一百首 二十六}{
行到楚江岸,苍茫人正迷。\\
只知秦塞远,格磔鹧鸪啼。\\
}
\poetry{江行无题一百首 二十七}{
月下江流静,村荒人语稀。\\
鹭鸶虽有伴,仍共影双飞。\\
}
\poetry{江行无题一百首 二十八}{
斗转月未落,舟行夜已深。\\
有村知不远,风便数声砧。\\
}
\poetry{江行无题一百首 二十九}{
櫂惊沙鸟迅,飞溅夕阳波。\\
不顾鱼多处,应防一目罗。\\
}
\poetry{江行无题一百首 三十}{
渐觉江天远,难逢故国书。\\
可能无往事,空食鼎中鱼。\\
}
\poetry{江行无题一百首 三十一}{
岸草连荒色,村声乐稔年。\\
晚晴初获稻,闲却采莲船。\\
}
\poetry{江行无题一百首 三十二}{
滩浅争游鹭,江清易见鱼。\\
怪来吟未足,秋物欠红蕖。\\
}
\poetry{江行无题一百首 三十三}{
蛩响依莎草,萤飞透水烟。\\
夜凉谁咏史,空泊运租船。\\
}
\poetry{江行无题一百首 三十四}{
睡稳叶舟轻,风微浪不惊。\\
任君芦苇岸,终夜动秋声。\\
}
\poetry{江行无题一百首 三十五}{
自念平生意,曾期一郡符。\\
岂知因谪宦,斑鬓入江湖。\\
}
\poetry{江行无题一百首 三十六}{
烟渚复烟渚,画屏休画屏。\\
引愁天末去,数点暮山青。\\
}
\poetry{江行无题一百首 三十七}{
水天凉夜月,不是惜清光。\\
好物随人秘,秦淮忆建康。\\
}
\poetry{江行无题一百首 三十八}{
古来多思客,摇落恨江潭。\\
今日秋风至,萧疏独沔南。\\
}
\poetry{江行无题一百首 三十九}{
暎竹疑村好,穿芦觉渚幽。\\
渐安无旷土,姜芋当农收。\\
}
\poetry{江行无题一百首 四十}{
秋风动客心,寂寂不成吟。\\
飞上危樯立,啼乌报好音。\\
}
\poetry{江行无题一百首 四十一}{
见底高秋水,开怀万里天。\\
旅吟还有伴,沙柳数枝蝉。\\
}
\poetry{江行无题一百首 四十二}{
九日自佳节,扁舟无一杯。\\
曹园旧尊酒,戏马忆高台。\\
}
\poetry{江行无题一百首 四十三}{
兵火有余烬,贫村才数家。\\
无人争晓渡,残月下寒沙。\\
}
\poetry{江行无题一百首 四十四}{
渚禽菱芡足,不向稻粱争。\\
静宿凉湾月,应无失侣声。\\
}
\poetry{江行无题一百首 四十五}{
轻云未护霜,树杪橘初黄。\\
信是知名物,微风过水香。\\
}
\poetry{江行无题一百首 四十六}{
渺渺望天涯,清涟浸赤霞。\\
难逢星汉使,乌鹊日乘槎。\\
}
\poetry{江行无题一百首 四十七}{
土旷深耕少,江平远钓多。\\
生平皆弃本,金革竟如何。\\
}
\poetry{江行无题一百首 四十八}{
海月非常物,等闲不可寻。\\
披沙应有地,浅处定无金。\\
}
\poetry{江行无题一百首 四十九}{
风晚冷飕飕,芦花已白头。\\
旧来红叶寺,堪忆玉京秋。\\
}
\poetry{江行无题一百首 五十}{
风好来无阵,云闲去有踪。\\
钓歌无远近,应喜罢艨艟。\\
}
\poetry{江行无题一百首 五十一}{
吴疆连楚甸,楚俗异吴乡。\\
漫把尊中物,无人啄蟹筐。\\
}
\poetry{江行无题一百首 五十二}{
岸绿野烟远,江红斜照微。\\
撑开小渔艇,应到月明归。\\
}
\poetry{江行无题一百首 五十三}{
雨余江始涨,漾漾见流薪。\\
曾叹河中木,斯言忆古人。\\
}
\poetry{江行无题一百首 五十四}{
叶舟维夏口,烟野独行时。\\
不见头陀寺,空怀幼妇碑。\\
}
\poetry{江行无题一百首 五十五}{
晚泊武昌岸,津亭疏柳风。\\
数株曾手植,好事忆陶公。\\
}
\poetry{江行无题一百首 五十六}{
坠露晓犹浓,秋花不易逢。\\
涉江虽已晚,高树搴芙蓉。\\
}
\poetry{江行无题一百首 五十七}{
舟航依浦定,星斗满江寒。\\
若比阴霾日,何妨夜未阑。\\
}
\poetry{江行无题一百首 五十八}{
近戍离金落,孤岑望火门。\\
唯将知命意,潇洒向乾坤。\\
}
\poetry{江行无题一百首 五十九}{
丛菊生堤上,此花长后时。\\
有人还采掇,何必在春期。\\
}
\poetry{江行无题一百首 六十}{
夕景残霞落,秋寒细雨晴。\\
短缨何用濯,舟在月中行。\\
}
\poetry{江行无题一百首 六十一}{
堤坏漏江水,地坳成野塘。\\
晚荷人不折,留取作秋香。\\
}
\poetry{江行无题一百首 六十二}{
左宦终何路,摅怀亦自宽。\\
襞笺嘲白鹭,无意喻枭鸾。\\
}
\poetry{江行无题一百首 六十三}{
楼空人不归,云似去时衣。\\
黄鹤无心下,长应笑令威。\\
}
\poetry{江行无题一百首 六十四}{
白帝朝惊浪,浔阳暮暎云。\\
等闲生险易,世路只如君。\\
}
\poetry{江行无题一百首 六十五}{
橹慢开轻浪,帆虚带白云。\\
客船虽狭小,容得庾将军。\\
}
\poetry{江行无题一百首 六十六}{
风雨正甘寝,云霓忽晚晴。\\
放歌虽自遣,一岁又峥嵘。\\
}
\poetry{江行无题一百首 六十七}{
静看秋江水,风微浪渐平。\\
人间驰竞处,尘土自波成。\\
}
\poetry{江行无题一百首 六十八}{
风劲帆方疾,风回櫂却迟。\\
较量人世事,不校一毫厘。\\
}
\poetry{江行无题一百首 六十九}{
咫尺愁风雨,匡庐不可登。\\
只疑云雾窟,犹有六朝僧。\\
}
\poetry{江行无题一百首 七十}{
幽思正迟迟,沙边濯弄时。\\
自怜非博物,犹未识凫葵。\\
}
\poetry{江行无题一百首 七十一}{
曾有烟波客,能歌西塞山。\\
落帆唯待月,一钓紫菱湾。\\
}
\poetry{江行无题一百首 七十二}{
千顷水纹细,一拳岚影孤。\\
君山寒树绿,曾过洞庭湖。\\
}
\poetry{江行无题一百首 七十三}{
光阔重湖水,低斜远雁行。\\
未曾无兴咏,多谢沈东阳。\\
}
\poetry{江行无题一百首 七十四}{
晚菊绕江垒,忽如开古屏。\\
莫言时节过,白日有余馨。\\
}
\poetry{江行无题一百首 七十五}{
秋寒鹰隼健,逐雀下云空。\\
知是江湖阔,无心击塞鸿。\\
}
\poetry{江行无题一百首 七十六}{
日落长亭晚,山门步障青。\\
可怜无酒分,处处有旗亭。\\
}
\poetry{江行无题一百首 七十七}{
江草何多思,冬青尚满洲。\\
谁能惊𫛳鸟,作赋为沙鸥。\\
}
\poetry{江行无题一百首 七十八}{
远岸无行树,经霜有半红。\\
停船搜好句,题叶赠江枫。\\
}
\poetry{江行无题一百首 七十九}{
身世比行舟,无风亦暂休。\\
敢言终破浪,唯愿稳乘流。\\
}
\poetry{江行无题一百首 八十}{
数亩苍苔石,烟蒙鹤卵洲。\\
定因词客遇,名字始风流。\\
}
\poetry{江行无题一百首 八十一}{
兴闲停桂楫,路好过松门。\\
不负佳山水,还开酒一尊。\\
}
\poetry{江行无题一百首 八十二}{
幽怀念烟水,长恨隔龙沙。\\
今日膝王阁,分明见落霞。\\
}
\poetry{江行无题一百首 八十三}{
短楫休敲桂,孤根自驻萍。\\
自怜非剑气,空向斗牛星。\\
}
\poetry{江行无题一百首 八十四}{
江流何渺渺,怀古独依依。\\
渔父非贤者,芦中但有矶。\\
}
\poetry{江行无题一百首 八十五}{
高浪如银屋,江风一发时。\\
笔端降太白,才大语终奇。\\
}
\poetry{江行无题一百首 八十六}{
细竹渔家路,晴阳看结缯。\\
喜来邀客坐,分与折腰菱。\\
}
\poetry{江行无题一百首 八十七}{
幸有烟波兴,宁辞笔砚劳。\\
缘情无怨刺,却似反离骚。\\
}
\poetry{江行无题一百首 八十八}{
平湖五百里,江水想通波。\\
不奈扁舟去,其如决计何。\\
}
\poetry{江行无题一百首 八十九}{
数峰云断处,去岸暎高山。\\
身到韦江日,犹应未得闲。\\
}
\poetry{江行无题一百首 九十}{
一湾斜照水,三版顺风船。\\
未敢相邀约,劳生只自怜。\\
}
\poetry{江行无题一百首 九十一}{
江雨正霏微,江村晚渡稀。\\
何曾妨钓艇,更待得鱼归。\\
}
\poetry{江行无题一百首 九十二}{
沙上独行时,高吟到楚词。\\
难将垂岸蓼,盈把当江蓠。\\
}
\poetry{江行无题一百首 九十三}{
新野旧楼名,浔阳胜赏情。\\
照人长一色,江月共凄清。\\
}
\poetry{江行无题一百首 九十四}{
愿饮西江水,那吟北渚愁。\\
莫教留滞迹,远比蔡昭侯。\\
}
\poetry{江行无题一百首 九十五}{
湖口分江水,东流独有情。\\
当时好风物,谁伴谢宣城。\\
}
\poetry{江行无题一百首 九十六}{
浔阳江畔菊,应似古来秋。\\
为问幽栖客,吟时得酒不。\\
}
\poetry{江行无题一百首 九十七}{
高峰有佳号,千尺倚寒松。\\
若使𬬻烟在,犹应为上公。\\
}
\poetry{江行无题一百首 九十八}{
万木已清霜,江边村事忙。\\
故溪黄稻熟,一夜梦中香。\\
}
\poetry{江行无题一百首 九十九}{
楚水苦萦回,征帆落又开。\\
可缘非直路,却有好风来。\\
}
\poetry{江行无题一百首 一百}{
远谪岁时晏,暮江风雨寒。\\
仍愁系舟处,惊梦近长滩。\\
}
\poetry{言怀}{
夜月霁未好,云泉堪梦归。\\
如何建章漏,催著早朝衣。\\
}
\poetry{和张仆射塞下曲}{
月黑雁飞高,单于夜遁逃。\\
欲将轻骑逐,大雪满弓刀。\\
}
\poetry{送李明府去官}{
谤言三至后,直道叹何如。\\
今日蓝溪水,无人不夜鱼。\\
}
\poetry{赴章陵酬李卿赠别}{
一官叨下秩,九棘谢知音。\\
芳草文园路,春愁满别心。\\
}
\poetry{逢侠者}{
燕赵悲歌士,相逢剧孟家。\\
寸心言不尽,前路日将斜。\\
}
\poetry{郎员外见寻不遇}{
轩骑来相访,渔樵悔晚归。\\
更怜垂露迹,花里点墙衣。\\
}
\poetry{过李侍御宅}{
不见承明客,愁闻长乐钟。\\
马卿何早世,汉主欲登封。\\
}
\poetry{宿洞口馆}{
野竹通溪冷,秋泉入户鸣。\\
乱来人不到,芳草上堦生。\\
}
\poetry{九日寄侄箊箕等}{
采菊偏相忆,传香寄便风。\\
今朝竹林下,莫使桂尊空。\\
}
\poetry{梨花}{
艳静如笼月,香寒未逐风。\\
桃花徒照地,终被笑妖红。\\
}
\poetry{题崔逸人山亭}{
药径深红藓,山窗满翠微。\\
羡君花下酒,蝴蝶梦中飞。\\
}
\poetry{蓝田溪杂咏二十二首 登台}{
望山登春台,目尽趣难极。\\
晚景下平阡,花际霞峰色。\\
}
\poetry{蓝田溪杂咏二十二首 板桥}{
静宜樵隐度,远与车马隔。\\
有时行药来,喜遇归山客。\\
}
\poetry{蓝田溪杂咏二十二首 石井}{
片霞照仙井,泉底桃花红。\\
那知幽石下,不与武陵通。\\
}
\poetry{蓝田溪杂咏二十二首 古藤}{
引蔓出云树,垂纶覆巢鹤。\\
幽人对酒时,苔上闲花落。\\
}
\poetry{蓝田溪杂咏二十二首 晚归鹭}{
池上静难厌,云间欲去晚。\\
忽背夕阳飞,乘兴清风远。\\
}
\poetry{蓝田溪杂咏二十二首 洞仙谣}{
几转到青山,数重度流水。\\
秦人入云去,知向桃源里。\\
}
\poetry{蓝田溪杂咏二十二首 药圃}{
春畦生百药,花叶香初霁。\\
好容似风光,偏来入丛蕙。\\
}
\poetry{蓝田溪杂咏二十二首 石上苔}{
净与溪色连,幽宜松雨滴。\\
谁知古石上,不染世人迹。\\
}
\poetry{蓝田溪杂咏二十二首 窗里山}{
远岫见如近,千里一窗里。\\
坐来石上云,乍谓壶中起。\\
}
\poetry{蓝田溪杂咏二十二首 竹间路}{
暗归草堂静,半入花园去。\\
有时载酒来,不与清风遇。\\
}
\poetry{蓝田溪杂咏二十二首 竹屿}{
幽鸟清涟上,兴来看不足。\\
新篁压水低,昨夜鸳鸯宿。\\
}
\poetry{蓝田溪杂咏二十二首 砌下泉}{
穿云来自远,激砌流偏驶。\\
能资庭户幽,更引海禽至。\\
}
\poetry{蓝田溪杂咏二十二首 戏鸥}{
乍依菱蔓聚,尽向芦花灭。\\
更喜好风来,数片翻晴雪。\\
}
\poetry{蓝田溪杂咏二十二首 远山钟}{
风送出山钟,云霞度水浅。\\
欲知声尽处,鸟灭寥天远。\\
}
\poetry{蓝田溪杂咏二十二首 东陂}{
永日兴难忘,掇芳春陂曲。\\
新晴花枝下,爱此苔水绿。\\
}
\poetry{蓝田溪杂咏二十二首 池上亭}{
临池构杏梁,待客归烟塘。\\
水上褰帘好,莲开杜若香。\\
}
\poetry{蓝田溪杂咏二十二首 衔鱼翠鸟}{
有意莲叶间,瞥然下高树。\\
擘波得潜鱼,一点翠光去。\\
}
\poetry{蓝田溪杂咏二十二首 石莲花}{
幽石生芙蓉,百花惭美色。\\
远笑越溪女,闻芳不可识。\\
}
\poetry{蓝田溪杂咏二十二首 潺湲声}{
乱石跳素波,寒声闻几处。\\
飕飕暝风引,散出空林去。\\
}
\poetry{蓝田溪杂咏二十二首 松下雪}{
虽因朔风至,不向瑶台侧。\\
唯助苦寒松,偏明后雕色。\\
}
\poetry{蓝田溪杂咏二十二首 田鹤}{
田鹤望碧霄,无风亦自举。\\
单飞后片云,早晚及前侣。\\
}
\poetry{蓝田溪杂咏二十二首 题南陂}{
家住凤城南,门临古陂曲。\\
时怜上林雁,半入池塘宿。\\
}
\poetry{伤秋}{
岁去人头白,秋来树叶黄。\\
搔头向黄叶,与尔共悲伤。\\
}
\poetry{失题}{
胡风迎马首,汉月学蛾眉。\\
久戍人将老,长征马不肥。\\
}
