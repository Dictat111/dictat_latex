\poetry{饮酒·其五}{
结庐在人境,而无车马喧。\\
问君何能尔?心远地自偏。\\
采菊东篱下,悠然见南山。\\
山气日夕佳,飞鸟相与还。\\
此中有真意,欲辨已忘言。\\
}

\poetry{归园田居·其一}{
少无适俗韵,性本爱丘山。\\
误落尘网中,一去三十年。\\
羁鸟恋旧林,池鱼思故渊。\\
开荒南野际,守拙归园田。\\
方宅十余亩,草屋八九间。\\
榆柳荫后檐,桃李罗堂前。\\
暧暧远人村,依依墟里烟。\\
狗吠深巷中,鸡鸣桑树颠。\\
户庭无尘杂,虚室有余闲。\\
久在樊笼里,复得返自然。\\
}

\poetry {归园田居・其三}{
种豆南山下,草盛豆苗稀。\\
晨兴理荒秽,带月荷锄归。\\
道狭草木长,夕露沾我衣。\\
衣沾不足惜,但使愿无违。\\
}

\poetry {读山海经・其十}{
精卫衔微木,将以填沧海。\\
刑天舞干戚,猛志固常在。\\
同物既无虑,化去不复悔。\\
徒设在昔心,良辰讵可待?\\
}