\documentclass{ctexart}
\usepackage{amsmath,}
\usepackage{tcolorbox}
\usepackage[fontsize=12pt]{fontsize}
\usepackage{fancyhdr}
\pagestyle{fancy}
\fancyhf{} % 清除当前头部和脚部的所有内容
\renewcommand{\headrulewidth}{0pt} % 去掉上边线
% \cfoot{\large\thepage} % 将页码置于中央位置

\usepackage{xeCJK} % 用于处理中文排版
\setCJKmainfont[AutoFakeBold=true]{STKaiti} % 设置中文主字体为华文楷体 %AutoFakeBold 添加加粗的效果(自动伪粗体),能使用 \textbf了
\usepackage{fontspec}
% 设置英文主字体为 Times New Roman
\setmainfont{Times New Roman}
\usepackage{graphicx}
% 定义定理环境
\newtheorem{definition}{定义}
\newtheorem{theorem}{定理} % 定义定理环境
\newtheorem{proof}{证} % 定义定理环境
\DeclareMathOperator{\Orb}{Orb} % 定义轨道算子
\DeclareMathOperator{\Fix}{Fix}
\begin{document}

\begin{definition}[群作用]
    设 \( G \) 是一个群,\( X \) 是一个集合。群 \( G \) 在集合 \( X \) 上的一个\textbf{左作用}是一个映射
    \[
    \cdot \colon G \times X \to X, \quad (g, x) \mapsto g \cdot x,
    \]
    满足以下两个条件:
    \begin{enumerate}
        \item \textbf{单位元作用}:对于所有 \( x \in X \),有 \( e \cdot x = x \),其中 \( e \) 是群 \( G \) 的单位元。
        \item \textbf{结合律}:对于所有 \( g, h \in G \) 和 \( x \in X \),有 \( g \cdot (h \cdot x) = (gh) \cdot x \)。
    \end{enumerate}
    类似地,群 \( G \) 在集合 \( X \) 上的一个\textbf{右作用}是一个映射
    \[
    \cdot \colon X \times G \to X, \quad (x, g) \mapsto x \cdot g,
    \]
    满足:
    \begin{enumerate}
        \item 对于所有 \( x \in X \),有 \( x \cdot e = x \)。
        \item 对于所有 \( g, h \in G \) 和 \( x \in X \),有 \( (x \cdot g) \cdot h = x \cdot (gh) \)。
    \end{enumerate}
\end{definition}


\begin{definition}[二面体群 \( D_4 \)]
    二面体群 \( D_4 \) 是正方形的对称群,包含8个元素。它可以通过以下两种方式定义:
    
    \textbf{几何定义:}
    \[
    D_4 = \{ e, r, r^2, r^3, s, sr, sr^2, sr^3 \}
    \]
    其中:
    \begin{itemize}
        \item \( e \) 是恒等变换,
        \item \( r \) 是逆时针旋转90度,满足 \( r^4 = e \),
        \item \( s \) 是关于某条对称轴的反射,满足 \( s^2 = e \),
        \item 反射与旋转满足关系 \( sr = r^{-1}s \)。
    \end{itemize}
    
    \textbf{生成元与关系定义:}
    \[
    D_4 = \langle r, s \mid r^4 = s^2 = e, sr = r^{-1}s \rangle
    \]
    \end{definition}

    \begin{definition}[群作用下的轨道]

        对于元素 \( x \in X \),其在群 \( G \) 作用下的轨道定义为:
        \[
        \Orb_G(x) = \{ g \cdot x \mid g \in G \}
        \]
    \end{definition}

    \begin{theorem}[轨道-稳定化子定理]
        设有限群 \( G \) 作用于集合 \( X \),对任意 \( x \in X \),其轨道 \( \Orb_G(x) \) 和稳定化子群 \( G_x \) 满足:
        \[
        |G| = |\Orb_G(x)| \cdot |G_x|
        \]
        其中稳定化子群 \( G_x = \{ g \in G \mid g \cdot x = x \} \)。
    \end{theorem}

    

\begin{proof}
我们通过构造双射来建立群基数的关系。


定义左陪集空间到轨道的映射:
\[
\phi: G / G_x \to \Orb_G(x), \quad \phi(gG_x) = g \cdot x
\]
其中 \( G / G_x \) 为 \( G_x \) 在 \( G \) 中的左陪集族。


设 \( g_1 G_x = g_2 G_x \),则存在 \( h \in G_x \) 使得 \( g_2 = g_1 h \)。  
由稳定化子群定义,\( h \cdot x = x \),故:
\[
g_2 \cdot x = (g_1 h) \cdot x = g_1 \cdot (h \cdot x) = g_1 \cdot x
\]
因此 \( \phi(g_1 G_x) = \phi(g_2 G_x) \),映射 \( \phi \) 良定。


若 \( \phi(g_1 G_x) = \phi(g_2 G_x) \),即 \( g_1 \cdot x = g_2 \cdot x \),  
左乘 \( g_1^{-1} \) 得:
\[
g_1^{-1} g_2 \cdot x = x \implies g_1^{-1} g_2 \in G_x \implies g_1 G_x = g_2 G_x
\]
故 \( \phi \) 为单射。

对任意 \( y \in \Orb_G(x) \),存在 \( g \in G \) 使得 \( g \cdot x = y \),  
显然 \( \phi(g G_x) = y \),故 \( \phi \) 为满射。

由双射 \( \phi \) 知:
\[
|G / G_x| = |\Orb_G(x)|
\]
根据拉格朗日定理:
\[
|G / G_x| = \frac{|G|}{|G_x|}
\]
联立得:
\[
\frac{|G|}{|G_x|} = |\Orb_G(x)| \implies |G| = |\Orb_G(x)| \cdot |G_x|
\]
定理得证。
\end{proof}

\begin{theorem}[Burnside引理]
    设有限群 \( G \) 作用于有限集合 \( X \),则轨道数 \( N \) 满足:
    \[
    N = \frac{1}{|G|} \sum_{g \in G} |\Fix(g)|
    \]
    其中 \( \Fix(g) = \{ x \in X \mid g \cdot x = x \} \) 表示元素 \( g \) 的不动点集。
    \end{theorem}



    \begin{proof}
        我们通过双重计数和轨道-稳定子定理来证明。
        
        \subsection*{步骤1:建立双重计数等式}
        考虑集合 \( S = \{ (g, x) \in G \times X \mid g \cdot x = x \} \)。我们可以从两个角度计算其基数:
        
        \subsubsection*{角度1:按群元素划分}
        对于每个 \( g \in G \),满足 \( g \cdot x = x \) 的 \( x \) 的数量为 \( |\Fix(g)| \),因此:
        \[
        |S| = \sum_{g \in G} |\Fix(g)|
        \]
        
        \subsubsection*{角度2:按集合元素划分}
        对于每个 \( x \in X \),满足 \( g \cdot x = x \) 的 \( g \) 的数量为稳定子群 \( G_x \) 的基数,因此:
        \[
        |S| = \sum_{x \in X} |G_x|
        \]
        
        联立两个角度的计数结果,得到:
        \[
        \sum_{g \in G} |\Fix(g)| = \sum_{x \in X} |G_x| \tag{1}
        \]
        
        \subsection*{步骤2:应用轨道-稳定子定理}
        根据轨道-稳定子定理,对每个 \( x \in X \),有:
        \[
        |G| = |\Orb(x)| \cdot |G_x| \implies |G_x| = \frac{|G|}{|\Orb(x)|}
        \]
        代入式(1)得:
        \[
        \sum_{g \in G} |\Fix(g)| = \sum_{x \in X} \frac{|G|}{|\Orb(x)|} = |G| \sum_{x \in X} \frac{1}{|\Orb(x)|} \tag{2}
        \]
        
        \subsection*{步骤3:按轨道合并求和}
        将 \( X \) 划分为互不相交的轨道 \( \Orb(x_1), \Orb(x_2), \dots, \Orb(x_N) \),其中 \( N \) 为轨道数。对于同一轨道 \( \Orb(x_i) \) 中的每个元素 \( x \),有 \( |\Orb(x)| = |\Orb(x_i)| \),且轨道中的元素个数为 \( |\Orb(x_i)| \),因此:
        \[
        \sum_{x \in \Orb(x_i)} \frac{1}{|\Orb(x)|} = \sum_{x \in \Orb(x_i)} \frac{1}{|\Orb(x_i)|} = |\Orb(x_i)| \cdot \frac{1}{|\Orb(x_i)|} = 1
        \]
        对所有 \( N \) 个轨道求和,得到:
        \[
        \sum_{x \in X} \frac{1}{|\Orb(x)|} = \sum_{i=1}^N \left( \sum_{x \in \Orb(x_i)} \frac{1}{|\Orb(x)|} \right) = \sum_{i=1}^N 1 = N \tag{3}
        \]
        
        \subsection*{步骤4:联立等式得出结论}
        将式(3)代入式(2),得:
        \[
        \sum_{g \in G} |\Fix(g)| = |G| \cdot N
        \]
        两边除以 \( |G| \),即得Burnside引理:
        \[
        N = \frac{1}{|G|} \sum_{g \in G} |\Fix(g)|
        \]
        \end{proof}
\end{document}