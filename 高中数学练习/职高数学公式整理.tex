\documentclass{article}

\usepackage{ctex}
\usepackage{amsmath}
\usepackage{amssymb}
\usepackage{geometry}
\geometry{a4paper, margin=1in}
\usepackage{enumitem}
\renewcommand{\labelitemi}{\textbullet} % 第一级,默认的实心圆点
\renewcommand{\labelitemii}{\textopenbullet} % 第二级,空心圆点
\renewcommand{\labelitemiii}{$\triangleright$} % 右尖三角形

\title{基础模块下数学公式整理}
\author{}
\date{}

\begin{document}
\maketitle

\section{指数函数与对数函数}

\subsection{指数函数的定义与性质}
\textbf{定义}:形如 \( f(x) = a^x \)(\( a > 0 \) 且 \( a \neq 1 \))的函数称为指数函数,其中 \( a \) 为底数,\( x \) 为指数。

\begin{itemize}
    \item \textbf{基本性质}:
    \begin{itemize}
        \item 定义域:\( (-\infty, +\infty) \)
        \item 值域:\( (0, +\infty) \)
        \item 过定点:\( (0, 1) \),即 \( x=0 \) 时,\( f(0) = 1 \)
        \item 单调性:
        \begin{itemize}
            \item 当 \( a > 1 \) 时,函数在定义域内单调递增;
            \item 当 \( 0 < a < 1 \) 时,函数在定义域内单调递减。
        \end{itemize}
    \end{itemize}
    
    \item \textbf{指数运算公式}(设 \( a, b > 0 \),\( m, n \in \mathbb{R} \)):
    \begin{enumerate}
        \item \( a^m \cdot a^n = a^{m+n} \)
        \item \( (a^m)^n = a^{mn} \)
        \item \( (ab)^n = a^n b^n \)
        \item \( \dfrac{a^m}{a^n} = a^{m-n} \)
        \item \( \left( \dfrac{a}{b} \right)^n = \dfrac{a^n}{b^n} \)
        \item \( a^0 = 1 \)(\( a \neq 0 \))
        \item \( a^{-n} = \dfrac{1}{a^n} \)
        \item \( a^{\frac{m}{n}} = \sqrt[n]{a^m} \)
    \end{enumerate}
\end{itemize}

\subsection{对数函数的定义与性质}
\textbf{定义}:形如 \( f(x) = \log_a x \)(\( a > 0 \) 且 \( a \neq 1 \))的函数称为对数函数,其中 \( a \) 为底数,\( x \) 为真数(\( x > 0 \))。

\begin{itemize}
    \item \textbf{基本性质}:
    \begin{itemize}
        \item 定义域:\( (0, +\infty) \)
        \item 值域:\( (-\infty, +\infty) \)
        \item 过定点:\( (1, 0) \),即 \( x=1 \) 时,\( f(1) = 0 \)
        \item 单调性:
        \begin{itemize}
            \item 当 \( a > 1 \) 时,函数在定义域内单调递增;
            \item 当 \( 0 < a < 1 \) 时,函数在定义域内单调递减。
        \end{itemize}
    \end{itemize}
    
    \item \textbf{对数运算公式}(设 \( a > 0 \) 且 \( a \neq 1 \),\( M, N > 0 \),\( b > 0 \) 且 \( b \neq 1 \),\( n \in \mathbb{R} \)):
    \begin{enumerate}
        \item 乘积法则:\( \log_a(MN) = \log_a M + \log_a N \)
        \item 商数法则:\( \log_a \dfrac{M}{N} = \log_a M - \log_a N \)
        \item 幂数法则:\( \log_a M^n = n \log_a M \)
        \item 换底公式 \footnote{了解即可}:\( \log_a b = \dfrac{\log_c b}{\log_c a} \)
        \item 自然对数与常用对数:
        \begin{itemize}
            \item 自然对数:\( \ln x = \log_e x \)
            \item 常用对数:\( \lg x = \log_{10} x \)
        \end{itemize}
        \item 对数恒等式:
        \begin{itemize}
            \item \( a^{\log_a x} = x \)
            \item \( \log_a a^x = x \)
        \end{itemize}
        \item 倒数关系\footnote{了解即可}:\( \log_a b = \dfrac{1}{\log_b a} \)
    \end{enumerate}
\end{itemize}


\begin{itemize}
    \item \textbf{指数与对数的转换}:
    \( a^b = N \iff \log_a N = b \)(\( a > 0 \), \( a \neq 1 \), \( N > 0 \))。
\end{itemize}

\begin{enumerate}
    \item \textbf{特殊值}:
    \begin{itemize}
        \item \( \log_a a = 1 \),\( \log_a 1 = 0 \)(转化为指数式理解一下);
        \item \( e^{\ln x} = x \),\( \ln e^x = x \);
        \item \( 10^{\lg x} = x \),\( \lg 10^x = x \)。
    \end{itemize}
\end{enumerate}

\section{直线和圆的方程}
\subsection{直线的方程}
\subsubsection{基本公式}
\begin{itemize}
    \item 两点间距离公式:\\
          设两点 \( P_1(x_1, y_1) \),\( P_2(x_2, y_2) \),则距离为:
          \[
          d = \sqrt{(x_2 - x_1)^2 + (y_2 - y_1)^2}
          \]
    \item 中点坐标公式:\\
          线段 \( P_1P_2 \) 的中点 \( M \) 的坐标为:
          \[
          M\left( \frac{x_1 + x_2}{2}, \frac{y_1 + y_2}{2} \right)
          \]
\end{itemize}

\subsubsection{直线的斜率}
\begin{itemize}
    \item 定义式:\\
          若直线过两点 \( P_1(x_1, y_1) \),\( P_2(x_2, y_2) \),则斜率为:
          \[
          k = \frac{y_2 - y_1}{x_2 - x_1} \quad (x_1 \neq x_2)
          \]
          
    \item 倾斜角与斜率关系:\\
          若直线倾斜角\footnote{倾斜角的范围是 $[0,\pi)$}为 \( \alpha \)(\( \alpha \neq \frac{\pi}{2}  \),即斜率存在时),则:
          \[
          k = \tan \alpha
          \]
\end{itemize}

\subsubsection{直线方程的形式}
\begin{itemize}
    \item 点斜式:\\
          过点 \( (x_0, y_0) \),斜率为 \( k \) 的直线方程:
          \[
          y - y_0 = k(x - x_0)
          \]
          
    \item 斜截式:\\
          斜率为 \( k \),$y$轴截距为 \( b \) 的直线方程:
          \[
          y = kx + b
          \]

          

    \item 一般式:
          \[
          Ax + By + C = 0 \quad (A^2 + B^2 \neq 0)
          \]
\end{itemize}

\subsection{两条直线的位置关系}
\subsubsection{平行与垂直的判定}
下面是$k$存在时的情况
\begin{itemize}
    \item 平行:\\
          直线 \( l_1: y = k_1x + b_1 \) 和 \( l_2: y = k_2x + b_2 \) 平行的充要条件:
          \[
          k_1 = k_2 \quad \text{且} \quad b_1 \neq b_2
          \]
          
    \item 垂直:\\
          直线 \( l_1 \) 和 \( l_2 \) 垂直的充要条件:
          \[
          k_1 \cdot k_2 = -1
          \]
\end{itemize}
思考:斜率不存在时要怎么判断?


\subsubsection{点到直线的距离}
点 \( (x_0, y_0) \) 到直线 \( Ax + By + C = 0 \) 的距离:
\[
d = \frac{|Ax_0 + By_0 + C|}{\sqrt{A^2 + B^2}}
\]

\subsubsection{两条平行直线间的距离}
直线 \( Ax + By + C_1 = 0 \) 和 \( Ax + By + C_2 = 0 \) 间的距离:
\[
d = \frac{|C_1 - C_2|}{\sqrt{A^2 + B^2}}
\]
也不需要记这个公式,在其中一条直线上取一点,计算该点到另一条直线的距离即可.

\subsection{圆的方程}
\subsubsection{标准方程}
\begin{itemize}
    \item 圆心为 \( (a, b) \),半径为 \( r \) 的圆:
          \[
          (x - a)^2 + (y - b)^2 = r^2
          \]
          
    \item 圆心在原点的圆:
          \[
          x^2 + y^2 = r^2
          \]
\end{itemize}

\subsubsection{一般方程}
圆的一般方程:
\[
x^2 + y^2 + Dx + Ey + F = 0 \quad (D^2 + E^2 - 4F > 0)
\]
圆心坐标为 \( \left( -\frac{D}{2}, -\frac{E}{2} \right) \),半径为 \( r = \frac{1}{2}\sqrt{D^2 + E^2 - 4F} \)。

不需要硬记,使用配方法即可将圆的一般方程转化为标准方程.



\subsection{直线与圆的位置关系}
设直线 \( l: Ax + By + C = 0 \),圆 \( (x - a)^2 + (y - b)^2 = r^2 \),圆心到直线的距离为 \( d = \frac{|Aa + Bb + C|}{\sqrt{A^2 + B^2}} \),则:
\begin{itemize}
    \item 相离:\( d > r \);
    \item 相切:\( d = r \);
    \item 相交:\( d < r \)。
\end{itemize}



\subsection{过圆外一点 \( P(x_0, y_0) \) 的切线}
已知圆 \( (x - a)^2 + (y - b)^2 = r^2 \),点 \( P(x_0, y_0) \) 在圆外,求过点 \( P \) 的圆的切线方程:
\begin{enumerate}
    \item 设切线方程为 \( y - y_0 = k(x - x_0) \),即 \( kx - y + (y_0 - kx_0) = 0 \);
    \item 根据圆心到切线的距离等于半径,列出方程:
          \[
          \frac{|ka  - b + (y_0 - kx_0)|}{\sqrt{k^2 + 1}} = r
          \]
    \item 解方程求出 \( k \) 的值(通常有两个解);
    \item 将 \( k \) 代入切线方程。
    
    \textbf{特殊情况}:若上述方程只有一个解,说明存在一条垂直于 \( x \)-轴的切线,其方程为 \( x = x_0 \)。
\end{enumerate}
\section{简单几何体}
% \subsection{基本几何体的定义与分类}
% \begin{itemize}
%     \item \textbf{棱柱}:有两个面互相平行,其余各面都是四边形,且每相邻两个四边形的公共边都互相平行。
%     \item \textbf{棱锥}:有一个面是多边形,其余各面都是有一个公共顶点的三角形。
%     \item \textbf{圆柱}:以矩形的一边所在直线为轴旋转,其余三边旋转形成的曲面所围成的几何体。
%     \item \textbf{圆锥}:以直角三角形的一条直角边为旋转轴,旋转一周所形成的几何体。
%     \item \textbf{球}:以半圆的直径所在直线为旋转轴,半圆面旋转一周形成的几何体。
% \end{itemize}

% \subsection{表面积与体积公式}
\begin{enumerate}
    \item \textbf{棱柱(直棱柱)}:
          \begin{itemize}
              \item 表面积 \( S = 2S_{\text{底}} + C_{\text{底}} \cdot h \)(\( S_{\text{底}} \)为底面积,\( C_{\text{底}} \)为底面周长,\( h \)为高)
              \item 体积 \( V = S_{\text{底}} \cdot h \)
              \item \textbf{特殊情况}:正方体(棱长为 \( a \))
                    \[
                    S = 6a^2, \quad V = a^3
                    \]
          \end{itemize}
          
    \item \textbf{棱锥(正棱锥)}:
          \begin{itemize}
              \item 表面积 \( S = S_{\text{底}} + \frac{1}{2}C_{\text{底}} \cdot h' \)(\( h' \)为斜高)
              \item 体积 \( V = \frac{1}{3}S_{\text{底}} \cdot h \)

          \end{itemize}
          
    \item \textbf{圆柱}:
          \begin{itemize}
              \item 表面积 \( S = 2\pi r^2 + 2\pi r h \)(\( r \)为底面半径,\( h \)为高)
              \item 体积 \( V = \pi r^2 h \)
          \end{itemize}
          
    \item \textbf{圆锥}:
          \begin{itemize}
              \item 表面积 \( S = \pi r^2 + \pi r l \)(\( l \)为母线长,\( l = \sqrt{r^2 + h^2} \))
              \item 体积 \( V = \frac{1}{3}\pi r^2 h \)
          \end{itemize}
          
    \item \textbf{球}:
          \begin{itemize}
              \item 表面积 \( S = 4\pi R^2 \)(\( R \)为球半径)
              \item 体积 \( V = \frac{4}{3}\pi R^3 \)
          \end{itemize}
\end{enumerate}

\end{document}