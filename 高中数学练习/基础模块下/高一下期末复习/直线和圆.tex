


\subsection{直线平行的定义与判定}

\begin{frame}{利用斜率判断直线平行}
    \begin{block}{定理}

      设两条直线 \( l_1 \) 和 \( l_2 \) 的斜率分别为 \( k_1 \) 和 \( k_2 \),则:
      \[
      l_1 \parallel l_2 \iff k_1 = k_2 \quad \text{且} \quad b_1 \neq b_2
      \]
      其中 \( b_1 \) 和 \( b_2 \) 分别为两直线在 \( y \)-轴上的截距。
    \end{block}
  \end{frame}

  \begin{frame}
    \begin{exampleblock}{示例 1}
      判断直线 \( l_1: y = 2x + 3 \) 和 \( l_2: 4x - 2y + 5 = 0 \) 是否平行。
      \begin{enumerate}
        \item 将 \( l_2 \) 化为斜截式:
          \[
          4x - 2y + 5 = 0 \implies y = 2x + \frac{5}{2}
          \]
        \item 比较斜率:
          \[
          k_1 = 2, \quad k_2 = 2 \implies k_1 = k_2
          \]
        \item 比较截距:
          \[
          b_1 = 3, \quad b_2 = \frac{5}{2} \implies b_1 \neq b_2
          \]
        \item 结论:\( l_1 \parallel l_2 \)。
      \end{enumerate}
    \end{exampleblock}
  \end{frame}


  \begin{frame}{特殊情况:直线斜率不存在}
    \begin{block}{判定规则}
      两条直线均垂直于 \( x \)轴时,即它们的斜率不存在,若它们的 \( x \)轴截距不同,则平行:
      \[
      l_1: x = a_1, \quad l_2: x = a_2 \quad (a_1 \neq a_2) \implies l_1 \parallel l_2
      \]
    \end{block}

    \begin{exampleblock}{示例 2}
      判断直线 \( l_1: x = 2 \) 和 \( l_2: x = -3 \) 是否平行。
      \begin{enumerate}
        \item 两直线斜率均不存在(垂直于 \( x \)-轴)。
        \item \( x \)-截距分别为 \( 2 \) 和 \( -3 \),即 \( a_1 \neq a_2 \)。
        \item 结论:\( l_1 \parallel l_2 \)。
      \end{enumerate}
    \end{exampleblock}
  \end{frame}

  \begin{frame}{利用直线方程的系数判断平行(一般式)}
    \begin{block}{定理}
      设两条直线的一般式方程为:
      \[
      l_1: A_1x + B_1y + C_1 = 0 \quad \text{和} \quad l_2: A_2x + B_2y + C_2 = 0
      \]
      则两直线平行的充要条件是:
      \[
      \frac{A_1}{A_2} = \frac{B_1}{B_2} \neq \frac{C_1}{C_2}
      \]
      若同时满足 \(\frac{A_1}{A_2} = \frac{B_1}{B_2} = \frac{C_1}{C_2}\),则两直线重合。
    \end{block}

  \end{frame}


  \begin{frame}

    \begin{exampleblock}{示例1:判断平行}
      判断直线 \( l_1: 2x + 4y - 3 = 0 \) 和 \( l_2: 4x + 8y + 5 = 0 \) 是否平行。
      \begin{enumerate}
        \item 计算系数比例:
          \[
          \frac{A_1}{A_2} = \frac{2}{4} = \frac{1}{2}, \quad \frac{B_1}{B_2} = \frac{4}{8} = \frac{1}{2}, \quad \frac{C_1}{C_2} = \frac{-3}{5}
          \]
        \item 比较比例:
          \[
          \frac{1}{2} = \frac{1}{2} \neq \frac{-3}{5} \implies l_1 \parallel l_2
          \]
      \end{enumerate}
    \end{exampleblock}
  \end{frame}


  \begin{frame}{示例}
    \begin{exampleblock}{示例2:判断重合}
      判断直线 \( l_1: 3x - y + 2 = 0 \) 和 \( l_2: 6x - 2y + 4 = 0 \) 的关系。
      \begin{enumerate}
        \item 计算系数比例:
          \[
          \frac{A_1}{A_2} = \frac{3}{6} = \frac{1}{2}, \quad \frac{B_1}{B_2} = \frac{-1}{-2} = \frac{1}{2}, \quad \frac{C_1}{C_2} = \frac{2}{4} = \frac{1}{2}
          \]
        \item 比较比例:
          \[
          \frac{1}{2} = \frac{1}{2} = \frac{1}{2} \implies l_1 \text{ 与 } l_2 \text{ 重合}
          \]
      \end{enumerate}
    \end{exampleblock}

  \end{frame}





  \subsection{直线垂直的定义与判定}
\begin{frame}
    \frametitle{直线垂直的定义}
    \begin{block}{}
        在平面直角坐标系中,两条直线 \( l_1 \) 和 \( l_2 \) 被称为互相垂直,当且仅当它们的夹角为 \( 90^\circ \),记作 \( l_1 \perp l_2 \)。
    \end{block}

    \begin{center}
        \begin{tikzpicture}[scale=1.5]
            % 绘制坐标系
            \draw[->] (-2,0) -- (2,0) node[right] {$x$};
            \draw[->] (0,-2) -- (0,2) node[above] {$y$};

            % 绘制第一条直线 y = x
            \draw[blue,thick] (-1,-2) -- (1,2) node[right] {$l_1: y = 2 x$};

            % 绘制第二条直线 y = -x
            \draw[red,thick] (-2,1) -- (2,-1) node[right] {$l_2: y = -\frac{1}{2}x$};

            % 标记直角
            \draw[black,thick] (0.1,0.2) -- (0.3,0.1) -- (0.2,-0.1);
            \node[right] at (0.3,0.1) {\(90^\circ\)};
        \end{tikzpicture}
    \end{center}
\end{frame}



% 直线垂直的判定方法
\section{直线垂直的判定方法}
\begin{frame}
    \frametitle{判定方法1:斜率乘积}
    \begin{block}{定理}
        设两条非垂直于坐标轴的直线 \( l_1 \) 和 \( l_2 \) 的斜率分别为 \( k_1 \) 和 \( k_2 \),则:
        \[
        l_1 \perp l_2 \iff k_1 \cdot k_2 = -1
        \]
    \end{block}

    \begin{example}
        直线 \( l_1: y = 2x + 3 \) 与直线 \( l_2: y = -\frac{1}{2}x + 1 \) 是否垂直?

        \textbf{解:}
        \[
        k_1 = 2, \quad k_2 = -\frac{1}{2} \implies k_1 \cdot k_2 = 2 \times \left(-\frac{1}{2}\right) = -1
        \]
        因此,\( l_1 \perp l_2 \)。
    \end{example}
\end{frame}

\begin{frame}
    \frametitle{判定方法2:一般式方程}
    \begin{block}{定理}
        设两条直线的一般式方程分别为:
        \[
        l_1: A_1x + B_1y + C_1 = 0, \quad l_2: A_2x + B_2y + C_2 = 0
        \]
        则:
        \[
        l_1 \perp l_2 \iff A_1A_2 + B_1B_2 = 0
        \]
    \end{block}

    \begin{example}
        直线 \( l_1: 3x - 2y + 5 = 0 \) 与直线 \( l_2: 2x + 3y - 1 = 0 \) 是否垂直?

        \textbf{解:}
        \[
        A_1 = 3, \ B_1 = -2, \ A_2 = 2, \ B_2 = 3 \implies A_1A_2 + B_1B_2 = 3 \times 2 + (-2) \times 3 = 0
        \]
        因此,\( l_1 \perp l_2 \)。
    \end{example}
\end{frame}





\begin{frame}
  \frametitle{特殊情况:两条直线中有直线斜率不存在}
  \begin{itemize}
      \item 垂直于 \( x \)-轴的直线方程为 \( x = a \),其斜率不存在。
      \item 垂直于 \( y \)-轴的直线方程为 \( y = b \),其斜率为 0。
      \item 垂直于 \( x \)-轴的直线与垂直于 \( y \)-轴的直线互相垂直。
  \end{itemize}

  \begin{center}
      \begin{tikzpicture}[scale=1.2]
          % 绘制坐标系
          \draw[->] (-2,0) -- (2,0) node[right] {$x$};
          \draw[->] (0,-2) -- (0,2) node[above] {$y$};

          % 绘制垂直于x轴的直线
          \draw[blue,thick] (1,-2) -- (1,2) node[above] {$x = 1$};

          % 绘制垂直于y轴的直线
          \draw[red,thick] (-2,-1) -- (2,-1) node[right] {$y = -1$};

          % 标记交点和直角
          \draw[black] (1,-1) circle (2pt);
          \draw[black] (0.8,-1) -- (0.8,-0.8) -- (1,-0.8);
      \end{tikzpicture}
  \end{center}
\end{frame}
















\subsection{点到直线距离公式}

\begin{frame}
  \frametitle{点到直线距离公式}
  在平面直角坐标系中,对于点\(P(x_0,y_0)\)和直线\(l:Ax + By+C = 0\)(\(A\)、\(B\)不同时为\(0\)),点\(P\)到直线\(l\)的距离\(d\)为:
  \[d=\frac{\vert Ax_0 + By_0 + C\vert}{\sqrt{A^{2}+B^{2}}}\]
  \begin{center}
      \begin{tikzpicture}[scale=1.2]
          \draw[->] (-3,0) -- (3,0) node[right] {\(x\)};
          \draw[->] (0,-3) -- (0,3) node[above] {\(y\)};
          \draw[thick,blue] (-3,-1) -- (3,1) node[midway,below right] {\(Ax + By+C = 0\)};
          \draw[thick, dashed] (1,1) -- (6/5,2/5) node[midway, right] {\(d\)};
          \filldraw[red] (1,1) circle (2pt) node[above right] {\(P(x_0,y_0)\)};
      \end{tikzpicture}
  \end{center}
\end{frame}

\begin{frame}
  \frametitle{公式拓展}
    两条平行直线\(l_1:Ax + By+C_1 = 0\),\(l_2:Ax + By+C_2 = 0\)(\(C_1\neq C_2\))间的距离\[d=\frac{\vert C_1 - C_2\vert}{\sqrt{A^{2}+B^{2}}}\]。
\end{frame}


\begin{frame}
  \frametitle{课堂练习}
    \begin{block}{}
      求点\(P(-1,4)\)到直线\(2x + 3y-6 = 0\)的距离。
    \end{block}
     
      \pause 
      \begin{solution}
          由点到直线距离公式\(d=\frac{\vert Ax_0 + By_0 + C\vert}{\sqrt{A^{2}+B^{2}}}\),其中\(A=2\),\(B=3\),\(C=-6\),\(x_0=-1\),\(y_0=4\)。
          \[
          d=\frac{\vert2\times(-1)+3\times4+(-6)\vert}{\sqrt{2^{2}+3^{2}}}=\frac{\vert-2 + 12-6\vert}{\sqrt{13}}=\frac{4}{\sqrt{13}}=\frac{4\sqrt{13}}{13}
          \]
      \end{solution}
\end{frame}
\begin{frame}
  \frametitle{课堂练习}
    \begin{block}{}
      已知点\(M(3,m)\)到直线\(x - y+1 = 0\)的距离为\(\sqrt{2}\),求\(m\)的值。
    \end{block}
     \pause

      \begin{solution}
          根据公式\(d=\frac{\vert3 - m+1\vert}{\sqrt{1^{2}+(-1)^{2}}}=\sqrt{2}\),即\(\frac{\vert4 - m\vert}{\sqrt{2}}=\sqrt{2}\)。
          两边乘以\(\sqrt{2}\)得\(\vert4 - m\vert=2\),解得\(4 - m=2\)或\(4 - m=-2\),即\(m=2\)或\(m=6\)。
      \end{solution}

      % \item 求两条平行直线\(l_1:3x + 4y - 1 = 0\)与\(l_2:3x + 4y+4 = 0\)间的距离。
      % \begin{solution}
      %     由平行直线距离公式\(d=\frac{\vert C_1 - C_2\vert}{\sqrt{A^{2}+B^{2}}}\),其中\(A=3\),\(B=4\),\(C_1=-1\),\(C_2=4\)。
      %     \[
      %     d=\frac{\vert-1 - 4\vert}{\sqrt{3^{2}+4^{2}}}=\frac{5}{5}=1
      %     \]
      % \end{solution}
\end{frame}










\subsection{直线和圆的位置关系}
% 第一题解析
\begin{frame}{例题1:求过点的切线方程}
    \begin{block}{题目}
        求过 \( M(3,1) \) 且与圆 \( (x-1)^2 + y^2 = 4 \) 相切的直线 \( l \) 方程。
    \end{block}
    \pause
    \begin{enumerate}
        \item \textbf{斜率不存在的情况}:\\
            直线方程为 \( x = 3 \),圆心到直线距离 \( |3-1| = 2 = r \),相切。

        \item \textbf{斜率存在的情况}:\\
            设方程为 \( y-1 = k(x-3) \),即 \( kx - y + (1-3k) = 0 \)。\\
            由圆心到直线距离等于半径:
            \[
            \frac{|k \cdot 1 - 0 + 1 - 3k|}{\sqrt{k^2 + 1}} = 2 \implies k = -\frac{3}{4}
            \]
            代入得直线方程:\( 3x + 4y - 13 = 0 \)。
    \end{enumerate}

    \begin{block}{答案}
        \[
        \boxed{x=3} \quad \text{和} \quad \boxed{3x +4y -13 =0}
        \]
    \end{block}
\end{frame}





% 第二题解析
\begin{frame}{例题2:圆上点到直线的最小距离}
  \begin{block}{题目}
      圆 \( x^2 + y^2 -2x = 0 \) 上点到直线 \( x - y +1 = 0 \) 的最小距离为?
  \end{block}
  \pause
  \begin{enumerate}
      \item \textbf{化圆为标准方程}:\\
          \( (x-1)^2 + y^2 = 1 \),圆心 \( C(1,0) \),半径 \( r=1 \)。

      \item \textbf{计算圆心到直线的距离}:\\
          \[
          d = \frac{|1 - 0 + 1|}{\sqrt{1^2 + (-1)^2}} = \sqrt{2}
          \]

      \item \textbf{最小距离公式}:\\
          当 \( d > r \) 时,最小距离为 \( d - r = \sqrt{2} - 1 \)。
  \end{enumerate}


  \begin{block}{答案}
      \[
      \boxed{\sqrt{2} -1}
      \]
  \end{block}
\end{frame}

% 第三题解析
\begin{frame}{例题3:直线平行求参数}
  \begin{block}{题目}
      已知直线 \( l_1 : (a-1) x + 2 y +1 = 0 \) 与 \( l_2 :x+ay+3 = 0 \) 平行,求 \( a \)。
  \end{block}
  \pause
  \begin{enumerate}
      \item \textbf{利用一般式平行条件}:\\
          \[
          \frac{a-1}{1} = \frac{2}{a} \implies a^2 - a - 2 = 0 \implies a = 2 \text{ 或 } -1
          \]

      \item \textbf{验证排除重合}:\\
          - \( a=2 \) 时,两直线为 \( x+2y+1=0 \) 和 \( x+2y+3=0 \),不重合。\\
          - \( a=-1 \) 时,两直线为 \( -2x+2y+1=0 \) 和 \( x-y+3=0 \),不重合。
  \end{enumerate}


  \begin{block}{答案}
      \[
      \boxed{2 \text{ 或 } -1}
      \]
  \end{block}
\end{frame}


% 第四题解析
\begin{frame}{例题4:直线过点求方程}
  \begin{block}{题目}
      若直线 \( l:mx - m^2 y = 1 \) 过点 \( P(2,1) \),求直线 \( l \) 的方程。
  \end{block}
  \pause
  \begin{enumerate}
      \item \textbf{代入点坐标求参数}:\\
          \[
          2m - m^2 = 1 \implies (m-1)^2 = 0 \implies m = 1
          \]

      \item \textbf{代入得直线方程}:\\
          \( x - y = 1 \)。
  \end{enumerate}


  \begin{block}{答案}
      \[
      \boxed{x - y = 1}
      \]
  \end{block}
\end{frame}
\begin{frame}{例题5:圆方程的参数范围}
  \begin{block}{题目}
      方程 \( x^2 + y^2 + ax + 2ay + 2a^2 + a - 1 = 0 \) 表示圆,求 \( a \) 的取值范围。
  \end{block}

  \begin{enumerate}
      \item \textbf{配方过程}:
          \[
          \begin{aligned}
          &\left(x + \frac{a}{2}\right)^2 + (y + a)^2 \\
          &= -\frac{3a^2}{4} - a + 1
          \end{aligned}
          \]

      \item \textbf{圆存在条件}:
          \[
          -\frac{3a^2}{4} - a + 1 > 0 \implies 3a^2 + 4a - 4 < 0
          \]

      \item \textbf{解不等式}:
          \[
          (3a - 2)(a + 2) < 0 \implies a \in (-2, \frac{2}{3})
          \]
  \end{enumerate}


\end{frame}

  \begin{frame}{}{}
      已知圆 $x^2 +y^2 - 2x +4y -4 = 0$,则过点$P(-3,4)$且与圆相切的切线长为?

      \pause
      $\sqrt{43}.$
  \end{frame}
