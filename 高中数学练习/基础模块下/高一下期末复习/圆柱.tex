
\subsection{圆柱}
\begin{frame}{圆柱的基本概念}
    \begin{block}{定义}
        圆柱是由两个平行且全等的圆面(底面)和一个连接两底面的曲面(侧面)所围成的几何体。
    \end{block}
    
    \begin{columns}
        \column{0.5\textwidth}
        \begin{exampleblock}{参数}
            \begin{itemize}
                \item 底面半径:\( r \)
                \item 高:\( h \)
                \item 底面面积:\( S_{\text{底}} = \pi r^2 \)
                \item 底面周长:\( C = 2\pi r \)
            \end{itemize}
        \end{exampleblock}
        
        \column{0.5\textwidth}
   
\begin{tikzpicture}[scale=1.5,
    y={(0cm,-0.3cm)}, % 设置 x 轴方向
    x={(1cm,0cm)},            % 设置 y 轴方向
    z={(0cm,1cm)}             % 设置 z 轴方向
]%斜二测画法
    % 定义圆柱的参数
    \def\radius{1} % 圆柱底面半径
    \def\height{2} % 圆柱的高度
    \coordinate (O') at (0,0,\height);
    \coordinate (O) at (0,0,0);

    %上圆上一点
    \coordinate (A') at ({\radius*sqrt(2)/2},{\radius*sqrt(2)/2},\height);

    %下圆上一点
    \coordinate (A) at ({\radius*sqrt(2)/2},{\radius*sqrt(2)/2},0);


    \fill[orange,opacity=0.5](A') -- (A) -- (O) -- (O') -- cycle; %填充颜色

        % 绘制母线
        \begin{scope}
            \draw[thick,red] (\radius,0,0) -- (\radius,0,\height);
            \draw[thick,red] (-\radius,0,0) -- (-\radius,0,\height);
        \end{scope}
        % 绘制高线
        \begin{scope}
            \fill[red] (0,0,0) circle (0.75pt);  % 绘制一个点测试一下 
            \fill[red] (0,0,\height) circle (0.75pt);  % 绘制一个点测试一下 
            \draw[thick,red,dashed] (0,0,0) -- (0,0,\height);
            \draw[thick,red] (A) -- (A');
            \end{scope}

        % 绘制顶面圆
        \begin{scope}
            %绘制一个圆
            % \draw(0,0,\height) circle (\radius);
            \draw [blue,thick] (\radius,0,\height) arc (0:-180:\radius) ;
            \draw [blue,thick] (\radius,0,\height) arc (0:180:\radius) ;
        \end{scope}

      
        % 绘制底面圆
        \begin{scope}[on background layer]
            \draw [blue,thick,dashed] (\radius,0,0) arc (0:-180:\radius) ;
            \draw [blue,thick] (\radius,0,0) arc (0:180:\radius) ;
        \end{scope}

        % 绘制底面顶面半径
        \begin{scope}[on background layer]
            %绘制一个圆
            % \draw(0,0,\height) circle (\radius);
            \draw [blue,thin,dashed] (O) -- (A) ;
            \draw [blue,thin,dashed] (O') -- (A') ;
            % \draw [blue,thick] (\radius,0,\height) arc (0:180:\radius) ;
        \end{scope}

    



        % % 绘制括号
        % \begin{scope}[on background layer]
        %     \draw[decorate,decoration={brace,amplitude=10pt},thick] (0-0.1,0,0) -- (0-0.1,0,\height);
        % \end{scope}



\end{tikzpicture}


    \end{columns}
\end{frame}



\begin{frame}{表面积与体积公式}
    \begin{block}{核心公式}
        \begin{itemize}
            \item 侧面积:\( S_{\text{侧}} = 2\pi r h \)
            \item 全面积:\( S_{\text{全}} = 2\pi r h + 2\pi r^2 \)
            \item 体积:\( V = \pi r^2 h \)
        \end{itemize}
    \end{block}
    
    \begin{exampleblock}{推导思路}
        \begin{itemize}
            \item 侧面积公式:将侧面展开为矩形,长为底面圆周长 \( 2\pi r \),宽为高 \( h \)
            \item 全面积公式:侧面积加上两个底面积
            \item 体积公式:底面积乘以高
        \end{itemize}
    \end{exampleblock}
\end{frame}

\begin{frame}{圆柱的截面性质}
    \begin{block}{定理}
        圆柱的截面有以下几种情况:
        \begin{enumerate}
            \item 平行于底面的截面:与底面全等的圆
            \item 过轴的截面(轴截面):矩形,长为底面直径 \( 2r \),宽为高 \( h \)
        \end{enumerate}
    \end{block}
    
    \begin{columns}
        \column{0.5\textwidth}
        \begin{exampleblock}{轴截面面积}
            \[
            S_{\text{轴}} = 2r \times h = 2rh
            \]
        \end{exampleblock}
        
        \column{0.5\textwidth}
        
\begin{tikzpicture}[scale=1.5,
    y={(0cm,-0.4cm)}, % 设置 x 轴方向
    x={(1cm,0cm)},            % 设置 y 轴方向
    z={(0cm,1cm)}             % 设置 z 轴方向
]%斜二测画法
    % 定义圆柱的参数
    \def\radius{1} % 圆柱底面半径
    \def\height{2} % 圆柱的高度
    \coordinate (O') at (0,0,\height);
    \coordinate (O) at (0,0,0);
    \coordinate (R') at (\radius,0,\height);
    \coordinate (R) at (\radius,0,0);
    \coordinate (R'') at (-\radius,0,\height);
    \coordinate (R''') at (-\radius,0,0);


    % %上圆上一点
    % \coordinate (A') at ({\radius*sqrt(2)/2},{\radius*sqrt(2)/2},\height);

    % %下圆上一点
    % \coordinate (A) at ({\radius*sqrt(2)/2},{\radius*sqrt(2)/2},0);


    % \fill[orange,opacity=0.5](A') -- (A) -- (O) -- (O') -- cycle; %填充颜色

    % \fill[orange,opacity=0.5](O') -- (O) -- (R) -- (R') -- cycle; %填充颜色
    \fill[orange,opacity=0.5](R) -- (R') -- (R'') -- (R''') -- cycle; %填充颜色
    \draw[blue,thick] (R) -- (R') -- (R'') -- (R''') -- cycle; %填充颜色



    \draw[<->] ($(R''')+(0,0,-0.1)$)-- ($(R)+(0,0,-0.1)$) node[midway,below]{\( 2r \)};

    \draw[<->] ($(R)+(0.1,0,0)$)-- ($(R')+(0.1,0,0)$) node[midway,right]{\( h \)};
    % ($(0,0) + (1,1)$)

        % 绘制母线
        \begin{scope}
            \draw[thick,red] (\radius,0,0) -- (\radius,0,\height);
            \draw[thick,red] (-\radius,0,0) -- (-\radius,0,\height);
        \end{scope}
        % 绘制高线
        \begin{scope}
            \fill[red] (0,0,0) circle (0.75pt);  % 绘制一个点测试一下 
            \fill[red] (0,0,\height) circle (0.75pt);  % 绘制一个点测试一下 
            \draw[thick,red] (0,0,0) -- (0,0,\height);
            \end{scope}

        % 绘制顶面圆
        \begin{scope}
            %绘制一个圆
            % \draw(0,0,\height) circle (\radius);
            %上半圆
            \draw [blue,thick] (\radius,0,\height) arc (0:-180:\radius) ;
            % \draw [blue,thick] (\radius,0,\height) arc (0:180:\radius) ;
        \end{scope}

      
        % 绘制底面圆
        \begin{scope}[on background layer]
            \draw [blue,thick,dashed] (\radius,0,0) arc (0:-180:\radius) ;
            % \draw [blue,thick] (\radius,0,0) arc (0:180:\radius) ;
        \end{scope}


\end{tikzpicture}

    \end{columns}
\end{frame}





\begin{frame}
    \frametitle{圆柱知识点}
    \begin{itemize}
        \item 圆柱侧面积公式:\( S_{\text{侧}} = 2\pi rh \)
        \item 圆柱表面积公式:\( S_{\text{表}} = 2\pi r(r + h) \)
        \item 圆柱体积公式:\( V = \pi r^2 h \)
    \end{itemize}
\end{frame}



\begin{frame}
    \frametitle{题目1:已知半径和高,求表面积和体积}
    \begin{block}{题目}
        圆柱底面半径为 \(2 \, \text{cm}\),高为 \(5 \, \text{cm}\),求:
        \begin{enumerate}
            \item 侧面积
            \item 表面积
            \item 体积
        \end{enumerate}
    \end{block}
    
    \vspace{0.5cm}
    \pause
    \begin{block}{解答}
        \begin{enumerate}
            \item
            \[
            S_{\text{侧}} = 2\pi \times 2 \times 5 = 20\pi \, \text{cm}^2
            \]
            
            \item 
            \[
            S_{\text{表}} = 2\pi \times 2 \times (2 + 5) = 28\pi \, \text{cm}^2
            \]
            
            \item 
            \[
            V = \pi \times 2^2 \times 5 = 20\pi \, \text{cm}^3
            \]
        \end{enumerate}
    \end{block}
\end{frame}



% 题目2
\begin{frame}
    \frametitle{题目2:已知直径和高,求体积}
    \begin{block}{题目}
        圆柱底面直径为 \(6 \, \text{dm}\),高为 \(4 \, \text{dm}\),求体积。
    \end{block}
    
    \vspace{0.5cm}
    \pause
    \begin{block}{解答}
        首先计算底面半径:
        \[
        r = \frac{d}{2} = \frac{6}{2} = 3 \, \text{dm}
        \]
        
        体积公式:\( V = \pi r^2 h \)
        \[
        V = \pi \times 3^2 \times 4 = 36\pi \, \text{dm}^3
        \]
    \end{block}
\end{frame}

% 题目3
\begin{frame}
    \frametitle{题目3:已知周长和高,求侧面积}
    \begin{block}{题目}
        圆柱底面周长为 \(8\pi \, \text{m}\),高为 \(3 \, \text{m}\),求侧面积。
    \end{block}
    
    \vspace{0.5cm}
    \pause
    
    \begin{block}{解答}
        侧面积公式:\( S_{\text{侧}} = C \times h \)(其中 \(C\) 为底面周长)
        \[
        S_{\text{侧}} = 8\pi \times 3 = 24\pi \, \text{m}^2
        \]
    \end{block}
\end{frame}


% 题目3:立体几何与代数结合
\begin{frame}
    \frametitle{题目4:立体几何与代数结合}
    \begin{block}{题目}
        圆柱的体积为 \(72\pi \, \text{cm}^3\),底面半径与高的比为 \(2:3\),求底面半径和高。
    \end{block}
    \pause
    
    \begin{block}{解析}
        设底面半径为 \(2k\),高为 \(3k\),则体积:
        \[
        V = \pi (2k)^2 \cdot 3k = 12\pi k^3
        \]
        由题可知:
        \[
        12\pi k^3 = 72\pi \implies k^3 = 6 \implies k = \sqrt[3]{6}
        \]
        因此:
        \[
        \text{底面半径} = 2k = 2\sqrt[3]{6} \, \text{cm}, \quad \text{高} = 3k = 3\sqrt[3]{6} \, \text{cm}
        \]
    \end{block}
    
    \begin{alertblock}{答案}
        底面半径:\(2\sqrt[3]{6} \, \text{cm}\),高:\(3\sqrt[3]{6} \, \text{cm}\)
    \end{alertblock}
\end{frame}



% 题目4:最值问题
\begin{frame}[allowframebreaks]
    \frametitle{题目6:最值问题(实际应用)}
    \begin{block}{题目}
        用一张长 \(20\pi \, \text{cm}\)、宽 \(10 \, \text{cm}\) 的矩形纸围成一个圆柱(接口处忽略不计),求围成圆柱的最大体积。
    \end{block}
    
    \begin{block}{解析}
        分两种情况:
        \begin{enumerate}
            \item 以长为底面周长:
            \[
            2\pi r = 20\pi \implies r = 10 \, \text{cm}, \quad h = 10 \, \text{cm}
            \]
            体积:\( V = \pi r^2 h = \pi \times 10^2 \times 10 = 1000\pi \, \text{cm}^3 \)
            
            \item 以宽为底面周长:
            \[
            2\pi r = 10 \implies r = \frac{5}{\pi} \, \text{cm}, \quad h = 20\pi \, \text{cm}
            \]
            体积:\( V = \pi \left(\frac{5}{\pi}\right)^2 \times 20\pi = 500 \, \text{cm}^3 \)
        \end{enumerate}
    \end{block}
    
    \begin{alertblock}{答案}
        最大体积为:\(1000\pi \, \text{cm}^3\)
    \end{alertblock}
\end{frame}

\begin{frame}
    \frametitle{题目7:立体几何与方程结合}
    
    \begin{block}{题目}
        圆柱的表面积为 \(100\pi \, \text{cm}^2\),底面半径与高的和为10 cm。求底面半径。
    \end{block}
    \begin{alertblock}{已知条件}
        \begin{enumerate}
            \item 圆柱表面积 \( S = 100\pi \, \text{cm}^2 \)
            \item 底面半径 \( r \) 与高 \( h \) 的和为10 cm,即 \( r + h = 10 \)
        \end{enumerate}
    \end{alertblock}
    

\end{frame}
\begin{frame}
    \frametitle{解析过程}
    
    \begin{block}{圆柱表面积公式}
        \[
        S = 2\pi r(r + h)
        \]
    \end{block}
    
    \begin{block}{代入已知条件}
        已知 \( S = 100\pi \) 和 \( r + h = 10 \),代入公式得:
        \[
        2\pi r \cdot 10 = 100\pi
        \]
        化简方程:
        \[
        20\pi r = 100\pi
        \]
        两边同时除以 \( 20\pi \):
        \[
        r = \frac{100\pi}{20\pi} = 5 \, \text{cm}
        \]
    \end{block}
\end{frame}