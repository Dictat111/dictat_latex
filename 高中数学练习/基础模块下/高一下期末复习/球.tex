
\subsection{球}
\begin{frame}{球的基本概念}
    \begin{block}{定义}
        空间中到定点(球心)的距离等于定长(半径 \(R\))的点的集合。
    \end{block}
    
    \begin{columns}
        \column{0.5\textwidth}
        \begin{exampleblock}{参数}
            \begin{itemize}
                \item 球心:\(O\)
                \item 半径:\(R\)
                \item 直径:\(D = 2R\)
            \end{itemize}
        \end{exampleblock}
        
        \column{0.5\textwidth}
        
\subsection{球}
\begin{frame}{球的基本概念}
    \begin{block}{定义}
        空间中到定点(球心)的距离等于定长(半径 \(R\))的点的集合。
    \end{block}
    
    \begin{columns}
        \column{0.5\textwidth}
        \begin{exampleblock}{参数}
            \begin{itemize}
                \item 球心:\(O\)
                \item 半径:\(R\)
                \item 直径:\(D = 2R\)
            \end{itemize}
        \end{exampleblock}
        
        \column{0.5\textwidth}
        
\subsection{球}
\begin{frame}{球的基本概念}
    \begin{block}{定义}
        空间中到定点(球心)的距离等于定长(半径 \(R\))的点的集合。
    \end{block}
    
    \begin{columns}
        \column{0.5\textwidth}
        \begin{exampleblock}{参数}
            \begin{itemize}
                \item 球心:\(O\)
                \item 半径:\(R\)
                \item 直径:\(D = 2R\)
            \end{itemize}
        \end{exampleblock}
        
        \column{0.5\textwidth}
        
\subsection{球}
\begin{frame}{球的基本概念}
    \begin{block}{定义}
        空间中到定点(球心)的距离等于定长(半径 \(R\))的点的集合。
    \end{block}
    
    \begin{columns}
        \column{0.5\textwidth}
        \begin{exampleblock}{参数}
            \begin{itemize}
                \item 球心:\(O\)
                \item 半径:\(R\)
                \item 直径:\(D = 2R\)
            \end{itemize}
        \end{exampleblock}
        
        \column{0.5\textwidth}
        \input{高一下期末复习/tikz/球.tex}
    \end{columns}
  \end{frame}
  
  \begin{frame}{球的截面性质}
    \begin{block}{定理}
        球心到截面的距离 \(d\)、截面圆半径 \(r\) 与球半径 \(R\) 满足:
        \[
        r = \sqrt{R^2 - d^2}
        \]
    \end{block} 
    
    \begin{columns}
        \column{0.6\textwidth}
        \begin{exampleblock}{特殊截面}
            \begin{enumerate}
                \item 当 \(d = 0\) 时,\(r = R\),截面为\textcolor{blue}{大圆}
                \item 当 \(0 < d < R\) 时,截面为\textcolor{red}{小圆}
            \end{enumerate}
        \end{exampleblock}
        
        \column{0.4\textwidth}
        \input{高一下期末复习/tikz/球的大圆和小圆.tex}
    \end{columns}
  \end{frame}
  
  
  
  
  \begin{frame}{表面积与体积公式}
    \begin{block}{核心公式}
        \begin{itemize}
            \item 表面积:\[ S = 4\pi R^2 \]
            \item 体积:\[ V = \frac{4}{3}\pi R^3 \]
        \end{itemize}
    \end{block}
  \end{frame}
  
  
  \begin{frame}{球与多面体的关系}
    \begin{block}{外接球}
        多面体的所有顶点都在球面上,球心到各顶点距离相等。
        \begin{itemize}
            \item 正方体的外接球半径:\( R = \frac{\sqrt{3}}{2}a \)(\(a\) 为棱长)
            \item 长方体的外接球半径:\( R = \frac{\sqrt{a^2 + b^2 + c^2}}{2} \)
        \end{itemize}
    \end{block}
    
    \begin{block}{内切球}
        球与多面体的各面都相切,球心到各面距离相等。
        \begin{itemize}
            % \item 正四面体的内切球半径:\( r = \frac{\sqrt{6}}{12}a \)(\(a\) 为棱长)
            \item 正方体的内切球半径:\( r = \frac{a}{2} \)
        \end{itemize}
    \end{block}
  \end{frame}
  
  
  \begin{frame}{典型例题}
    \begin{exampleblock}{例题1}
        已知球的表面积为 \(100\pi\),求其体积。
    \end{exampleblock}
    
    \pause
    
    \begin{alertblock}{解答}
        \[
        S = 4\pi R^2 = 100\pi \implies R^2 = 25 \implies R = 5
        \]
        \[
        V = \frac{4}{3}\pi R^3 = \frac{4}{3}\pi \times 5^3 = \frac{500}{3}\pi
        \]
    \end{alertblock}
  \end{frame}
  
  
  \begin{frame}{典型例题}
    \begin{exampleblock}{例题2}
        球的截面圆面积为 \(9\pi\),球心到截面的距离为 \(4\),求球的半径。
    \end{exampleblock}
    
    \pause
    
    \begin{alertblock}{解答}
        \[
        \text{截面圆半径:} \pi r^2 = 9\pi \implies r = 3
        \]
        \[
        \text{由勾股定理:} R = \sqrt{r^2 + d^2} = \sqrt{3^2 + 4^2} = 5
        \]
    \end{alertblock}
  \end{frame}
  
  \begin{frame}{典型例题}
    \begin{exampleblock}{例题3}
        已知球的半径为 \(5\),求其表面积和体积之比。
    \end{exampleblock}
    
    \pause
    
    \begin{alertblock}{解答}
        \[
        \text{表面积:} S = 4\pi R^2 = 4\pi \times 5^2 = 100\pi
        \]
        \[
        \text{体积:} V = \frac{4}{3}\pi R^3 = \frac{4}{3}\pi \times 5^3 = \frac{500}{3}\pi
        \]
        \[
        \text{比值:} \frac{S}{V} = \frac{100\pi}{\frac{500}{3}\pi} = \frac{3}{5}
        \]
    \end{alertblock}
  \end{frame}
  
  
  \begin{frame}{典型例题}
    \begin{exampleblock}{例题4}
        正方体的外接球体积为 \(36\pi\),求正方体的棱长。
    \end{exampleblock}
    
    \pause
    
    \begin{alertblock}{解答}
        \[
        \text{设球半径为 } R,\frac{4}{3}\pi R^3 = 36\pi \implies R^3 = 27 \implies R = 3
        \]
        \[
        \text{正方体体对角线长 } l = 2R = 6
        \]
        \[
        \text{设正方体棱长为 } a,则 \sqrt{3}a = 6 \implies a = \frac{6}{\sqrt{3}} = 2\sqrt{3}
        \]
    \end{alertblock}
  \end{frame}
  
  
  \begin{frame}{典型例题}
    \begin{exampleblock}{例题5}
        两个球的表面积之比为 \(4:9\),求它们的体积之比。
    \end{exampleblock}
    
    \pause
    
    \begin{alertblock}{解答}
        \[
        \text{设两球半径为 } R_1, R_2,\frac{4\pi R_1^2}{4\pi R_2^2} = \frac{4}{9} \implies \left(\frac{R_1}{R_2}\right)^2 = \frac{4}{9} \implies \frac{R_1}{R_2} = \frac{2}{3}
        \]
        \[
        \text{体积之比:} \frac{\frac{4}{3}\pi R_1^3}{\frac{4}{3}\pi R_2^3} = \left(\frac{R_1}{R_2}\right)^3 = \left(\frac{2}{3}\right)^3 = \frac{8}{27}
        \]
    \end{alertblock}
  \end{frame}
  
  
  
  
  \begin{frame}{典型例题}
    \begin{exampleblock}{例题6}
        已知正方体内切球的表面积为 \(36\pi\),求正方体的棱长和体积。
    \end{exampleblock}
    
    \pause
    
    \begin{alertblock}{解答}
        \[
        \text{设内切球半径为 } r,\quad 4\pi r^2 = 36\pi \quad \Rightarrow \quad r^2 = 9 \quad \Rightarrow \quad r = 3
        \]
        \[
        \text{正方体棱长:} \quad a = 2r = 6
        \]
        \[
        \text{正方体体积:} \quad V = a^3 = 6^3 = 216
        \]
    \end{alertblock}
  \end{frame}
  
  
    \end{columns}
  \end{frame}
  
  \begin{frame}{球的截面性质}
    \begin{block}{定理}
        球心到截面的距离 \(d\)、截面圆半径 \(r\) 与球半径 \(R\) 满足:
        \[
        r = \sqrt{R^2 - d^2}
        \]
    \end{block} 
    
    \begin{columns}
        \column{0.6\textwidth}
        \begin{exampleblock}{特殊截面}
            \begin{enumerate}
                \item 当 \(d = 0\) 时,\(r = R\),截面为\textcolor{blue}{大圆}
                \item 当 \(0 < d < R\) 时,截面为\textcolor{red}{小圆}
            \end{enumerate}
        \end{exampleblock}
        
        \column{0.4\textwidth}
        \begin{tikzpicture}[scale=2,
    y={(-0,-0.2cm)}, % 设置 x 轴方向
    x={(1cm,0cm)},            % 设置 y 轴方向
    z={(0cm,1cm)}             % 设置 z 轴方向
]%斜二测画法
    % 定义圆柱的参数
    \def\radius{1} % 圆锥底面半径
    \def\height{2} % 圆锥的高度
    %这只是个一个文本替换
    % \def\littleradius{{\radius*sqrt(3)/2}} % 底面小圆半径
    \pgfmathsetmacro{\littleradius}{\radius*sqrt(3)/2}


    %球心
    \coordinate (O) at (0,0,0);
    %小圆圆心
    \coordinate (O') at (0,0,-\radius/2);
    %小圆上一点
    \coordinate (A) at ({\littleradius*sqrt(2)/2},{\littleradius*sqrt(2)/2},-\radius/2);


    \node[above] at (O) {$O$};
    \node[left] at (O') {$O'$};
    % \node[below] at (A) {$A$};
    \draw[dashed,red,thick] (O) -- node[above]{$R$} (A);
    \draw[dashed,blue,thick] (O') -- node[above]{$r$} (A);
    \draw[dashed,blue,thick] (O') -- node[left]{$d$} (O);



      
        % 绘制正面圆
        \begin{scope}[canvas is xz plane at y=0]
            \draw[thick] (0,0) circle (\radius);
        \end{scope}


        % 绘制水平圆
        \begin{scope}[canvas is xy plane at z=0]
            \draw [blue,thick,dashed] (\radius,0) arc (0:-180:\radius) ;
            \draw [blue,thick] (\radius,0) arc (0:180:\radius) ;
        \end{scope}


        % 绘制水平小圆 这个小圆刚好是圆心下降一半的半径的位置
        \begin{scope}[canvas is xy plane at z=-\radius/2]
            \draw [red,thick,dashed] (\littleradius,0) arc (0:-180:\littleradius) ;
            \draw [red,thick] (\littleradius,0) arc (0:180:\littleradius) ;
        \end{scope}


        % 绘制圆心
        \begin{scope}
            \fill[fill=black] (O) circle (1pt);
            \fill[fill=black] (O') circle (1pt);
            \fill[fill=red] (A) circle (1pt);
        \end{scope}
        

        
        %问题 怎么根据比例绘制圆弧上的点,我想要在圆弧上描点来连接球心.


        % 绘制竖直圆 % 竖直圆有问题,不绘制
        % 因为书本上的球,并不是使用斜二测画法绘制的
        % \begin{scope}[canvas is yz plane at x=0]
        %     \draw [blue,thick,dashed] (\radius,0,0) arc (0:-180:\radius) ;
            
        %     \draw [blue,thick] (\radius,0,0) arc (0:180:\radius) ;
        % \end{scope}
\end{tikzpicture}
    \end{columns}
  \end{frame}
  
  
  
  
  \begin{frame}{表面积与体积公式}
    \begin{block}{核心公式}
        \begin{itemize}
            \item 表面积:\[ S = 4\pi R^2 \]
            \item 体积:\[ V = \frac{4}{3}\pi R^3 \]
        \end{itemize}
    \end{block}
  \end{frame}
  
  
  \begin{frame}{球与多面体的关系}
    \begin{block}{外接球}
        多面体的所有顶点都在球面上,球心到各顶点距离相等。
        \begin{itemize}
            \item 正方体的外接球半径:\( R = \frac{\sqrt{3}}{2}a \)(\(a\) 为棱长)
            \item 长方体的外接球半径:\( R = \frac{\sqrt{a^2 + b^2 + c^2}}{2} \)
        \end{itemize}
    \end{block}
    
    \begin{block}{内切球}
        球与多面体的各面都相切,球心到各面距离相等。
        \begin{itemize}
            % \item 正四面体的内切球半径:\( r = \frac{\sqrt{6}}{12}a \)(\(a\) 为棱长)
            \item 正方体的内切球半径:\( r = \frac{a}{2} \)
        \end{itemize}
    \end{block}
  \end{frame}
  
  
  \begin{frame}{典型例题}
    \begin{exampleblock}{例题1}
        已知球的表面积为 \(100\pi\),求其体积。
    \end{exampleblock}
    
    \pause
    
    \begin{alertblock}{解答}
        \[
        S = 4\pi R^2 = 100\pi \implies R^2 = 25 \implies R = 5
        \]
        \[
        V = \frac{4}{3}\pi R^3 = \frac{4}{3}\pi \times 5^3 = \frac{500}{3}\pi
        \]
    \end{alertblock}
  \end{frame}
  
  
  \begin{frame}{典型例题}
    \begin{exampleblock}{例题2}
        球的截面圆面积为 \(9\pi\),球心到截面的距离为 \(4\),求球的半径。
    \end{exampleblock}
    
    \pause
    
    \begin{alertblock}{解答}
        \[
        \text{截面圆半径:} \pi r^2 = 9\pi \implies r = 3
        \]
        \[
        \text{由勾股定理:} R = \sqrt{r^2 + d^2} = \sqrt{3^2 + 4^2} = 5
        \]
    \end{alertblock}
  \end{frame}
  
  \begin{frame}{典型例题}
    \begin{exampleblock}{例题3}
        已知球的半径为 \(5\),求其表面积和体积之比。
    \end{exampleblock}
    
    \pause
    
    \begin{alertblock}{解答}
        \[
        \text{表面积:} S = 4\pi R^2 = 4\pi \times 5^2 = 100\pi
        \]
        \[
        \text{体积:} V = \frac{4}{3}\pi R^3 = \frac{4}{3}\pi \times 5^3 = \frac{500}{3}\pi
        \]
        \[
        \text{比值:} \frac{S}{V} = \frac{100\pi}{\frac{500}{3}\pi} = \frac{3}{5}
        \]
    \end{alertblock}
  \end{frame}
  
  
  \begin{frame}{典型例题}
    \begin{exampleblock}{例题4}
        正方体的外接球体积为 \(36\pi\),求正方体的棱长。
    \end{exampleblock}
    
    \pause
    
    \begin{alertblock}{解答}
        \[
        \text{设球半径为 } R,\frac{4}{3}\pi R^3 = 36\pi \implies R^3 = 27 \implies R = 3
        \]
        \[
        \text{正方体体对角线长 } l = 2R = 6
        \]
        \[
        \text{设正方体棱长为 } a,则 \sqrt{3}a = 6 \implies a = \frac{6}{\sqrt{3}} = 2\sqrt{3}
        \]
    \end{alertblock}
  \end{frame}
  
  
  \begin{frame}{典型例题}
    \begin{exampleblock}{例题5}
        两个球的表面积之比为 \(4:9\),求它们的体积之比。
    \end{exampleblock}
    
    \pause
    
    \begin{alertblock}{解答}
        \[
        \text{设两球半径为 } R_1, R_2,\frac{4\pi R_1^2}{4\pi R_2^2} = \frac{4}{9} \implies \left(\frac{R_1}{R_2}\right)^2 = \frac{4}{9} \implies \frac{R_1}{R_2} = \frac{2}{3}
        \]
        \[
        \text{体积之比:} \frac{\frac{4}{3}\pi R_1^3}{\frac{4}{3}\pi R_2^3} = \left(\frac{R_1}{R_2}\right)^3 = \left(\frac{2}{3}\right)^3 = \frac{8}{27}
        \]
    \end{alertblock}
  \end{frame}
  
  
  
  
  \begin{frame}{典型例题}
    \begin{exampleblock}{例题6}
        已知正方体内切球的表面积为 \(36\pi\),求正方体的棱长和体积。
    \end{exampleblock}
    
    \pause
    
    \begin{alertblock}{解答}
        \[
        \text{设内切球半径为 } r,\quad 4\pi r^2 = 36\pi \quad \Rightarrow \quad r^2 = 9 \quad \Rightarrow \quad r = 3
        \]
        \[
        \text{正方体棱长:} \quad a = 2r = 6
        \]
        \[
        \text{正方体体积:} \quad V = a^3 = 6^3 = 216
        \]
    \end{alertblock}
  \end{frame}
  
  
    \end{columns}
  \end{frame}
  
  \begin{frame}{球的截面性质}
    \begin{block}{定理}
        球心到截面的距离 \(d\)、截面圆半径 \(r\) 与球半径 \(R\) 满足:
        \[
        r = \sqrt{R^2 - d^2}
        \]
    \end{block} 
    
    \begin{columns}
        \column{0.6\textwidth}
        \begin{exampleblock}{特殊截面}
            \begin{enumerate}
                \item 当 \(d = 0\) 时,\(r = R\),截面为\textcolor{blue}{大圆}
                \item 当 \(0 < d < R\) 时,截面为\textcolor{red}{小圆}
            \end{enumerate}
        \end{exampleblock}
        
        \column{0.4\textwidth}
        \begin{tikzpicture}[scale=2,
    y={(-0,-0.2cm)}, % 设置 x 轴方向
    x={(1cm,0cm)},            % 设置 y 轴方向
    z={(0cm,1cm)}             % 设置 z 轴方向
]%斜二测画法
    % 定义圆柱的参数
    \def\radius{1} % 圆锥底面半径
    \def\height{2} % 圆锥的高度
    %这只是个一个文本替换
    % \def\littleradius{{\radius*sqrt(3)/2}} % 底面小圆半径
    \pgfmathsetmacro{\littleradius}{\radius*sqrt(3)/2}


    %球心
    \coordinate (O) at (0,0,0);
    %小圆圆心
    \coordinate (O') at (0,0,-\radius/2);
    %小圆上一点
    \coordinate (A) at ({\littleradius*sqrt(2)/2},{\littleradius*sqrt(2)/2},-\radius/2);


    \node[above] at (O) {$O$};
    \node[left] at (O') {$O'$};
    % \node[below] at (A) {$A$};
    \draw[dashed,red,thick] (O) -- node[above]{$R$} (A);
    \draw[dashed,blue,thick] (O') -- node[above]{$r$} (A);
    \draw[dashed,blue,thick] (O') -- node[left]{$d$} (O);



      
        % 绘制正面圆
        \begin{scope}[canvas is xz plane at y=0]
            \draw[thick] (0,0) circle (\radius);
        \end{scope}


        % 绘制水平圆
        \begin{scope}[canvas is xy plane at z=0]
            \draw [blue,thick,dashed] (\radius,0) arc (0:-180:\radius) ;
            \draw [blue,thick] (\radius,0) arc (0:180:\radius) ;
        \end{scope}


        % 绘制水平小圆 这个小圆刚好是圆心下降一半的半径的位置
        \begin{scope}[canvas is xy plane at z=-\radius/2]
            \draw [red,thick,dashed] (\littleradius,0) arc (0:-180:\littleradius) ;
            \draw [red,thick] (\littleradius,0) arc (0:180:\littleradius) ;
        \end{scope}


        % 绘制圆心
        \begin{scope}
            \fill[fill=black] (O) circle (1pt);
            \fill[fill=black] (O') circle (1pt);
            \fill[fill=red] (A) circle (1pt);
        \end{scope}
        

        
        %问题 怎么根据比例绘制圆弧上的点,我想要在圆弧上描点来连接球心.


        % 绘制竖直圆 % 竖直圆有问题,不绘制
        % 因为书本上的球,并不是使用斜二测画法绘制的
        % \begin{scope}[canvas is yz plane at x=0]
        %     \draw [blue,thick,dashed] (\radius,0,0) arc (0:-180:\radius) ;
            
        %     \draw [blue,thick] (\radius,0,0) arc (0:180:\radius) ;
        % \end{scope}
\end{tikzpicture}
    \end{columns}
  \end{frame}
  
  
  
  
  \begin{frame}{表面积与体积公式}
    \begin{block}{核心公式}
        \begin{itemize}
            \item 表面积:\[ S = 4\pi R^2 \]
            \item 体积:\[ V = \frac{4}{3}\pi R^3 \]
        \end{itemize}
    \end{block}
  \end{frame}
  
  
  \begin{frame}{球与多面体的关系}
    \begin{block}{外接球}
        多面体的所有顶点都在球面上,球心到各顶点距离相等。
        \begin{itemize}
            \item 正方体的外接球半径:\( R = \frac{\sqrt{3}}{2}a \)(\(a\) 为棱长)
            \item 长方体的外接球半径:\( R = \frac{\sqrt{a^2 + b^2 + c^2}}{2} \)
        \end{itemize}
    \end{block}
    
    \begin{block}{内切球}
        球与多面体的各面都相切,球心到各面距离相等。
        \begin{itemize}
            % \item 正四面体的内切球半径:\( r = \frac{\sqrt{6}}{12}a \)(\(a\) 为棱长)
            \item 正方体的内切球半径:\( r = \frac{a}{2} \)
        \end{itemize}
    \end{block}
  \end{frame}
  
  
  \begin{frame}{典型例题}
    \begin{exampleblock}{例题1}
        已知球的表面积为 \(100\pi\),求其体积。
    \end{exampleblock}
    
    \pause
    
    \begin{alertblock}{解答}
        \[
        S = 4\pi R^2 = 100\pi \implies R^2 = 25 \implies R = 5
        \]
        \[
        V = \frac{4}{3}\pi R^3 = \frac{4}{3}\pi \times 5^3 = \frac{500}{3}\pi
        \]
    \end{alertblock}
  \end{frame}
  
  
  \begin{frame}{典型例题}
    \begin{exampleblock}{例题2}
        球的截面圆面积为 \(9\pi\),球心到截面的距离为 \(4\),求球的半径。
    \end{exampleblock}
    
    \pause
    
    \begin{alertblock}{解答}
        \[
        \text{截面圆半径:} \pi r^2 = 9\pi \implies r = 3
        \]
        \[
        \text{由勾股定理:} R = \sqrt{r^2 + d^2} = \sqrt{3^2 + 4^2} = 5
        \]
    \end{alertblock}
  \end{frame}
  
  \begin{frame}{典型例题}
    \begin{exampleblock}{例题3}
        已知球的半径为 \(5\),求其表面积和体积之比。
    \end{exampleblock}
    
    \pause
    
    \begin{alertblock}{解答}
        \[
        \text{表面积:} S = 4\pi R^2 = 4\pi \times 5^2 = 100\pi
        \]
        \[
        \text{体积:} V = \frac{4}{3}\pi R^3 = \frac{4}{3}\pi \times 5^3 = \frac{500}{3}\pi
        \]
        \[
        \text{比值:} \frac{S}{V} = \frac{100\pi}{\frac{500}{3}\pi} = \frac{3}{5}
        \]
    \end{alertblock}
  \end{frame}
  
  
  \begin{frame}{典型例题}
    \begin{exampleblock}{例题4}
        正方体的外接球体积为 \(36\pi\),求正方体的棱长。
    \end{exampleblock}
    
    \pause
    
    \begin{alertblock}{解答}
        \[
        \text{设球半径为 } R,\frac{4}{3}\pi R^3 = 36\pi \implies R^3 = 27 \implies R = 3
        \]
        \[
        \text{正方体体对角线长 } l = 2R = 6
        \]
        \[
        \text{设正方体棱长为 } a,则 \sqrt{3}a = 6 \implies a = \frac{6}{\sqrt{3}} = 2\sqrt{3}
        \]
    \end{alertblock}
  \end{frame}
  
  
  \begin{frame}{典型例题}
    \begin{exampleblock}{例题5}
        两个球的表面积之比为 \(4:9\),求它们的体积之比。
    \end{exampleblock}
    
    \pause
    
    \begin{alertblock}{解答}
        \[
        \text{设两球半径为 } R_1, R_2,\frac{4\pi R_1^2}{4\pi R_2^2} = \frac{4}{9} \implies \left(\frac{R_1}{R_2}\right)^2 = \frac{4}{9} \implies \frac{R_1}{R_2} = \frac{2}{3}
        \]
        \[
        \text{体积之比:} \frac{\frac{4}{3}\pi R_1^3}{\frac{4}{3}\pi R_2^3} = \left(\frac{R_1}{R_2}\right)^3 = \left(\frac{2}{3}\right)^3 = \frac{8}{27}
        \]
    \end{alertblock}
  \end{frame}
  
  
  
  
  \begin{frame}{典型例题}
    \begin{exampleblock}{例题6}
        已知正方体内切球的表面积为 \(36\pi\),求正方体的棱长和体积。
    \end{exampleblock}
    
    \pause
    
    \begin{alertblock}{解答}
        \[
        \text{设内切球半径为 } r,\quad 4\pi r^2 = 36\pi \quad \Rightarrow \quad r^2 = 9 \quad \Rightarrow \quad r = 3
        \]
        \[
        \text{正方体棱长:} \quad a = 2r = 6
        \]
        \[
        \text{正方体体积:} \quad V = a^3 = 6^3 = 216
        \]
    \end{alertblock}
  \end{frame}
  
  
    \end{columns}
  \end{frame}
  
  \begin{frame}{球的截面性质}
    \begin{block}{定理}
        球心到截面的距离 \(d\)、截面圆半径 \(r\) 与球半径 \(R\) 满足:
        \[
        r = \sqrt{R^2 - d^2}
        \]
    \end{block} 
    
    \begin{columns}
        \column{0.6\textwidth}
        \begin{exampleblock}{特殊截面}
            \begin{enumerate}
                \item 当 \(d = 0\) 时,\(r = R\),截面为\textcolor{blue}{大圆}
                \item 当 \(0 < d < R\) 时,截面为\textcolor{red}{小圆}
            \end{enumerate}
        \end{exampleblock}
        
        \column{0.4\textwidth}
        \begin{tikzpicture}[scale=2,
    y={(-0,-0.2cm)}, % 设置 x 轴方向
    x={(1cm,0cm)},            % 设置 y 轴方向
    z={(0cm,1cm)}             % 设置 z 轴方向
]%斜二测画法
    % 定义圆柱的参数
    \def\radius{1} % 圆锥底面半径
    \def\height{2} % 圆锥的高度
    %这只是个一个文本替换
    % \def\littleradius{{\radius*sqrt(3)/2}} % 底面小圆半径
    \pgfmathsetmacro{\littleradius}{\radius*sqrt(3)/2}


    %球心
    \coordinate (O) at (0,0,0);
    %小圆圆心
    \coordinate (O') at (0,0,-\radius/2);
    %小圆上一点
    \coordinate (A) at ({\littleradius*sqrt(2)/2},{\littleradius*sqrt(2)/2},-\radius/2);


    \node[above] at (O) {$O$};
    \node[left] at (O') {$O'$};
    % \node[below] at (A) {$A$};
    \draw[dashed,red,thick] (O) -- node[above]{$R$} (A);
    \draw[dashed,blue,thick] (O') -- node[above]{$r$} (A);
    \draw[dashed,blue,thick] (O') -- node[left]{$d$} (O);



      
        % 绘制正面圆
        \begin{scope}[canvas is xz plane at y=0]
            \draw[thick] (0,0) circle (\radius);
        \end{scope}


        % 绘制水平圆
        \begin{scope}[canvas is xy plane at z=0]
            \draw [blue,thick,dashed] (\radius,0) arc (0:-180:\radius) ;
            \draw [blue,thick] (\radius,0) arc (0:180:\radius) ;
        \end{scope}


        % 绘制水平小圆 这个小圆刚好是圆心下降一半的半径的位置
        \begin{scope}[canvas is xy plane at z=-\radius/2]
            \draw [red,thick,dashed] (\littleradius,0) arc (0:-180:\littleradius) ;
            \draw [red,thick] (\littleradius,0) arc (0:180:\littleradius) ;
        \end{scope}


        % 绘制圆心
        \begin{scope}
            \fill[fill=black] (O) circle (1pt);
            \fill[fill=black] (O') circle (1pt);
            \fill[fill=red] (A) circle (1pt);
        \end{scope}
        

        
        %问题 怎么根据比例绘制圆弧上的点,我想要在圆弧上描点来连接球心.


        % 绘制竖直圆 % 竖直圆有问题,不绘制
        % 因为书本上的球,并不是使用斜二测画法绘制的
        % \begin{scope}[canvas is yz plane at x=0]
        %     \draw [blue,thick,dashed] (\radius,0,0) arc (0:-180:\radius) ;
            
        %     \draw [blue,thick] (\radius,0,0) arc (0:180:\radius) ;
        % \end{scope}
\end{tikzpicture}
    \end{columns}
  \end{frame}
  
  
  
  
  \begin{frame}{表面积与体积公式}
    \begin{block}{核心公式}
        \begin{itemize}
            \item 表面积:\[ S = 4\pi R^2 \]
            \item 体积:\[ V = \frac{4}{3}\pi R^3 \]
        \end{itemize}
    \end{block}
  \end{frame}
  
  
  \begin{frame}{球与多面体的关系}
    \begin{block}{外接球}
        多面体的所有顶点都在球面上,球心到各顶点距离相等。
        \begin{itemize}
            \item 正方体的外接球半径:\( R = \frac{\sqrt{3}}{2}a \)(\(a\) 为棱长)
            \item 长方体的外接球半径:\( R = \frac{\sqrt{a^2 + b^2 + c^2}}{2} \)
        \end{itemize}
    \end{block}
    
    \begin{block}{内切球}
        球与多面体的各面都相切,球心到各面距离相等。
        \begin{itemize}
            % \item 正四面体的内切球半径:\( r = \frac{\sqrt{6}}{12}a \)(\(a\) 为棱长)
            \item 正方体的内切球半径:\( r = \frac{a}{2} \)
        \end{itemize}
    \end{block}
  \end{frame}
  
  
  \begin{frame}{典型例题}
    \begin{exampleblock}{例题1}
        已知球的表面积为 \(100\pi\),求其体积。
    \end{exampleblock}
    
    \pause
    
    \begin{alertblock}{解答}
        \[
        S = 4\pi R^2 = 100\pi \implies R^2 = 25 \implies R = 5
        \]
        \[
        V = \frac{4}{3}\pi R^3 = \frac{4}{3}\pi \times 5^3 = \frac{500}{3}\pi
        \]
    \end{alertblock}
  \end{frame}
  
  
  \begin{frame}{典型例题}
    \begin{exampleblock}{例题2}
        球的截面圆面积为 \(9\pi\),球心到截面的距离为 \(4\),求球的半径。
    \end{exampleblock}
    
    \pause
    
    \begin{alertblock}{解答}
        \[
        \text{截面圆半径:} \pi r^2 = 9\pi \implies r = 3
        \]
        \[
        \text{由勾股定理:} R = \sqrt{r^2 + d^2} = \sqrt{3^2 + 4^2} = 5
        \]
    \end{alertblock}
  \end{frame}
  
  \begin{frame}{典型例题}
    \begin{exampleblock}{例题3}
        已知球的半径为 \(5\),求其表面积和体积之比。
    \end{exampleblock}
    
    \pause
    
    \begin{alertblock}{解答}
        \[
        \text{表面积:} S = 4\pi R^2 = 4\pi \times 5^2 = 100\pi
        \]
        \[
        \text{体积:} V = \frac{4}{3}\pi R^3 = \frac{4}{3}\pi \times 5^3 = \frac{500}{3}\pi
        \]
        \[
        \text{比值:} \frac{S}{V} = \frac{100\pi}{\frac{500}{3}\pi} = \frac{3}{5}
        \]
    \end{alertblock}
  \end{frame}
  
  
  \begin{frame}{典型例题}
    \begin{exampleblock}{例题4}
        正方体的外接球体积为 \(36\pi\),求正方体的棱长。
    \end{exampleblock}
    
    \pause
    
    \begin{alertblock}{解答}
        \[
        \text{设球半径为 } R,\frac{4}{3}\pi R^3 = 36\pi \implies R^3 = 27 \implies R = 3
        \]
        \[
        \text{正方体体对角线长 } l = 2R = 6
        \]
        \[
        \text{设正方体棱长为 } a,则 \sqrt{3}a = 6 \implies a = \frac{6}{\sqrt{3}} = 2\sqrt{3}
        \]
    \end{alertblock}
  \end{frame}
  
  
  \begin{frame}{典型例题}
    \begin{exampleblock}{例题5}
        两个球的表面积之比为 \(4:9\),求它们的体积之比。
    \end{exampleblock}
    
    \pause
    
    \begin{alertblock}{解答}
        \[
        \text{设两球半径为 } R_1, R_2,\frac{4\pi R_1^2}{4\pi R_2^2} = \frac{4}{9} \implies \left(\frac{R_1}{R_2}\right)^2 = \frac{4}{9} \implies \frac{R_1}{R_2} = \frac{2}{3}
        \]
        \[
        \text{体积之比:} \frac{\frac{4}{3}\pi R_1^3}{\frac{4}{3}\pi R_2^3} = \left(\frac{R_1}{R_2}\right)^3 = \left(\frac{2}{3}\right)^3 = \frac{8}{27}
        \]
    \end{alertblock}
  \end{frame}
  
  
  
  
  \begin{frame}{典型例题}
    \begin{exampleblock}{例题6}
        已知正方体内切球的表面积为 \(36\pi\),求正方体的棱长和体积。
    \end{exampleblock}
    
    \pause
    
    \begin{alertblock}{解答}
        \[
        \text{设内切球半径为 } r,\quad 4\pi r^2 = 36\pi \quad \Rightarrow \quad r^2 = 9 \quad \Rightarrow \quad r = 3
        \]
        \[
        \text{正方体棱长:} \quad a = 2r = 6
        \]
        \[
        \text{正方体体积:} \quad V = a^3 = 6^3 = 216
        \]
    \end{alertblock}
  \end{frame}
  
  