\documentclass[UTF8]{beamer}
\usepackage{ctex}
\usepackage{amsmath,amssymb}
\usepackage{graphicx}
\usepackage{tikz}
\usepackage{booktabs}
\usepackage{xcolor}

% 设置主题
\usetheme{Madrid}
% \usecolortheme{beaver}
\setbeamertemplate{navigation symbols}{} % 移除导航栏

% 自定义颜色
\definecolor{titlecolor}{RGB}{42, 100, 150}
\definecolor{accentcolor}{RGB}{237, 125, 49}

% 自定义标题页样式
\setbeamertemplate{title page}
{
  \begin{center}
    \vspace*{1cm}
    \usebeamercolor[fg]{title}
    \usebeamerfont{title}
    \inserttitle\par
    \vspace*{0.5cm}
    \usebeamercolor[fg]{subtitle}
    \usebeamerfont{subtitle}
    \insertsubtitle\par
    \vspace*{2cm}
    \usebeamercolor[fg]{author}
    \usebeamerfont{author}
    \insertauthor\par
    \vspace*{0.2cm}
    \usebeamercolor[fg]{date}
    \usebeamerfont{date}
    \insertdate\par
  \end{center}
}

% 自定义块样式
\setbeamertemplate{blocks}[rounded][shadow=true]

\begin{document}

% 标题页
\title{充要条件}
\subtitle{逻辑与数学中的核心概念}
\author{豆包编程助手}
\date{\today}
\begin{frame}
  \titlepage
\end{frame}

% 目录页
\begin{frame}{目录}
  \tableofcontents
\end{frame}

% 定义与符号
\section{定义与符号}
\begin{frame}{定义与符号}
  \begin{block}{基本概念}
    \begin{itemize}
      \item<1-> \textbf{充分条件(Sufficient Condition)}:若命题 \( P \) 成立时,命题 \( Q \) 一定成立,则 \( P \) 是 \( Q \) 的充分条件,记作 \( P \Rightarrow Q \)。
      \item<2-> \textbf{必要条件(Necessary Condition)}:若命题 \( Q \) 成立时,命题 \( P \) 一定成立,则 \( P \) 是 \( Q \) 的必要条件,记作 \( Q \Rightarrow P \)(或 \( P \Leftarrow Q \))。
      \item<3-> \textbf{充要条件(Necessary and Sufficient Condition)}:若 \( P \) 既是 \( Q \) 的充分条件又是必要条件,即 \( P \Leftrightarrow Q \),读作“\( P \) 当且仅当 \( Q \)”(iff)。
    \end{itemize}
  \end{block}
  
  \begin{exampleblock}{举例}
    \begin{itemize}
      \item<4-> “一个数是偶数”是“该数能被2整除”的充分条件。
      \item<5-> “能被2整除”是“一个数是偶数”的必要条件。
      \item<6-> “一个数是偶数”当且仅当“该数能被2整除”。
    \end{itemize}
  \end{exampleblock}
\end{frame}

% 逻辑结构
\section{充要条件的逻辑结构}
\begin{frame}{逻辑结构}
  \begin{block}{真值表分析}
    \begin{center}
      \begin{tabular}{cccccc}
        \toprule
        \( P \) & \( Q \) & \( P \Rightarrow Q \) & \( Q \Rightarrow P \) & \( P \Leftrightarrow Q \) \\
        \midrule
        T & T & T & T & T \\
        T & F & F & T & F \\
        F & T & T & F & F \\
        F & F & T & T & T \\
        \bottomrule
      \end{tabular}
    \end{center}
  \end{block}
  
  \begin{alertblock}{关键结论}
    \( P \Leftrightarrow Q \) 当且仅当 \( P \Rightarrow Q \) 且 \( Q \Rightarrow P \)。
  \end{alertblock}
  
  \begin{tikzpicture}[overlay, remember picture, xshift=5cm, yshift=-3cm]
    \node[opacity=0.2] {
      \begin{tikzpicture}
        \draw[thick, titlecolor] (0,0) circle (1cm);
        \draw[thick, accentcolor] (1.5,0) circle (1cm);
        \draw[thick, red] (-0.5,-1.5) -- (2.5,1.5);
        \draw[thick, blue] (2.5,-1.5) -- (-0.5,1.5);
      \end{tikzpicture}
    };
  \end{tikzpicture}
\end{frame}

% 举例说明
\section{举例说明}
\begin{frame}{举例说明(数学篇)}
  \begin{exampleblock}{例1:三角形全等}
    \begin{itemize}
      \item \( P \): 两个三角形三边对应相等(SSS)。
      \item \( Q \): 两个三角形全等。
      \item \textbf{结论}: \( P \Leftrightarrow Q \),即“三边对应相等”是“三角形全等”的充要条件。
    \end{itemize}
  \end{exampleblock}
  
  \begin{exampleblock}{例2:一元二次方程}
    \begin{itemize}
      \item \( P \): 方程 \( ax^2 + bx + c = 0 \) 有实根(\( a \neq 0 \))。
      \item \( Q \): 判别式 \( \Delta = b^2 - 4ac \geq 0 \)。
      \item \textbf{结论}: \( P \Leftrightarrow Q \)。
    \end{itemize}
  \end{exampleblock}
  
  \begin{tikzpicture}[overlay, remember picture, xshift=8cm, yshift=-2cm]
    \node[opacity=0.2] {
      \begin{tikzpicture}
        \draw[thick] (0,0) -- (2,0) -- (1,1.73) -- cycle;
        \draw[thick] (3,0) -- (5,0) -- (4,1.73) -- cycle;
        \draw[<->] (1,0) -- (4,0);
      \end{tikzpicture}
    };
  \end{tikzpicture}
\end{frame}

\begin{frame}{举例说明(逻辑与生活篇)}
  \begin{exampleblock}{例3:闰年的定义}
    \begin{itemize}
      \item \( P \): 年份是闰年。
      \item \( Q \): 年份能被4整除但不能被100整除,或能被400整除。
      \item \textbf{结论}: \( P \Leftrightarrow Q \)。
    \end{itemize}
  \end{exampleblock}
  
  \begin{alertblock}{例4:充要条件的误用}
    \begin{itemize}
      \item \textbf{错误表述}: “下雨”是“地湿”的充要条件。
      \item \textbf{分析}:
        \begin{itemize}
          \item 充分性:下雨会导致地湿(\( P \Rightarrow Q \))。
          \item 必要性:地湿可能由洒水等原因引起(\( Q \nRightarrow P \))。
        \end{itemize}
      \item \textbf{结论}: “下雨”仅是“地湿”的充分条件,而非必要条件。
    \end{itemize}
  \end{alertblock}
\end{frame}

% 证明策略
\section{充要条件的证明策略}
\begin{frame}{充要条件的证明策略}
  \begin{block}{方法1:分别证明充分性和必要性}
    \begin{enumerate}
      \item<1-> \textbf{证明充分性(\( P \Rightarrow Q \))}:假设 \( P \) 成立,推导 \( Q \) 成立。
      \item<2-> \textbf{证明必要性(\( Q \Rightarrow P \))}:假设 \( Q \) 成立,推导 \( P \) 成立。
    \end{enumerate}
  \end{block}
  
  \begin{block}{方法2:等价转换}
    通过逻辑等价式或定理,将 \( P \) 和 \( Q \) 转化为同一条件。
    \begin{example}
      证明“函数 \( f(x) \) 在 \( x_0 \) 连续”的充要条件是“左右极限存在且等于 \( f(x_0) \)”。
    \end{example}
  \end{block}
  
  \begin{tikzpicture}[overlay, remember picture, xshift=5cm, yshift=-3cm]
    \node[opacity=0.2] {
      \begin{tikzpicture}
        \draw[thick, titlecolor, ->] (0,0) -- (2,0);
        \draw[thick, accentcolor, ->] (2,0) -- (0,0);
        \node at (1,0.5) {\( P \Leftrightarrow Q \)};
      \end{tikzpicture}
    };
  \end{tikzpicture}
\end{frame}

% 应用场景
\section{应用场景}
\begin{frame}{应用场景}
  \begin{exampleblock}{1. 数学定理}
    \begin{itemize}
      \item 微积分:函数可导的充要条件(左右导数存在且相等)。
      \item 线性代数:矩阵可逆的充要条件(行列式不为零)。
    \end{itemize}
  \end{exampleblock}
  
  \begin{exampleblock}{2. 计算机科学}
    \begin{itemize}
      \item 算法正确性:算法终止的充要条件(循环不变式成立且终止条件满足)。
      \item 数据库:函数依赖中的充要条件(BC范式的判定)。
    \end{itemize}
  \end{exampleblock}
  
  \begin{exampleblock}{3. 日常生活}
    \begin{itemize}
      \item 法律:犯罪成立的充要条件(主体、主观方面、客体、客观方面均满足)。
    \end{itemize}
  \end{exampleblock}
\end{frame}

% 总结
\section{总结与练习}
\begin{frame}{总结}
  \begin{block}{核心要点}
    \begin{itemize}
      \item<1-> 充要条件是逻辑等价的最强关系:\( P \Leftrightarrow Q \)。
      \item<2-> 证明时需分别验证充分性和必要性。
      \item<3-> 避免混淆充分条件与必要条件(如“只有…才…”表示必要性)。
    \end{itemize}
  \end{block}
  
  \begin{alertblock}{公式回顾}
    \[
    P \Leftrightarrow Q \quad \equiv \quad (P \Rightarrow Q) \land (Q \Rightarrow P)
    \]
  \end{alertblock}
  
  \begin{tikzpicture}[overlay, remember picture, xshift=5cm, yshift=-2cm]
    \node[opacity=0.2] {
      \begin{tikzpicture}
        \draw[thick, titlecolor] (0,0) circle (1.5cm);
        \draw[thick, accentcolor] (0,0) circle (1cm);
        \node at (0,0) {\( P \Leftrightarrow Q \)};
      \end{tikzpicture}
    };
  \end{tikzpicture}
\end{frame}

% 致谢
\begin{frame}
  \begin{center}
    \Huge{\textcolor{titlecolor}{感谢观看!}}
    \vspace{2cm}
    
    \Large{\textcolor{accentcolor}{问题与讨论}}
    \vspace{1cm}
    
    \textcolor{gray}{\(\heartsuit\) 豆包编程助手}
  \end{center}
\end{frame}

\end{document}
