% \usetikzlibrary{decorations.pathreplacing}%使用大括号
% \usetikzlibrary{calligraphy} %calligraphy 书法风格
% % \usetikzlibrary{positioning}
% \usetikzlibrary{calc}
% \usetikzlibrary{backgrounds} %特定的绘图操作置于背景层
% \usetikzlibrary{3d}

% 将代码中的顶点名称 A、B、C、D 替换为 D、C、B、A
\begin{tikzpicture}[scale=3,
    y={(-0.353cm,-0.353cm)}, % 设置 x 轴方向
    x={(1cm,0cm)},            % 设置 y 轴方向
    z={(0cm,1cm)}             % 设置 z 轴方向
]%斜二测画法
\coordinate (D) at (0,0,0);
\coordinate (C) at (1,0,0);
\coordinate (B) at (1,1,0);
\coordinate (A) at (0,1,0);
\coordinate (V) at (0.5,0.5,1);

\coordinate (H) at (0.5,0.5,0);
% 计算中点
\coordinate (H') at ($(A)!0.5!(B)$);
\coordinate (H'') at ($(C)!0.5!(B)$);
%填充底面
\fill[lightgray,opacity=0.5] (D) -- (C) -- (B) -- (A) -- cycle;

% 绘制底面
\draw (C) -- (B) -- (A);

% 绘制侧面
\draw[dashed] (D) -- (V);
\draw[dashed] (D) -- (A);
\draw[dashed] (D) -- (C);
\draw (C) -- (V);
\draw (B) -- (V);
\draw (A) -- (V);


\draw[thick,blue,dashed] (H) -- (V); %棱柱的高

\draw[thick,red,dashed] (H') -- (V); %棱柱的斜高
\draw[thick,red,dashed] (H'') -- (V); %棱柱的斜高

% 标记顶点
\node[ left] at (D) {$D$};
\node[ right] at (C) {$C$};
\node[ right] at (B) {$B$};
\node[left] at (A) {$A$};
\node[above] at (V) {$V$};
\node[right] at (H) {$H$};

%绘制圆点
\fill[blue] (V) circle (0.75pt);
\fill[blue] (H) circle (0.75pt);
\fill[red] (H') circle (0.75pt);
\fill[red] (H'') circle (0.75pt);
% \node[circle, fill=blue, inner sep=1.5pt] at (V) {};
% \node[circle, fill=blue, inner sep=1.5pt] at (H) {};
\end{tikzpicture}
