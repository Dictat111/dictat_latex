\documentclass{article}
\usepackage{ctex}
\usepackage{xcolor}
\usepackage[]{exercise} % exerciseonly 表示不显示答案
\usepackage{amsmath}

\title{圆的方程练习题}
\date{\today} % 自动生成当前日期(可自定义)

% 重新定义例题和解答的名称
\renewcommand{\ExerciseName}{题目}
\renewcommand{\ExerciseListName}{题}
\renewcommand{\AnswerName}{解答}
\renewcommand{\AnswerListName}{\textcolor{blue}{解}}
\begin{document}
% \thispagestyle{empty} %禁用所有页码

% \section*{}
\maketitle % 生成标题




\begin{ExerciseList}
\Exercise
求以点\((2, -3)\)为圆心,半径为\(5\)的圆的标准方程。
\Answer
根据圆标准方程\((x - a)^2+(y - b)^2 = r^2\),\(a = 2\),\(b=-3\),\(r = 5\),得\((x - 2)^2+(y + 3)^2 = 25\)。

\Exercise
已知圆的方程为\((x + 1)^2+(y - 2)^2 = 9\),求圆心坐标和半径。
\Answer
圆标准方程\((x - a)^2+(y - b)^2 = r^2\),此方程中\(a=-1\),\(b = 2\),\(r = 3\),所以圆心\((-1,2)\),半径\(3\)。

\Exercise
若圆经过点\(A(1,2)\),\(B(3,4)\),且圆心在直线\(x - y + 1 = 0\)上,求该圆的方程。
\Answer
设圆方程\((x - a)^2+(y - b)^2 = r^2\),由已知得\(\begin{cases}(1 - a)^2+(2 - b)^2 = r^2\\(3 - a)^2+(4 - b)^2 = r^2\\a - b+1 = 0\end{cases}\),前两式相减得\(a + b = 5\),联立\(\begin{cases}a + b = 5\\a - b+1 = 0\end{cases}\)解得\(a = 2\),\(b = 3\),\(r^2 = 2\),圆方程为\((x - 2)^2+(y - 3)^2 = 2\)。

\Exercise
求过点\((0,0)\),\((1,1)\),\((2,0)\)的圆的方程。
\Answer
设圆一般方程\(x^{2}+y^{2}+Dx + Ey+F = 0\),代入三点得\(\begin{cases}F = 0\\1 + 1+D + E+F = 0\\4+2D+F = 0\end{cases}\),解得\(D=-2\),\(E = 0\),\(F = 0\),圆方程为\(x^{2}+y^{2}-2x = 0\)。

\Exercise
圆\(x^2 + y^2 - 4x + 6y - 3 = 0\)的圆心坐标和半径分别是多少?
\Answer
配方得\(x^{2}-4x + 4+y^{2}+6y+9=3 + 4+9\),即\((x - 2)^2+(y + 3)^2 = 16\),圆心\((2,-3)\),半径\(4\)。
\end{ExerciseList}

\end{document}