
\subsection{棱锥}




\begin{frame}{题目1:表面积、体积计算}
  已知正四棱锥的底面边长为 \(a = 6 \, \text{cm}\),侧棱长为 \(L = 5 \, \text{cm}\),求其体积和表面积。

  \pause

  计算底面中心到边的距离 \(r\):
  \[
  r = \frac{a}{2} = \frac{6}{2} = 3 \, \text{cm}
  \]

  计算斜高 \(l\):
 \[
 l = \sqrt{L^2 - r^2} = \sqrt{5^2 - 3^2} = 4 \, \text{cm}
 \]

 计算四棱锥的高 \(h\):
 \[
 h = \sqrt{l^2 - r^2} = \sqrt{4^2 - 3^2} = \sqrt{7} \, \text{cm}
 \]

 计算体积 \(V\):
 \[
 V = \frac{1}{3} \cdot a^2 \cdot h = \frac{1}{3} \times 6^2 \times \sqrt{7} = 12\sqrt{7} \, \text{cm}^3
 \]

\end{frame}


\begin{frame}{题目2:表面积计算}
  \begin{block}{题目}
      正四棱锥底面边长 \(a = 5 \, \text{cm}\),斜高 \(l = 4 \, \text{cm}\),求其表面积。
  \end{block}
  
  \pause
  
  \begin{alertblock}{解答步骤}
      \begin{enumerate}
          \item 底面面积:\( S_{\text{底}} = a^2 = 5^2 = 25 \, \text{cm}^2 \)
          \item 侧面面积:\( S_{\text{侧}} = \frac{1}{2} a l = \frac{1}{2} \times 5 \times 4 = 10 \, \text{cm}^2 \)
          \item 总表面积:\( S = S_{\text{底}} + 4 S_{\text{侧}} = 25 + 4 \times 10 = 65 \, \text{cm}^2 \)
      \end{enumerate}
  \end{alertblock}
\end{frame}


\begin{frame}{题目3:侧棱长计算}
  \begin{block}{题目}
      已知正四棱锥底面边长 \(a = 6 \, \text{cm}\),高 \(h = 4 \, \text{cm}\),求侧棱长 \(L\)。
  \end{block}
  
  \pause
  
  \begin{alertblock}{解答步骤}
      \begin{enumerate}
          \item 底面中心到边的距离:\( r = \frac{a}{2} = \frac{6}{2} = 3 \, \text{cm} \)
          \item 斜高 :\( l = \sqrt{h^2 + r^2} = 5 \text{cm} \)
          \item $L = \sqrt{l^2 + r^2} = \sqrt{34} \text{cm} $
      \end{enumerate}
  \end{alertblock}
\end{frame}


\begin{frame}{题目4:综合应用}
  \begin{block}{题目}
      正四棱锥表面积为 \(144 \, \text{cm}^2\),底面边长 \(a = 6 \, \text{cm}\),求体积。
  \end{block}
  
  \pause
  
  \begin{alertblock}{解答步骤}
      \begin{enumerate}
          \item 底面面积:\( S_{\text{底}} = a^2 = 6^2 = 36 \, \text{cm}^2 \)
          \item 侧面积总和:\( S_{\text{侧总}} = 144 - 36 = 108 \, \text{cm}^2 \)
          \item 单个侧面面积:\( S_{\text{侧}} = \frac{108}{4} = 27 \, \text{cm}^2 \)
          \item 斜高:\( l = \frac{2 S_{\text{侧}}}{a} = \frac{2 \times 27}{6} = 9 \, \text{cm} \)
          \item 高:\( h = \sqrt{l^2 - \left(\frac{a}{2}\right)^2} = \sqrt{9^2 - 3^2} = 6\sqrt{2} \, \text{cm} \)
          \item 体积:\( V = \frac{1}{3} S_{\text{底}} h = \frac{1}{3} \times 36 \times 6\sqrt{2} = 72\sqrt{2} \, \text{cm}^3 \)
      \end{enumerate}
  \end{alertblock}
\end{frame}




\subsubsection{正三棱锥知识点}


\begin{frame}{正三棱锥结构图}

  \begin{minipage}[t][\textheight][t]{\textwidth}
    \centering
\begin{columns}
    \begin{column}{0.48\textwidth}
    \centering
    \begin{figure}
      \resizebox{!}{3cm}{ % 高度固定为5cm,宽度自动调整
      
% \begin{figure}[H]
    \begin{tikzpicture}[scale=2.5,
        y={(-0.353cm,-0.353cm)}, % 设置 x 轴方向
        x={(1cm,0cm)},            % 设置 y 轴方向
        z={(0cm,1cm)}             % 设置 z 轴方向
    ]%斜二测画法
    % 定义底面正三角形的顶点
    \coordinate (A) at (0,0,0);
    \coordinate (C) at (1,0,0);
    \coordinate (B) at (0.5,{sqrt(3)/2},0);
    
    % 定义顶点 V
    \coordinate (V) at (0.5,{sqrt(3)/6},1);
    
    % 计算底面中心 O
    \coordinate (O) at (0.5,{sqrt(3)/6},0);
    
    % 计算各边中点
    \coordinate (O') at ($(A)!0.5!(C)$);
    \coordinate (M) at ($(C)!0.5!(B)$);
    \coordinate (O''') at ($(B)!0.5!(A)$);
    
    % 绘制底面
    \draw[dashed] (A) -- (C);
    \draw (A) -- (B);
    \draw (C) -- (B);
  
    %%高
    \draw[dashed,red,thick] (A) -- (O);
    \draw[dashed,blue,thick] (O) -- (M);
  
  
    
    % 绘制侧面
    \draw (A) -- (V);
    \draw (C) -- (V);
    \draw (B) -- (V);
    %% 斜高
    \draw[blue,thick] (V) -- (M);
    % 棱锥的高
    \draw[thick,blue,dashed] (O) -- (V); 
    
    % 标记顶点
    \node[left] at (A) {$A$};
    \node[right] at (C) {$C$};
    \node[left] at (B) {$B$};
    \node[above] at (V) {$V$};
    \node[below] at (O) {$O$};
  
    \node[right] at (M) {$M$};
    % 绘制圆点
    \fill[blue] (V) circle (0.75pt);
    \fill[blue] (O) circle (0.75pt);
    \fill[blue] (M) circle (0.75pt);
    \end{tikzpicture}
    % \caption{正三棱锥立体图}
% \end{figure} 
  
      }
      \caption{正三棱锥立体图}
\end{figure}

    \end{column}
    \hfill % 两栏之间的间隔
    \begin{column}{0.48\textwidth}
    \centering
    \begin{figure}
      \resizebox{!}{3cm}{ % 高度固定为5cm,宽度自动调整
      \begin{tikzpicture}[scale=1]
    % 定义正三角形的边长
    \def\sideLength{3}
    % 定义点(按 B、C、A 顺序)
    \coordinate (B) at (0,0);        % 原 A 点改为 B
    \coordinate (C) at (\sideLength,0);  % 原 B 点改为 C
    \coordinate (A) at (\sideLength/2,{sqrt(3)*\sideLength/2}); % 原 C 点改为 A
    
    % 计算 BC 的中点(原 AB 中点改为 BC 中点,H 改为 M)
    \coordinate (M) at ($(B)!0.5!(C)$); % 中点 M 对应新的边 BC
    \coordinate (O) at ($(M)!0.3333!(A)$); % 重心 O(原指向 C,现指向新顶点 A)

    % 填涂区域(保持原逻辑,随顶点变化自动调整)
    \fill[blue!20] (B) -- (O) -- (M) -- cycle; % 顶点改为 B、O、M

    % 绘制正三角形(边改为 B-C-A-B)
    \draw (B) -- (C) -- (A) -- cycle;
    
    % 绘制辅助线(调整指向新顶点)
    \draw[dashed,ultra thick, red] (A) -- (O); % 原 C-O 改为 A-O(红色)
    \draw[dashed,ultra thick,blue] (O) -- (M); % 保持 O-M 不变(蓝色)
    \draw[dashed] (B) -- ($(A)!0.5!(C)$); % 原 A 到 BC 中点,现 B 到 AC 中点
    \draw[dashed] (C) -- ($(A)!0.5!(B)$); % 原 B 到 AC 中点,现 C 到 AB 中点

    % 标记顶点(全部改为新标签)
    \node[below left] at (B) {$B$};  % 原 A 标签改为 B
    \node[below right] at (C) {$C$};  % 原 B 标签改为 C
    \node[above] at (A) {$A$};       % 原 C 标签改为 A
    \node[below] at (M) {$M$}; % 中点 M 标签(原 H 改为 M)
    \node[right=3pt,red] at (O) {$O$}; % 重心 O 标签不变
    \fill[blue] (O) circle (2pt);% 填涂测试点
    \fill[blue] (M) circle (2pt);% 填涂测试点
\end{tikzpicture}
% \captionof{这是一个简单的圆形}  % 添加标题}
      \caption{底面图}
    \end{figure}
    \end{column}
  \end{columns}
  \end{minipage}
  \end{frame}


\subsubsection{正三棱锥}



\begin{frame}{正三棱锥体积计算}
    \begin{block}{题目}
      正三棱锥底面边长为 \(6\),侧棱长为 \(5\),求其体积。
    \end{block}
  
    \pause
    \begin{enumerate}
      \item[1.] \textbf{求底面中心到顶点的距离}  
        底面为正三角形,边长 \(a = 6\),中心到顶点的距离:  
        \[
        d = \frac{\sqrt{3}}{3}a = \frac{\sqrt{3}}{3} \times 6 = 2\sqrt{3}
        \]
  
      \item[2.] \textbf{求三棱锥的高}  
        侧棱长 \(l = 5\),由勾股定理:  
        \[
        h = \sqrt{l^2 - d^2} = \sqrt{5^2 - (2\sqrt{3})^2} = \sqrt{25 - 12} = \sqrt{13}
        \]
  
      \item[3.] \textbf{求底面积}  
        正三角形面积公式:  
        \[
        S = \frac{\sqrt{3}}{4}a^2 = \frac{\sqrt{3}}{4} \times 6^2 = 9\sqrt{3}
        \]
  
      \item[4.] \textbf{计算体积}  
        体积公式 \(V = \frac{1}{3}Sh\):  
        \[
        V = \frac{1}{3} \times 9\sqrt{3} \times \sqrt{13} = 3\sqrt{39}
        \]
    \end{enumerate}
  
  \end{frame}
  
  \begin{frame}{正三棱锥表面积计算}
    \begin{block}{题目}
      正三棱锥底面边长为 \(4\),斜高为 \(3\),求其表面积。
    \end{block}
    \pause
  
    \begin{enumerate}
      \item[1.] \textbf{计算底面积 \(S_{\text{底}}\)}  
        底面为正三角形,边长 \(a = 4\):  
        \[
        S_{\text{底}} = \frac{\sqrt{3}}{4}a^2 = \frac{\sqrt{3}}{4} \times 4^2 = 4\sqrt{3}
        \]
  
      \item[2.] \textbf{计算单个侧面面积 \(S_{\text{侧}}\)}  
        侧面为等腰三角形,底为 \(a = 4\),高为斜高 \(l = 3\):  
        \[
        S_{\text{侧}} = \frac{1}{2} \times a \times l = \frac{1}{2} \times 4 \times 3 = 6
        \]
  
      \item[3.] \textbf{计算总侧面积 \(S_{\text{侧总}}\)}  
        正三棱锥有3个全等的侧面:  
        \[
        S_{\text{侧总}} = 3 \times S_{\text{侧}} = 3 \times 6 = 18
        \]
  
      \item[4.] \textbf{计算表面积 \(S_{\text{表}}\)}  
        表面积等于底面积加总侧面积:  
        \[
        S_{\text{表}} = S_{\text{底}} + S_{\text{侧总}} = 4\sqrt{3} + 18
        \]
    \end{enumerate}

  \end{frame}
  


  \begin{frame}{正三棱锥求高计算}
    \begin{block}{题目}
      正三棱锥体积为 \(16\sqrt{3}\),底面边长为 \(6\),求其高。
    \end{block}
    \pause
    \begin{enumerate}
      \item[1.] \textbf{计算底面积 \(S_{\text{底}}\)}  

        底面为正三角形,边长 \(a = 6\):  
        \[
        S_{\text{底}} = \frac{\sqrt{3}}{4}a^2 = \frac{\sqrt{3}}{4} \times 6^2 = 9\sqrt{3}
        \]
  
      \item[2.] \textbf{利用体积公式反推高 \(h\)}  
      
        体积公式 \(V = \frac{1}{3}S_{\text{底}}h\),变形得:  
        \[
        h = \frac{3V}{S_{\text{底}}} = \frac{3 \times 16\sqrt{3}}{9\sqrt{3}} = \frac{48\sqrt{3}}{9\sqrt{3}} = \frac{16}{3}
        \]
    \end{enumerate}
  
    \begin{exampleblock}{答案}
      \[
      \boxed{\dfrac{16}{3}}
      \]
    \end{exampleblock}
  \end{frame}
  
  \begin{frame}{正三棱锥侧棱长计算}
    \begin{block}{题目}
      正三棱锥高为 \(4\),底面边长为 \(6\),求其侧棱长。
    \end{block}
  \pause
    \begin{enumerate}
      \item[1.] \textbf{计算底面中心到顶点的距离 \(d\)}  
      
        底面为正三角形,边长 \(a = 6\):  
        \[
        d = \frac{\sqrt{3}}{3}a = \frac{\sqrt{3}}{3} \times 6 = 2\sqrt{3}
        \]
  
      \item[2.] \textbf{利用勾股定理求侧棱长 \(l\)}  
      
        高 \(h = 4\),侧棱长、高与底面中心到顶点距离构成直角三角形:  
        \[
        l = \sqrt{h^2 + d^2} = \sqrt{4^2 + (2\sqrt{3})^2} = \sqrt{16 + 12} = \sqrt{28} = 2\sqrt{7}
        \]
    \end{enumerate}
  
    \begin{exampleblock}{答案}
      \[
      \boxed{2\sqrt{7}}
      \]
    \end{exampleblock}
  \end{frame}
  