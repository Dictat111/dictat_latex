\documentclass[]{ctexart}
\usepackage[]{exercise} % exerciseonly 表示不显示答案
\usepackage{amsmath} % 提供 \implies 命令 逻辑蕴涵符号
\usepackage{xcolor} % 或 \usepackage{color}
\usepackage{hyperref}
\usepackage[most]{tcolorbox} %使用 most来 使用 breakable 选项,即自动分页功能
% \usepackage{draftwatermark}         % 所有页加水印
% \SetWatermarkText{钱皓楠}           % 设置水印内容
\renewcommand{\DifficultyMarker}{$\star$} %重新定义难度为五角星标
\renewcommand{\ExerciseName}{题目}
\renewcommand{\ExerciseListName}{题}
\renewcommand{\AnswerHeader}{\color{blue}\medskip\centerline{\textbf{
 \ExerciseName\ \ExerciseHeaderNB  ~答案}\smallskip}} %自定义题目标题

 \newenvironment{MyAnswer}[1][] %设置了一个参数,用来
{
    \begin{tcolorbox}[breakable, colframe=blue]
    \begin{Answer}[#1] \color{blue} \kaishu
        }  % 开始部分
    {\end{Answer}
\end{tcolorbox}
}      

\newtcbox{\mybox}[1][red]{on line,
arc=0pt,outer arc=0pt,colback=#1!10!white,colframe=#1!50!black,
boxsep=0pt,left=1pt,right=1pt,top=2pt,bottom=2pt,
boxrule=0pt,bottomrule=1pt,toprule=1pt} % 用来开头高亮表示答案

\def\listexercisename{练习目录}

% \renewcommand{\AnswerName}{解答}
% \renewcommand{\AnswerListName}{\textcolor{blue}{解}}
% \renewcommand{\AtBeginAnswer}{ \color{blue} \kaishu}

% \AtBeginEnvironment{Answer}{  \begin{tcolorbox}}
% \AtEndEnvironment{Answer}{  \end{tcolorbox}} %有问题

\begin{document}
% 1.还需要做的,添加一个题目的目录
% 怎么设置没有答案的时候 tcolorbox 就不产生
% 解决 linux 粘贴出现 ^[[200~tcolorbox~ 的问题
% 解决没有答案显示时方框仍会被显示的问题

\listofexercises %有点问题
% \ListOfExerciseInToc


\section{指数运算}

\begin{Exercise}[title={基础根式与指数式互换练习}, label={ex:basic - radical - exponential} ,difficulty = 1]
    \Question 将\(\sqrt{2}\)化为指数式。
    \Question 把\(5^{\frac{1}{2}}\)写成根式形式。
    \Question 将\(\sqrt[3]{7}\)化为指数式。
    \Question 把\(10^{\frac{1}{3}}\)写成根式形式。
    \Question 将\(\sqrt[4]{9}\)化为指数式。
    \Question 将\(\frac{1}{\sqrt{3}}\)化为指数式。
    \Question 把\(4^{-\frac{1}{2}}\)写成根式形式。
    \Question 将\(\frac{1}{\sqrt[3]{5}}\)化为指数式。
    \Question 把\(6^{-\frac{1}{3}}\)写成根式形式。
    \Question 将\(\frac{1}{\sqrt[4]{11}}\)化为指数式。
    \Question 将 $\sqrt[3]{5^2}$ 化为指数式.
    \Question 把 $3^{\frac{2}{3}}$ 写成根式形式.
\end{Exercise}
\begin{MyAnswer}[ref={ex:basic - radical - exponential}]
    \Question \mybox{答案为\(2^{\frac{1}{2}}\);}\\
    解:根据根式与指数式的转换规则\(\sqrt[n]{a}=a^{\frac{1}{n}}\),对于\(\sqrt{2}\),这里\(n = 2\),\(a = 2\),所以\(\sqrt{2}=2^{\frac{1}{2}}\)。
    \Question \mybox{答案为\(\sqrt{5}\);}\\
    解:由\(a^{\frac{1}{n}}=\sqrt[n]{a}\),对于\(5^{\frac{1}{2}}\),其中\(n = 2\),\(a = 5\),所以\(5^{\frac{1}{2}}=\sqrt{5}\)。
    \Question \mybox{答案为\(7^{\frac{1}{3}}\);}\\
    解:依据\(\sqrt[n]{a}=a^{\frac{1}{n}}\),对于\(\sqrt[3]{7}\),\(n = 3\),\(a = 7\),则\(\sqrt[3]{7}=7^{\frac{1}{3}}\)。
    \Question \mybox{答案为\(\sqrt[3]{10}\);}\\
    解:因为\(a^{\frac{1}{n}}=\sqrt[n]{a}\),对于\(10^{\frac{1}{3}}\),\(n = 3\),\(a = 10\),所以\(10^{\frac{1}{3}}=\sqrt[3]{10}\)。
    \Question \mybox{答案为\(9^{\frac{1}{4}}\);}\\
    解:根据\(\sqrt[n]{a}=a^{\frac{1}{n}}\),对于\(\sqrt[4]{9}\),\(n = 4\),\(a = 9\),故\(\sqrt[4]{9}=9^{\frac{1}{4}}\)。
    \Question \mybox{答案为\(3^{-\frac{1}{2}}\);}\\
    解:因为\(\frac{1}{\sqrt{3}}=\frac{1}{3^{\frac{1}{2}}}\),根据负整数指数幂的规则\(a^{-p}=\frac{1}{a^{p}}(a\neq0)\),所以\(\frac{1}{3^{\frac{1}{2}}}=3^{-\frac{1}{2}}\)。
    \Question \mybox{答案为\(\frac{1}{\sqrt{4}}\);}\\
    解:根据负整数指数幂的规则\(a^{-p}=\frac{1}{a^{p}}(a\neq0)\),\(4^{-\frac{1}{2}}=\frac{1}{4^{\frac{1}{2}}}\),又因为\(4^{\frac{1}{2}}=\sqrt{4}\),所以\(4^{-\frac{1}{2}}=\frac{1}{\sqrt{4}}\)。
    \Question \mybox{答案为\(5^{-\frac{1}{3}}\);}\\
    解:由于\(\frac{1}{\sqrt[3]{5}}=\frac{1}{5^{\frac{1}{3}}}\),根据负整数指数幂的规则\(a^{-p}=\frac{1}{a^{p}}(a\neq0)\),可得\(\frac{1}{5^{\frac{1}{3}}}=5^{-\frac{1}{3}}\)。
    \Question \mybox{答案为\(\frac{1}{\sqrt[3]{6}}\);}\\
    解:依据负整数指数幂的规则\(a^{-p}=\frac{1}{a^{p}}(a\neq0)\),\(6^{-\frac{1}{3}}=\frac{1}{6^{\frac{1}{3}}}\),而\(6^{\frac{1}{3}}=\sqrt[3]{6}\),所以\(6^{-\frac{1}{3}}=\frac{1}{\sqrt[3]{6}}\)。
    \Question \mybox{答案为\(11^{-\frac{1}{4}}\);}\\
    解:因为\(\frac{1}{\sqrt[4]{11}}=\frac{1}{11^{\frac{1}{4}}}\),根据负整数指数幂的规则\(a^{-p}=\frac{1}{a^{p}}(a\neq0)\),所以\(\frac{1}{11^{\frac{1}{4}}}=11^{-\frac{1}{4}}\)。
    \Question \mybox{答案为 $5^{\frac{2}{3}}$;}\\ 解:根据根式与指数式的互换规则,\(\sqrt[n]{a^m}=a^{\frac{m}{n}}\),对于\(\sqrt[3]{5^2}\),这里\(n = 3\),\(m = 2\),所以\(\sqrt[3]{5^2}=5^{\frac{2}{3}}\)。

    \Question \mybox{答案为 $\sqrt[3]{3^2}$;}\\ 解:由\(a^{\frac{m}{n}}=\sqrt[n]{a^m}\),对于\(3^{\frac{2}{3}}\),其中\(a = 3\),\(m = 2\),\(n = 3\),所以\(3^{\frac{2}{3}}=\sqrt[3]{3^2}\)。

\end{MyAnswer}





\begin{Exercise}[title={根式与指数式互换进阶练习}, label={ex:radical - exponential - advanced},difficulty = 3]

    \Question 已知 $a^{\frac{3}{4}} = 8$,求 $a$ 的值.
    \Question 化简\((\sqrt[4]{x^3}\cdot\sqrt{x})^{\frac{1}{3}}\)为指数式。
    \Question 已知\(\sqrt[5]{a^2}\cdot a^{-\frac{3}{5}} = a^k\),求\(k\)的值。
    \Question 若\(x^{\frac{1}{2}}+x^{-\frac{1}{2}} = 3\),求\(x + x^{-1}\)的值。
\end{Exercise}
\begin{MyAnswer}[ref={ex:radical - exponential - advanced}]
        \Question  \mybox{答案为 $a = 16$;}\\  解:已知\(a^{\frac{3}{4}} = 8\),将等式两边同时进行\(\frac{4}{3}\)次方运算,即\((a^{\frac{3}{4}})^{\frac{4}{3}}=8^{\frac{4}{3}}\)。根据指数运算法则\((a^m)^n=a^{mn}\),左边\((a^{\frac{3}{4}})^{\frac{4}{3}}=a^{\frac{3}{4}\times\frac{4}{3}} = a\)。对于右边\(8^{\frac{4}{3}}\),因为\(8 = 2^3\),所以\(8^{\frac{4}{3}}=(2^3)^{\frac{4}{3}}\),再根据指数运算法则\((a^m)^n=a^{mn}\),可得\((2^3)^{\frac{4}{3}}=2^{3\times\frac{4}{3}} = 2^4 = 16\),所以\(a = 16\)。

        \Question \mybox{答案为\(x^{\frac{5}{12}}\);}\\ 解:先将原式中的根式化为指数式,\(\sqrt[4]{x^3}=x^{\frac{3}{4}}\),\(\sqrt{x}=x^{\frac{1}{2}}\),则\((\sqrt[4]{x^3}\cdot\sqrt{x})^{\frac{1}{3}}=(x^{\frac{3}{4}}\cdot x^{\frac{1}{2}})^{\frac{1}{3}}\)。根据指数运算法则\(a^m\cdot a^n=a^{m + n}\),可得\(x^{\frac{3}{4}}\cdot x^{\frac{1}{2}}=x^{\frac{3}{4}+\frac{1}{2}}=x^{\frac{3 + 2}{4}}=x^{\frac{5}{4}}\),那么\((x^{\frac{5}{4}})^{\frac{1}{3}}\),再根据指数运算法则\((a^m)^n=a^{mn}\),得到\((x^{\frac{5}{4}})^{\frac{1}{3}}=x^{\frac{5}{4}\times\frac{1}{3}}=x^{\frac{5}{12}}\)。

        \Question \mybox{答案为\(k = -\frac{1}{5}\);}\\ 解:将\(\sqrt[5]{a^2}\)化为指数式为\(a^{\frac{2}{5}}\),则\(\sqrt[5]{a^2}\cdot a^{-\frac{3}{5}}=a^{\frac{2}{5}}\cdot a^{-\frac{3}{5}}\)。根据指数运算法则\(a^m\cdot a^n=a^{m + n}\),可得\(a^{\frac{2}{5}}\cdot a^{-\frac{3}{5}}=a^{\frac{2}{5}+(-\frac{3}{5})}=a^{\frac{2 - 3}{5}}=a^{-\frac{1}{5}}\),因为\(\sqrt[5]{a^2}\cdot a^{-\frac{3}{5}} = a^k\),所以\(k = -\frac{1}{5}\)。

        \Question  \mybox{答案为\(x + x^{-1}=7\);}\\  解:已知\(x^{\frac{1}{2}}+x^{-\frac{1}{2}} = 3\),将等式两边同时平方,\((x^{\frac{1}{2}}+x^{-\frac{1}{2}})^2 = 3^2\)。根据完全平方公式\((a + b)^2=a^2 + 2ab + b^2\),这里\(a = x^{\frac{1}{2}}\),\(b = x^{-\frac{1}{2}}\),则\((x^{\frac{1}{2}}+x^{-\frac{1}{2}})^2=(x^{\frac{1}{2}})^2+2\cdot x^{\frac{1}{2}}\cdot x^{-\frac{1}{2}}+(x^{-\frac{1}{2}})^2\)。根据指数运算法则\((a^m)^n=a^{mn}\),可得\((x^{\frac{1}{2}})^2=x\),\((x^{-\frac{1}{2}})^2=x^{-1}\),又根据指数运算法则\(a^m\cdot a^n=a^{m + n}\),\(x^{\frac{1}{2}}\cdot x^{-\frac{1}{2}}=x^{\frac{1}{2}+(-\frac{1}{2})}=x^0 = 1\),所以\((x^{\frac{1}{2}}+x^{-\frac{1}{2}})^2=x + 2 + x^{-1}\)。因为\((x^{\frac{1}{2}}+x^{-\frac{1}{2}})^2 = 9\),即\(x + 2 + x^{-1}=9\),移项可得\(x + x^{-1}=9 - 2 = 7\)。
\end{MyAnswer}






\begin{Exercise}[title={指数运算小练习}, label={ex:exponent},difficulty = 3]
    \Question 已知 $2^x = 3$,$4^y = 5$,求 $2^{x - 2y}$ 的值.
    \Question 若函数 $y = a^{2x - 2}+1$($a>0$ 且 $a\neq1$)的图象恒过定点 $P$,求点 $P$ 的坐标.
    \Question 已知 $3^a = 5^b = 15$,求 $\frac{1}{a}+\frac{1}{b}$ 的值.

    \Question 计算 $4^{3/2} + 27^{2/3}$ 的值。
    \Question 已知 $2^x = 5$ 和 $2^y = 3$,求 $2^{2x - y + 1}$。
    \Question 解方程 $3^{x+2} - 3^x = 24$。
\end{Exercise}

\begin{MyAnswer}[ref={ex:exponent}]
    \Question \mybox{答案为 $\frac{3}{5}$;}\\ 解:因为 $4^y=(2^2)^y = 2^{2y}=5$,又已知 $2^x = 3$,根据指数运算法则 $a^m\div a^n=a^{m - n}$,可得 $2^{x - 2y}=2^x\div2^{2y}=\frac{3}{5}$。
    \Question \mybox{答案为 $(1,2)$;}\\ 解:令 $2x - 2 = 0$,解得 $x = 1$。当 $x = 1$ 时,$y=a^{0}+1=1 + 1=2$,所以函数 $y = a^{2x - 2}+1$($a>0$ 且 $a\neq1$)的图象恒过定点 $P(1,2)$。
    \Question \mybox{答案为 $1$;}\\ 解:因为 $3^a = 15$,所以 $a=\log_3{15}$,则 $\frac{1}{a}=\log_{15}3$;因为 $5^b = 15$,所以 $b=\log_5{15}$,则 $\frac{1}{b}=\log_{15}5$。根据对数运算法则 $\log_aM+\log_aN=\log_a(MN)$,可得 $\frac{1}{a}+\frac{1}{b}=\log_{15}3+\log_{15}5=\log_{15}(3\times5)=\log_{15}15 = 1$。

    \Question \mybox{答案为 $17$;}\\ 解:$4^{3/2} = (4^{1/2})^3 = 2^3 = 8$,$27^{2/3} = (27^{1/3})^2 = 3^2 = 9$,故 $4^{3/2} + 27^{2/3} = 8 + 9 = 17$。
    \Question \mybox{答案为 $\dfrac{50}{3}$;}\\ 解:由题意,$2^{2x} = (2^x)^2 = 25$,$2^{-y} = \dfrac{1}{2^y} = \dfrac{1}{3}$,$2^1 = 2$。因此:
    $$
    2^{2x - y + 1} = 2^{2x} \cdot 2^{-y} \cdot 2^1 = 25 \times \dfrac{1}{3} \times 2 = \dfrac{50}{3} 。
    $$
    \Question \mybox{答案为 $x = 1$;}\\ 解:方程变形为 $3^x (3^2 - 1) = 24$,即 $3^x \times 8 = 24$,故 $3^x = 3$,解得 $x = 1$。
\end{MyAnswer}

\begin{Exercise}[title={指数运算小练习2}, label={ex:exponent2},difficulty = 2]
    \Question 计算 $8^{2/3} + 16^{3/4} - 81^{1/2}$ 的值。
    \Question 解方程 $5^{x} - 5^{x-2} = 120$。
    \Question 化简表达式 $\frac{4^{x} \cdot 8^{y}}{2^{2x + 3y}}$ 为最简形式。
\end{Exercise}


\begin{MyAnswer}[ref={ex:exponent2}]
    \Question \mybox{答案为 $3$;}\\ 
    解:逐项化简:
    $$
    8^{2/3} = (2^3)^{2/3} = 2^2 = 4, \quad 
    16^{3/4} = (2^4)^{3/4} = 2^3 = 8, \quad 
    81^{1/2} = 9.
    $$
    因此,$4 + 8 - 9 = 3$。  
 
    \Question \mybox{答案为 $x = 3$;}\\ 
    解:提取公因式 $5^{x-2}$:
    $$
    5^{x} - 5^{x-2} = 5^{x-2}(5^2 - 1) = 5^{x-2} \times 24 = 120.
    $$
    化简得 $5^{x-2} = 5$,故 $x - 2 = 1$,解得 $x = 3$。

    \Question \mybox{答案为 $1$;}\\ 
    解:将所有项转化为以 $2$ 为底的指数:
    $$
    4^x = (2^2)^x = 2^{2x}, \quad 8^y = (2^3)^y = 2^{3y}, \quad 2^{2x + 3y} \text{ 保持不变}.
    $$
    因此
    $
    \frac{2^{2x} \cdot 2^{3y}}{2^{2x + 3y}} = \frac{2^{2x + 3y}}{2^{2x + 3y}} = 1.
    $ %太长会报错
\end{MyAnswer}



\clearpage

\section{对数运算}

\begin{Exercise}[title={对数式与指数式转化练习}, label={ex:logarithm},difficulty=1]
    \Question 将指数式\(2^5 = 32\)转化为对数式。
    \Question 将对数式\(\log_{4}16 = 2\)转化为指数式。
    \Question 已知指数式\(3^{-2}=\frac{1}{9}\),写出其对应的对数式。
    \Question 若对数式\(\log_{7}x = - 1\),求\(x\)的值并将该对数式转化为指数式。
    \Question 把指数式\(10^3 = 1000\)转化为对数式。
    \Question 已知对数式\(\log_{a}8 = 3\),求\(a\)的值并将该对数式转化为指数式。
\end{Exercise}
\begin{MyAnswer}[ref={ex:logarithm}]
        \Question \mybox{对数式为\(\log_{2}32 = 5\);}\\解:根据指数式\(a^b = N\)可转化为对数式\(\log_{a}N = b\),对于\(2^5 = 32\),这里\(a = 2\),\(b = 5\),\(N = 32\),所以转化后的对数式为\(\log_{2}32 = 5\)。

        \Question \mybox{指数式为\(4^2 = 16\);}\\解:因为对数式\(\log_{a}N = b\)可转化为指数式\(a^b = N\),对于\(\log_{4}16 = 2\),其中\(a = 4\),\(b = 2\),\(N = 16\),所以对应的指数式是\(4^2 = 16\)。

        \Question \mybox{对数式为\(\log_{3}\frac{1}{9}=-2\);}\\解:由指数式\(a^b = N\)转化为对数式\(\log_{a}N = b\),对于\(3^{-2}=\frac{1}{9}\),这里\(a = 3\),\(b=-2\),\(N=\frac{1}{9}\),故对应的对数式为\(\log_{3}\frac{1}{9}=-2\)。

        \Question \mybox{\(x=\frac{1}{7}\),指数式为\(7^{-1}=\frac{1}{7}\);}\\解:因为对数式\(\log_{a}N = b\)可转化为指数式\(a^b = N\),对于\(\log_{7}x = - 1\),即\(7^{-1}=x\),所以\(x=\frac{1}{7}\),其指数式为\(7^{-1}=\frac{1}{7}\)。

        \Question \mybox{对数式为\(\log_{10}1000 = 3\)(也可写成\(\lg1000 = 3\));}\\解:依据指数式\(a^b = N\)转化为对数式\(\log_{a}N = b\),对于\(10^3 = 1000\),这里\(a = 10\),\(b = 3\),\(N = 1000\),所以对数式为\(\log_{10}1000 = 3\),常用对数可写成\(\lg1000 = 3\)。

        \Question \mybox{\(a = 2\),指数式为\(2^3 = 8\);}\\解:由对数式\(\log_{a}N = b\)转化为指数式\(a^b = N\),对于\(\log_{a}8 = 3\),即\(a^3 = 8\),因为\(2^3 = 8\),所以\(a = 2\),其指数式为\(2^3 = 8\)。
\end{MyAnswer}

\begin{Exercise}[title={对数运算基础练习}, label={ex:basic_logarithm} ,difficulty = 1]
    \Question 计算\(\log_3 9\)的值。
    \Question 计算\(\log_2 32\)的值。
    \Question 计算\(\log_5 125\)的值。
    \Question 计算\(\log_{10}100\)的值。
    \Question 计算\(\log_4 64\)的值。
    \Question 已知\(a^{\log_a 5}\)(\(a>0,a\neq1\)),求其值。
    \Question 已知\(3^{\log_3 8}\),求其值。
    \Question 已知\(10^{\log_{10}15}\),求其值。
    \Question 计算\(\log_8 9\cdot\log_{27}32\)的值(使用换底公式)。
\end{Exercise}
\begin{MyAnswer}[ref={ex:basic_logarithm}]
    \Question \mybox{答案为\(2\);}\\解:因为\(9 = 3^2\),根据对数的定义,\(\log_3 9=\log_3 3^2 = 2\)。
    \Question \mybox{答案为\(5\);}\\解:因为\(32 = 2^5\),所以\(\log_2 32=\log_2 2^5 = 5\)。
    \Question \mybox{答案为\(3\);}\\解:由于\(125 = 5^3\),则\(\log_5 125=\log_5 5^3 = 3\)。
    \Question \mybox{答案为\(2\);}\\解:因为\(100 = 10^2\),故\(\log_{10}100=\log_{10}10^2 = 2\)。
    \Question \mybox{答案为\(3\);}\\解:鉴于\(64 = 4^3\),所以\(\log_4 64=\log_4 4^3 = 3\)。
    \Question \mybox{答案为\(5\);}\\解:根据对数恒等式\(a^{\log_a N}=N\)(\(a>0,a\neq1,N>0\)),这里\(N = 5\),所以\(a^{\log_a 5}=5\)。
    \Question \mybox{答案为\(8\);}\\解:依据对数恒等式\(a^{\log_a N}=N\),此时\(a = 3\),\(N = 8\),则\(3^{\log_3 8}=8\)。
    \Question \mybox{答案为\(15\);}\\解:由对数恒等式\(a^{\log_a N}=N\),这里\(a = 10\),\(N = 15\),所以\(10^{\log_{10}15}=15\)。
    \Question \mybox{答案为\(\frac{10}{9}\);}\\解:根据换底公式\(\log_8 9\cdot\log_{27}32=\frac{\lg 9}{\lg 8}\cdot\frac{\lg 32}{\lg 27}=\frac{\lg 3^2}{\lg 2^3}\cdot\frac{\lg 2^5}{\lg 3^3}=\frac{2\lg 3}{3\lg 2}\cdot\frac{5\lg 2}{3\lg 3}=\frac{10}{9}\)。
\end{MyAnswer}



\begin{Exercise}[title={常用对数和自然对数练习}, label={ex:logarithm},difficulty =3]
    \Question 计算 $\lg 1000+\ln e^2$ 的值。
    \Question 已知 $\lg 2 = a$,$\lg 3 = b$,求 $\lg 18$(用 $a$,$b$ 表示)。
    \Question 计算 $\lg 0.001+\ln 1$ 的值。
    \Question 已知 $\ln 2 = m$,$\ln 3 = n$,求 $\ln 12$(用 $m$,$n$ 表示)。
    \Question 若 $\lg (x - 1)+\lg (x + 1)= \lg 5$,求 $x$ 的值。
\end{Exercise}
\begin{MyAnswer}[ref={ex:logarithm}]
        \Question \mybox{答案为 $5$;}\\ 解:根据对数运算法则,$\lg 1000=\lg 10^3 = 3$,$\ln e^2 = 2$,所以 $\lg 1000+\ln e^2=3 + 2=5$。

        \Question \mybox{答案为 $a + 2b$;}\\ 解:因为 $\lg 18=\lg(2\times3^2)$,根据对数运算法则 $\lg(MN)=\lg M+\lg N$ 和 $\lg M^p = p\lg M$,可得 $\lg 18=\lg 2 + 2\lg 3$,又已知 $\lg 2 = a$,$\lg 3 = b$,所以 $\lg 18=a + 2b$。


        \Question \mybox{答案为 $-3$;}\\ 解:根据对数运算法则,$\lg 0.001=\lg 10^{-3} = - 3$,$\ln 1 = 0$,所以 $\lg 0.001+\ln 1=-3+0=-3$。

        \Question \mybox{答案为 $2m + n$;}\\ 解:因为 $\ln 12=\ln(3\times2^2)$,根据对数运算法则 $\ln(MN)=\ln M+\ln N$ 和 $\ln M^p = p\ln M$,可得 $\ln 12=\ln 3 + 2\ln 2$,又已知 $\ln 2 = m$,$\ln 3 = n$,所以 $\ln 12=2m + n$。

        \Question  \mybox{答案为 $x=\sqrt{6}$;}\\  解:根据对数运算法则 $\lg M+\lg N=\lg(MN)$,则 $\lg (x - 1)+\lg (x + 1)=\lg((x - 1)(x + 1))=\lg(x^2 - 1)$。已知 $\lg(x^2 - 1)=\lg 5$,则 $x^2 - 1 = 5$,移项可得 $x^2 = 6$,解得 $x=\pm\sqrt{6}$。因为对数函数中真数须大于 $0$,当 $x = -\sqrt{6}$ 时,$x - 1=-\sqrt{6}-1<0$,$x + 1=-\sqrt{6}+1<0$,不满足定义域要求,舍去;当 $x=\sqrt{6}$ 时,$x - 1=\sqrt{6}-1>0$,$x + 1=\sqrt{6}+1>0$,满足要求,所以 $x=\sqrt{6}$。
\end{MyAnswer}


\begin{Exercise}[title={对数运算练习1}, label={ex:logarithm},difficulty = 2]
    \Question 计算 $\log_2 8 + \log_3 27$ 的值.
    \Question 已知 $\log_a 2 = m$,$\log_a 3 = n$,求 $\log_a 12$(用 $m$,$n$ 表示).
    \Question 若 $\log_5 (x + 1) - \log_5 (x - 1) = 1$,求 $x$ 的值.
    \Question 计算 $\log_{10} 1000 - \log_{10} 10$ 的值.
    \Question 已知 $\log_b 5 = p$,$\log_b 2 = q$,求 $\log_b 20$(用 $p$,$q$ 表示).
    \Question 若 $\log_3 (x^2 - 1) = 2$,求 $x$ 的值.
\end{Exercise}
\begin{MyAnswer}[ref={ex:logarithm}]
        \Question \mybox{答案为 $6$;}\\ 解:根据对数运算法则,$\log_2 8=\log_2 2^3 = 3$,$\log_3 27=\log_3 3^3 = 3$,所以 $\log_2 8+\log_3 27=3 + 3=6$.

        \Question \mybox{答案为 $n + 2m$;}\\ 解:因为 $\log_a 12=\log_a(3\times2^2)$,根据对数运算法则 $\log_a(MN)=\log_a M+\log_a N$ 和 $\log_a M^p = p\log_a M$,可得 $\log_a 12=\log_a 3 + 2\log_a 2$,又已知 $\log_a 2 = m$,$\log_a 3 = n$,所以 $\log_a 12=n + 2m$.

        \Question  \mybox{答案为 $x=\frac{3}{2}$;}\\  解:根据对数运算法则 $\log_a M-\log_a N=\log_a\frac{M}{N}$,则 $\log_5 (x + 1)-\log_5 (x - 1)=\log_5\frac{x + 1}{x - 1}$.已知 $\log_5\frac{x + 1}{x - 1}=1$,即 $\frac{x + 1}{x - 1}=5^1 = 5$.
        方程两边同乘 $x - 1$ 得:$x + 1 = 5(x - 1)$,展开得 $x + 1 = 5x-5$,移项可得 $4x = 6$,解得 $x=\frac{3}{2}$.经检验,当 $x=\frac{3}{2}$ 时,$x + 1=\frac{5}{2}>0$,$x - 1=\frac{1}{2}>0$,满足对数函数的定义域要求.

        \Question \mybox{答案为 $2$;}\\ 解:根据对数运算法则,$\log_{10} 1000=\log_{10} 10^3 = 3$,$\log_{10} 10 = 1$,所以 $\log_{10} 1000 - \log_{10} 10=3 - 1 = 2$.

        \Question \mybox{答案为 $p + 2q$;}\\ 解:因为 $\log_b 20=\log_b(5\times2^2)$,根据对数运算法则 $\log_b(MN)=\log_b M+\log_b N$ 和 $\log_b M^p = p\log_b M$,可得 $\log_b 20=\log_b 5 + 2\log_b 2$,又已知 $\log_b 5 = p$,$\log_b 2 = q$,所以 $\log_b 20=p + 2q$.

        \Question \mybox{答案为 $x=\pm\sqrt{10}$;}\\ 解:已知 $\log_3 (x^2 - 1) = 2$,根据对数的定义可得 $x^2 - 1 = 3^2 = 9$,移项可得 $x^2 = 10$,解得 $x=\pm\sqrt{10}$。经检验,当 $x=\pm\sqrt{10}$ 时,$x^2 - 1 = 9>0$,满足对数函数的定义域要求。
\end{MyAnswer}



\begin{Exercise}[title={对数运算练习2}, label={ex:logarithm2},difficulty = 2]
    \Question 计算 $\log_4 16 + \log_9 81$ 的值.
    \Question 已知 $\log_b 5 = p$, $\log_b 7 = q$, 求 $\log_b 35$(用 $p$, $q$ 表示).
    \Question 解方程 $2^{x} = 5$, 用对数表示 $x$ 的值.
\end{Exercise}
\begin{MyAnswer}[ref={ex:logarithm2}]
        \Question \mybox{答案为 $4$;}\\ 解:根据对数运算法则, $\log_4 16 = \log_4 4^2 = 2$, $\log_9 81 = \log_9 9^2 = 2$, 所以 $\log_4 16 + \log_9 81 = 2 + 2 = 4$.

        \Question \mybox{答案为 $p + q$;}\\ 解:因为 $\log_b 35 = \log_b (5 \times 7)$, 根据对数运算法则 $\log_b(MN) = \log_b M + \log_b N$, 可得 $\log_b 35 = \log_b 5 + \log_b 7$, 又已知 $\log_b 5 = p$, $\log_b 7 = q$, 所以 $\log_b 35 = p + q$.


        \Question \mybox{答案为 $x = \log_2 5$;}\\ 解:对等式 $2^x = 5$ 两边取以 $2$ 为底的对数, 得 $x = \log_2 5$.
\end{MyAnswer}





















\begin{Exercise}[title={对数运算进阶练习}, label={ex:logarithm-advanced},difficulty = 3]
    \Question 计算 $\log_4 9 \cdot \log_3 16$ 的值.
    \Question 解不等式 $\log_{\frac{1}{2}}(2x-1) > \log_{\frac{1}{2}}(x+3)$.
    \Question 已知 $\log_{12} 3 = a$, 用 $a$ 表示 $\log_{\sqrt{3}} 4$.
    \Question 设 $a > 0$ 且 $a \neq 1$,若 $\log_a x = 2\log_a 3 - \log_a 5$,求 $x$ 的值.
    \Question 解方程 $\log_2(x^2 - 3x) = 1 + \log_2(1 - x)$.
\end{Exercise}

\begin{MyAnswer}[ref={ex:logarithm-advanced}]
    \Question \mybox{答案为 $4$;}\\ 
    解:利用换底公式 $\log_a b = \frac{\log_c b}{\log_c a}$:
    $$
    \log_4 9 \cdot \log_3 16 = \frac{\log 9}{\log 4} \cdot \frac{\log 16}{\log 3} = \frac{2\log 3}{2\log 2} \cdot \frac{4\log 2}{\log 3} = 4
    $$

    \Question \mybox{答案为 $\frac{1}{2} < x < 4$;}\\ 
    解:由于底数 $\frac{1}{2} \in (0,1)$,对数函数单调递减:
    $$
    \begin{cases}
        2x-1 > 0 \\
        x+3 > 0 \\
        2x-1 < x+3
    \end{cases}
    \Rightarrow
    \begin{cases}
        x > \frac{1}{2} \\
        x > -3 \\
        x < 4
    \end{cases}
    $$
    最终解集为 $\frac{1}{2} < x < 4$.

    \Question \mybox{答案为 $\frac{2(1-a)}{a}$;}\\ 
    解:由 $\log_{12} 3 = a$ 得:
    $$
    \frac{\log 3}{\log 12} = a \Rightarrow \log 3 = a(\log 3 + 2\log 2)
    $$
    设 $\log 3 = k$,则 $\log 2 = \frac{(1-a)k}{2a}$。所求表达式为:
    $$
    \log_{\sqrt{3}} 4 = \frac{\log 4}{\frac{1}{2}\log 3} = \frac{4\log 2}{\log 3} = \frac{4 \cdot \frac{(1-a)k}{2a}}{k} = \frac{2(1-a)}{a}
    $$

    \Question \mybox{答案为 $x = \frac{9}{5}$;}\\ 
    解:根据对数运算法则:
    $$
    \log_a x = \log_a 3^2 - \log_a 5 = \log_a \left( \frac{9}{5} \right)
    $$
    因此 $x = \frac{9}{5}$.

    \Question \mybox{答案为 $x = -1$;}\\ 
    解:首先确定定义域:
    $$
    \begin{cases}
        x^2 - 3x > 0 \\
        1 - x > 0
    \end{cases}
    \Rightarrow x < 0
    $$
    原方程化为:
    $$
    \log_2(x^2 - 3x) = \log_2 2 + \log_2(1 - x) = \log_2[2(1 - x)]
    $$
    得到 $x^2 - 3x = 2 - 2x$,整理得 $x^2 - x - 2 = 0$,解得 $x = -1$ 或 $x = 2$(舍去)。经检验 $x = -1$ 满足定义域。
\end{MyAnswer}



\begin{Exercise}[title={对数函数和指数函数定义域练习}, label={ex:logarithm}]
    \Question 求函数\(y = \log_2(x - 3)\)的定义域。
    \Question 求函数\(y = \log_{0.5}(4 - x^2)\)的定义域。
    \Question 求函数\(y = \sqrt{\log_3(x - 1)}\)的定义域。
    \Question 求函数\(y = 2^{\frac{1}{x - 2}}\)的定义域。
    \Question 求函数\(y = \log_{\frac{1}{3}}(x^2 - 2x - 3)\)的定义域。
    \Question 求函数\(y = \sqrt{3^{x - 1}- \frac{1}{9}}\)的定义域。
\end{Exercise}
\begin{MyAnswer}[ref={ex:logarithm}]
        \Question \mybox{答案为\((3,+\infty)\);}\\ 解:对于对数函数\(y = \log_a u\),要求\(u>0\)。在函数\(y = \log_2(x - 3)\)中,\(u=x - 3\),则\(x - 3>0\),解得\(x>3\),所以函数的定义域为\((3,+\infty)\)。

        \Question \mybox{答案为\((-2,2)\);}\\ 解:对于函数\(y = \log_{0.5}(4 - x^2)\),因为对数函数中真数须大于\(0\),即\(4 - x^2>0\),变形为\(x^2 - 4<0\),因式分解得\((x + 2)(x - 2)<0\),解得\(-2<x<2\),所以函数的定义域为\((-2,2)\)。

        \Question  \mybox{答案为\([2,+\infty)\);}\\  解:要使函数\(y = \sqrt{\log_3(x - 1)}\)有意义,则\(\log_3(x - 1)\ge0\)。因为\(\log_3(x - 1)\ge\log_3 1\),且对数函数\(y = \log_3 u\)在\((0,+\infty)\)上单调递增,所以\(x - 1\ge1\),解得\(x\ge2\),同时对数函数中\(x - 1>0\)(此条件包含在\(x\ge2\)中),所以函数的定义域为\([2,+\infty)\)。

        \Question \mybox{答案为\((-\infty,2)\cup(2,+\infty)\);}\\ 解:对于函数\(y = 2^{\frac{1}{x - 2}}\),因为指数函数的指数部分分母不能为\(0\),即\(x - 2\neq0\),解得\(x\neq2\),所以函数的定义域为\((-\infty,2)\cup(2,+\infty)\)。

        \Question \mybox{答案为\((-\infty,-1)\cup(3,+\infty)\);}\\ 解:对于函数\(y = \log_{\frac{1}{3}}(x^2 - 2x - 3)\),对数函数真数要大于\(0\),即\(x^2 - 2x - 3>0\),因式分解得\((x + 1)(x - 3)>0\),解得\(x<-1\)或\(x>3\),所以函数的定义域为\((-\infty,-1)\cup(3,+\infty)\)。

        \Question  \mybox{答案为\([-1,+\infty)\);}\\  解:要使函数\(y = \sqrt{3^{x - 1}- \frac{1}{9}}\)有意义,则\(3^{x - 1}- \frac{1}{9}\ge0\),即\(3^{x - 1}\ge\frac{1}{9}=3^{-2}\)。因为指数函数\(y = 3^x\)在\(R\)上单调递增,所以\(x - 1\ge - 2\),解得\(x\ge - 1\),所以函数的定义域为\([-1,+\infty)\)。
\end{MyAnswer}


% \clearpage

% 
\begin{Exercise}[title={两点间距离与中点坐标计算练习}, label={ex:distance_midpoint}]
    \Question 已知点\(A(1, 2)\)和点\(B(4, 6)\),求\(A\)、\(B\)两点间的距离以及线段\(AB\)的中点坐标。
    \Question 若点\(C(-3, 5)\)与点\(D(1, -1)\),计算\(C\)、\(D\)两点间的距离和线段\(CD\)的中点坐标。
    \Question 设点\(E(2, -3)\)和点\(F(-4, 1)\),求两点间的距离和线段\(EF\)的中点坐标。
    \Question 已知点\(M(0, 0)\)与点\(N(3, 4)\),计算\(M\)、\(N\)两点间的距离以及线段\(MN\)的中点坐标。
    \Question 若点\(P(-2, -3)\)和点\(Q(4, 5)\),求\(P\)、\(Q\)两点间的距离和线段\(PQ\)的中点坐标。
\end{Exercise}
\begin{MyAnswer}[ref={ex:distance_midpoint}]
    \Question 
        \mybox{两点间距离为\(5\),中点坐标为\((\frac{5}{2}, 4)\);}\\
        解:
        1. 两点间距离公式为\(d = \sqrt{(x_2 - x_1)^2+(y_2 - y_1)^2}\),对于\(A(1, 2)\)和\(B(4, 6)\),\(x_1 = 1,y_1 = 2,x_2 = 4,y_2 = 6\),则\(AB=\sqrt{(4 - 1)^2+(6 - 2)^2}=\sqrt{9 + 16}=\sqrt{25}=5\)。
        2. 中点坐标公式为\((\frac{x_1 + x_2}{2},\frac{y_1 + y_2}{2})\),所以线段\(AB\)中点坐标为\((\frac{1 + 4}{2},\frac{2 + 6}{2}) = (\frac{5}{2}, 4)\)。
    \Question 
        \mybox{两点间距离为\(2\sqrt{13}\),中点坐标为\((-1, 2)\);}\\
        解:
        1. 由两点间距离公式,对于\(C(-3, 5)\)与\(D(1, -1)\),\(x_1=-3,y_1 = 5,x_2 = 1,y_2=-1\),则\(CD=\sqrt{(1 - (-3))^2+((-1)-5)^2}=\sqrt{16 + 36}=\sqrt{52}=2\sqrt{13}\)。
        2. 根据中点坐标公式,线段\(CD\)中点坐标为\((\frac{-3 + 1}{2},\frac{5 + (-1)}{2})=(-1, 2)\)。
    \Question 
        \mybox{两点间距离为\(2\sqrt{13}\),中点坐标为\((-1, -1)\);}\\
        解:
        1. 对于\(E(2, -3)\)和\(F(-4, 1)\),由两点间距离公式,\(x_1 = 2,y_1=-3,x_2=-4,y_2 = 1\),则\(EF=\sqrt{((-4)-2)^2+(1 - (-3))^2}=\sqrt{36 + 16}=\sqrt{52}=2\sqrt{13}\)。
        2. 依据中点坐标公式,线段\(EF\)中点坐标为\((\frac{2 + (-4)}{2},\frac{-3 + 1}{2})=(-1, -1)\)。
    \Question 
        \mybox{两点间距离为\(5\),中点坐标为\((\frac{3}{2}, 2)\);}\\
        解:
        1. 已知\(M(0, 0)\)与\(N(3, 4)\),由两点间距离公式,\(x_1 = 0,y_1 = 0,x_2 = 3,y_2 = 4\),则\(MN=\sqrt{(3 - 0)^2+(4 - 0)^2}=\sqrt{9 + 16}=\sqrt{25}=5\)。
        2. 根据中点坐标公式,线段\(MN\)中点坐标为\((\frac{0 + 3}{2},\frac{0 + 4}{2})=(\frac{3}{2}, 2)\)。
    \Question 
        \mybox{两点间距离为\(10\),中点坐标为\((1, 1)\);}\\
        解:
        1. 对于\(P(-2, -3)\)和\(Q(4, 5)\),由两点间距离公式,\(x_1=-2,y_1=-3,x_2 = 4,y_2 = 5\),则\(PQ=\sqrt{(4 - (-2))^2+(5 - (-3))^2}=\sqrt{36 + 64}=\sqrt{100}=10\)。
        2. 依据中点坐标公式,线段\(PQ\)中点坐标为\((\frac{-2 + 4}{2},\frac{-3 + 5}{2})=(1, 1)\)。
\end{MyAnswer}



\begin{Exercise}[title={倾斜角相关题目}, label={ex:inclination_angle_radian}] %全是弧度制
    \Question 直线的倾斜角为\(0\),求该直线的斜率。
    \Question 已知直线的倾斜角为\(\frac{\pi}{3}\),求该直线的斜率。
    \Question 直线的斜率为\(\sqrt{3}\),求其倾斜角。
    \Question 若直线的斜率为\(-\frac{\sqrt{3}}{3}\),求其倾斜角。
    \Question 已知直线过点\((1,2)\)与\((3,4)\),求该直线的倾斜角。
\end{Exercise}
\begin{MyAnswer}[ref={ex:inclination_angle_radian}]
    \Question \mybox{答案为\(0\);}\\ 解:已知直线倾斜角\(\alpha = 0\),根据直线斜率\(k = \tan\alpha\),可得\(k=\tan0 = 0\)。
    \Question \mybox{答案为\(\sqrt{3}\);}\\ 解:已知直线倾斜角\(\alpha=\frac{\pi}{3}\),根据直线斜率\(k = \tan\alpha\),可得\(k = \tan\frac{\pi}{3}=\sqrt{3}\)。
    \Question \mybox{答案为\(\frac{\pi}{3}\);}\\ 解:设倾斜角为\(\alpha\),\(0\leq\alpha<\pi\),已知直线斜率\(k = \sqrt{3}\),因为\(k = \tan\alpha\),所以\(\tan\alpha=\sqrt{3}\),则\(\alpha=\frac{\pi}{3}\)。
    \Question \mybox{答案为\(\frac{5\pi}{6}\);}\\ 解:设倾斜角为\(\alpha\),\(0\leq\alpha<\pi\),已知直线斜率\(k = -\frac{\sqrt{3}}{3}\),因为\(k = \tan\alpha\),所以\(\tan\alpha=-\frac{\sqrt{3}}{3}\),则\(\alpha=\frac{5\pi}{6}\)。
    \Question \mybox{答案为\(\frac{\pi}{4}\);}\\ 解:设过点\((x_1,y_1)=(1,2)\)与\((x_2,y_2)=(3,4)\)直线的斜率为\(k\),根据斜率公式\(k=\frac{y_2 - y_1}{x_2 - x_1}\),则\(k=\frac{4 - 2}{3 - 1}=\frac{2}{2}=1\)。设倾斜角为\(\alpha\),\(0\leq\alpha<\pi\),且\(k = \tan\alpha\),所以\(\tan\alpha = 1\),可得\(\alpha=\frac{\pi}{4}\)。
\end{MyAnswer}












\begin{Exercise}[title={直线斜率计算小练习}, label={ex:line_slope}]
    \Question 已知直线经过点\(A(2, 3)\)和\(B(4, 7)\),求该直线的斜率。
    \Question 若直线过点\(M(-1, 5)\)与\(N(3, -3)\),计算此直线的斜率。
    \Question 设直线经过\(P(5, 2)\)和\(Q(5, -4)\),求直线的斜率。
    \Question 已知直线经过点\(C(0, 1)\)和\(D(3, 4)\),求该直线的斜率。
    \Question 设直线经过\(G(1, 2)\)和\(H(1, 5)\),求直线的斜率。
\end{Exercise}
\begin{MyAnswer}[ref={ex:line_slope}]
    \Question \mybox{答案为\(2\);}\\ 解:根据直线斜率公式\(k = \frac{y_2 - y_1}{x_2 - x_1}\),已知点\(A(2, 3)\)和\(B(4, 7)\),则直线\(AB\)的斜率\(k_{AB}=\frac{7 - 3}{4 - 2}=\frac{4}{2}=2\)。

    \Question \mybox{答案为\(-2\);}\\ 解:由直线斜率公式,对于点\(M(-1, 5)\)与\(N(3, -3)\),直线\(MN\)的斜率\(k_{MN}=\frac{-3 - 5}{3 - (-1)}=\frac{-8}{4}=-2\)。

    \Question  \mybox{直线斜率不存在;}\\  解:对于点\(P(5, 2)\)和\(Q(5, -4)\),此时\(x_1 = x_2 = 5\),在直线斜率公式\(k = \frac{y_2 - y_1}{x_2 - x_1}\)中,分母\(x_2 - x_1 = 0\),因为分母不能为\(0\),所以该直线的斜率不存在。

    \Question \mybox{答案为\(1\);}\\ 解:根据直线斜率公式,对于点\(C(0, 1)\)和\(D(3, 4)\),直线\(CD\)的斜率\(k_{CD}=\frac{4 - 1}{3 - 0}=\frac{3}{3}=1\)。


    \Question \mybox{直线斜率不存在;}\\ 解:对于点\(G(1, 2)\)和\(H(1, 5)\),此时\(x_1 = x_2 = 1\),在直线斜率公式\(k = \frac{y_2 - y_1}{x_2 - x_1}\)中,分母\(x_2 - x_1 = 0\),所以该直线的斜率不存在(这两个点在同一条竖线上!!)。

\end{MyAnswer}






\begin{Exercise}[title={直线方程相关练习}, label={ex:line_equations}]
    \Question 已知直线过点\((2,3)\),斜率为\(4\),求直线的点斜式方程,并将其化为一般式方程。
    \Question 直线斜率为\(-2\),在\(y\)轴上的截距为\(5\),写出直线的斜截式方程,并转化为一般式方程。
    \Question 已知直线过点\((-1,4)\),倾斜角为\(135^{\circ}\),求直线的点斜式方程,再化为斜截式方程。
    \Question 直线经过点\((3, -1)\)与\((1,2)\),先求直线的斜率,再写出直线的点斜式方程,最后化为一般式方程。
    \Question 写出直线\(3x - 2y + 6 = 0\)的斜截式方程,求出直线的斜率以及在\(y\)轴上的截距。
\end{Exercise}
\begin{MyAnswer}[ref={ex:line_equations}]
    \Question
        \mybox{点斜式方程为\(y - 3 = 4(x - 2)\),一般式方程为\(4x - y - 5 = 0\);}\\
        解:
        1. 点斜式方程的形式为\(y - y_1 = k(x - x_1)\),已知点\((2,3)\),斜率\(k = 4\),则点斜式方程为\(y - 3 = 4(x - 2)\)。
        2. 将点斜式方程\(y - 3 = 4(x - 2)\)展开并移项化为一般式:
        \(y - 3 = 4x - 8\),移项可得\(4x - y - 8 + 3 = 0\),即\(4x - y - 5 = 0\)。
    \Question
        \mybox{斜截式方程为\(y = -2x + 5\),一般式方程为\(2x + y - 5 = 0\);}\\
        解:
        1. 斜截式方程的形式为\(y = kx + b\),已知斜率\(k = -2\),\(y\)轴截距\(b = 5\),则斜截式方程为\(y = -2x + 5\)。
        2. 将斜截式方程\(y = -2x + 5\)移项化为一般式:\(2x + y - 5 = 0\)。
    \Question
        \mybox{点斜式方程为\(y - 4 = -(x + 1)\),斜截式方程为\(y = -x + 3\);}\\
        解:
        1. 已知倾斜角为\(135^{\circ}\),则斜率\(k=\tan135^{\circ}=-1\)。直线过点\((-1,4)\),根据点斜式方程\(y - y_1 = k(x - x_1)\),可得点斜式方程为\(y - 4 = -(x + 1)\)。
        2. 将点斜式方程\(y - 4 = -(x + 1)\)展开并整理为斜截式:\(y - 4 = -x - 1\),移项可得\(y = -x + 3\)。
    \Question
        \mybox{斜率为\(-\frac{3}{2}\),点斜式方程为\(y + 1 = -\frac{3}{2}(x - 3)\),}\\
        \mybox{一般式方程为\(3x + 2y - 7 = 0\);}\\
        解:
        1. 直线斜率\(k=\frac{y_2 - y_1}{x_2 - x_1}\),已知点\((3, -1)\)与\((1,2)\),则\(k=\frac{2 - (-1)}{1 - 3}=\frac{3}{-2}=-\frac{3}{2}\)。
        2. 直线过点\((3, -1)\),斜率\(k = -\frac{3}{2}\),根据点斜式方程\(y - y_1 = k(x - x_1)\),可得点斜式方程为\(y + 1 = -\frac{3}{2}(x - 3)\)。
        3. 将点斜式方程\(y + 1 = -\frac{3}{2}(x - 3)\)展开并移项化为一般式:
        \(y + 1 = -\frac{3}{2}x+\frac{9}{2}\),两边同乘\(2\)得\(2y + 2 = -3x + 9\),移项可得\(3x + 2y - 7 = 0\)。
    \Question
        \mybox{斜截式方程为\(y=\frac{3}{2}x + 3\),斜率为\(\frac{3}{2}\),\(y\)轴截距为\(3\);}\\
        解:
        1. 将直线方程\(3x - 2y + 6 = 0\)移项化为斜截式:
        \(-2y=-3x - 6\),两边同除以\(-2\)得\(y=\frac{3}{2}x + 3\)。
        2. 由斜截式方程\(y=\frac{3}{2}x + 3\)可知,直线的斜率\(k=\frac{3}{2}\),在\(y\)轴上的截距\(b = 3\)。
\end{MyAnswer}




\begin{Exercise}[title={两条直线的位置关系小练习}, label={ex:line_position},difficulty=1]
    \Question 判断直线 $y = 2x + 3$ 与直线 $y = 2x - 5$ 的位置关系。
    \Question 判断直线 $y = 3x + 2$ 与直线 $y=-\frac{1}{3}x - 1$ 的位置关系。
    \Question 判断直线 $2x - y + 1 = 0$ 与直线 $4x - 2y + 2 = 0$ 的位置关系。
    \Question 判断直线 $x + y - 3 = 0$ 与直线 $x - y + 1 = 0$ 的位置关系。
    \Question 已知直线 $l_1:y = k_1x + 1$ 与直线 $l_2:y = k_2x - 2$ 平行,且 $k_1 = 3$,求 $k_2$ 的值。
    \Question 已知直线 $l_1:2x + my - 1 = 0$ 与直线 $l_2:mx + 8y + 2 = 0$ 垂直,求 $m$ 的值。
    \Question 判断直线 $3x - 4y + 5 = 0$ 与直线 $6x - 8y + 10 = 0$ 的位置关系。
    \Question 判断直线 $x - 2y + 3 = 0$ 与直线 $2x - 4y - 1 = 0$ 的位置关系。
    \Question 已知直线 $l_1:y = 2x + b_1$ 与直线 $l_2:y = 2x + b_2$ 重合,且 $b_1 = 4$,求 $b_2$ 的值。
    \Question 判断直线 $5x + 12y - 1 = 0$ 与直线 $12x - 5y + 3 = 0$ 的位置关系。
\end{Exercise}
\begin{MyAnswer}[ref={ex:line_position}]
    \Question \mybox{答案为平行;}\\ 解:对于直线 $y = k_1x + b_1$ 和直线 $y = k_2x + b_2$,若 $k_1 = k_2$ 且 $b_1\neq b_2$,则两直线平行。在直线 $y = 2x + 3$ 与直线 $y = 2x - 5$ 中,$k_1 = k_2 = 2$,$b_1 = 3$,$b_2=-5$,$b_1\neq b_2$,所以两直线平行。
    \Question \mybox{答案为垂直;}\\ 解:若两条直线的斜率 $k_1$ 和 $k_2$ 满足 $k_1k_2=-1$,则两直线垂直。直线 $y = 3x + 2$ 的斜率 $k_1 = 3$,直线 $y=-\frac{1}{3}x - 1$ 的斜率 $k_2=-\frac{1}{3}$,$k_1k_2=3\times(-\frac{1}{3})=-1$,所以两直线垂直。
    \Question \mybox{答案为重合;}\\ 解:将直线 $4x - 2y + 2 = 0$ 两边同时除以 $2$,可得 $2x - y + 1 = 0$,与另一条直线方程完全相同,所以两直线重合。
    \Question \mybox{答案为相交;}\\ 解:直线 $x + y - 3 = 0$ 可化为 $y=-x + 3$,斜率 $k_1=-1$;直线 $x - y + 1 = 0$ 可化为 $y=x + 1$,斜率 $k_2 = 1$。因为 $k_1\neq k_2$,所以两直线相交。
    \Question \mybox{答案为 $3$;}\\ 解:若两条直线 $y = k_1x + b_1$ 与 $y = k_2x + b_2$ 平行,则 $k_1 = k_2$。已知 $k_1 = 3$,所以 $k_2 = 3$。
    \Question \mybox{答案为 $m = 0$;}\\ 解:当两条直线 $A_1x + B_1y + C_1 = 0$ 与 $A_2x + B_2y + C_2 = 0$ 垂直时,有 $A_1A_2 + B_1B_2 = 0$。对于直线 $l_1:2x + my - 1 = 0$ 与直线 $l_2:mx + 8y + 2 = 0$,则 $2m+8m = 0$,即 $10m = 0$,解得 $m = 0$。
    \Question \mybox{答案为重合;}\\ 解:将直线 $6x - 8y + 10 = 0$ 两边同时除以 $2$,得到 $3x - 4y + 5 = 0$,与另一条直线方程相同,所以两直线重合。
    \Question \mybox{答案为平行;}\\ 解:直线 $x - 2y + 3 = 0$ 可化为 $y=\frac{1}{2}x+\frac{3}{2}$,直线 $2x - 4y - 1 = 0$ 可化为 $y=\frac{1}{2}x-\frac{1}{4}$,两直线斜率相等都为 $\frac{1}{2}$,但截距 $\frac{3}{2}\neq-\frac{1}{4}$,所以两直线平行。
    \Question \mybox{答案为 $4$;}\\ 解:若两条直线 $y = k_1x + b_1$ 与 $y = k_2x + b_2$ 重合,则 $k_1 = k_2$ 且 $b_1 = b_2$。已知 $k_1 = k_2 = 2$,$b_1 = 4$,所以 $b_2 = 4$。
    \Question \mybox{答案为垂直;}\\ 解:直线 $5x + 12y - 1 = 0$ 可化为 $y=-\frac{5}{12}x+\frac{1}{12}$,斜率 $k_1=-\frac{5}{12}$;直线 $12x - 5y + 3 = 0$ 可化为 $y=\frac{12}{5}x+\frac{3}{5}$,斜率 $k_2=\frac{12}{5}$。$k_1k_2=-\frac{5}{12}\times\frac{12}{5}=-1$,所以两直线垂直。
\end{MyAnswer}










\begin{Exercise}[title={圆的基本性质练习}, label={ex:circle-properties},difficulty=2]
    \Question 求以点$(2, -3)$为圆心,半径为$5$的圆的标准方程.
    \Question 已知圆的方程为$(x + 1)^2+(y - 2)^2 = 9$,求圆心坐标和半径.
    \Question 若圆经过点$A(1,2)$,$B(3,4)$,且圆心在直线$x - y + 1 = 0$上,求该圆的方程.
    \Question 求过点$(0,0)$,$(1,1)$,$(2,0)$的圆的方程.
    \Question 圆$x^2 + y^2 - 4x + 6y - 3 = 0$的圆心坐标和半径分别是多少?
\end{Exercise}

\begin{MyAnswer}[ref={ex:circle-properties}]
    \Question 解:根据圆标准方程$(x - a)^2+(y - b)^2 = r^2$,$a = 2$,$b=-3$,$r = 5$,得$(x - 2)^2+(y + 3)^2 = 25$.
    
    \Question 解:圆标准方程$(x - a)^2+(y - b)^2 = r^2$,此方程中$a=-1$,$b = 2$,$r = 3$,所以圆心$(-1,2)$,半径$3$.
    
    \Question 解:设圆方程$(x - a)^2+(y - b)^2 = r^2$,由已知得
    $$
    \begin{cases}
    (1 - a)^2+(2 - b)^2 = r^2 \\
    (3 - a)^2+(4 - b)^2 = r^2 \\
    a - b+1 = 0
    \end{cases}
    $$
    前两式相减得$a + b = 5$,联立
    $$
    \begin{cases}
    a + b = 5 \\
    a - b+1 = 0
    \end{cases}
    $$
    解得$a = 2$,$b = 3$,$r^2 = 2$,圆方程为$(x - 2)^2+(y - 3)^2 = 2$.
    
    \Question 解:设圆一般方程$x^{2}+y^{2}+Dx + Ey+F = 0$,代入三点得
    $$
    \begin{cases}
    F = 0 \\
    1 + 1+D + E+F = 0 \\
    4+2D+F = 0
    \end{cases}
    $$
    解得$D=-2$,$E = 0$,$F = 0$,圆方程为$x^{2}+y^{2}-2x = 0$.
    
    \Question 解:配方得
    $$
    x^{2}-4x + 4+y^{2}+6y+9=3 + 4+9
    $$
    即$(x - 2)^2+(y + 3)^2 = 16$,圆心$(2,-3)$,半径$4$.
\end{MyAnswer}







\begin{Exercise}[title={圆与直线位置关系练习1}, label={ex:line-circle}]
    \Question 判断直线 $ y = x + 1 $ 与圆 $ x^2 + y^2 = 2 $ 的位置关系(相交、相切、相离).
    \Question 求直线 $ 3x + 4y - 5 = 0 $ 与圆 $ (x-1)^2 + (y+2)^2 = 4 $ 的圆心到直线的距离,并判断位置关系.
    \Question 若直线 $ y = kx + 2 $ 与圆 $ x^2 + y^2 = 1 $ 相切,求实数 $ k $ 的值.
\end{Exercise}
\begin{MyAnswer}[ref={ex:line-circle}]
        \Question 解:将直线方程代入圆的方程:
        $$
        x^2 + (x + 1)^2 = 2 \implies 2x^2 + 2x - 1 = 0
        $$
        判别式 $ \Delta = 4 + 8 = 12 > 0 $,因此直线与圆相交.

        或者圆心到直线的距离为 
        $$    
        d = \frac{|0-0+1|}{\sqrt{1^2 + (-1)^2}} = \frac{\sqrt{2}}{2} < \sqrt{2} =r
        $$
        所以相交.
        \Question 解:圆心为 $ (1, -2) $,半径 $ r = 2 $.计算圆心到直线的距离:
        $$
        d = \frac{|3 \times 1 + 4 \times (-2) - 5|}{\sqrt{3^2 + 4^2}} = \frac{10}{5} = 2
        $$
        因为 $ d = r $,所以直线与圆相切.

        \Question 解:直线与圆相切时,距离等于半径:
        $$
        \frac{|2|}{\sqrt{k^2 + 1}} = 1 \implies \sqrt{k^2 + 1} = 2 \implies k^2 = 3 \implies k = \pm \sqrt{3}
        $$
\end{MyAnswer}


\begin{Exercise}[title={圆与直线位置关系练习2}, label={ex:circle-line2}]
    \Question 已知直线 $y = kx + 1$ 与圆 $x^2 + y^2 = 4$ 相交,求 $k$ 的取值范围.
    \Question 求圆心在 $(2,-1)$ 且与直线 $3x - 4y + 5 = 0$ 相切的圆的方程.
\end{Exercise}

\begin{MyAnswer}[ref={ex:circle-line2}]


    \Question \mybox{答案为 $-\frac{\sqrt{3}}{2} < k < \frac{\sqrt{3}}{2}$;}\\ 
    解:将直线方程代入圆的方程:
    $$
    x^2 + (kx + 1)^2 = 4 \Rightarrow (1 + k^2)x^2 + 2kx - 3 = 0
    $$
    由判别式 $\Delta > 0$ 得:
    $$
    (2k)^2 - 4(1 + k^2)(-3) > 0 \Rightarrow 16k^2 < 12
    $$
    解得 $k$ 的范围.

    \Question \mybox{答案为 $(x-2)^2 + (y+1)^2 = 1$;}\\ 
    解:计算圆心到直线的距离即为半径:
    $$
    r = \frac{|3 \times 2 - 4 \times (-1) + 5|}{\sqrt{3^2 + 4^2}} = \frac{15}{5} = 3
    $$
    故圆的方程为 $(x-2)^2 + (y+1)^2 = 9$.
\end{MyAnswer}
% 根式 和 分数要怎么比较大小








\end{document}


%  不要出先类似## **的 markdown 语法, 要使用严格的latex语言.