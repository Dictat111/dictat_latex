\documentclass{ctexart}
\usepackage{exercise}
\usepackage{amsmath} % 提供 \implies 命令 逻辑蕴涵符号
\usepackage{xcolor} % 或 \usepackage{color}
\usepackage[most]{tcolorbox} %使用 most来 使用 breakable 选项,即自动分页功能

\renewcommand{\ExerciseName}{题目}
\renewcommand{\ExerciseListName}{题}
\renewcommand{\AnswerHeader}{\color{blue}\medskip\centerline{\textbf{
 \ExerciseName\ \ExerciseHeaderNB  ~答案}\smallskip}} %自定义题目标题

 \newenvironment{MyAnswer}[1][] %设置了一个参数,用来
 {\begin{tcolorbox}[breakable, colframe=blue] \begin{Answer}[#1] \color{blue} \kaishu}  % 开始部分
 {\end{Answer}\end{tcolorbox}}      

% \renewcommand{\AnswerName}{解答}
% \renewcommand{\AnswerListName}{\textcolor{blue}{解}}
% \renewcommand{\AtBeginAnswer}{ \color{blue} \kaishu}

% \AtBeginEnvironment{Answer}{  \begin{tcolorbox}}
% \AtEndEnvironment{Answer}{  \end{tcolorbox}} %有问题

\begin{document}

\begin{Exercise}[title={圆的基本性质练习}, label={ex:circle-properties}]
    \Question 求以点$(2, -3)$为圆心,半径为$5$的圆的标准方程。
    \Question 已知圆的方程为$(x + 1)^2+(y - 2)^2 = 9$,求圆心坐标和半径。
    \Question 若圆经过点$A(1,2)$,$B(3,4)$,且圆心在直线$x - y + 1 = 0$上,求该圆的方程。
    \Question 求过点$(0,0)$,$(1,1)$,$(2,0)$的圆的方程。
    \Question 圆$x^2 + y^2 - 4x + 6y - 3 = 0$的圆心坐标和半径分别是多少?
\end{Exercise}

\begin{MyAnswer}[ref={ex:circle-properties}]
    \Question 解:根据圆标准方程$(x - a)^2+(y - b)^2 = r^2$,$a = 2$,$b=-3$,$r = 5$,得$(x - 2)^2+(y + 3)^2 = 25$。
    
    \Question 解:圆标准方程$(x - a)^2+(y - b)^2 = r^2$,此方程中$a=-1$,$b = 2$,$r = 3$,所以圆心$(-1,2)$,半径$3$。
    
    \Question 解:设圆方程$(x - a)^2+(y - b)^2 = r^2$,由已知得
    $$
    \begin{cases}
    (1 - a)^2+(2 - b)^2 = r^2 \\
    (3 - a)^2+(4 - b)^2 = r^2 \\
    a - b+1 = 0
    \end{cases}
    $$
    前两式相减得$a + b = 5$,联立
    $$
    \begin{cases}
    a + b = 5 \\
    a - b+1 = 0
    \end{cases}
    $$
    解得$a = 2$,$b = 3$,$r^2 = 2$,圆方程为$(x - 2)^2+(y - 3)^2 = 2$。
    
    \Question 解:设圆一般方程$x^{2}+y^{2}+Dx + Ey+F = 0$,代入三点得
    $$
    \begin{cases}
    F = 0 \\
    1 + 1+D + E+F = 0 \\
    4+2D+F = 0
    \end{cases}
    $$
    解得$D=-2$,$E = 0$,$F = 0$,圆方程为$x^{2}+y^{2}-2x = 0$。
    
    \Question 解:配方得
    $$
    x^{2}-4x + 4+y^{2}+6y+9=3 + 4+9
    $$
    即$(x - 2)^2+(y + 3)^2 = 16$,圆心$(2,-3)$,半径$4$。
\end{MyAnswer}

\clearpage
\begin{Exercise}[title={直线与圆的位置关系小练习}, label={ex:line-circle}]
    \Question 判断直线 $ y = x + 1 $ 与圆 $ x^2 + y^2 = 2 $ 的位置关系(相交、相切、相离)。
    \Question 求直线 $ 3x + 4y - 5 = 0 $ 与圆 $ (x-1)^2 + (y+2)^2 = 4 $ 的圆心到直线的距离,并判断位置关系。
    \Question 若直线 $ y = kx + 2 $ 与圆 $ x^2 + y^2 = 1 $ 相切,求实数 $ k $ 的值。
\end{Exercise}
\begin{MyAnswer}[ref={ex:line-circle}]
        \Question 解:将直线方程代入圆的方程:
        $$
        x^2 + (x + 1)^2 = 2 \implies 2x^2 + 2x - 1 = 0
        $$
        判别式 $ \Delta = 4 + 8 = 12 > 0 $,因此直线与圆相交。

        \Question 解:圆心为 $ (1, -2) $,半径 $ r = 2 $。计算圆心到直线的距离:
        $$
        d = \frac{|3 \times 1 + 4 \times (-2) - 5|}{\sqrt{3^2 + 4^2}} = \frac{10}{5} = 2
        $$
        因为 $ d = r $,所以直线与圆相切。

        \Question 解:直线与圆相切时,距离等于半径:
        $$
        \frac{|2|}{\sqrt{k^2 + 1}} = 1 \implies \sqrt{k^2 + 1} = 2 \implies k^2 = 3 \implies k = \pm \sqrt{3}
        $$
\end{MyAnswer}





\end{document}


%  不要出先类似## **的 markdown 语法, 要使用严格的latex语言.