\documentclass{ctexart}
\usepackage[]{exercise} % exerciseonly 表示不显示答案
\usepackage{amsmath} % 提供 \implies 命令 逻辑蕴涵符号
\usepackage{xcolor} % 或 \usepackage{color}
\usepackage[most]{tcolorbox} %使用 most来 使用 breakable 选项,即自动分页功能
\renewcommand{\DifficultyMarker}{$\star$} %重新定义难度为五角星标
\renewcommand{\ExerciseName}{题目}
\renewcommand{\ExerciseListName}{题}
\renewcommand{\AnswerHeader}{\color{blue}\medskip\centerline{\textbf{
 \ExerciseName\ \ExerciseHeaderNB  ~答案}\smallskip}} %自定义题目标题

 \newenvironment{MyAnswer}[1][] %设置了一个参数,用来
{
    \begin{tcolorbox}[breakable, colframe=blue]
    \begin{Answer}[#1] \color{blue} \kaishu
        }  % 开始部分
    {\end{Answer}
\end{tcolorbox}
}      

\newtcbox{\mybox}[1][red]{on line,
arc=0pt,outer arc=0pt,colback=#1!10!white,colframe=#1!50!black,
boxsep=0pt,left=1pt,right=1pt,top=2pt,bottom=2pt,
boxrule=0pt,bottomrule=1pt,toprule=1pt} % 用来开头高亮表示答案

% \def\listexercisename{练习目录}

% \renewcommand{\AnswerName}{解答}
% \renewcommand{\AnswerListName}{\textcolor{blue}{解}}
% \renewcommand{\AtBeginAnswer}{ \color{blue} \kaishu}

% \AtBeginEnvironment{Answer}{  \begin{tcolorbox}}
% \AtEndEnvironment{Answer}{  \end{tcolorbox}} %有问题

\begin{document}
% 1.还需要做的,添加一个题目的目录
% 怎么设置没有答案的时候 tcolorbox 就不产生
% 解决 linux 粘贴出现 ^[[200~tcolorbox~ 的问题
% 解决没有答案显示时方框仍会被显示的问题

% \listofexercises %有点问题
% \ListOfExerciseInToc

\section{指数运算}
\begin{Exercise}[title={指数运算小练习}, label={ex:exponent}]
    \Question 已知 $2^x = 3$,$4^y = 5$,求 $2^{x - 2y}$ 的值.
    \Question 若函数 $y = a^{2x - 2}+1$($a>0$ 且 $a\neq1$)的图象恒过定点 $P$,求点 $P$ 的坐标.
    \Question 已知 $3^a = 5^b = 15$,求 $\frac{1}{a}+\frac{1}{b}$ 的值.

    \Question 计算 $4^{3/2} + 27^{2/3}$ 的值。
    \Question 已知 $2^x = 5$ 和 $2^y = 3$,求 $2^{2x - y + 1}$。
    \Question 解方程 $3^{x+2} - 3^x = 24$。
\end{Exercise}

\begin{MyAnswer}[ref={ex:exponent}]
    \Question \mybox{答案为 $\frac{3}{5}$;}\\ 解:因为 $4^y=(2^2)^y = 2^{2y}=5$,又已知 $2^x = 3$,根据指数运算法则 $a^m\div a^n=a^{m - n}$,可得 $2^{x - 2y}=2^x\div2^{2y}=\frac{3}{5}$。
    \Question \mybox{答案为 $(1,2)$;}\\ 解:令 $2x - 2 = 0$,解得 $x = 1$。当 $x = 1$ 时,$y=a^{0}+1=1 + 1=2$,所以函数 $y = a^{2x - 2}+1$($a>0$ 且 $a\neq1$)的图象恒过定点 $P(1,2)$。
    \Question \mybox{答案为 $1$;}\\ 解:因为 $3^a = 15$,所以 $a=\log_3{15}$,则 $\frac{1}{a}=\log_{15}3$;因为 $5^b = 15$,所以 $b=\log_5{15}$,则 $\frac{1}{b}=\log_{15}5$。根据对数运算法则 $\log_aM+\log_aN=\log_a(MN)$,可得 $\frac{1}{a}+\frac{1}{b}=\log_{15}3+\log_{15}5=\log_{15}(3\times5)=\log_{15}15 = 1$。

    \Question \mybox{答案为 $17$;}\\ 解:$4^{3/2} = (4^{1/2})^3 = 2^3 = 8$,$27^{2/3} = (27^{1/3})^2 = 3^2 = 9$,故 $4^{3/2} + 27^{2/3} = 8 + 9 = 17$。
    \Question \mybox{答案为 $\dfrac{50}{3}$;}\\ 解:由题意,$2^{2x} = (2^x)^2 = 25$,$2^{-y} = \dfrac{1}{2^y} = \dfrac{1}{3}$,$2^1 = 2$。因此:
    $$
    2^{2x - y + 1} = 2^{2x} \cdot 2^{-y} \cdot 2^1 = 25 \times \dfrac{1}{3} \times 2 = \dfrac{50}{3} 。
    $$
    \Question \mybox{答案为 $x = 1$;}\\ 解:方程变形为 $3^x (3^2 - 1) = 24$,即 $3^x \times 8 = 24$,故 $3^x = 3$,解得 $x = 1$。
\end{MyAnswer}

\begin{Exercise}[title={指数运算小练习2}, label={ex:exponent2}]
    \Question 计算 $8^{2/3} + 16^{3/4} - 81^{1/2}$ 的值。
    \Question 解方程 $5^{x} - 5^{x-2} = 120$。
    \Question 化简表达式 $\frac{4^{x} \cdot 8^{y}}{2^{2x + 3y}}$ 为最简形式。
\end{Exercise}


\begin{MyAnswer}[ref={ex:exponent2}]
    \Question \mybox{答案为 $3$;}\\ 
    解:逐项化简:
    $$
    8^{2/3} = (2^3)^{2/3} = 2^2 = 4, \quad 
    16^{3/4} = (2^4)^{3/4} = 2^3 = 8, \quad 
    81^{1/2} = 9.
    $$
    因此,$4 + 8 - 9 = 3$。  
 
    \Question \mybox{答案为 $x = 3$;}\\ 
    解:提取公因式 $5^{x-2}$:
    $$
    5^{x} - 5^{x-2} = 5^{x-2}(5^2 - 1) = 5^{x-2} \times 24 = 120.
    $$
    化简得 $5^{x-2} = 5$,故 $x - 2 = 1$,解得 $x = 3$。

    \Question \mybox{答案为 $1$;}\\ 
    解:将所有项转化为以 $2$ 为底的指数:
    $$
    4^x = (2^2)^x = 2^{2x}, \quad 8^y = (2^3)^y = 2^{3y}, \quad 2^{2x + 3y} \text{ 保持不变}.
    $$
    因此
    $
    \frac{2^{2x} \cdot 2^{3y}}{2^{2x + 3y}} = \frac{2^{2x + 3y}}{2^{2x + 3y}} = 1.
    $ %太长会报错
\end{MyAnswer}


\begin{Exercise}[title={指数运算小练习3}, label={ex:exponent3}]
    \Question  化简:$\sqrt{(e - \pi)^2}$.
\end{Exercise}



\begin{MyAnswer}[ref={ex:exponent3}]
    \Question \mybox{答案为 $\pi-e$;}\\ 


\end{MyAnswer}
\clearpage

\
\begin{Exercise}[title={对数运算小练习1}, label={ex:logarithm}]
    \Question 计算 $\log_2 8 + \log_3 27$ 的值.
    \Question 已知 $\log_a 2 = m$,$\log_a 3 = n$,求 $\log_a 12$(用 $m$,$n$ 表示).
    \Question 若 $\log_5 (x + 1) - \log_5 (x - 1) = 1$,求 $x$ 的值.
    \Question 计算 $\log_{10} 1000 - \log_{10} 10$ 的值.
    \Question 已知 $\log_b 5 = p$,$\log_b 2 = q$,求 $\log_b 20$(用 $p$,$q$ 表示).
    \Question 若 $\log_3 (x^2 - 1) = 2$,求 $x$ 的值.
\end{Exercise}
\begin{MyAnswer}[ref={ex:logarithm}]
        \Question \mybox{答案为 $6$;}\\ 解:根据对数运算法则,$\log_2 8=\log_2 2^3 = 3$,$\log_3 27=\log_3 3^3 = 3$,所以 $\log_2 8+\log_3 27=3 + 3=6$.

        \Question \mybox{答案为 $n + 2m$;}\\ 解:因为 $\log_a 12=\log_a(3\times2^2)$,根据对数运算法则 $\log_a(MN)=\log_a M+\log_a N$ 和 $\log_a M^p = p\log_a M$,可得 $\log_a 12=\log_a 3 + 2\log_a 2$,又已知 $\log_a 2 = m$,$\log_a 3 = n$,所以 $\log_a 12=n + 2m$.

        \Question  \mybox{答案为 $x=\frac{3}{2}$;}\\  解:根据对数运算法则 $\log_a M-\log_a N=\log_a\frac{M}{N}$,则 $\log_5 (x + 1)-\log_5 (x - 1)=\log_5\frac{x + 1}{x - 1}$.已知 $\log_5\frac{x + 1}{x - 1}=1$,即 $\frac{x + 1}{x - 1}=5^1 = 5$.
        方程两边同乘 $x - 1$ 得:$x + 1 = 5(x - 1)$,展开得 $x + 1 = 5x-5$,移项可得 $4x = 6$,解得 $x=\frac{3}{2}$.经检验,当 $x=\frac{3}{2}$ 时,$x + 1=\frac{5}{2}>0$,$x - 1=\frac{1}{2}>0$,满足对数函数的定义域要求.

        \Question \mybox{答案为 $2$;}\\ 解:根据对数运算法则,$\log_{10} 1000=\log_{10} 10^3 = 3$,$\log_{10} 10 = 1$,所以 $\log_{10} 1000 - \log_{10} 10=3 - 1 = 2$.

        \Question \mybox{答案为 $p + 2q$;}\\ 解:因为 $\log_b 20=\log_b(5\times2^2)$,根据对数运算法则 $\log_b(MN)=\log_b M+\log_b N$ 和 $\log_b M^p = p\log_b M$,可得 $\log_b 20=\log_b 5 + 2\log_b 2$,又已知 $\log_b 5 = p$,$\log_b 2 = q$,所以 $\log_b 20=p + 2q$.

        \Question \mybox{答案为 $x=\pm\sqrt{10}$;}\\ 解:已知 $\log_3 (x^2 - 1) = 2$,根据对数的定义可得 $x^2 - 1 = 3^2 = 9$,移项可得 $x^2 = 10$,解得 $x=\pm\sqrt{10}$。经检验,当 $x=\pm\sqrt{10}$ 时,$x^2 - 1 = 9>0$,满足对数函数的定义域要求。
\end{MyAnswer}



\begin{Exercise}[title={对数运算小练习2}, label={ex:logarithm2}]
    \Question 计算 $\log_4 16 + \log_9 81$ 的值.
    \Question 已知 $\log_b 5 = p$, $\log_b 7 = q$, 求 $\log_b 35$(用 $p$, $q$ 表示).
    \Question 解方程 $2^{x} = 5$, 用对数表示 $x$ 的值.
\end{Exercise}
\begin{MyAnswer}[ref={ex:logarithm2}]
        \Question \mybox{答案为 $4$;}\\ 解:根据对数运算法则, $\log_4 16 = \log_4 4^2 = 2$, $\log_9 81 = \log_9 9^2 = 2$, 所以 $\log_4 16 + \log_9 81 = 2 + 2 = 4$.

        \Question \mybox{答案为 $p + q$;}\\ 解:因为 $\log_b 35 = \log_b (5 \times 7)$, 根据对数运算法则 $\log_b(MN) = \log_b M + \log_b N$, 可得 $\log_b 35 = \log_b 5 + \log_b 7$, 又已知 $\log_b 5 = p$, $\log_b 7 = q$, 所以 $\log_b 35 = p + q$.


        \Question \mybox{答案为 $x = \log_2 5$;}\\ 解:对等式 $2^x = 5$ 两边取以 $2$ 为底的对数, 得 $x = \log_2 5$.
\end{MyAnswer}


\begin{Exercise}[title={对数运算进阶练习}, label={ex:logarithm-advanced}]
    \Question 计算 $\log_4 9 \cdot \log_3 16$ 的值.
    \Question 解不等式 $\log_{\frac{1}{2}}(2x-1) > \log_{\frac{1}{2}}(x+3)$.
    \Question 已知 $\log_{12} 3 = a$, 用 $a$ 表示 $\log_{\sqrt{3}} 4$.
    \Question 设 $a > 0$ 且 $a \neq 1$,若 $\log_a x = 2\log_a 3 - \log_a 5$,求 $x$ 的值.
    \Question 解方程 $\log_2(x^2 - 3x) = 1 + \log_2(1 - x)$.
\end{Exercise}

\begin{MyAnswer}[ref={ex:logarithm-advanced}]
    \Question \mybox{答案为 $4$;}\\ 
    解:利用换底公式 $\log_a b = \frac{\log_c b}{\log_c a}$:
    $$
    \log_4 9 \cdot \log_3 16 = \frac{\log 9}{\log 4} \cdot \frac{\log 16}{\log 3} = \frac{2\log 3}{2\log 2} \cdot \frac{4\log 2}{\log 3} = 4
    $$

    \Question \mybox{答案为 $\frac{1}{2} < x < 4$;}\\ 
    解:由于底数 $\frac{1}{2} \in (0,1)$,对数函数单调递减:
    $$
    \begin{cases}
        2x-1 > 0 \\
        x+3 > 0 \\
        2x-1 < x+3
    \end{cases}
    \Rightarrow
    \begin{cases}
        x > \frac{1}{2} \\
        x > -3 \\
        x < 4
    \end{cases}
    $$
    最终解集为 $\frac{1}{2} < x < 4$.

    \Question \mybox{答案为 $\frac{2(1-a)}{a}$;}\\ 
    解:由 $\log_{12} 3 = a$ 得:
    $$
    \frac{\log 3}{\log 12} = a \Rightarrow \log 3 = a(\log 3 + 2\log 2)
    $$
    设 $\log 3 = k$,则 $\log 2 = \frac{(1-a)k}{2a}$。所求表达式为:
    $$
    \log_{\sqrt{3}} 4 = \frac{\log 4}{\frac{1}{2}\log 3} = \frac{4\log 2}{\log 3} = \frac{4 \cdot \frac{(1-a)k}{2a}}{k} = \frac{2(1-a)}{a}
    $$

    \Question \mybox{答案为 $x = \frac{9}{5}$;}\\ 
    解:根据对数运算法则:
    $$
    \log_a x = \log_a 3^2 - \log_a 5 = \log_a \left( \frac{9}{5} \right)
    $$
    因此 $x = \frac{9}{5}$.

    \Question \mybox{答案为 $x = -1$;}\\ 
    解:首先确定定义域:
    $$
    \begin{cases}
        x^2 - 3x > 0 \\
        1 - x > 0
    \end{cases}
    \Rightarrow x < 0
    $$
    原方程化为:
    $$
    \log_2(x^2 - 3x) = \log_2 2 + \log_2(1 - x) = \log_2[2(1 - x)]
    $$
    得到 $x^2 - 3x = 2 - 2x$,整理得 $x^2 - x - 2 = 0$,解得 $x = -1$ 或 $x = 2$(舍去)。经检验 $x = -1$ 满足定义域。
\end{MyAnswer}



\begin{Exercise}[title={圆的基本性质练习}, label={ex:circle-properties},difficulty=3]
    \Question 求以点$(2, -3)$为圆心,半径为$5$的圆的标准方程.
    \Question 已知圆的方程为$(x + 1)^2+(y - 2)^2 = 9$,求圆心坐标和半径.
    \Question 若圆经过点$A(1,2)$,$B(3,4)$,且圆心在直线$x - y + 1 = 0$上,求该圆的方程.
    \Question 求过点$(0,0)$,$(1,1)$,$(2,0)$的圆的方程.
    \Question 圆$x^2 + y^2 - 4x + 6y - 3 = 0$的圆心坐标和半径分别是多少?
\end{Exercise}

\begin{MyAnswer}[ref={ex:circle-properties}]
    \Question 解:根据圆标准方程$(x - a)^2+(y - b)^2 = r^2$,$a = 2$,$b=-3$,$r = 5$,得$(x - 2)^2+(y + 3)^2 = 25$.
    
    \Question 解:圆标准方程$(x - a)^2+(y - b)^2 = r^2$,此方程中$a=-1$,$b = 2$,$r = 3$,所以圆心$(-1,2)$,半径$3$.
    
    \Question 解:设圆方程$(x - a)^2+(y - b)^2 = r^2$,由已知得
    $$
    \begin{cases}
    (1 - a)^2+(2 - b)^2 = r^2 \\
    (3 - a)^2+(4 - b)^2 = r^2 \\
    a - b+1 = 0
    \end{cases}
    $$
    前两式相减得$a + b = 5$,联立
    $$
    \begin{cases}
    a + b = 5 \\
    a - b+1 = 0
    \end{cases}
    $$
    解得$a = 2$,$b = 3$,$r^2 = 2$,圆方程为$(x - 2)^2+(y - 3)^2 = 2$.
    
    \Question 解:设圆一般方程$x^{2}+y^{2}+Dx + Ey+F = 0$,代入三点得
    $$
    \begin{cases}
    F = 0 \\
    1 + 1+D + E+F = 0 \\
    4+2D+F = 0
    \end{cases}
    $$
    解得$D=-2$,$E = 0$,$F = 0$,圆方程为$x^{2}+y^{2}-2x = 0$.
    
    \Question 解:配方得
    $$
    x^{2}-4x + 4+y^{2}+6y+9=3 + 4+9
    $$
    即$(x - 2)^2+(y + 3)^2 = 16$,圆心$(2,-3)$,半径$4$.
\end{MyAnswer}

\clearpage
\begin{Exercise}[title={直线与圆的位置关系小练习}, label={ex:line-circle}]
    \Question 判断直线 $ y = x + 1 $ 与圆 $ x^2 + y^2 = 2 $ 的位置关系(相交、相切、相离).
    \Question 求直线 $ 3x + 4y - 5 = 0 $ 与圆 $ (x-1)^2 + (y+2)^2 = 4 $ 的圆心到直线的距离,并判断位置关系.
    \Question 若直线 $ y = kx + 2 $ 与圆 $ x^2 + y^2 = 1 $ 相切,求实数 $ k $ 的值.
\end{Exercise}
\begin{MyAnswer}[ref={ex:line-circle}]
        \Question 解:将直线方程代入圆的方程:
        $$
        x^2 + (x + 1)^2 = 2 \implies 2x^2 + 2x - 1 = 0
        $$
        判别式 $ \Delta = 4 + 8 = 12 > 0 $,因此直线与圆相交.

        或者圆心到直线的距离为 
        $$    
        d = \frac{|0-0+1|}{\sqrt{1^2 + (-1)^2}} = \frac{\sqrt{2}}{2} < \sqrt{2} =r
        $$
        所以相交.
        \Question 解:圆心为 $ (1, -2) $,半径 $ r = 2 $.计算圆心到直线的距离:
        $$
        d = \frac{|3 \times 1 + 4 \times (-2) - 5|}{\sqrt{3^2 + 4^2}} = \frac{10}{5} = 2
        $$
        因为 $ d = r $,所以直线与圆相切.

        \Question 解:直线与圆相切时,距离等于半径:
        $$
        \frac{|2|}{\sqrt{k^2 + 1}} = 1 \implies \sqrt{k^2 + 1} = 2 \implies k^2 = 3 \implies k = \pm \sqrt{3}
        $$
\end{MyAnswer}


\begin{Exercise}[title={圆与直线位置关系练习}, label={ex:circle-line}]
    \Question 已知直线 $y = kx + 1$ 与圆 $x^2 + y^2 = 4$ 相交,求 $k$ 的取值范围.
    \Question 求圆心在 $(2,-1)$ 且与直线 $3x - 4y + 5 = 0$ 相切的圆的方程.
\end{Exercise}

\begin{MyAnswer}[ref={ex:circle-line}]


    \Question \mybox{答案为 $-\frac{\sqrt{3}}{2} < k < \frac{\sqrt{3}}{2}$;}\\ 
    解:将直线方程代入圆的方程:
    $$
    x^2 + (kx + 1)^2 = 4 \Rightarrow (1 + k^2)x^2 + 2kx - 3 = 0
    $$
    由判别式 $\Delta > 0$ 得:
    $$
    (2k)^2 - 4(1 + k^2)(-3) > 0 \Rightarrow 16k^2 < 12
    $$
    解得 $k$ 的范围.

    \Question \mybox{答案为 $(x-2)^2 + (y+1)^2 = 1$;}\\ 
    解:计算圆心到直线的距离即为半径:
    $$
    r = \frac{|3 \times 2 - 4 \times (-1) + 5|}{\sqrt{3^2 + 4^2}} = \frac{15}{5} = 3
    $$
    故圆的方程为 $(x-2)^2 + (y+1)^2 = 9$.
\end{MyAnswer}



\end{document}


%  不要出先类似## **的 markdown 语法, 要使用严格的latex语言.