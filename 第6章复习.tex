
\begin{Exercise}[title={两点间距离与中点坐标计算练习}, label={ex:distance_midpoint}]
    \Question 已知点\(A(1, 2)\)和点\(B(4, 6)\),求\(A\)、\(B\)两点间的距离以及线段\(AB\)的中点坐标。
    \Question 若点\(C(-3, 5)\)与点\(D(1, -1)\),计算\(C\)、\(D\)两点间的距离和线段\(CD\)的中点坐标。
    \Question 设点\(E(2, -3)\)和点\(F(-4, 1)\),求两点间的距离和线段\(EF\)的中点坐标。
    \Question 已知点\(M(0, 0)\)与点\(N(3, 4)\),计算\(M\)、\(N\)两点间的距离以及线段\(MN\)的中点坐标。
    \Question 若点\(P(-2, -3)\)和点\(Q(4, 5)\),求\(P\)、\(Q\)两点间的距离和线段\(PQ\)的中点坐标。
\end{Exercise}
\begin{MyAnswer}[ref={ex:distance_midpoint}]
    \Question 
        \mybox{两点间距离为\(5\),中点坐标为\((\frac{5}{2}, 4)\);}\\
        解:
        1. 两点间距离公式为\(d = \sqrt{(x_2 - x_1)^2+(y_2 - y_1)^2}\),对于\(A(1, 2)\)和\(B(4, 6)\),\(x_1 = 1,y_1 = 2,x_2 = 4,y_2 = 6\),则\(AB=\sqrt{(4 - 1)^2+(6 - 2)^2}=\sqrt{9 + 16}=\sqrt{25}=5\)。
        2. 中点坐标公式为\((\frac{x_1 + x_2}{2},\frac{y_1 + y_2}{2})\),所以线段\(AB\)中点坐标为\((\frac{1 + 4}{2},\frac{2 + 6}{2}) = (\frac{5}{2}, 4)\)。
    \Question 
        \mybox{两点间距离为\(2\sqrt{13}\),中点坐标为\((-1, 2)\);}\\
        解:
        1. 由两点间距离公式,对于\(C(-3, 5)\)与\(D(1, -1)\),\(x_1=-3,y_1 = 5,x_2 = 1,y_2=-1\),则\(CD=\sqrt{(1 - (-3))^2+((-1)-5)^2}=\sqrt{16 + 36}=\sqrt{52}=2\sqrt{13}\)。
        2. 根据中点坐标公式,线段\(CD\)中点坐标为\((\frac{-3 + 1}{2},\frac{5 + (-1)}{2})=(-1, 2)\)。
    \Question 
        \mybox{两点间距离为\(2\sqrt{13}\),中点坐标为\((-1, -1)\);}\\
        解:
        1. 对于\(E(2, -3)\)和\(F(-4, 1)\),由两点间距离公式,\(x_1 = 2,y_1=-3,x_2=-4,y_2 = 1\),则\(EF=\sqrt{((-4)-2)^2+(1 - (-3))^2}=\sqrt{36 + 16}=\sqrt{52}=2\sqrt{13}\)。
        2. 依据中点坐标公式,线段\(EF\)中点坐标为\((\frac{2 + (-4)}{2},\frac{-3 + 1}{2})=(-1, -1)\)。
    \Question 
        \mybox{两点间距离为\(5\),中点坐标为\((\frac{3}{2}, 2)\);}\\
        解:
        1. 已知\(M(0, 0)\)与\(N(3, 4)\),由两点间距离公式,\(x_1 = 0,y_1 = 0,x_2 = 3,y_2 = 4\),则\(MN=\sqrt{(3 - 0)^2+(4 - 0)^2}=\sqrt{9 + 16}=\sqrt{25}=5\)。
        2. 根据中点坐标公式,线段\(MN\)中点坐标为\((\frac{0 + 3}{2},\frac{0 + 4}{2})=(\frac{3}{2}, 2)\)。
    \Question 
        \mybox{两点间距离为\(10\),中点坐标为\((1, 1)\);}\\
        解:
        1. 对于\(P(-2, -3)\)和\(Q(4, 5)\),由两点间距离公式,\(x_1=-2,y_1=-3,x_2 = 4,y_2 = 5\),则\(PQ=\sqrt{(4 - (-2))^2+(5 - (-3))^2}=\sqrt{36 + 64}=\sqrt{100}=10\)。
        2. 依据中点坐标公式,线段\(PQ\)中点坐标为\((\frac{-2 + 4}{2},\frac{-3 + 5}{2})=(1, 1)\)。
\end{MyAnswer}



\begin{Exercise}[title={倾斜角相关题目}, label={ex:inclination_angle_radian}] %全是弧度制
    \Question 直线的倾斜角为\(0\),求该直线的斜率。
    \Question 已知直线的倾斜角为\(\frac{\pi}{3}\),求该直线的斜率。
    \Question 直线的斜率为\(\sqrt{3}\),求其倾斜角。
    \Question 若直线的斜率为\(-\frac{\sqrt{3}}{3}\),求其倾斜角。
    \Question 已知直线过点\((1,2)\)与\((3,4)\),求该直线的倾斜角。
\end{Exercise}
\begin{MyAnswer}[ref={ex:inclination_angle_radian}]
    \Question \mybox{答案为\(0\);}\\ 解:已知直线倾斜角\(\alpha = 0\),根据直线斜率\(k = \tan\alpha\),可得\(k=\tan0 = 0\)。
    \Question \mybox{答案为\(\sqrt{3}\);}\\ 解:已知直线倾斜角\(\alpha=\frac{\pi}{3}\),根据直线斜率\(k = \tan\alpha\),可得\(k = \tan\frac{\pi}{3}=\sqrt{3}\)。
    \Question \mybox{答案为\(\frac{\pi}{3}\);}\\ 解:设倾斜角为\(\alpha\),\(0\leq\alpha<\pi\),已知直线斜率\(k = \sqrt{3}\),因为\(k = \tan\alpha\),所以\(\tan\alpha=\sqrt{3}\),则\(\alpha=\frac{\pi}{3}\)。
    \Question \mybox{答案为\(\frac{5\pi}{6}\);}\\ 解:设倾斜角为\(\alpha\),\(0\leq\alpha<\pi\),已知直线斜率\(k = -\frac{\sqrt{3}}{3}\),因为\(k = \tan\alpha\),所以\(\tan\alpha=-\frac{\sqrt{3}}{3}\),则\(\alpha=\frac{5\pi}{6}\)。
    \Question \mybox{答案为\(\frac{\pi}{4}\);}\\ 解:设过点\((x_1,y_1)=(1,2)\)与\((x_2,y_2)=(3,4)\)直线的斜率为\(k\),根据斜率公式\(k=\frac{y_2 - y_1}{x_2 - x_1}\),则\(k=\frac{4 - 2}{3 - 1}=\frac{2}{2}=1\)。设倾斜角为\(\alpha\),\(0\leq\alpha<\pi\),且\(k = \tan\alpha\),所以\(\tan\alpha = 1\),可得\(\alpha=\frac{\pi}{4}\)。
\end{MyAnswer}












\begin{Exercise}[title={直线斜率计算小练习}, label={ex:line_slope}]
    \Question 已知直线经过点\(A(2, 3)\)和\(B(4, 7)\),求该直线的斜率。
    \Question 若直线过点\(M(-1, 5)\)与\(N(3, -3)\),计算此直线的斜率。
    \Question 设直线经过\(P(5, 2)\)和\(Q(5, -4)\),求直线的斜率。
    \Question 已知直线经过点\(C(0, 1)\)和\(D(3, 4)\),求该直线的斜率。
    \Question 设直线经过\(G(1, 2)\)和\(H(1, 5)\),求直线的斜率。
\end{Exercise}
\begin{MyAnswer}[ref={ex:line_slope}]
    \Question \mybox{答案为\(2\);}\\ 解:根据直线斜率公式\(k = \frac{y_2 - y_1}{x_2 - x_1}\),已知点\(A(2, 3)\)和\(B(4, 7)\),则直线\(AB\)的斜率\(k_{AB}=\frac{7 - 3}{4 - 2}=\frac{4}{2}=2\)。

    \Question \mybox{答案为\(-2\);}\\ 解:由直线斜率公式,对于点\(M(-1, 5)\)与\(N(3, -3)\),直线\(MN\)的斜率\(k_{MN}=\frac{-3 - 5}{3 - (-1)}=\frac{-8}{4}=-2\)。

    \Question  \mybox{直线斜率不存在;}\\  解:对于点\(P(5, 2)\)和\(Q(5, -4)\),此时\(x_1 = x_2 = 5\),在直线斜率公式\(k = \frac{y_2 - y_1}{x_2 - x_1}\)中,分母\(x_2 - x_1 = 0\),因为分母不能为\(0\),所以该直线的斜率不存在。

    \Question \mybox{答案为\(1\);}\\ 解:根据直线斜率公式,对于点\(C(0, 1)\)和\(D(3, 4)\),直线\(CD\)的斜率\(k_{CD}=\frac{4 - 1}{3 - 0}=\frac{3}{3}=1\)。


    \Question \mybox{直线斜率不存在;}\\ 解:对于点\(G(1, 2)\)和\(H(1, 5)\),此时\(x_1 = x_2 = 1\),在直线斜率公式\(k = \frac{y_2 - y_1}{x_2 - x_1}\)中,分母\(x_2 - x_1 = 0\),所以该直线的斜率不存在(这两个点在同一条竖线上!!)。

\end{MyAnswer}






\begin{Exercise}[title={直线方程相关练习}, label={ex:line_equations}]
    \Question 已知直线过点\((2,3)\),斜率为\(4\),求直线的点斜式方程,并将其化为一般式方程。
    \Question 直线斜率为\(-2\),在\(y\)轴上的截距为\(5\),写出直线的斜截式方程,并转化为一般式方程。
    \Question 已知直线过点\((-1,4)\),倾斜角为\(135^{\circ}\),求直线的点斜式方程,再化为斜截式方程。
    \Question 直线经过点\((3, -1)\)与\((1,2)\),先求直线的斜率,再写出直线的点斜式方程,最后化为一般式方程。
    \Question 写出直线\(3x - 2y + 6 = 0\)的斜截式方程,求出直线的斜率以及在\(y\)轴上的截距。
\end{Exercise}
\begin{MyAnswer}[ref={ex:line_equations}]
    \Question
        \mybox{点斜式方程为\(y - 3 = 4(x - 2)\),一般式方程为\(4x - y - 5 = 0\);}\\
        解:
        1. 点斜式方程的形式为\(y - y_1 = k(x - x_1)\),已知点\((2,3)\),斜率\(k = 4\),则点斜式方程为\(y - 3 = 4(x - 2)\)。
        2. 将点斜式方程\(y - 3 = 4(x - 2)\)展开并移项化为一般式:
        \(y - 3 = 4x - 8\),移项可得\(4x - y - 8 + 3 = 0\),即\(4x - y - 5 = 0\)。
    \Question
        \mybox{斜截式方程为\(y = -2x + 5\),一般式方程为\(2x + y - 5 = 0\);}\\
        解:
        1. 斜截式方程的形式为\(y = kx + b\),已知斜率\(k = -2\),\(y\)轴截距\(b = 5\),则斜截式方程为\(y = -2x + 5\)。
        2. 将斜截式方程\(y = -2x + 5\)移项化为一般式:\(2x + y - 5 = 0\)。
    \Question
        \mybox{点斜式方程为\(y - 4 = -(x + 1)\),斜截式方程为\(y = -x + 3\);}\\
        解:
        1. 已知倾斜角为\(135^{\circ}\),则斜率\(k=\tan135^{\circ}=-1\)。直线过点\((-1,4)\),根据点斜式方程\(y - y_1 = k(x - x_1)\),可得点斜式方程为\(y - 4 = -(x + 1)\)。
        2. 将点斜式方程\(y - 4 = -(x + 1)\)展开并整理为斜截式:\(y - 4 = -x - 1\),移项可得\(y = -x + 3\)。
    \Question
        \mybox{斜率为\(-\frac{3}{2}\),点斜式方程为\(y + 1 = -\frac{3}{2}(x - 3)\),}\\
        \mybox{一般式方程为\(3x + 2y - 7 = 0\);}\\
        解:
        1. 直线斜率\(k=\frac{y_2 - y_1}{x_2 - x_1}\),已知点\((3, -1)\)与\((1,2)\),则\(k=\frac{2 - (-1)}{1 - 3}=\frac{3}{-2}=-\frac{3}{2}\)。
        2. 直线过点\((3, -1)\),斜率\(k = -\frac{3}{2}\),根据点斜式方程\(y - y_1 = k(x - x_1)\),可得点斜式方程为\(y + 1 = -\frac{3}{2}(x - 3)\)。
        3. 将点斜式方程\(y + 1 = -\frac{3}{2}(x - 3)\)展开并移项化为一般式:
        \(y + 1 = -\frac{3}{2}x+\frac{9}{2}\),两边同乘\(2\)得\(2y + 2 = -3x + 9\),移项可得\(3x + 2y - 7 = 0\)。
    \Question
        \mybox{斜截式方程为\(y=\frac{3}{2}x + 3\),斜率为\(\frac{3}{2}\),\(y\)轴截距为\(3\);}\\
        解:
        1. 将直线方程\(3x - 2y + 6 = 0\)移项化为斜截式:
        \(-2y=-3x - 6\),两边同除以\(-2\)得\(y=\frac{3}{2}x + 3\)。
        2. 由斜截式方程\(y=\frac{3}{2}x + 3\)可知,直线的斜率\(k=\frac{3}{2}\),在\(y\)轴上的截距\(b = 3\)。
\end{MyAnswer}




\begin{Exercise}[title={两条直线的位置关系小练习}, label={ex:line_position},difficulty=1]
    \Question 判断直线 $y = 2x + 3$ 与直线 $y = 2x - 5$ 的位置关系。
    \Question 判断直线 $y = 3x + 2$ 与直线 $y=-\frac{1}{3}x - 1$ 的位置关系。
    \Question 判断直线 $2x - y + 1 = 0$ 与直线 $4x - 2y + 2 = 0$ 的位置关系。
    \Question 判断直线 $x + y - 3 = 0$ 与直线 $x - y + 1 = 0$ 的位置关系。
    \Question 已知直线 $l_1:y = k_1x + 1$ 与直线 $l_2:y = k_2x - 2$ 平行,且 $k_1 = 3$,求 $k_2$ 的值。
    \Question 已知直线 $l_1:2x + my - 1 = 0$ 与直线 $l_2:mx + 8y + 2 = 0$ 垂直,求 $m$ 的值。
    \Question 判断直线 $3x - 4y + 5 = 0$ 与直线 $6x - 8y + 10 = 0$ 的位置关系。
    \Question 判断直线 $x - 2y + 3 = 0$ 与直线 $2x - 4y - 1 = 0$ 的位置关系。
    \Question 已知直线 $l_1:y = 2x + b_1$ 与直线 $l_2:y = 2x + b_2$ 重合,且 $b_1 = 4$,求 $b_2$ 的值。
    \Question 判断直线 $5x + 12y - 1 = 0$ 与直线 $12x - 5y + 3 = 0$ 的位置关系。
\end{Exercise}
\begin{MyAnswer}[ref={ex:line_position}]
    \Question \mybox{答案为平行;}\\ 解:对于直线 $y = k_1x + b_1$ 和直线 $y = k_2x + b_2$,若 $k_1 = k_2$ 且 $b_1\neq b_2$,则两直线平行。在直线 $y = 2x + 3$ 与直线 $y = 2x - 5$ 中,$k_1 = k_2 = 2$,$b_1 = 3$,$b_2=-5$,$b_1\neq b_2$,所以两直线平行。
    \Question \mybox{答案为垂直;}\\ 解:若两条直线的斜率 $k_1$ 和 $k_2$ 满足 $k_1k_2=-1$,则两直线垂直。直线 $y = 3x + 2$ 的斜率 $k_1 = 3$,直线 $y=-\frac{1}{3}x - 1$ 的斜率 $k_2=-\frac{1}{3}$,$k_1k_2=3\times(-\frac{1}{3})=-1$,所以两直线垂直。
    \Question \mybox{答案为重合;}\\ 解:将直线 $4x - 2y + 2 = 0$ 两边同时除以 $2$,可得 $2x - y + 1 = 0$,与另一条直线方程完全相同,所以两直线重合。
    \Question \mybox{答案为相交;}\\ 解:直线 $x + y - 3 = 0$ 可化为 $y=-x + 3$,斜率 $k_1=-1$;直线 $x - y + 1 = 0$ 可化为 $y=x + 1$,斜率 $k_2 = 1$。因为 $k_1\neq k_2$,所以两直线相交。
    \Question \mybox{答案为 $3$;}\\ 解:若两条直线 $y = k_1x + b_1$ 与 $y = k_2x + b_2$ 平行,则 $k_1 = k_2$。已知 $k_1 = 3$,所以 $k_2 = 3$。
    \Question \mybox{答案为 $m = 0$;}\\ 解:当两条直线 $A_1x + B_1y + C_1 = 0$ 与 $A_2x + B_2y + C_2 = 0$ 垂直时,有 $A_1A_2 + B_1B_2 = 0$。对于直线 $l_1:2x + my - 1 = 0$ 与直线 $l_2:mx + 8y + 2 = 0$,则 $2m+8m = 0$,即 $10m = 0$,解得 $m = 0$。
    \Question \mybox{答案为重合;}\\ 解:将直线 $6x - 8y + 10 = 0$ 两边同时除以 $2$,得到 $3x - 4y + 5 = 0$,与另一条直线方程相同,所以两直线重合。
    \Question \mybox{答案为平行;}\\ 解:直线 $x - 2y + 3 = 0$ 可化为 $y=\frac{1}{2}x+\frac{3}{2}$,直线 $2x - 4y - 1 = 0$ 可化为 $y=\frac{1}{2}x-\frac{1}{4}$,两直线斜率相等都为 $\frac{1}{2}$,但截距 $\frac{3}{2}\neq-\frac{1}{4}$,所以两直线平行。
    \Question \mybox{答案为 $4$;}\\ 解:若两条直线 $y = k_1x + b_1$ 与 $y = k_2x + b_2$ 重合,则 $k_1 = k_2$ 且 $b_1 = b_2$。已知 $k_1 = k_2 = 2$,$b_1 = 4$,所以 $b_2 = 4$。
    \Question \mybox{答案为垂直;}\\ 解:直线 $5x + 12y - 1 = 0$ 可化为 $y=-\frac{5}{12}x+\frac{1}{12}$,斜率 $k_1=-\frac{5}{12}$;直线 $12x - 5y + 3 = 0$ 可化为 $y=\frac{12}{5}x+\frac{3}{5}$,斜率 $k_2=\frac{12}{5}$。$k_1k_2=-\frac{5}{12}\times\frac{12}{5}=-1$,所以两直线垂直。
\end{MyAnswer}










\begin{Exercise}[title={圆的基本性质练习}, label={ex:circle-properties},difficulty=2]
    \Question 求以点$(2, -3)$为圆心,半径为$5$的圆的标准方程.
    \Question 已知圆的方程为$(x + 1)^2+(y - 2)^2 = 9$,求圆心坐标和半径.
    \Question 若圆经过点$A(1,2)$,$B(3,4)$,且圆心在直线$x - y + 1 = 0$上,求该圆的方程.
    \Question 求过点$(0,0)$,$(1,1)$,$(2,0)$的圆的方程.
    \Question 圆$x^2 + y^2 - 4x + 6y - 3 = 0$的圆心坐标和半径分别是多少?
\end{Exercise}

\begin{MyAnswer}[ref={ex:circle-properties}]
    \Question 解:根据圆标准方程$(x - a)^2+(y - b)^2 = r^2$,$a = 2$,$b=-3$,$r = 5$,得$(x - 2)^2+(y + 3)^2 = 25$.
    
    \Question 解:圆标准方程$(x - a)^2+(y - b)^2 = r^2$,此方程中$a=-1$,$b = 2$,$r = 3$,所以圆心$(-1,2)$,半径$3$.
    
    \Question 解:设圆方程$(x - a)^2+(y - b)^2 = r^2$,由已知得
    $$
    \begin{cases}
    (1 - a)^2+(2 - b)^2 = r^2 \\
    (3 - a)^2+(4 - b)^2 = r^2 \\
    a - b+1 = 0
    \end{cases}
    $$
    前两式相减得$a + b = 5$,联立
    $$
    \begin{cases}
    a + b = 5 \\
    a - b+1 = 0
    \end{cases}
    $$
    解得$a = 2$,$b = 3$,$r^2 = 2$,圆方程为$(x - 2)^2+(y - 3)^2 = 2$.
    
    \Question 解:设圆一般方程$x^{2}+y^{2}+Dx + Ey+F = 0$,代入三点得
    $$
    \begin{cases}
    F = 0 \\
    1 + 1+D + E+F = 0 \\
    4+2D+F = 0
    \end{cases}
    $$
    解得$D=-2$,$E = 0$,$F = 0$,圆方程为$x^{2}+y^{2}-2x = 0$.
    
    \Question 解:配方得
    $$
    x^{2}-4x + 4+y^{2}+6y+9=3 + 4+9
    $$
    即$(x - 2)^2+(y + 3)^2 = 16$,圆心$(2,-3)$,半径$4$.
\end{MyAnswer}







\clearpage

\section{下面内容期中考试不考}
\begin{Exercise}[title={圆与直线位置关系练习1}, label={ex:line-circle}]
    \Question 判断直线 $ y = x + 1 $ 与圆 $ x^2 + y^2 = 2 $ 的位置关系(相交、相切、相离).
    \Question 求直线 $ 3x + 4y - 5 = 0 $ 与圆 $ (x-1)^2 + (y+2)^2 = 4 $ 的圆心到直线的距离,并判断位置关系.
    \Question 若直线 $ y = kx + 2 $ 与圆 $ x^2 + y^2 = 1 $ 相切,求实数 $ k $ 的值.
\end{Exercise}
\begin{MyAnswer}[ref={ex:line-circle}]
        \Question 解:将直线方程代入圆的方程:
        $$
        x^2 + (x + 1)^2 = 2 \implies 2x^2 + 2x - 1 = 0
        $$
        判别式 $ \Delta = 4 + 8 = 12 > 0 $,因此直线与圆相交.

        或者圆心到直线的距离为 
        $$    
        d = \frac{|0-0+1|}{\sqrt{1^2 + (-1)^2}} = \frac{\sqrt{2}}{2} < \sqrt{2} =r
        $$
        所以相交.
        \Question 解:圆心为 $ (1, -2) $,半径 $ r = 2 $.计算圆心到直线的距离:
        $$
        d = \frac{|3 \times 1 + 4 \times (-2) - 5|}{\sqrt{3^2 + 4^2}} = \frac{10}{5} = 2
        $$
        因为 $ d = r $,所以直线与圆相切.

        \Question 解:直线与圆相切时,距离等于半径:
        $$
        \frac{|2|}{\sqrt{k^2 + 1}} = 1 \implies \sqrt{k^2 + 1} = 2 \implies k^2 = 3 \implies k = \pm \sqrt{3}
        $$
\end{MyAnswer}


\begin{Exercise}[title={圆与直线位置关系练习2}, label={ex:circle-line2}]
    \Question 已知直线 $y = kx + 1$ 与圆 $x^2 + y^2 = 4$ 相交,求 $k$ 的取值范围.
    \Question 求圆心在 $(2,-1)$ 且与直线 $3x - 4y + 5 = 0$ 相切的圆的方程.
\end{Exercise}

\begin{MyAnswer}[ref={ex:circle-line2}]


    \Question \mybox{答案为 $-\frac{\sqrt{3}}{2} < k < \frac{\sqrt{3}}{2}$;}\\ 
    解:将直线方程代入圆的方程:
    $$
    x^2 + (kx + 1)^2 = 4 \Rightarrow (1 + k^2)x^2 + 2kx - 3 = 0
    $$
    由判别式 $\Delta > 0$ 得:
    $$
    (2k)^2 - 4(1 + k^2)(-3) > 0 \Rightarrow 16k^2 < 12
    $$
    解得 $k$ 的范围.

    \Question \mybox{答案为 $(x-2)^2 + (y+1)^2 = 1$;}\\ 
    解:计算圆心到直线的距离即为半径:
    $$
    r = \frac{|3 \times 2 - 4 \times (-1) + 5|}{\sqrt{3^2 + 4^2}} = \frac{15}{5} = 3
    $$
    故圆的方程为 $(x-2)^2 + (y+1)^2 = 9$.
\end{MyAnswer}
% 根式 和 分数要怎么比较大小