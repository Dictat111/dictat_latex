\documentclass{ctexart}
\usepackage{amsmath}
\usepackage[fontsize=20pt]{fontsize}
\usepackage{fancyhdr}
\pagestyle{fancy}
\fancyhf{} % 清除当前头部和脚部的所有内容
\renewcommand{\headrulewidth}{0pt} % 去掉上边线
% \cfoot{\large\thepage} % 将页码置于中央位置

\usepackage{xeCJK} % 用于处理中文排版
\setCJKmainfont[AutoFakeBold=true]{STKaiti} % 设置中文主字体为华文楷体 %AutoFakeBold 添加加粗的效果(自动伪粗体),能使用 \textbf了
\usepackage{fontspec}
% 设置英文主字体为 Times New Roman
\setmainfont{Times New Roman}
\begin{document}
唐朝书法家:
\begin{enumerate}
    \item \textbf{欧阳询}:初唐书法家,其书法风格险峻严谨,结构精妙,被称为“欧体”。代表作有《九成宫醴泉铭》,此碑书法高华庄重,法度森严,用笔刚劲峻拔,结构紧凑,是楷书的典范之作,对后世书法影响深远。
    \item \textbf{虞世南}:初唐书法家,其书法风格温润典雅,外柔内刚。代表作《孔子庙堂碑》,笔画圆润,气息宁静,展现出一种雍容华贵的气质。
    \item \textbf{褚遂良}:初唐书法家,他的书法线条流畅,变化丰富,具有独特的艺术魅力。代表作《雁塔圣教序》,字体清丽刚劲,灵动多姿,在书法史上具有重要地位。
    \item \textbf{颜真卿}:中唐书法家,其书法风格雄浑壮美,气势磅礴,对后世书法发展产生了深远影响。他的楷书代表作《颜勤礼碑》,结体端庄,宽博厚重,笔画刚健有力,具有强烈的艺术感染力;行书代表作《祭侄文稿》,被誉为“天下第二行书”,此作情感真挚,笔势奔放,是书法与情感完美结合的典范。
    \item \textbf{柳公权}:晚唐书法家,以楷书著称,其书法风格骨力劲健,结构严谨,有“颜筋柳骨”之称。代表作《玄秘塔碑》,笔画刚劲挺拔,结构紧凑规整,是学习楷书的经典范本之一。
    \item \textbf{张旭}:盛唐时期书法家,以草书闻名,被尊称为“草圣”。他的草书风格豪放不羁,气势恢宏,常常在醉酒后挥毫泼墨,其作品《古诗四帖》笔画连绵不绝,跌宕起伏,充满了艺术的张力和激情。
    \item \textbf{怀素}:唐代著名的草书书法家,其草书风格灵动飞逸,笔法瘦劲。代表作《自叙帖》,是他草书的巅峰之作,该帖用笔圆转自如,线条流畅,如行云流水,展现出极高的艺术造诣。
\end{enumerate}
\end{document}


% convert -density 150 -quality 90 input.pdf output-%3d.jpg

% 3d用于指定生成图像的文件名格式,生成的图像将被命名为output-001.jpg、output-002.jpg等

% 转化第三页为图片
% convert input.pdf[2] output.jpg